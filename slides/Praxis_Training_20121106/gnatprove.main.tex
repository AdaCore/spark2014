\documentclass{beamer}

\usepackage{xspace}
\usepackage{xcolor}
\usepackage{eurosym}
\usepackage[utf8]{inputenc}
\usepackage{tikz}
\usetikzlibrary{arrows,positioning}
\usepackage{pgflibraryshapes} % for ellipse shape

%\usepackage{beamerthemesplit}

\usepackage[absolute,overlay]{textpos}
\TPGrid{15}{10}

\newcommand{\vs}{\vspace{0.5cm}}

\newcommand{\resp}{resp.\xspace}
\newcommand{\etal}{\textit{et al.}\xspace}
\newcommand{\etc}{\textit{etc.}\xspace}
\newcommand{\ie}{\textit{i.e.,}\xspace}
\newcommand{\eg}{\textit{e.g.,}\xspace}
\newcommand{\wrt}{w.r.t.\xspace}

\definecolor{mygreen}{rgb}{0,0.7,0}

\usepackage{listings}
\usepackage{color}
\lstset{
	language=Ada,
	keywordstyle=\bfseries\ttfamily\color[rgb]{0,0,1},
	identifierstyle=\ttfamily,
	commentstyle=\color[rgb]{0.133,0.545,0.133},
	stringstyle=\ttfamily\color[rgb]{0.627,0.126,0.941},
        morekeywords=[1]some,
	showstringspaces=false,
	basicstyle=\tiny,
	numberstyle=\tiny,
	numbers=left,
	stepnumber=1,
	numbersep=10pt,
	tabsize=2,
	breaklines=true,
	prebreak = \raisebox{0ex}[0ex][0ex]{\ensuremath{\hookleftarrow}},
	breakatwhitespace=false,
	aboveskip={1.5\baselineskip},
  columns=fixed,
  extendedchars=true,
% frame=single,
% backgroundcolor=\color{lbcolor},
}

% special frames used to put source-code listings
\newenvironment{specialframe}{%
  \begin{frame}[fragile,environment=specialframe]}{\end{frame}}

\xdefinecolor{adacoreblue}{rgb}{0,0.34,0.59}
\xdefinecolor{adacoregrey}{rgb}{0.53,0.68,0.84}

\AtBeginSection[]{\frame{\frametitle{Outline}
\tableofcontents[current]}}
\AtBeginSubsection[]{\frame{\frametitle{Outline}
\tableofcontents[currentsection,currentsubsection]}}

\setbeamertemplate{footline}[page number]
\setbeamercolor{frametitle}{bg=adacoreblue!40!adacoregrey, fg=white}
\setbeamercolor{section in toc}{fg=adacoreblue}
\setbeamercolor{block title}{bg=adacoregrey, fg=white}
\setbeamertemplate{navigation symbols}{}
\setbeamercovered{transparent}
\setbeamertemplate{footline}
{%
  \hfill \insertframenumber\ / \inserttotalframenumber%
}

\begin{document}

%%%%%%%%%%%%%%%%%%%%%%%%%%%%%%%%%%%%%%%%%%%%%%%%%%%%%%%%%%%%%%%%%%

% types: abstract types, predefined types, records, algebraic data types
% logic: function declarations, definitions, predicates, axioms
% theories

\section{gnatwhy3}

\begin{specialframe}\frametitle{Overview}
   gnatwhy3 drives...
      \begin{itemize}
      \item generation of VCs
      \item prover runs
      \item persistence of prover results
      \item user-readable error messages
      \end{itemize}
      It does this for one Why file (the \verb|__package.mlw| file for an Ada
      unit)
\end{specialframe}

\begin{specialframe}\frametitle{How it works (1)}
   \begin{block}{Structure}
   \begin{itemize}
   \item gnatwhy3 is written in OCaml
   \item gnatwhy3 is a client of the Why library in OCaml
   \item calls the why3 parser to read the file
   \item calls the why3 VCgen to obtain formulas
   \end{itemize}
   \end{block}

   \begin{block}{VCs and objectives}
   \begin{itemize}
   \item Why VCgen generates one formula per subprogram
   \item \emph{splitting} that formula generates a formula per path and per
     \emph{check}
   \item we group all VCs by \emph{check} on the Ada side (objective)
   \end{itemize}
   \end{block}

\end{specialframe}

\begin{specialframe}\frametitle{How it works (2)}

   \begin{block}{Objective computation}
      \begin{itemize}
         \item split big formula to obtain VCs for paths + checks
         \item look at the VC labels to see which objective it corresponds to
         \item possibly filter VCs (by suprogram, by line, \etc)
      \end{itemize}
   \end{block}
   \begin{block}{For each objective}
      \begin{itemize}
         \item try to prove all VCs, stop at the first unproved
         \item if possible, split conjunctions and try again
         \item issue message (proved/not proved) for the objective
      \end{itemize}
   \end{block}

\end{specialframe}

\begin{specialframe}\frametitle{Fast WP (software present tense)}

   Changes shape and size of formula
      \begin{itemize}
      \item size of the regular formula is exponential (linear in number of paths)
      \item new scheme is mostly linear (quadratic in theory)
      \item splitting that new formula will provide one VC per objective
      \item splitting conjuncts still possible
      \item left-splitting will provide paths (but is exponential again)
      \end{itemize}

\end{specialframe}

\section{gnatprove}

\begin{specialframe}\frametitle{Overview}
   \begin{itemize}
      \item gnatprove is the project aware wrapper for gcc, gnat2why and
      gnatwhy3
      \item it does nothing intelligent by itself
   \end{itemize}

\vs

   \begin{block}{Phases}
      \begin{enumerate}
         \item ALI file generation - gnatprove calls gprbuild, which calls gcc
         on all relevant Ada files
         \item Translation to Why - gnatprove calls gprbuild, which calls
         gnat2why on all relevant Ada files
         \item Proving - gnatprove calls gprbuild, which calls gnatwhy3 on all
         relevant \verb|.mlw| files
      \end{enumerate}
   \end{block}
\end{specialframe}

\begin{specialframe}\frametitle{Dependency Computation}
it is a subtle point
      \begin{itemize}
         \item is different for gnat2why because of effect computation
         \item we use a special mode of gprbuild for that: \verb|ALI_Closure|
         \item still a bit buggy ...
      \end{itemize}
\end{specialframe}

\begin{specialframe}\frametitle{Statistics}
really meant for helping early users
      \begin{itemize}
         \item always run and result saved in \verb|gnatprove.out|
         \item fine-grain details in separate files \verb|unit.alfa|
         \item number and percentage of subprograms in Alfa or not in Alfa
         \item units with largest scores
      \end{itemize}
\vs

it could be useful for final users
\end{specialframe}

\end{document}
