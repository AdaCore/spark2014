\documentclass{beamer}

\usepackage{amsthm}
\usepackage{xcolor}
\usepackage{eurosym}
\usepackage[utf8]{inputenc}
\usepackage{tikz}
\usetikzlibrary{arrows,positioning}
\usepackage{pgflibraryshapes} % for ellipse shape

%\usepackage{beamerthemesplit}

\usepackage[absolute,overlay]{textpos}
\TPGrid{15}{10}

\newcommand{\vs}{\vspace{0.5cm}}

\definecolor{mygreen}{rgb}{0,0.7,0}

\usepackage{listings}
\usepackage{color}
\lstset{
	language=Ada,
	keywordstyle=\bfseries\ttfamily\color[rgb]{0,0,1},
	identifierstyle=\ttfamily,
	commentstyle=\color[rgb]{0.133,0.545,0.133},
	stringstyle=\ttfamily\color[rgb]{0.627,0.126,0.941},
        morekeywords=[1]some,
	showstringspaces=false,
	basicstyle=\tiny,
	numberstyle=\tiny,
	numbers=left,
	stepnumber=1,
	numbersep=10pt,
	tabsize=2,
	breaklines=true,
	prebreak = \raisebox{0ex}[0ex][0ex]{\ensuremath{\hookleftarrow}},
	breakatwhitespace=false,
	aboveskip={1.5\baselineskip},
  columns=fixed,
  extendedchars=true,
% frame=single,
% backgroundcolor=\color{lbcolor},
}

% special frames used to put source-code listings
\newenvironment{specialframe}{%
  \begin{frame}[fragile,environment=specialframe]}{\end{frame}}

\setbeamertemplate{theorems}[numbered]
\newtheorem{exercise}{Exercise}

\newenvironment{answer}[1][Answer]{\begin{trivlist}
\item[\hskip \labelsep {\bfseries #1}]}{\end{trivlist}}

\xdefinecolor{adacoreblue}{rgb}{0,0.34,0.59}
\xdefinecolor{adacoregrey}{rgb}{0.53,0.68,0.84}

\AtBeginSection[]{\frame{\frametitle{Outline}
\tableofcontents[current]}}
\AtBeginSubsection[]{\frame{\frametitle{Outline}
\tableofcontents[currentsection,currentsubsection]}}

\setbeamertemplate{footline}[page number]
\setbeamercolor{frametitle}{bg=adacoreblue!40!adacoregrey, fg=white}
\setbeamercolor{section in toc}{fg=adacoreblue}
\setbeamercolor{block title}{bg=adacoregrey, fg=white}
\setbeamertemplate{navigation symbols}{}
\setbeamercovered{transparent}
\setbeamertemplate{footline}
{%
  \hfill \insertframenumber\ / \inserttotalframenumber%
}

\begin{document}

%%%%%%%%%%%%%%%%%%%%%%%%%%%%%%%%%%%%%%%%%%%%%%%%%%%%%%%%%%%%%%%%%%
\begin{specialframe}
  \frametitle{Compilation Phases}

\begin{itemize}
\item lexing: \verb|Scn| (mostly need to modify \verb|snames.ads| for S14)
\item parsing: \verb|Par| (no work for S14 except for error recovery)
  \begin{itemize}
  \item \verb|Par.Ch2| .. \verb|Par.Ch13| match RM chapters
  \item \verb|Par.Ch4| for attributes
  \item \verb|Par.Ch13| for aspects
  \item \verb|Par.Prag| for pragmas
  \end{itemize}
\item semantic analysis: \verb|Sem| (where most work for S14 is expected)
  \begin{itemize}
  \item \verb|Sem_Ch2| .. \verb|Sem_Ch13| match RM chapters
  \item \verb|Sem_Prag| for pragmas/aspects
  \item \verb|Sem_Attr| for attributes
  \end{itemize}
\item expansion: \verb|Exp| (little work for S14)
  \begin{itemize}
  \item \verb|Exp_Ch2| .. \verb|Exp_Ch13| match RM chapters
  \item \verb|Exp_Prag| for pragmas/aspects
  \item \verb|Exp_Attr| for attributes
  \end{itemize}
\end{itemize}

\end{specialframe}

%%%%%%%%%%%%%%%%%%%%%%%%%%%%%%%%%%%%%%%%%%%%%%%%%%%%%%%%%%%%%%%%%%
\begin{specialframe}
  \frametitle{\textit{Possibly} and \textbf{Certainly} Relevant RM Chapters}

\begin{enumerate}
\item General
\item Lexical Elements
\item Declarations and Types
\item \textbf{Names and Expressions}
\item \textbf{Statements}
\item \textbf{Subprograms}
\item \textbf{Packages}
\item \textit{Visibility Rules}
\item Tasks and Synchronisation
\item Program Structure and Compilation Issues
\item Exceptions
\item Generic Units
\item \textit{Representation Issues}
\end{enumerate}

\end{specialframe}

%%%%%%%%%%%%%%%%%%%%%%%%%%%%%%%%%%%%%%%%%%%%%%%%%%%%%%%%%%%%%%%%%%
\begin{specialframe}
  \frametitle{Exercises}

\begin{exercise}
  Which procedures do the semantic analysis of expression functions?
\end{exercise}

\visible<2>{\begin{answer}
in \texttt{sem\_ch6.adb}: \texttt{Analyze\_Expression\_Function},
\texttt{Analyze\_Subprogram\_Declaration},
\texttt{Analyze\_Subprogram\_Specification},
\texttt{Analyze\_Subprogram\_Body},
\texttt{Analyze\_Subprogram\_Body\_Helper}
\end{answer}}

\begin{exercise}
  Which procedure does the semantic analysis of attribute \verb|'Old|?
\end{exercise}

\visible<2>{\begin{answer}
\texttt{Sem\_Attr.Analyze\_Attribute}
\end{answer}}

\begin{exercise}
  Which procedure does the expansion of attribute \verb|'Old|?
\end{exercise}

\visible<2>{\begin{answer}
\texttt{Exp\_Attr.Expand\_N\_Attribute\_Reference}
\end{answer}}

\end{specialframe}

%%%%%%%%%%%%%%%%%%%%%%%%%%%%%%%%%%%%%%%%%%%%%%%%%%%%%%%%%%%%%%%%%%
\begin{specialframe}
  \frametitle{GNAT Abstract Syntax Tree}

\begin{block}{internal nodes}
  \begin{itemize}
  \item \verb|Sinfo|, 226 values in kind \verb|Sinfo.Node_Kind|
  \item tree structure provided by syntactic fields
  \item function \verb|Parent| points to tree root
  \item additional semantic fields
  \item 5 general purpose fields for other nodes, list of nodes, names
    literals, universal integers, floats, character codes
  \end{itemize}
\end{block}

\begin{block}{entity nodes}
  \begin{itemize}
  \item \verb|Einfo|, 79 values in kind \verb|Einfo.Entity_Kind|
  \item for all identifiers
  \item 23 general purpose fields
  \item many boolean flags
  \end{itemize}
\end{block}

\end{specialframe}

%%%%%%%%%%%%%%%%%%%%%%%%%%%%%%%%%%%%%%%%%%%%%%%%%%%%%%%%%%%%%%%%%%
\begin{specialframe}
  \frametitle{Internal AST Nodes}

how to describe a new node?
\begin{itemize}
\item GNAT Book (2.2.1 The Abstract Syntax Tree)
\item file \verb|sinfo.ads| (starts with 7500 lines of comments)
\end{itemize}

\vs

what to describe?
\begin{itemize}
\item source location (sloc) if any
\item syntactic fields and semantic fields
\item field default value if any
\end{itemize}

\vs

what else to know?
\begin{itemize}
\item same fields in 2 nodes use same general purpose field
\item constructors \verb|Make_<node>| auto-generated in \verb|Nmake|
\item files auto-generated by \verb|xsinfo|, \verb|xnmake| and \verb|xtreeprs|
\end{itemize}

\end{specialframe}

%%%%%%%%%%%%%%%%%%%%%%%%%%%%%%%%%%%%%%%%%%%%%%%%%%%%%%%%%%%%%%%%%%
\begin{specialframe}
  \frametitle{Exercises}

\begin{exercise}
  What are the node kinds for a subprogram declaration and specification?
\end{exercise}

\visible<2>{\begin{answer}
\texttt{N\_Subprogram\_Declaration, N\_Procedure\_Specification} and \texttt{N\_Function\_Specification}
\end{answer}}

\begin{exercise}
  What is the entity kind \verb|E_Subprogram_Body| used for?
\end{exercise}

\visible<2>{\begin{answer}
when subprogram body is separate from declaration
\end{answer}}

\begin{exercise}
  Add a new expression node \verb|N_Unknown|. What to update?
\end{exercise}

\visible<2>{\begin{answer}
in \texttt{sinfo.ads}: comments, \texttt{Node\_Kind}, node class definitions,
\texttt{Is\_Syntatic\_Field}, plus case statements to complete in
\texttt{sem.adb}, \texttt{sprint.adb}, \texttt{exp\_util.adb},
\texttt{sem\_res.adb}
\end{answer}}

\end{specialframe}

%%%%%%%%%%%%%%%%%%%%%%%%%%%%%%%%%%%%%%%%%%%%%%%%%%%%%%%%%%%%%%%%%%
\begin{specialframe}
  \frametitle{Error Detection}

see comments in \verb|errout.ads|

\begin{itemize}
\item redundant errors are suppressed
\item
  22 special characters used to insert information in the error message (sloc,
  name, etc.)
\item \verb|CODEFIX| comment used to indicate autofix in GPS/GNATbench
\end{itemize}

\vs

error messages are typically one line, but
\begin{itemize}
\item they can be continued on multiple lines (with $\backslash$ and $\backslash\backslash$)
\item messages on generic instances include prefix lines giving locations for
  the chain of instantiations
\end{itemize}

\vs

2 most useful error procedures:
\vspace{-0.5cm}
\begin{lstlisting}[language=ada]
  procedure Error_Msg_N (Msg : String; N : Node_Or_Entity_Id);
   -- Output a message at the Sloc of the given node

  procedure Error_Msg_F (Msg : String; N : Node_Id);
   --  Similar to Error_Msg_N except that the message is placed on the first
   --  node of the construct N (First_Node (N)).
\end{lstlisting}
\end{specialframe}

%%%%%%%%%%%%%%%%%%%%%%%%%%%%%%%%%%%%%%%%%%%%%%%%%%%%%%%%%%%%%%%%%%
\begin{specialframe}
  \frametitle{Alfa Mode}

mode for GNATprove
\begin{itemize}
\item \verb|gcc| called in this mode to generate cross-refs
\item \verb|gnat2why| called in this mode
\end{itemize}

\vs

triggered by debug switch \verb|-gnatd.F|

\vs

other relevant debug switches (see \verb|debug.adb|)
\begin{itemize}
\item \verb|-gnatd.V| - extensions for S14 (\verb|'Loop_Entry|, etc.)
\item \verb|-gnatd.D| - strict Alfa mode (switch \verb|--pedantic|)
\item \verb|-gnatd.E| - force Alfa mode (GNATprove mode \verb|force|)
\item \verb|-gnatd.G| - precondition only mode (GNATprove mode \verb|check|)
\item \verb|-gnatd.H| - special mode for package \verb|Standard|
\item \verb|-gnatd.K| - Alfa detection mode (GNATprove mode \verb|detect|)
\end{itemize}

\end{specialframe}

%%%%%%%%%%%%%%%%%%%%%%%%%%%%%%%%%%%%%%%%%%%%%%%%%%%%%%%%%%%%%%%%%%
\begin{specialframe}
  \frametitle{Alfa Mode in GNAT Frontend}

\verb|Opt.Alfa_Mode| set to True

\verb|Opt.Full_Expander_Active| returns False

\vs

47 uses of \verb|Alfa_Mode| in frontend

49 used of \verb|Full_Expander_Active| in frontend

\vs

regular scanning/parsing/semantic analysis
\verb|Exp_Alfa.Expand_Alfa| called for expansion

\vs

special expansion does little
\begin{itemize}
\item identifiers: add suffix to homonyms
\item calls: introduce temporaries for IN OUT arguments
\item renamings: replace by object being renamed
\end{itemize}

\end{specialframe}

%%%%%%%%%%%%%%%%%%%%%%%%%%%%%%%%%%%%%%%%%%%%%%%%%%%%%%%%%%%%%%%%%%
\begin{specialframe}
  \frametitle{Cross-Refs Generated in Alfa Mode }

ALI files store compiler-generated info on units (cross-refs)

general info on xrefs in \verb|lib-xref.ads|

info on special Alfa xrefs in \verb|alfa.ads|

\vs

regular xrefs not sufficient because:
\begin{itemize}
\item they do not identify subprogram scopes (for entities and refs)
\item they do not handle refs through pointers
\item they lack some refs (instances, renamings)
\end{itemize}

\vs

special xrefs based on regular ones:
\begin{itemize}
\item \verb|Lib.Xref.Generate_Definition| called on defs (as usual)
\item \verb|Lib.Xref.Generate_Reference| called on refs (as usual)
\item \verb|Generate_Dereference| called on dereference
\item references not ignored anymore (instances, renamings)
\item defs/refs include info on scope
\item \verb|Collect_Alfa| generates internal Alfa xrefs
\end{itemize}

\end{specialframe}

%%%%%%%%%%%%%%%%%%%%%%%%%%%%%%%%%%%%%%%%%%%%%%%%%%%%%%%%%%%%%%%%%%
\begin{specialframe}
  \frametitle{Exercises}

\begin{exercise}
  Compile a file normally and in Alfa mode. In the ALI file, what is the
  starting character for Alfa xrefs lines?
\end{exercise}

\visible<2>{\begin{answer}
F
\end{answer}}

\begin{exercise}
  Find xrefs for reads/writes/calls in the regular xrefs, and in Alfa xrefs.
\end{exercise}

\begin{exercise}
  Review the code of \verb|Gnat1drv.Adjust_Global_Switches| for the Alfa mode.
\end{exercise}

\end{specialframe}

%%%%%%%%%%%%%%%%%%%%%%%%%%%%%%%%%%%%%%%%%%%%%%%%%%%%%%%%%%%%%%%%%%
\begin{specialframe}
  \frametitle{Adding A New Pragma}

add two names in \verb|snames.ads-tmpl|

\vs

parsing in \verb|par-prag.adb|: usually nothing to do

\vs

semantic analysis in \verb|sem_prag.adb|
\begin{itemize}
\item add case in \verb|Analyze_Pragma|
\item initial comment gives the syntax in BNF
\item check cases in which pragma applies (Ada version? GNAT?)
\item check legality rules (parameters, placement)
\item if error, call \verb|Error_Pragma| that does not return
\end{itemize}

\vs

expansion in \verb|exp_prag.adb|
\begin{itemize}
\item initial comment gives expanded pseudo-code
\item build nodes with \verb|Make_<node>|
\item insert statements with \verb|Insert_Action|
\item rewrite node with \verb|Rewrite|
\end{itemize}

\end{specialframe}

%%%%%%%%%%%%%%%%%%%%%%%%%%%%%%%%%%%%%%%%%%%%%%%%%%%%%%%%%%%%%%%%%%
\begin{specialframe}
  \frametitle{Exercises}

\begin{exercise}
  How do we check that a pragma is only allowed in S14?
\end{exercise}

\visible<2>{\begin{answer}
call \texttt{S14\_Pragma}
\end{answer}}

\begin{exercise}
  What are the procedures to check arguments number/types?
\end{exercise}

\visible<2>{\begin{answer}
    \texttt{Check\_Arg\_Count},
    \texttt{Check\_At\_Least\_N\_Arguments},
    \texttt{Check\_At\_Most\_N\_Arguments},
    \texttt{Check\_Arg\_Order},
    \texttt{Check\_Arg\_Is\_...}
\end{answer}}

\begin{exercise}
  Review the code for analysis/expansion of \verb|Loop_Assertion|.
\end{exercise}

\end{specialframe}

%%%%%%%%%%%%%%%%%%%%%%%%%%%%%%%%%%%%%%%%%%%%%%%%%%%%%%%%%%%%%%%%%%
\begin{specialframe}
  \frametitle{Adding A New Aspect}

add one name in \verb|snames.ads-tmpl|

\vs

add an aspect kind in \verb|aspects.ads|, and follow instructions in comments

\vs

semantic analysis in \verb|sem_ch13.adb|
\begin{itemize}
\item most aspect rewritten as pragma or attribute definition clause
\item syntax usually given only for pragma
\item analysis is delayed for some aspects, to the freeze point of the
  corresponding entity
\end{itemize}

\vs

expansion in \verb|exp_ch13.adb|, \verb|exp_prag.adb|, \verb|sem_ch6.adb|, etc.

\end{specialframe}

%%%%%%%%%%%%%%%%%%%%%%%%%%%%%%%%%%%%%%%%%%%%%%%%%%%%%%%%%%%%%%%%%%
\begin{specialframe}
  \frametitle{Exercises}

\begin{exercise}
  Give two aspects do not have a corresponding pragma/attribute definition
  clause.
\end{exercise}

\visible<2>{\begin{answer}
e.g. \texttt{Default\_Value} and \texttt{Dimension}
\end{answer}}

\begin{exercise}
  How does one know when a pragma comes from an aspect?
\end{exercise}

\visible<2>{\begin{answer}
    \texttt{From\_Aspect\_Specification} returns True
\end{answer}}

\begin{exercise}
  Review the code for analysis/expansion of \verb|Pre| and \verb|Post| in
  \verb|sem_ch13.adb|, \verb|sem_prag.adb|, \verb|sem_ch6.adb|
  (\verb|Process_PCCs|), \verb|exp_prag.adb|.
\end{exercise}

\end{specialframe}

%%%%%%%%%%%%%%%%%%%%%%%%%%%%%%%%%%%%%%%%%%%%%%%%%%%%%%%%%%%%%%%%%%
\begin{specialframe}
  \frametitle{Adding A New Attribute}

add two names in \verb|snames.ads-tmpl|

\vs

parsing in \verb|par-ch4.adb| and \verb|par-ch13.adb|: usually nothing to do

\vs

semantic analysis in \verb|sem_attr.adb|
\begin{itemize}
\item entry in array \verb|Attribute_Impl_Def| in \verb|sem_attr.ads|
  together with a description
\item add case in \verb|Analyze_Attribute|
\item check cases in which pragma applies (Ada version? GNAT?)
\item check legality rules (parameters, prefix)
\item if error, call \verb|Error_Attr| procedures that does not return
\end{itemize}

\vs

expansion in \verb|exp_attr.adb|

\end{specialframe}

%%%%%%%%%%%%%%%%%%%%%%%%%%%%%%%%%%%%%%%%%%%%%%%%%%%%%%%%%%%%%%%%%%
\begin{specialframe}
  \frametitle{Exercises}

\begin{exercise}
  How does one check that an attribute is only used in Ada 2012?
\end{exercise}

\visible<2>{\begin{answer}
    call \texttt{Check\_Ada\_2012\_Attribute}
\end{answer}}

\begin{exercise}
  Where is the attribute \verb|'Old| allowed?
\end{exercise}

\visible<2>{\begin{answer}
in postcondition aspect/pragma, \texttt{Ensures} clause, and in the generated
\texttt{\_Postcondition} procedure
\end{answer}}

\begin{exercise}
  Review the code for analysis/expansion of \verb|Loop_Entry|.
\end{exercise}

\end{specialframe}

%%%%%%%%%%%%%%%%%%%%%%%%%%%%%%%%%%%%%%%%%%%%%%%%%%%%%%%%%%%%%%%%%%
\begin{specialframe}
  \frametitle{Adding A New Restriction}

add a value in \verb|Restriction_Id| in \verb|s-rident.ads|
\begin{itemize}
\item partition-wide restrictions listed first
\item then unit-level restrictions without value
\item then unit-level restrictions with value
\end{itemize}

\vs

semantic analysis in \verb|restrict.adb|
\begin{itemize}
\item setting of restriction with \verb|Set_Restriction|
\item check of restriction with \verb|Check_Restriction|
\end{itemize}

\end{specialframe}

%%%%%%%%%%%%%%%%%%%%%%%%%%%%%%%%%%%%%%%%%%%%%%%%%%%%%%%%%%%%%%%%%%
\begin{specialframe}
  \frametitle{Adding A New Profile}

add a value in \verb|Profile_Name| in \verb|s-rident.ads|

\vs

required updates are listed in comments in \verb|s-rident.ads|

\end{specialframe}

%%%%%%%%%%%%%%%%%%%%%%%%%%%%%%%%%%%%%%%%%%%%%%%%%%%%%%%%%%%%%%%%%%
\begin{specialframe}
  \frametitle{Exercises}

\begin{exercise}
  What kind of restriction is \verb|SPARK|?
\end{exercise}

\visible<2>{\begin{answer}
unit-level restriction without value
\end{answer}}

\begin{exercise}
  Find a SPARK violation detected by the \verb|SPARK| restriction, and one not
  detected.
\end{exercise}

\visible<2>{\begin{answer}
e.g. missing end labels are detected; errors in annotations are not detected
\end{answer}}

\begin{exercise}
  Review the code for the \verb|SPARK| restriction.
\end{exercise}

\begin{exercise}
  Add a profile \verb|SPARK_2014| which defines the restriction \verb|SPARK|
  and a few others.
\end{exercise}

\end{specialframe}

%%%%%%%%%%%%%%%%%%%%%%%%%%%%%%%%%%%%%%%%%%%%%%%%%%%%%%%%%%%%%%%%%%
\begin{specialframe}
  \frametitle{Adding A New Warning}

procedure at AdaCore
\begin{itemize}
\item open TN and discuss proposed semantics
\item implement under a debug switch (in \verb|debug.adb|)
\item assess impact
\item decide if warning kept, if so under which switch(es)
\end{itemize}

\vs

switches defined in \verb|warnsw.adb|, flags defined in \verb|opt.ads|

\vs

implementation partly in \verb|Sem_Warn|, otherwise scattered in semantic
analysis units

\end{specialframe}

%%%%%%%%%%%%%%%%%%%%%%%%%%%%%%%%%%%%%%%%%%%%%%%%%%%%%%%%%%%%%%%%%%
\begin{specialframe}
  \frametitle{Exercises}

\begin{exercise}
  How to get the list of GNAT compiler warnings?
\end{exercise}

\visible<2>{\begin{answer}
call \texttt{gnatmake} or browse GNAT User's Guide
\end{answer}}

\begin{exercise}
  How to enable/disable warning on suspicious contracts?
\end{exercise}

\visible<2>{\begin{answer}
    \texttt{-gnatw.t}/\texttt{-gnatw.T}
\end{answer}}

\begin{exercise}
  Review the code of \verb|Sem_Ch6.Check_Subprogram_Contract| for suspicious
  contracts.
\end{exercise}

\end{specialframe}

%%%%%%%%%%%%%%%%%%%%%%%%%%%%%%%%%%%%%%%%%%%%%%%%%%%%%%%%%%%%%%%%%%
\begin{specialframe}
  \frametitle{Check Marks}

frontend marks each node with required run-time checks
\begin{itemize}
\item checks defined in \verb|types.ads|
\item checks suppression recorded in \verb|Opt.Suppress_Options|
\end{itemize}

\vs

handling defined in \verb|checks.adb|
\begin{itemize}
\item check marks set during semantic analysis (\verb|Enable_..._Check| and
  \verb|Apply_..._Check| procedures)
\item checks then expanded into code (not in Alfa mode)
\end{itemize}

\vs

GNATprove can rely on frontend check marks to generate VCs

\end{specialframe}

%%%%%%%%%%%%%%%%%%%%%%%%%%%%%%%%%%%%%%%%%%%%%%%%%%%%%%%%%%%%%%%%%%
\begin{specialframe}
  \frametitle{Overflow Checking Modes}

three different modes for checking overflows on intermediate arithmetic
computations
\begin{enumerate}
\item STRICT: RM specified base type
\item MINIMIZED: use larger base type (currently 64-bits
  \verb|Long_Long_Integer|) if needed
\item ELIMINATED: use larger base type or arbitrary precision integers if needed
\end{enumerate}

\vs

set separately for assertions/code

\vs

separate from suppression of overflow checks

\vs

implemented in
\begin{itemize}
\item \verb|Minimize_Eliminate_Overflows| in \verb|checks.adb|
\item run-time unit \verb|System.Bignum| for arbitrary precision integers
\end{itemize}
\end{specialframe}

%%%%%%%%%%%%%%%%%%%%%%%%%%%%%%%%%%%%%%%%%%%%%%%%%%%%%%%%%%%%%%%%%%
\begin{specialframe}
  \frametitle{Special Testing Modes}

extended validity checking for mix test+proof
\begin{itemize}
\item \verb|-gnateA|: aliasing checks on subprogram parameters
\item \verb|-gnateV|: validity checks on subprogram parameters
\end{itemize}

\vs

aliasing checks in \verb|Checks.Apply_Parameter_Aliasing_Checks|

\vs

validity checks in \verb|Checks.Apply_Parameter_Validity_Checks|

\end{specialframe}

\end{document}
