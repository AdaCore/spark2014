\documentclass{beamer}

\usepackage{xcolor}
\usepackage{eurosym}
\usepackage[utf8]{inputenc}
\usepackage{tikz}
\usetikzlibrary{arrows,positioning}
\usepackage{pgflibraryshapes} % for ellipse shape

%\usepackage{beamerthemesplit}

\usepackage[absolute,overlay]{textpos}
\TPGrid{15}{10}

\newcommand{\vs}{\vspace{0.5cm}}

\definecolor{mygreen}{rgb}{0,0.7,0}

\usepackage{listings}
\usepackage{color}
\lstset{
	language=Ada,
	keywordstyle=\bfseries\ttfamily\color[rgb]{0,0,1},
	identifierstyle=\ttfamily,
	commentstyle=\color[rgb]{0.133,0.545,0.133},
	stringstyle=\ttfamily\color[rgb]{0.627,0.126,0.941},
        morekeywords=[1]some,
	showstringspaces=false,
	basicstyle=\tiny,
	numberstyle=\tiny,
	numbers=left,
	stepnumber=1,
	numbersep=10pt,
	tabsize=2,
	breaklines=true,
	prebreak = \raisebox{0ex}[0ex][0ex]{\ensuremath{\hookleftarrow}},
	breakatwhitespace=false,
	aboveskip={1.5\baselineskip},
  columns=fixed,
  extendedchars=true,
% frame=single,
% backgroundcolor=\color{lbcolor},
}

% special frames used to put source-code listings
\newenvironment{specialframe}{%
  \begin{frame}[fragile,environment=specialframe]}{\end{frame}}

\xdefinecolor{adacoreblue}{rgb}{0,0.34,0.59}
\xdefinecolor{adacoregrey}{rgb}{0.53,0.68,0.84}

\AtBeginSection[]{\frame{\frametitle{Outline}
\tableofcontents[current]}}
\AtBeginSubsection[]{\frame{\frametitle{Outline}
\tableofcontents[currentsection,currentsubsection]}}

\setbeamertemplate{footline}[page number]
\setbeamercolor{frametitle}{bg=adacoreblue!40!adacoregrey, fg=white}
\setbeamercolor{section in toc}{fg=adacoreblue}
\setbeamercolor{block title}{bg=adacoregrey, fg=white}
\setbeamertemplate{navigation symbols}{}
\setbeamercovered{transparent}
\setbeamertemplate{footline}
{%
  \hfill \insertframenumber\ / \inserttotalframenumber%
}

\begin{document}

%%%%%%%%%%%%%%%%%%%%%%%%%%%%%%%%%%%%%%%%%%%%%%%%%%%%%%%%%%%%%%%%%%

% types: abstract types, predefined types, records, algebraic data types
% logic: function declarations, definitions, predicates, axioms
% theories

\section{Encoding of Ada into Why3}

\begin{specialframe}\frametitle{Types - Scalar types}
   \begin{block}{Generalities}
      All entities are translated into their own module, sometimes two
   \end{block}

   \begin{block}{Discrete types}
   \begin{itemize}
      \item abstract type
      \item conversion to/from int
      \item axioms about these conversion functions
      \item constants for first/last/modulus, with definition when not dynamic
      \item \verb|of_int_| program function with precondition that argument is
         in range
      \item equality ...
   \end{itemize}
   \end{block}
   \vspace{-1em}
   \begin{block}{Floating point types}
      \begin{itemize}
         \item very similar, but conversion to/from real
      \end{itemize}
   \end{block}
\end{specialframe}

\begin{specialframe}\frametitle{Types - Record types}
   \begin{block}{Why3 model}
      \begin{itemize}
         \item why3 record type with one field for all present fields
               (even in case of variant records)
         \item program functions for access with discriminant check
      \end{itemize}
   \end{block}
\end{specialframe}

\begin{specialframe}\frametitle{Types - Array types}
   \begin{block}{Why3 record with fields:}
      \begin{itemize}
         \item map - contains the data
         \item first
         \item last
         \item offset - to allow for sliding
      \end{itemize}
   \end{block}
\end{specialframe}

\section{gnat2why passes}

\begin{specialframe}\frametitle{Different Passes (1)}

   \begin{block}{Alfa filter}
      \begin{itemize}
         \item Compute globals for all subprograms
         \item Traverse the entire tree to find all entities, and to decide
            whether they are in Alfa or not
         \item Define lists:
            \begin{itemize}
                  \item types in alfa
                  \item objects in alfa
                  \item subprograms in alfa
            \end{itemize}
      \end{itemize}
   \end{block}
\end{specialframe}


\begin{specialframe}\frametitle{Different Passes (2) }
   \begin{block}{Translation}
      \begin{itemize}
         \item Dispatch over type of entity and call corresponding translation
            procedure:
            \begin{itemize}
                  \item \verb|Translate_Type|
                  \item \verb|Translate_Constant|
                  \item \verb|Translate_Variable|
                  \item \verb|Translate_Subprogram_Spec|
            \end{itemize}
         \item Iterate over entities to generate "VCs" (ie Why3 programs that
            will trigger VCs)
            \begin{itemize}
               \item Today, only calls \verb|Generate_VCs_For_Subprogram_Body|
            \end{itemize}
      \end{itemize}
   \end{block}
\end{specialframe}

\end{document}
