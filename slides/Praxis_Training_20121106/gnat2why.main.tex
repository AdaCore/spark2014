\documentclass{beamer}

\usepackage{xspace}
\usepackage{xcolor}
\usepackage{eurosym}
\usepackage[utf8]{inputenc}
\usepackage{tikz}
\usetikzlibrary{arrows,positioning}
\usepackage{pgflibraryshapes} % for ellipse shape

%\usepackage{beamerthemesplit}

\usepackage[absolute,overlay]{textpos}
\TPGrid{15}{10}

\newcommand{\vs}{\vspace{0.5cm}}

\newcommand{\resp}{resp.\xspace}
\newcommand{\etal}{\textit{et al.}\xspace}
\newcommand{\etc}{\textit{etc.}\xspace}
\newcommand{\ie}{\textit{i.e.,}\xspace}
\newcommand{\eg}{\textit{e.g.,}\xspace}
\newcommand{\wrt}{w.r.t.\xspace}

\definecolor{mygreen}{rgb}{0,0.7,0}

\usepackage{listings}
\usepackage{color}
\lstset{
	language=Ada,
	keywordstyle=\bfseries\ttfamily\color[rgb]{0,0,1},
	identifierstyle=\ttfamily,
	commentstyle=\color[rgb]{0.133,0.545,0.133},
	stringstyle=\ttfamily\color[rgb]{0.627,0.126,0.941},
        morekeywords=[1]some,
	showstringspaces=false,
	basicstyle=\tiny,
	numberstyle=\tiny,
	numbers=left,
	stepnumber=1,
	numbersep=10pt,
	tabsize=2,
	breaklines=true,
	prebreak = \raisebox{0ex}[0ex][0ex]{\ensuremath{\hookleftarrow}},
	breakatwhitespace=false,
	aboveskip={1.5\baselineskip},
  columns=fixed,
  extendedchars=true,
% frame=single,
% backgroundcolor=\color{lbcolor},
}

% special frames used to put source-code listings
\newenvironment{specialframe}{%
  \begin{frame}[fragile,environment=specialframe]}{\end{frame}}

\xdefinecolor{adacoreblue}{rgb}{0,0.34,0.59}
\xdefinecolor{adacoregrey}{rgb}{0.53,0.68,0.84}

\AtBeginSection[]{\frame{\frametitle{Outline}
\tableofcontents[current]}}
\AtBeginSubsection[]{\frame{\frametitle{Outline}
\tableofcontents[currentsection,currentsubsection]}}

\setbeamertemplate{footline}[page number]
\setbeamercolor{frametitle}{bg=adacoreblue!40!adacoregrey, fg=white}
\setbeamercolor{section in toc}{fg=adacoreblue}
\setbeamercolor{block title}{bg=adacoregrey, fg=white}
\setbeamertemplate{navigation symbols}{}
\setbeamercovered{transparent}
\setbeamertemplate{footline}
{%
  \hfill \insertframenumber\ / \inserttotalframenumber%
}

\begin{document}

%%%%%%%%%%%%%%%%%%%%%%%%%%%%%%%%%%%%%%%%%%%%%%%%%%%%%%%%%%%%%%%%%%

% types: abstract types, predefined types, records, algebraic data types
% logic: function declarations, definitions, predicates, axioms
% theories

\section{Encoding of Ada Into Why}

\begin{specialframe}\frametitle{A Modular Translation}
  1 Ada entity $\longleftrightarrow$ 1 or 2 Why modules [+ 1 or 2 modules for VCs]

  \vs

  dependences between Ada entities $\longleftrightarrow$ includes of Why modules

\pause
  \vs

  1 Ada unit $\longleftrightarrow$ 6 Why files (one is for VCs)

  \vs

  with-clause of Ada unit $\longleftrightarrow$ (nothing special)

\pause
  \vs

  1 GNAT project $\longleftrightarrow$ 1 output dir \verb|gnatprove|

  \vs

  subproject relations $\longleftrightarrow$ (nothing special)

\end{specialframe}

\begin{specialframe}\frametitle{Dispatching of Entities}

Why files cannot be mutually dependent

\vs

solution: 6 layers of Why files
      \begin{itemize}
         \item \verb|unit__types_in_spec.mlw|: types defined in spec
         \item \verb|unit__types_in_body.mlw|: types defined in body
         \item \verb|unit__variables.mlw|: global/local variables defined in unit
         \item \verb|unit__context_in_spec.mlw|: subprograms decl in spec
         \item \verb|unit__context_in_body.mlw|: subprograms decl in body
         \item \verb|unit__package.mlw|: subprogram bodies defined in unit
      \end{itemize}

\vs

no possible circular dependencies
      \begin{itemize}
         \item dependency on file in upper layers is fine
         \item no possible dependency on file in layers below
         \item dependencies within a layer follow with-clauses in Ada specs
         \item recursion? calls in \verb|unit__package.mlw| only, functions in
           \verb|unit__context_in_<spec or body>.mlw| only
      \end{itemize}
\end{specialframe}

\begin{specialframe}\frametitle{Scalar Types}

   \begin{block}{Discrete types}
   \begin{itemize}
      \item abstract type
      \item conversion to/from native mathematical \verb|int|
      \item axioms about these conversion functions
      \item constants for first/last/modulus, with definition when static
      \item \verb|of_int_| program function with precondition for the range
      \item equality
   \end{itemize}
   \end{block}

   \begin{block}{Floating point types}
      \begin{itemize}
         \item very similar, but conversion to/from real
      \end{itemize}
   \end{block}
\end{specialframe}

\begin{specialframe}\frametitle{Composite Types}
   \begin{block}{Record types}
      \begin{itemize}
         \item Why record type with corresponding fields
               (for all fields, even in case of variant records)
         \item program functions for access (with discriminant check)
      \end{itemize}
   \end{block}

   \begin{block}{Array types}
      \begin{itemize}
      \item Why record with 4 fields
        \begin{itemize}
        \item map - contains the data
        \item first
        \item last
        \item offset - to allow for sliding
        \end{itemize}
      \item program functions for access (with index check)
      \end{itemize}
   \end{block}
\end{specialframe}

\begin{specialframe}\frametitle{Constants}
   \begin{itemize}
      \item declare a logic variable
      \item value defined by an axiom
      \item value for deferred constant defined in separate module available
        only in enclosing unit
   \end{itemize}
\end{specialframe}

\begin{specialframe}\frametitle{Variables}
   \begin{itemize}
      \item declare a program (always mutable) variable
      \item no axiom for initialization
      \item declare a type for the type of the variable (used for effects)
   \end{itemize}
\end{specialframe}

\begin{specialframe}\frametitle{Subprograms}

1 module in \verb|unit__context_in_<spec or body>.mlw|
   \begin{itemize}
      \item a function declaration
      \item a program declaration, with pre/post/effects, but no body
      \end{itemize}

\vs

2 modules in \verb|unit__package.mlw|
     \begin{itemize}
      \item a program that computes the precondition, used to get VCs
for checks in the precondition
      \item the actual translation of the subprogram, used to get
VCs for the body + postcondition
   \end{itemize}
\end{specialframe}

\begin{specialframe}\frametitle{Other Modules}

   \begin{block}{Array aggregates}
      We generate:
   \begin{itemize}
      \item a function with \emph{holes} for the right-hand-side
      \item an axiom defining the aggregate
   \end{itemize}
   \end{block}
\vs
   \begin{block}{Array slices}
      We generate:
   \begin{itemize}
      \item a function taking the array and bounds as parameters
      \item an axiom defining the slice
   \end{itemize}
   \end{block}
\vs
   \begin{block}{String literals}
      We only generate the variable, no knowledge about the string for now
   \end{block}
\end{specialframe}


\section{gnat2why Phases}

\begin{specialframe}\frametitle{Translation Phases}

  documented in \verb|gnat2why.ads|

  \vs

  \begin{enumerate}
  \item Framing: compute SPARK \emph{globals}
  \item Detection: separate Alfa and non-Alfa entity views
  \item Transformation: from GNAT tree to Why tree
  \item Printing: generate Why files
  \end{enumerate}
\end{specialframe}

\begin{specialframe}\frametitle{Phase 1: Framing}

  implemented in \verb|Alfa.Frame_Conditions|

  \vs

  \begin{itemize}
  \item read Alfa xrefs from ALI files (using \verb|Get_Alfa| in GNAT sources)
  \item entities only known by their name
  \item construct call graph
  \item start with local reads/writes
  \item iterate over call graph to compute final reads/writes
  \end{itemize}
\end{specialframe}

\begin{specialframe}\frametitle{Phase 2: Detection}

  implemented in \verb|Alfa.Detection|

  \vs

  \begin{itemize}
  \item single traversal of the tree
  \item detect every violation of Alfa
  \item mark entity views in Alfa or not in Alfa
  \item define two lists of entities
    \begin{itemize}
    \item \verb|Spec_Entities| of entities defined in spec
    \item \verb|Body_Entities| of entities defined in body
    \end{itemize}
  \end{itemize}
\end{specialframe}

\begin{specialframe}\frametitle{Phase 3: Transformation}
  implemented in \verb|Gnat2Why.Decls|, \verb|Gnat2Why.Types|,
  \verb|Gnat2Why.Subprograms|, \verb|Gnat2Why.Expr|

  \vs

      \begin{itemize}
      \item one translation procedure per entity kind
            \begin{itemize}
                  \item \verb|Translate_Type|
                  \item \verb|Translate_Constant|
                  \item \verb|Translate_Variable|
                  \item \verb|Translate_Subprogram_Spec|
            \end{itemize}
\vs
         \item procedures to generate VCs (\ie Why programs that
            will trigger VCs)
            \begin{itemize}
               \item \verb|Generate_VCs_For_Subprogram_Spec|
               \item \verb|Generate_VCs_For_Subprogram_Body|
            \end{itemize}
\vs
          \item special handling for aggregates/slices/string literals
            \begin{itemize}
            \item \verb|Transform_Aggregate|
            \item \verb|Transform_Slice|
            \item \verb|Transform_String_Literal|
            \end{itemize}
      \end{itemize}
\end{specialframe}


\begin{specialframe}\frametitle{Phase 4: Printing}
  implemented in \verb|Why.Atree.Sprint|

  \vs

      \begin{itemize}
      \item general traversal based on visitor pattern
      \item special care taken to avoid too deep trees, leading to stack
        overflow
      \item pretty printer to facilitate debugging
        \begin{itemize}
        \item Ada identifiers are preserved
        \item code is indented
        \item initial comment for modules points to source Ada code and
          translation code in \verb|gnat2why|
        \item Why labels contain source location information
        \item we could intersperse Ada code...
        \end{itemize}

      \end{itemize}
\end{specialframe}

\begin{specialframe}\frametitle{Translation of Statements and Expressions}
   \begin{block}{Generalities}
      \begin{itemize}
      \item Why makes no difference between statements and expressions. This
        simplifies things (\eg to insert an assertion inside an expression).
      \item All annotations (pre/post, assert, \etc) are translated (at least)
        twice into Why: once as a program expression to get the checks, once as
        an annotation.
      \end{itemize}

   \end{block}

   \begin{block}{Loops}
      \begin{itemize}
         \item transform Assert-like invariant into Hoare invariant
         \item use infinite loop + exception to allow multiple exits
      \end{itemize}
   \end{block}

\end{specialframe}

\begin{specialframe}\frametitle{Translation of Statements and Expressions (2)}
   \begin{block}{Conversions}
      \begin{itemize}
         \item every Ada (sub)type is translated into a different Why type
         \item need to insert conversions
         \item conversions go through the \emph{Why base type}
      \end{itemize}
   \end{block}
   \begin{block}{Why base type}
      \begin{itemize}
         \item for discrete types: \verb|int|
         \item for floating point types: \verb|real|
         \item for arrays: generic array type
         \item for record types: the Why type of the Ada root type
      \end{itemize}
   \end{block}
\end{specialframe}

\begin{specialframe}\frametitle{Translation of Statements and Expressions (3)}
   \begin{block}{Arithmetic expressions}
      \begin{itemize}
         \item insert conversions to \verb|int| to allow arithmetic operators
         \item insert overflow check calls
         \item range check on conversion to Ada type
      \end{itemize}
   \end{block}
\end{specialframe}

\begin{specialframe}\frametitle{Special case - \emph{invisible} globals}
   \begin{block}{Almost everything is driven by entities}
      \begin{itemize}
         \item name generation
         \item dispatching to correct Why file
         \item dispatching on the kind of the entity
      \end{itemize}
   \end{block}
   \begin{block}{Except reads/writes to \emph{invisible} variables}
      \begin{itemize}
      \item  \eg they are declared in a different package body
         \item we only have their name from ALI file
         \item that's enough to refer to the variable ...
         \item ... and to its type
      \end{itemize}
   \end{block}
\end{specialframe}

\begin{specialframe}\frametitle{Traceability}
   \begin{block}{Labels in Why code}
      \begin{itemize}
         \item One can label any Why entity or expression with a string
         \item This label will stick to that expression through VCgen,
         simplifications, \etc
      \end{itemize}
   \end{block}
   \begin{block}{Use of labels in \texttt{gnat2why}}
      \begin{itemize}
         \item all VCs (kind of check, source location)
         \item all statements (for ``Show Path'')
         \item all subprograms (for ``Prove Subprogram'')
      \end{itemize}
   \end{block}
\end{specialframe}

\section{gnat2why Code}

\begin{specialframe}\frametitle{Code Organization}
   \begin{itemize}
      \item \verb|alfa-*|       units to compute Alfa subset
      \item \verb|why-atree-*|  units to define and manipulate Why tree
      \item \verb|why-gen-*|    units to generate specific Why subtrees
      \item \verb|gnat2why-*|   units that traverse the GNAT tree and generate
         Why trees
      \item \verb|why-inter|    unit that does a lot of work \emph{in between}
   \end{itemize}
\end{specialframe}

\begin{specialframe}\frametitle{Why Tree Code}
  input files
  \begin{itemize}
  \item \verb|why-sinfo.ads|  defines enumeration \verb|Why_Node_Kind| and some subtypes
  \item \verb|xgen/xtree_sinfo.adb| defines the tree structure
  \end{itemize}

\vs
  generated files
  \begin{itemize}
  \item \verb|why-classes.ads| defines main subtypes of node kind
  \item \verb|why-ids.ads|    defines subtypes of nodes (for list, array, option, valid)
  \item \verb|why-conversions.ads|    defines valid conversions between subtypes (operator ``+'')
  \item \verb|why-atree-*|   units that manipulate the Why tree
   \end{itemize}
\end{specialframe}

\begin{specialframe}\frametitle{Code Pointers}

  Entities in Alfa and not in Alfa are translated differently. Switch is done
  in \verb|Gnat2Why.Driver.Translate_Entity|

\vs

Standard package is translated by
\verb|Gnat2Why.Driver.Translate_Standard_Package|

\vs

  Same transformation is called multiple times on the same entity. Context is
  given by \verb|Gnat2Why.Util.Transformation_Params|

\vs

Special treatment for containers in
\verb|Gnat2Why.Decls.Translate_Container_Package| based on ad-hoc \verb|sed|
rewriting. TBD: use new Why clones of modules.

\end{specialframe}


\end{document}
