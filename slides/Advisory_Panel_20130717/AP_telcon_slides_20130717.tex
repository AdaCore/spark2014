\documentclass{beamer}
\input{praxis-beamer}

\usepackage{tikz}
\usetikzlibrary{positioning,calc}

\title{SPARK 2014 Advisory Panel}
\subtitle{Teleconference 2}

% Inspired by http://tex.stackexchange.com/questions/15237
\makeatletter
\newenvironment{btHighlight}[1][]
{\begingroup\tikzset{bt@Highlight@par/.style={#1}}\begin{lrbox}{\@tempboxa}}
{\end{lrbox}\bt@HL@box[bt@Highlight@par]{\@tempboxa}\endgroup}

\newcommand\btHL[1][]{%
  \begin{btHighlight}[#1]\bgroup\aftergroup\bt@HL@endenv%
}
\def\bt@HL@endenv{%
  \end{btHighlight}%
  \egroup
}
\newcommand{\bt@HL@box}[2][]{%
  \tikz[remember picture]{%
    \pgfpathrectangle{\pgfpoint{1pt}{0pt}}{\pgfpoint{\wd #2}{\ht #2}}%
    \pgfusepath{use as bounding box}%
    \node[anchor=base west,%
          outer sep=0pt,%
          inner xsep=1pt,%
          inner ysep=0pt,%
          rounded corners=2pt,%
          minimum height=\ht\strutbox+1pt,%
          #1]%
          {%
            \raisebox{1pt}{\strut}\strut\usebox{#2}%
          };%
  }%
}
\makeatother

\date{July 17, 2013}
\begin{document}

\begin{altrantitle}
\titleprismlabela{SPARK 2014}
\titleprismlabelb{Safe and Secure programming}
\titleprismlabelc{
  \resizebox{4cm}{!}{
    \includegraphics{partnership}
  }
}
\end{altrantitle}

\begin{frame}{Introduction}

  \begin{itemize}

  \item Introductions
  \item Round-up of key SPARK 2014 RM review comments:
    \begin{itemize}
    \item Access types and Allocators
    \item Global and Depends Aspects
    \item Ghost Functions
    \item Compiler Portability
    \end{itemize}
  \item Update on Language Issues: 
    \begin{itemize}
    \item Refined\_Post
    \item Refined\_Pre
    \item Externals
    \end{itemize}
  \item Early Preview Release:
    \begin{itemize}
    \item How to Download
    \item Demo
    \end{itemize}
  \item Conclusion

  \end{itemize}

\end{frame}

\begin{frame}{Access types and Allocators}

  \begin{itemize}

  \item Access types are not available in release 1 but there are workarounds
  \item More may be possible in future releases --- perhaps access to constants or to subprogram types. 
  \item In the meantime, possible workarounds:
   \begin{itemize}
    \item Use an indefinite container - we hope to give a SPARK
      version of indefinite containers and/or provide a SPARK
      interface to the Ada Container library
   \item Make the type to contain the constants private and provide
     subprograms to read the values of the constants. The body of the
     package defining the private type will not be in SPARK
  \end{itemize}
  \item Limited support of allocators - particularly used for context
    switching.  We will consider this when we add tasking. A possible
    workaround is:
    \begin{itemize}
    \item use a private type as the context 
    \item mark the body of the package defining the type as not in SPARK
    \end{itemize}
  \end{itemize}

\end{frame}

\begin{frame}[fragile]{Using a Private Type Wrapper}

  \begin{itemize}

    \item For example, consider an array of access to String
    \item Write a SPARK compatible visible part and a non-spark private part and body
    \item Use the declared functions to get pointers and strings.
    \item Alternatively use a SPARK interface to Ada.Strings.Unbounded. 
  \end{itemize}

  \begin{pxcode}[language=SPARK,style=tinystyle,gobble=4]
    package String_Constants
    is
        subtype String_Index is Positive range 1 .. 512;
        type Ref_String is private;

        function Get_A_Ref_String (I : String_Index) return ref_String;

        function Get_A_String (S : Ref_String) return String;
    private
       pragma SPARK_Mode (Off);
       type Ref_String is access String;
    end String_Constants; 
  \end{pxcode}

\end{frame}

\begin{frame}{Global and Depends aspects}

  \begin{itemize}

  \item Global and Depends aspects: we say in the RM "for mode out
    parameters and globals things like array attributes,
    discriminants and tags are ignored", what we mean by this is
    that flow analysis treats them as constants.
  \item We will consider after release 1 if this model is adequate
  \item Also after release 1 we will consider whether we can provide more
  precise flow relations for composite objects which are only
  partially updated by a subprogram, but never read
  \end{itemize}

\end{frame}

\begin{frame}{Ghost Functions}

  \begin{itemize}

  \item Ghost functions can be used in any assertion expression this
    includes Pre and Post conditions.  We will make this clearer in the
    RM.
  \item SPARK will raise an error if a Ghost
    function is used outside of an assertion expression.  
  \item the compiler will be able to not generate
    code for Ghost functions if assertion expressions are not executed
    (Ignored).
  \item gnatcoverage will be updated to ignore ghost functions if
    they are not executed.
 \end{itemize}

\end{frame}

\begin{frame}{Compiler Independence}

  \begin{itemize}

  \item SPARK 2014 aspects cannot be processed by a non SPARK aware compiler  
  \item All SPARK aspects can be expressed as pragmas which will be
    ignored by such a compiler
 \end{itemize}

\end{frame}

\begin{frame}{Update on Language Issues: Refined\_Post}

  \begin{itemize}
  \item The aspect Refined\_Post is in SPARK 2014.  It facilitates
    writing a post condition in terms of the implementation of a subprogram. 
  \end{itemize}
 
\end{frame}

\begin{frame}{Refined\_Pre}

  \begin{itemize}
  \item Refined\_Pre is being postponed until after Release 1
  \item In traditional refinement techniques a precondition implies
    its refined precondition.  In principle this means that the
    refined precondition may be weaker
  \item Ada requires that the precondition is evaluated each time a
    subprogram is called
  \item This is equivalent to requiring that both the precondition and
    the refined precondition must be satisfied whenever the subprogram
    is called.  The refined precondition cannot be weaker it must be equivalent.
  \item This difference has led to much discussion and the deferral of
    the refined precondition until after release 1.
  \end{itemize}
 
\end{frame}

\begin{frame}{Refined\_Pre - Workarounds}

  \begin{itemize}
    \item There are usually workarounds in the absence of Refine\_Pre
    \item Simple case where the preconditions are given in terms of
      functions without postconditions \ldots
  \end{itemize}

\end{frame}

\begin{frame}[fragile]{Refined\_Pre - Example Spec}

  \begin{pxcode}[language=SPARK,style=tinystyle,gobble=4]
    package Stacks
    is
       type Stack is private;
 
       function Is_Empty (S : Stack) return Boolean;
       function Is_Full (S : Stack) return Boolean;

       function New_Stack return Stack
       with Post => Is_Empty (New_Stack'Result) and 
                             not Is_Full (New_Stack'Result);

       procedure Push (S : in out Stack; I : in Integer)
          with Depends => (S => S, I),
                  Pre         => not Is_Full (S);

       procedure Pop (S : in out Stack)
          with Depends => (S => S),
                  Pre         => not Is_Emptyl (S);

       function Top (S : Stack) return Integer
           with Pre => not Is_Empty;

    private
        Max_Stack : constant 100;
        type Stack_Pointer is range 0 .. Max_Stack;
        subtype Stack_Index is Stack_Pointer range 1 .. Max_Stack;
        Vector is array (Stack_Index) of Integer; 
        type Stack is record
            P : Stack_Pointer;
            V : Vector;
         end record;
    end Stacks;
          
 \end{pxcode}

\end{frame}

\begin{frame}[fragile]{Refined\_Pre - Example Body}

  \begin{itemize}

    \item The specification (and implementation) of the functions used
      in the preconditions are given as expression functions
    \item The default Refined\_Post condition of an expression
      function is F'Result = \emph{expression}
   \item The expression can be in terms of the full view of type Stack \ldots
  \end{itemize}

\end{frame}
\begin{frame}[fragile]{Refined\_Pre - Example Body}

  \begin{pxcode}[language=SPARK,style=tinystyle,gobble=4]
    package Stacks
    is
       type Stack is private;
 
       function Is_Empty (S : Stack) return Boolean is (S.P = 0);
       function Is_Full (S : Stack) return Boolean is (S.P /= Max_Stack);

       function New_Stack return Stack is (Stack'((P => 0, V => (others => 0)));

       procedure Push (S : in out Stack; I : in Integer)
       -- Uses the precondition specified in the visible view
       -- but in terms of the refined postcondition of Is\_Full
       is
       begin
          S.P := S.P + 1;
          S.V := I;
       end Push;

       procedure Pop (S : in out Stack)
       -- Uses the precondition specified in the visible view
       -- but in terms of the refined postcondition of Is\_Empty
       is
       begin
            S.P := S.P - 1;
       end Pop;

       function Top (S : Stack) return Integer is (S.V (S.P));
       -- Uses the precondition specified in the visible view
       -- but in terms of the refined postcondition of Is\_Empty

    end Stacks;          
 \end{pxcode}

\end{frame}

\begin{frame}{Refined\_Pre - Feedback}

  \begin{itemize}
  \item Is the proposed workaround sufficient to support typical use cases?
  \item How important is the ability to be able to express a weaker precondition on the refined subprogram?
  \end{itemize}
 
\end{frame}

\begin{frame}{Externals - Ada Context}

  \begin{itemize}
  \item In Ada interfacing to an external device or subsystem normally
    entails using one or more volatile variables to ensure that writes
    and reads to the device are not optimised by the compiler into
    internal register reads and writes.
  \item Ada also has the concept of \emph {external readers} and \emph
    {external writers} to represent external entities reading or
    writing to the program through volatlie variables
  \end{itemize}
 
\end{frame}

\begin{frame}{Externals - SPARK 2014 Overview}

  \begin{itemize}
 \item In SPARK we are specifically concerned with asynchronous
   external readers and writers - ones we essentially have no direct
   control over when they read or write and object.
  \item We use the concept of an asynchronous writer to represent an
    external entity writing to the program, an external input; and
 \item the concept of an asychronous reader to represent an external
   entity reading from the program, an external output
 \item Unlike SPARK 2005 an external object may be both an input and
   an output
 \item A state abstraction representing one or more volatile objects
   is called an \emph {External} state abstraction.  
  \end{itemize}
 
\end{frame}

\begin{frame}{Externals - Language Definition}

  \begin{itemize}
  \item We introduce four new Boolean aspects:
    \begin{itemize}
    \item Async\_Writers
    \item Async\_Readers
    \item Read\_Only
    \item Write\_Only
    \end{itemize}
  \item These aspects may be applied to volatile object declarations
  \item Async\_Writers and Async\_Readers may also be applied to an
    External state abstraction
  \item Further aspects will be added after release 1 to allow modelling of other types of interface
  \end{itemize}
 
\end{frame}

\begin{frame}{Early Preview Release}

  \begin{itemize}
  \item Early preview of the SPARK 2014 toolset announced end of May
  \item To obtain the latest version, submit a request through the GNAT Tracker report mechanism
  \item Wavefront will be made available to download from you GNAT Tracker account
  \item Please report any comments or issues via GNAT Tracker
  \end{itemize}
 
\end{frame}

\begin{frame}{Demo}

  \begin{itemize}
  \item Based on the Linear Search example shown in the Getting Started section of the User Guide \ldots
  \end{itemize}
 
\end{frame}

\begin{frame}{Conclusion}

  \begin{itemize}

  \item Key dates: 
    \begin{itemize}
    \item News email 4 - September 2013
    \item Teleconference 3 - October 2013
    \item Preview Release Tutorial - October/November 2013
    \end{itemize}

  \item AOB

  \end{itemize}

\end{frame}

\end{document}

