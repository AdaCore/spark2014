\documentclass[11pt,a4paper]{article}
\usepackage[utf8]{inputenc}
\usepackage{url}
\usepackage{xspace}
\usepackage{amsmath}
\usepackage{hyperref}
\usepackage[usenames,dvipsnames]{color}
\usepackage{listings}
\newcommand{\CodeSymbol}[1]{\textcolor{Bittersweet}{#1}}
\lstset{
   language=Ada,
   keywordstyle=\color{RedViolet}\ttfamily\bf,
   showspaces=false,
   basicstyle={\scriptsize \sffamily},
   commentstyle=\color{red}\textit,
   stringstyle=\color{MidnightBlue}\ttfamily,
   showtabs=false,
   showstringspaces=false,
   morekeywords=[1]Pre,
   morekeywords=[1]Post,
   morekeywords=[1]Test\_Case,
   morekeywords=[1]Contract\_Cases,
   morekeywords=[1]some,
   morekeywords=[1]Old,
   morekeywords=[1]Global,
   morekeywords=[1]Depends,
   morekeywords=[1]Loop\_Invariant,
   morekeywords=[1]Loop\_Variant,
   morekeywords=[1]Loop\_Entry,
   morekeywords=[1]Increases,
   literate={(}{{\CodeSymbol{(}}}1
            {)}{{\CodeSymbol{)}}}1
            {>}{{\CodeSymbol{$>$}}}1
            {>=}{{\CodeSymbol{$\ge$}}}1
            {<}{{\CodeSymbol{$<$}}}1
            {<=}{{\CodeSymbol{$\le$}}}1
            {=}{{\CodeSymbol{$=$}}}1
            {:}{{\CodeSymbol{$:$}}}1
            {.}{{\CodeSymbol{$.$}}}1
            {;}{{\CodeSymbol{$;$}}}1
            {/=}{{\CodeSymbol{$\ne$}}}1
            {=>}{{\CodeSymbol{$\Rightarrow$}}}1
            {->}{{\CodeSymbol{$\rightarrow$}}}1
            {<->}{{\CodeSymbol{$\leftrightarrow$}}}1
}

\newcommand{\DO}{\textsc{do-178}\xspace}
\newcommand{\DOB}{\textsc{do-178b}\xspace}
\newcommand{\DOC}{\textsc{do-178c}\xspace}
\newcommand{\hilite}{Hi-Lite\xspace}
\newcommand{\openetcs}{openETCS\xspace}
\newcommand{\gnatprove}{GNATprove\xspace}
\newcommand{\oldspark}{SPARK~2005\xspace}
\newcommand{\newspark}{SPARK~2014\xspace}
\newcommand{\spark}{SPARK\xspace}
\newcommand{\ada}{Ada\xspace}
\newcommand{\adatwtw}{Ada~2012\xspace}
\newcommand{\altergo}{Alt-Ergo\xspace}

\newcommand{\etc}{\textit{etc.}\xspace}
\newcommand{\ie}{\textit{i.e.}\xspace}
\newcommand{\adhoc}{\textit{ad hoc}\xspace}
\newcommand{\Eg}{\textit{E.g.}\xspace}
\newcommand{\eg}{\textit{e.g.}\xspace}
\newcommand{\etal}{\textit{et al.}\xspace}
\newcommand{\wrt}{w.r.t.\xspace}
\newcommand{\aka}{a.k.a.\xspace}
\newcommand{\resp}{resp.\xspace}

\urlstyle{sf}

\begin{document}

\title{Auto-Active Proof of Red-Black Trees in SPARK}

\author{%
\large Claire Dross and Yannick Moy\\
\normalsize AdaCore, 46 rue d'Amsterdam, F-75009 Paris (France)}

\date{}

\maketitle

\paragraph{Abstract}
SPARK is a subset of the Ada programming language targeted at safety- and
security-critical applications. SPARK formal verification toolset allows to
guarantee that a SPARK program is free from broad classes of errors (like reads
of uninitialized data and run-time errors) and that it complies with its
specification. While the former is a well adopted practice among SPARK users,
the latter is used much more narrowly, owing to the cost of specifying the
behavior of programs and even more the cost of achieving proof of such
specifications. SPARK relies on automatic provers to keep the cost of formal
verification reasonable, and on the techniques of auto-active verification for
interacting with automatic provers. In this paper, we present how we applied
auto-active verification to formally verify a library of red-black trees. To
the best of our knowledge, this is the most advanced use of auto-active
verification so far.

\paragraph{Keywords}
System formal development, Verification and validation,
Certification and dependability

\section{Introduction}

\section{Preliminaries}
\subsection{SPARK 2014}
\subsection{Auto-active Verification}
\subsection{Red-Black Trees}

\section{Red-Black Trees in SPARK}
\subsection{Invariants and Models}
% invariants of data structures
% relation between invariants and models
% public model of RBT
% internal model of tree
\subsection{Proof Principles}
% principle of inductive proofs on size of path
% principles of dealing with the frame condition: unmodified trees in forest
\subsection{Ghost Code}
% different uses of ghost code
\subsection{Specification}
\subsection{Implementation}
\subsection{Auto-active Verification}

\section{Development and Verification Data}
% number of assertions, loc of ghost code, etc.
% data on automatic verification
% feedback from development and verification cycles

\section{Related Work}

\section{Conclusion}


\paragraph*{Acknowledgements}


\bibliographystyle{plain}
\bibliography{nfm_2017}

\end{document}
