\documentclass{llncs}
\usepackage[utf8]{inputenc}
\usepackage{url}
\usepackage{xspace}
\usepackage{hyperref}
\usepackage{listings}
\usepackage{amsmath}
\usepackage{graphicx}
\usepackage{zed-csp}

\newcommand{\DO}{\textsc{do-178}\xspace}
\newcommand{\DOB}{\textsc{do-178b}\xspace}
\newcommand{\DOC}{\textsc{do-178c}\xspace}
\newcommand{\hilite}{Hi-Lite\xspace}
\newcommand{\gnatprove}{GNATprove\xspace}
\newcommand{\oldspark}{SPARK~2005\xspace}
\newcommand{\newspark}{SPARK~2014\xspace}
\newcommand{\spark}{SPARK\xspace}
\newcommand{\ada}{Ada\xspace}
\newcommand{\adatwtw}{Ada~2012\xspace}
\newcommand{\vectorcast}{VectorCast\xspace}

\newcommand{\etc}{\textit{etc.}\xspace}
\newcommand{\ie}{\textit{i.e.,}\xspace}
\newcommand{\adhoc}{\textit{ad hoc}\xspace}
\newcommand{\Eg}{\textit{E.g.,}\xspace}
\newcommand{\eg}{\textit{e.g.,}\xspace}
\newcommand{\etal}{\textit{et al.}\xspace}
\newcommand{\wrt}{w.r.t.\xspace}
\newcommand{\aka}{a.k.a.\xspace}
\newcommand{\resp}{resp.\xspace}

\lstdefinestyle{tinystyle} {basicstyle=\scriptsize\tt,
  keywordstyle=\bf, commentstyle=\rmfamily\it, escapeinside={(*}{*)}}
\lstset{style=tinystyle}

\lstdefinelanguage{SPARK}{ language = [95]Ada, morekeywords = {pre,
    post, assert, assume, check, derives, hide, global, inherit, from,
    own, initializes, main_program, input, output, in_out,
    refined_pre, refined_post, some, depends}, comment=[l][commentstyle]{--\ },
  showstringspaces=false }

\lstset{language=SPARK}

\begin{document}

\title{Optimising Verification Effort with \newspark}

\author{Pavlos Efstathopoulos, Andrew Hawthorn}
\institute{ALTRAN, 22 St Lawrence Street, Bath BA1 1AN (United Kingdom)}

\maketitle

\begin{abstract}
\end{abstract}

\section{Abstract}
The introduction of executable contracts in Ada 2012 brings a new
dimension to the debate over which is most efficient: proof or
test. The Hi-Lite project was designed to demonstrate this extension
and coverered the question of how proven subprograms can reliably call
tested subprograms and visa versa \cite{hiliteERTS2012}. This paper
looks at the issue from an industrial perspective to try to identify
where it is most efficient to prove or test subprograms.

Test is clearly the most preferred method of verifying software at the
moment but the regulated industries have been trying to use formal
methods where they can ever since software was first written. The
acceptance of formal methods seems to be increasing, for example, the
latest version of the aerospace standard \DO provides explicit
guidance on the use of formal methods.

The paper will consider the sorts of questions that need to be
asked at the start of a project with respect to choosing and using a
programming language and as part of this will introduce \newspark and the
capabilities of the \newspark tool set focusing on the features that
support the formal methods supplement of \DOC. These are:
\begin{itemize}
   \item the language subset;
   \item the use of flow analysis to check for uninitialized, ineffective
         and unused variables;
   \item the use of theorem provers to prove the absence of run-time errors;
   \item the use of pre- and post-conditions to specify subprograms and
   \item the use of global contracts to check for non-interference between 
         subprograms.
\end{itemize}

We then briefly walk through relevant aspects of the traditional 
V-lifecycle explaining how REVEAL \cite{reveal} and INFORMED 
\cite{informed} help to understand how pre- and post-condition
specificatons can fit into a development model.

We then consider the practical expressiveness of \newspark pre- and
post-conditions in comparison with Z using Tokeneer as an example
application. Examples of how writing pre- and post-conditions affect the
development process and how they may be written to minimise or maximise
the semantic gap between the specification and the code will also be given.

With clear specifications on our subprograms we are now able to decide
whether or not to prove, test or use a combination of the two methods.
We will present examples of the following situations and demonstrate how
a combination of proof and test can resolve them all:
\begin {enumerate}
\item \emph{Prove by strengthening assertions} - try to fully prove
  the code by strengthening preconditions and adding assertions until
  full proof is achieved. No testing is involved in this scenario.

\item \emph{Test without proof} - decide that proof is too difficult
  to achieve (for example, due to the code having complicated
  constructs like multiple nested loops) and resort to testing right
  from the beginning. No proof is involved in this scenario.

\item \emph{Prove some, test some} - attempt to prove the code, fail
  and resort to testing. At this point the user has two different
  options:

  \begin{enumerate}
  \item Perform exclusively testing (prove no properties).

  \item Prove only a subset of the properties of the code and test the
    remaining. For instance, prove freedom of run-time exceptions and
    test functional behaviour-related contracts.
  \end{enumerate}

\item \emph{Use test to debug contracts} - make a first unsuccessful
  attempt to prove the code and afterwards utilize testing to identify
  potential issues with the contracts or the implementation. The tests
  might provide hints as to how the user can alter the code to render
  it provable. In this scenario, testing reveals the actions that need
  be performed to achieve full proof. 

\item \emph{Contracts can not be written} - not able to formulate a
  contract. 

\item \emph{Proof results not useable} - be required (by a standard)
  to test the code. However, as an addition, performing proof could
  improve the safety case and grant more confidence on the code. In
  this scenario the user would try to prove as much as possible, but
  would not insist when proof is too hard.

\end{enumerate}


By walking through the examples, we will demonstrate how a combination
of the use of the SPARK 2014 tool set and a commercially available unit 
testing tool have the potential to dramatically reduce subprogram 
verification effort.

We conclude by summarising our views on the relative merits of
alternative styles of contracts. Specifically, concluding where
different types of contracts are most productively used. We also 
share our experiences and views of the limitations of current tool
support.



\paragraph{Keywords}
Formal methods, Verification and validation, Certification,
Dependability, DO-178C


\bibliographystyle{plain}
\bibliography{proof_and_test}

\end{document}
