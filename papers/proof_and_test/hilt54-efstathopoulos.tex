\documentclass[paperletter]{sig-alternate-2013}
\usepackage[utf8]{inputenc}
\usepackage{url}
\usepackage{xspace}
\usepackage{hyperref}
\usepackage{listings}
\usepackage{amsmath}
\usepackage{graphicx}
\usepackage{zed-csp}

\newcommand{\DO}{\textsc{do-178}\xspace}
\newcommand{\DOB}{\textsc{do-178b}\xspace}
\newcommand{\DOC}{\textsc{do-178c}\xspace}
\newcommand{\hilite}{Hi-Lite\xspace}
\newcommand{\gnatprove}{GNATprove\xspace}
\newcommand{\oldspark}{SPARK~2005\xspace}
\newcommand{\newspark}{SPARK~2014\xspace}
\newcommand{\spark}{SPARK\xspace}
\newcommand{\ada}{Ada\xspace}
\newcommand{\adatwtw}{Ada~2012\xspace}
\newcommand{\vectorcast}{VectorCast\xspace}

\newcommand{\etc}{\textit{etc.}\xspace}
\newcommand{\ie}{\textit{i.e.,}\xspace}
\newcommand{\adhoc}{\textit{ad hoc}\xspace}
\newcommand{\Eg}{\textit{E.g.,}\xspace}
\newcommand{\eg}{\textit{e.g.,}\xspace}
\newcommand{\etal}{\textit{et al.}\xspace}
\newcommand{\wrt}{w.r.t.\xspace}
\newcommand{\aka}{a.k.a.\xspace}
\newcommand{\resp}{resp.\xspace}

\lstdefinestyle{tinystyle} {basicstyle=\scriptsize\tt,
  keywordstyle=\bf, commentstyle=\rmfamily\it, escapeinside={(*}{*)}}
\lstset{style=tinystyle}

\lstdefinelanguage{SPARK}{ language = [95]Ada, morekeywords = {pre,
    post, assert, assume, check, derives, hide, global, inherit, from,
    own, initializes, main_program, input, output, in_out,
    refined_pre, refined_post, some, depends}, comment=[l][commentstyle]{--\ },
  showstringspaces=false }

\lstset{language=SPARK}

\begin{document}

\CopyrightYear{2013}
\crdata{978-1-4503-2466-3/13/11}

\title{Optimising Verification Effort with \newspark}

\numberofauthors{2}
\author{
% 1st. author
  \alignauthor
  Pavlos Efstathopoulos\\
  \affaddr{ALTRAN}\\
  \affaddr{22 St Lawrence Street}\\
  \affaddr{Bath BA1 1AN (United Kingdom)}\\
  \email{pavlos.efstathopoulos@altran.com}
  \alignauthor Andrew Hawthorn\\
  \affaddr{ALTRAN}\\
  \affaddr{22 St Lawrence Street}\\
  \affaddr{Bath BA1 1AN (United Kingdom)}\\
  \email{andrew.hawthorn@altran.com} }

\maketitle

\terms{Languages; Verification}
\keywords{Formal methods; Verification and validation; Certification;
Dependability; DO-178C}

\section{Extended Abstract}
The introduction of executable contracts in Ada 2012 brings a new
dimension to the debate over which is most efficient: proof or
test. The Hi-Lite project was designed to demonstrate the use of
executable contracts extension and covered the question of how proven
subprograms can reliably call tested subprograms and visa versa. This
paper looks at the issue from an industrial perspective to try to
identify where it is most efficient to prove or test subprograms.

Test is clearly the most preferred method of verifying software at the
moment but the regulated industries have been trying to introduce
formal methods ever since software was first written. The acceptance
of formal methods seems to be increasing, for example, the latest
version of the aerospace standard \DO provides explicit guidance on
the use of formal methods.

The paper will consider the sorts of questions that need to be asked
at the start of a project with respect to choosing and using a
programming language and as part of this will introduce the \newspark
language and the capabilities of the \newspark tool set focusing on
the features that support the formal methods supplement of \DOC. These
are the language subset, the use of flow analysis to check for
uninitialized, ineffective and unused variables, the use of theorem
provers to prove the absence of run-time errors, the use of pre- and
post-conditions to specify subprograms and the use of global contracts
to check for non-interference between subprograms.

We then briefly walk through relevant aspects of the traditional
V-lifecycle explaining how \newspark can be used as an architectural
design tool and specification language as well as a coding language.

With the assumption that \newspark has been used as a specification
language we are now able to decide whether or not to prove, test or
use a combination of the two methods to verify our software.  We will
present examples of the following situations and demonstrate how a
combination of proof and test can resolve them all:
\begin {enumerate}
\item \emph{Prove by strengthening assertions} - try to fully prove
  the code by strengthening preconditions and adding assertions until
  full proof is achieved. No testing is involved in this scenario.

\item \emph{Test without proof} - decide that developing tests is
  likely to be easier than completing full proof and resort to testing
  right from the beginning. No proof is involved in this scenario.

\item \emph{Prove some, test some} - attempt to prove the code, find
  that some verification conditions can not be discharged, and resort
  to testing. At this point the user can either perform exclusively
  testing (prove no properties) or prove only a subset of the
  properties of the code and test the remaining. For instance, prove
  freedom of run-time exceptions and test contracts related with
  functional behaviour.

\item \emph{Use test to debug contracts} - make a first unsuccessful
  attempt to prove the code and afterwards utilize testing to identify
  potential issues with the contracts or the implementation. The tests
  might provide hints as to how the user can alter the code to render
  it provable. In this scenario, testing reveals the actions that need
  be performed to achieve full proof.

\item \emph{Contracts can not be written} - not able to formulate a
  contract, for example, because the subprogram interfaces with an
  external device that is not modelled.

\item \emph{Proof results not useable} - standards require code to
  be fully tested. However, as an addition, performing proof could
  improve the safety case and grant more confidence on the code. In
  this scenario the user would try to prove as much as possible, but
  would not insist when proof is too hard. Also, proof of some
  properties can often be used to support an argument that
  unit testing of all subprograms is not required.

\end{enumerate}


By walking through the examples, we will demonstrate how a combination
of the use of the SPARK 2014 tool set and a commercially available unit
testing tool have the potential to dramatically reduce subprogram
verification effort.

We conclude by summarising our views on the relative merits of
alternative styles of contracts. Specifically, concluding where
different types of contracts are most productively used.


\bibliographystyle{plain}
\bibliography{proof_and_test}

\end{document}
