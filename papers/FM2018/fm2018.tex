\documentclass{eceasst}

% Volume frontmatter for AVoCS 2018
% =====================================
\volume{XXX}{2018} % Volume number and year
\volumetitle{% Title of the volume (optional)
Automated Verification of Critical Systems 2018\\
(AVoCS 2018)}
\volumeshort{% Short title of the volume (optional)
AVoCS 2018}
\guesteds{% Multiple guest editors
David Pichardie, Mihaela Sighireanu}


\usepackage[utf8]{inputenc}
\usepackage{graphicx}
\usepackage{url}
\usepackage{xspace}
\usepackage{amsmath}
\usepackage{hyperref}
\usepackage{color}
\usepackage{listings}
\usepackage[table]{xcolor}
\newcommand{\CodeSymbol}[1]{#1}
%\usepackage{lineno}
%\pagestyle{plain}
%\linenumbers
\lstset{
   language=Ada,
   keywordstyle=\ttfamily\bf,
   showspaces=false,
   basicstyle={\scriptsize \sffamily},
   commentstyle=\color{red}\textit,
   stringstyle=\ttfamily,
   string=[b]",  % remove ' from string delimiter as it interfers with attributes
   showtabs=false,
   showstringspaces=false,
   morekeywords=[1]Pre,
   morekeywords=[1]Post,
   morekeywords=[1]Test\_Case,
   morekeywords=[1]Contract\_Cases,
   morekeywords=[1]some,
   morekeywords=[1]Old,
   morekeywords=[1]Global,
   morekeywords=[1]Depends,
   morekeywords=[1]Loop\_Invariant,
   morekeywords=[1]Loop\_Variant,
   morekeywords=[1]Loop\_Entry,
   morekeywords=[1]Increases,
   literate={(}{{\CodeSymbol{(}}}1
            {)}{{\CodeSymbol{)}}}1
            {>}{{\CodeSymbol{$>$}}}1
            {>=}{{\CodeSymbol{$\ge$}}}1
            {<}{{\CodeSymbol{$<$}}}1
            {<=}{{\CodeSymbol{$\le$}}}1
            {=}{{\CodeSymbol{$=$}}}1
            {:}{{\CodeSymbol{$:$}}}1
            {.}{{\CodeSymbol{$.$}}}1
            {;}{{\CodeSymbol{$;$}}}1
            {/=}{{\CodeSymbol{$\ne$}}}1
            {=>}{{\CodeSymbol{$\Rightarrow$}}}1
            {->}{{\CodeSymbol{$\rightarrow$}}}1
            {<->}{{\CodeSymbol{$\leftrightarrow$}}}1
}

\newcommand{\DO}{\textsc{do-178}\xspace}
\newcommand{\DOB}{\textsc{do-178b}\xspace}
\newcommand{\DOC}{\textsc{do-178c}\xspace}
\newcommand{\hilite}{Hi-Lite\xspace}
\newcommand{\openetcs}{openETCS\xspace}
\newcommand{\gnatprove}{GNATprove\xspace}
\newcommand{\oldspark}{SPARK~2005\xspace}
\newcommand{\newspark}{SPARK~2014\xspace}
\newcommand{\spark}{SPARK\xspace}
\newcommand{\ada}{Ada\xspace}
\newcommand{\adatwtw}{Ada~2012\xspace}
\newcommand{\altergo}{Alt-Ergo\xspace}

\newcommand{\etc}{\textit{etc.}\xspace}
\newcommand{\ie}{\textit{i.e.}\xspace}
\newcommand{\adhoc}{\textit{ad hoc}\xspace}
\newcommand{\Eg}{\textit{E.g.}\xspace}
\newcommand{\eg}{\textit{e.g.}\xspace}
\newcommand{\etal}{\textit{et al.}\xspace}
\newcommand{\wrt}{w.r.t.\xspace}
\newcommand{\aka}{a.k.a.\xspace}
\newcommand{\resp}{resp.\xspace}

\urlstyle{sf}

\title{Climbing the Software Assurance Ladder - Practical Formal Verification
  for Reliable Software}
\short{Climbing the Software Assurance Ladder}

\author{Claire~Dross\autref{1}\sponsor{Work partly supported by the Joint Laboratory
    ProofInUse (ANR-13-LAB3-0007, \url{https://why3.gitlabpages.inria.fr/proofinuse/})} ,
  Guillaume~Foliard\autref{2},
  Théo~Jouanny\autref{3}, \\ Lionel~Matias\autref{2},
  Stuart~Matthews\autref{4}, Jean-Marc~Mota\autref{5}, \\
  Yannick~Moy\autref{1}, Pascal~Pignard\autref{3}, Romain~Soulat\autref{5}}

\institute{\autlabel{1} AdaCore, F-75009 Paris\\ \autlabel{2} Thales Air Systems, F-91470 Limours\\
  \autlabel{3} Thales Communications \& Security, F-49300 Cholet\\ \autlabel{4}Altran, Bath BA1 1AN,
  United Kingdom\\ \autlabel{5} Thales Research \& Technologies, F-91767 Palaiseau}

\keywords{Certification, Formal methods, Programming by contract}

\abstract{
There is a strong link between software quality and software reliability. By
decreasing the probability of imperfection in the software, we can augment its
reliability guarantees. At one extreme, software with one unknown bug is not
reliable. At the other extreme, perfect software is fully reliable. Formal
verification with SPARK has been used for years to get as close as possible to
zero-defect software. We present the well-established processes surrounding the
use of SPARK at Altran UK, as well as the deployment experiments performed at
Thales to fine-tune the gradual insertion of formal verification techniques in
existing processes.  Experience of both long-term and new users helped us
define adoption and usage guidelines for SPARK based on five levels of
increasing assurance that map well with industrial needs in practice.
}

\date{}

\begin{document}
\sloppy
\hbadness=9999
\maketitle


% Keywords
%

\section{Introduction}

Taken literally, reliable software is the notion that we can rely on software
to perform as intended. This is also how the international standard bodies and
academic experts define it, as phrased in IEC 60050 terms applied to software:
\textit{``reliability [is the] ability to perform as required, without failure,
  for a given time interval, under given conditions''}. Currently, almost no
software is reliable in this sense, because the intention is usually expressed
in ambiguous natural language, and the confidence that software behaves as
intended is obtained by a combination of development discipline (to avoid
introducing errors) and partial testing of all the possible software behaviors
(to detect errors that were introduced). Hence, reliable software today is more
an aspiration when building the software than a quality of the software
produced.  However, a link between software quality and reliability does exist,
and it was clarified by researcher John Rushby~\cite{RushbySEFM2009}:
\textit{``probability of (im)perfection [..]  provides a bridge between
  correctness, which is the goal of software verification (and especially
  formal verification), and the probabilistic properties such as reliability
  that are the targets for system level assurance.''}

This interpretation of reliable software as probably perfect software has been
taken seriously in some companies like Altran UK, where specifications are
routinely expressed with the precision of a formal language, and confidence is
obtained by a combination of classical techniques plus the guarantees provided
by the use of formal verification. Tools for formal verification of software
have reached a degree of automation and usability that makes them suitable for
use in commercial contexts across a large set of industries, from the
well-established - space, railway, aerospace \& defense - to industries that
more recently included software as a critical component like automotive and
medical. The main tool used at Altran UK for formal specification, programming
and formal verification of software is SPARK~\cite{mccormick15}, a subset of
the Ada programming language targeted at safety- and security-critical
applications. The use of SPARK allows Altran UK to provide assurance that
software will not crash or behave erratically, and that critical properties are
satisfied, which it demonstrates by committing to these properties with its
customers.

While the benefits obtained by formal verification at Altran UK are clearly
desirable, it may be intimidating for companies without formal verification
know-how to start on this path. Knowing that others have replicated these
benefits in other contexts is an important argument to make.  Here, we are
describing the experiments done at Thales with the support of AdaCore,
during the years 2016 and 2017, to
assess the costs and benefits of using formal verification of software using
SPARK. With little investment in training (2 days in one experiment, self-training
only in the other) and consulting (20 days in one experiment, online support only
in the other), either provided
externally or through self-training, the operational teams managed to specify
intended behavior formally. The engineers in these teams were knowledgeable
about Ada but not familiar with formal methods.
The teams also proved critical
properties of their software. Specifically: in multiple case studies the code was
fully proved to be free of run-time errors (like buffer overflows and divisions
by zero); in a subset of these case studies the code was proved to implement
functional API specifications; while in another case study the code was proved to
follow a specified safety automaton.

In addition, the collaboration of AdaCore and Thales resulted in a set of
guidelines~\cite{AdaCoreThalesSPARK} that should be followed for an easier
adoption of formal verification in existing projects, codebases and
processes. These guidelines are based on five levels of assurance that can be
achieved on software, in increasing order of costs and benefits. These
guidelines are a major result of this work, as there is very little available
guidance on the use of specific formal methods and tools in industrial
context. It could be used as an inspiration by other formal verification
platforms for software.

In section~\ref{sec:SPARK}, we introduce the SPARK formal verification
platform. In section~\ref{sec:practice}, we present the experience of Altran
UK, a long-time user of formal verification with SPARK, and how it relates to
traditional assurance levels (DAL and SIL) considered in industry. In
section~\ref{sec:adoption}, we present two studies carried at Thales, a recent
user of formal verification with SPARK, and how such adoption can be
facilitated by the use of suitable guidelines. We finish with related works in
section~\ref{sec:related-works} and conclude.

\section{SPARK: Formal Verification Focused on Practical Use}
\label{sec:SPARK}

For his PhD defense in 1969 on ``A Program Verifier'', J. King submitted a
manuscript that started with these sentences~\cite{King1970PhD}:

\begin{quote}
This research is a first step toward developing a ``verifying compiler''. Such
a compiler, as well as doing the standard translation of a program to machine
executable form, attempts to prove that the program is ``correct''. In order to
do this a program must be annotated with propositions in a mathematical
notation which define the ``correct'' relations among the program
variables. The verifying compiler then checks for consistency between the
program and its propositions.
\end{quote}

To most programmers, this may sound like a naive dream, whose illusory nature
is exemplified by C.A.R. Hoare’s call in 2003 for researchers to tackle the
``verifying compiler'' as a Grand Challenge, more than thirty years later. Yet,
``verifying compilers'' are available today. For example, the formal
development environments Coq and Isabelle/HOL have been used to create a
compiler for C~\cite{Leroy-backend} and a microkernel~\cite{Klein2009SOSP}
which are guaranteed to be ``correct'' (related to a set of requirements and
assumptions).

The problem is that these ``verifying compilers'' are operating on proof
languages that are reserved for experts. Since King's PhD defense, there have
been numerous attempts at defining practical ``verifying compilers'' for
programming languages used in industry (first Pascal, then Ada, more recently
Java and C\#), none of which has succeeded in gaining industrial adoption. It
is difficult to prove that a program is ``correct'', and it will remain so for
the foreseeable future. As Rustan Leino, a prominent researcher in formal
program verification, put it in 2010: \textit{``Program verification is
  unusable. But perhaps not useless.''}~\cite{Leino10usableauto-active}

Departing from this academic tradition, SPARK has been focused on practical
formal verification from the start. SPARK has been adopted in numerous large
industrial projects and only critical parts of the software were proved
``correct'' with respect to full functional (\ie behavioral) specification. SPARK was used to
prove specific properties of interest about the software, like the absence of
all possible run-time errors (no division by zero, no buffer overflow, etc.) and
some user-specified safety or security properties.

SPARK is both a language and a toolset, supported by specific development and
verification processes. In this article, we are focusing on the latest
generation of SPARK technology, called SPARK 2014~\cite{sparkERTS2014}, in
which the specification language and the programming language have been unified
as a subset of the programming language Ada 2012. Constraints on both program
data and control can be specified using respectively type contracts (predicates
and invariants) and function contracts (preconditions and
postconditions).

The concept of program contracts was invented by the researcher C.A.R. Hoare in
1969 in the context of reasoning about programs. In the mid-1980s, another
researcher, Bertrand Meyer, introduced the modern function contract and type
invariant in the Eiffel programming language~\cite{meyer:1988:OSC}.  In its
simplest formulation, a function contract consists of two Boolean expressions:
a precondition to specify input constraints and a postcondition to specify
output constraints.  Function contracts have subsequently been included in many
other languages, either as part of the language (\eg contracts for SPARK),
as part of the standard library (\eg CodeContracts for .NET~\cite{CodeContracts}) or as an annotation language
(such as JML for Java~\cite{JML} or ACSL for C~\cite{acsl}). Type invariants
may come in two forms, depending on whether they can be temporarily violated
(type invariants in SPARK) or not (type predicates in SPARK). Contracts can be
executed as runtime assertions, interpreted as logic formulas by analysis
tools, or both.

The latest version of SPARK has opted for both.
This design choice has far-reaching consequences. First, specifying properties
of programs is similar to programming: there is no additional language to learn
and the tools available to the programmer also work on specifications. Second,
contracts are executable, which means that they can be tested and debugged like
code. Another important design choice was to allow SPARK and Ada code to
coexist in the same files. Hybrid verification is obtained by using proof on
SPARK code and test on Ada code. This is possible because contracts can be
executed, and because test and proof use the exact same semantics for
contracts~\cite{tseChalin10}. Other formal program verification technologies
like Frama-C for C programs have made similar although not identical
choices~\cite{kosmatov:hal-01344110}.

SPARK toolset focuses on automation and usability. Generation of implicit
specifications lowers the cost of writing specifications, and generation of
loop invariants, use of multiple state-of-the-art automatic provers, possible
generation of counterexamples when proof fails, combination of static analysis
and proof, all lower the cost of proof by reducing the time and effort required
to prove that the code respects its contracts. Usability is similar to other
tools in the developer’s toolbox, mostly because formal verification can be
performed by developers while they are developing, using their personal
computers, thanks to the modularity and parallelization of the analysis.

We identify five levels of assurance that can be achieved with SPARK, which are
- in increasing order:

\begin{enumerate}
\item Stone level - valid SPARK
\item Bronze level - initialization and correct data flow
\item Silver level - absence of run-time errors (AoRTE)
\item Gold level - proof of key properties
\item Platinum level - full functional correctness
\end{enumerate}

At Stone level, strict SPARK rules are enforced on the code, having the effect
of ensuring that a strong semantic coding standard is followed, which leads to
better code quality and maintainability. At Bronze level, the SPARK code is
guaranteed to be free from a number of defects like reads of uninitialized
variables. At Silver level, the SPARK code is guaranteed to be free of run-time
errors. At Gold level, the SPARK code is guaranteed to respect key integrity
properties. At Platinum level, the SPARK code is guaranteed to implement a
complete specification of intended behavior. Note that each level builds on the
previous one, so that at Platinum level the guarantees given by all the lower
levels are also achieved.

\section{The Practice of Formal Verification}
\label{sec:practice}

Altran UK has a special relationship with the SPARK technology, being the heir
of both PVL and then Praxis, the companies which have developed SPARK since
1987~\cite{Chapman2014ITP}. Along the years, Altran has used SPARK both directly
and in partnership with our customers - through training, support and
consulting - in a number of project domains which range across air traffic
management, airborne systems, avionics, railway control \& protection, security
and defense systems.

SPARK is used at Altran as an efficient means to both get as close as possible
to zero-defect software and as a means to address the objectives of the
relevant standards. This technical strategy has been subject to careful
evaluation of costs and benefits, in order to apply formal verification where
it brings more value to the business. At Altran UK, SPARK fits within an
overall software development philosophy known as Correctness By
Construction~\cite{Croxford2005Manifesto}. The key principles of this approach
are:

\begin{itemize}
\item to use techniques that prevent the introduction of errors (e.g. language
  subsets);
\item to maximize the ability to detect defects early (e.g. through the use of
  formal techniques);
\item to generate assurance evidence as you progress.
\end{itemize}

The detail of how SPARK is applied varies from project to project, depending on
factors which include the required integrity level, applicable standards, and
the overall verification strategy for the system (in which SPARK will play a
part amongst other techniques and tools). Together, these considerations will
lead to a set of verification objectives for SPARK, which will be documented in
the technical plan at the start of a project (and which in turn support the
assurance case either implicitly or explicitly if it is a formal deliverable).

In spite of these variations, one can identify certain typical ways in which
SPARK is applied on projects which have been shown to deliver high value in
relation to the effort required. The table in Figure~\ref{fig:levels}
summarizes Altran's experience of how best to apply the different assurance levels
possible with SPARK vs. the relative design integrity level of the software
under development.  Stone level is not represented as it is more an
intermediate level during adoption of SPARK than a target assurance level.

\begin{figure}

\begin{center}
\begin{tabular}{|p{1.5cm}|p{1.5cm}|p{1.5cm}|p{1.5cm}|p{1.5cm}|p{1.5cm}|} \hline
\multicolumn{2}{|c|}{Software Integrity Level} & \multicolumn{4}{c|}{SPARK Verification Objective} \\  \hline
DAL & SIL & Bronze & Silver & Gold & Platinum \\ \hline
A   & 4   &        & \cellcolor{black} & \cellcolor{black} & \cellcolor{black} \\ \hline
B   & 3   &        & \cellcolor{black} & \cellcolor{black} & \cellcolor{black} \\ \hline
C   & 2   &        & \cellcolor{black!50} & \cellcolor{black!50} & \\ \hline
D   & 1   &        & \cellcolor{black!50} & \cellcolor{black!50} & \\ \hline
E   & 0   & \cellcolor{gray!30} & \cellcolor{gray!30} & & \\ \hline
\end{tabular}
\end{center}

\caption{Technical Planning Guidelines for the Application of SPARK.  The
  filled cells denote the three most common categories of application.}
\label{fig:levels}
\end{figure}

The way to understand this table is as both a summary of experience on
industrial projects at Altran UK and as a starting point for how Altran UK
approaches new projects. Every project at Altran will tailor its own
approach. However, one would expect new projects to fall within the typical
region(s) indicated in the table; any which did not would require justification
in the planning phase.

We have chosen to represent the software integrity level using two commonly
understood scales: DAL (Design Assurance Level) is the terminology from DO-178
and SIL (Software Integrity Level) is the terminology used in DEFSTAN 00-55,
IEC 61508, EN 50128 et al. The correspondence between DAL and SIL is
necessarily informal because different standards define the levels according to
different criteria. Note also that while DAL-E is defined by DO-178 its
counterpart ``SIL-0'' is an informal but widely used term taken here to mean
software below SIL-1 but which is still well-engineered.

Experience shows that projects can be grouped into three broad categories, shown by the three
filled regions in the table. Category 1, shown in black, represents our
practice at the highest levels of integrity SIL-3 and SIL-4. Within this
category, Silver (AoRTE proof) is considered the ``default level'', but may be
increased to Gold or even Platinum depending on whether key properties and
functional correctness respectively are verified by other means. Targeting
Platinum (full functional proof) becomes less likely for a SIL-3 system where
verification by testing could more easily be argued to be sufficient.

Category 2, shown in gray, captures our practice at medium levels of integrity
SIL-1 and SIL-2. Silver is still the default level, and it is very unlikely
that Platinum would be employed on systems below SIL-3.  However, proof of key
properties (Gold) should still be strongly considered. There may be some key
property where proof represents a very efficient means of verification, \ie it
is relatively easy to prove and relatively difficult to verify by any other
means. The nature of such properties will vary from system to system, but could
include even one key safety property (``the lift will not move when the doors
are open'') or security property (``the account details cannot be accessed when
the user is not logged in''). While testing can provide some level of
confidence in such properties it can never provide a complete guarantee for any
realistically-sized system, due to the impossibility of covering all possible
states and input combinations.

Category 3, shown in light gray, represents the lowest levels of integrity,
so-called SIL-0. Even here, Silver is the default objective,
but this could be weakened to Bronze if there is enough confidence
that AoRTE was being sufficiently-well assured by other means or mitigation.

The table shows that - for all but SIL-0 software - SPARK code will as a
minimum be checked for AoRTE. Note that this level of verification implicitly
means that all SPARK code has also been shown to be free of references to
uninitialized variables and basic data flow errors.  Experience shows that the
presence of this kind of flaw - which can have far-reaching consequences - can
be immensely difficult to detect by other forms of
verification~\cite{King2000TSE}.

A key part of the software engineering process which maximizes the benefit of
SPARK is a careful delineation of the ``SPARK boundary'' \ie choosing which
parts of the application software will be written in SPARK. Although the
benefits of SPARK would push towards maximizing the proportion of the software
written in SPARK, other factors are likely to affect this engineering
decision. For example, there may be pre-existing libraries to support the user
interface or other external communication protocols that one wishes to use and
which are qualified by alternative means. It is not unusual even to use
different levels of SPARK verification within the same application. For example,
SHOLIS~\cite{Croxford2005Manifesto} used this approach with SIL-4 parts of the
application attaining full functional proof (Platinum level) while in
lower-integrity functions (SIL-2) they verified only up to AoRTE (Silver level).
The non-interference between different sections of the code was assured by the
use of information flow analysis: a contract was attached to each subprogram
specifying which global data items it could access in accordance with its SIL
and the SPARK tools were used to check that the implementation respected these
contracts. More generally, consideration has to be given to the assumptions
that are made to support the verification objectives - how these are satisfied
or mitigated by other activities in the overall V\&V
strategy~\cite{kanig2014tap}.

The use of SPARK within the Correctness By Construction framework as described
above has been demonstrated to produce software with very low defect density
when compared to other high-integrity
processes~\cite{Croxford2005Manifesto}. Although the above approach is the
standard approach within Altran UK, the company has continued to explore new ways in
which benefits can be gained from the use of SPARK, in particular the
possibility of so-called ``hybrid'' approaches to verification, where a mixture
of static and dynamic verification techniques are used to exploit the SPARK
contracts.

The hybrid approach that Altran is currently pioneering, called ConTestor, uses
SPARK verification at Silver level \ie assurance of AoRTE using proof. In
addition, SPARK contracts are used to add a functional specification to the
code. Rather than verifying these contracts by proof using the SPARK tools (as
per the standard Platinum approach), they are verified dynamically by
testing. To perform these tests a fully-integrated version of the code is
compiled with the run-time checks enabled for the functional contracts. Test
cases for the integrated code are generated using constrained-random test
generation and if no exceptions are raised during execution then the code has
passed this functional test. The contracts effectively provide a test oracle
\ie an independent calculation of the expected outcome for each test
case. However, rather than having to manually calculate the expected outcomes
per test case, the contracts are written once and provide an implicit
definition of the expected outcome for all possible test cases.

\section{The Adoption of Formal Verification}
\label{sec:adoption}

Contary to Altran UK, Thales has no established use of formal verification, but
different units in Thales have been experimenting since 2015 with formal
verification of programs using SPARK and Frama-C. The two case studies in this
section describe the experiments with SPARK in the context of two different
units working respectively in the domains of air defense systems software and cryptography.

\subsection{First Study: Define an Adoption Strategy}

One trait of established industrial software development processes is their
inertia in accepting new practices which could be considered as too disrupting,
either by lack of understanding and know-how or, mostly for early adopters,
because of the difficulty to assess costs and benefits. In the latter case, the
upfront adoption effort is hiding the longer term process optimisation
opportunities. In order to get a first idea of the possibilities SPARK-based
formal verification could provide at Thales, a study was carried throughout
2016 with the aim of producing a first set of deployment guidelines supported
by real life experiments on actual software applications.

As formal verification with SPARK is not a widespread technology in the
software industry, a prerequisite is to picture the range of its capabilities
with a simple to remember concept. This led to the definition of the five
levels of assurance previously introduced. As part of this study, AdaCore and
Thales wrote a guidance document~\cite{AdaCoreThalesSPARK} describing how SPARK
could be adopted at these different levels. This is further conditioned by the
phase of the software development lifecycle, which has a significant impact on
the definition of activities to be performed when deploying SPARK.

Given the current state of progress of some ongoing software development
projects, four case studies were identified as potential targets for SPARK
deployment experiments, from teams working on air defense systems software.

\paragraph{The first case study} meant to assess the effort for transitioning from Ada to
SPARK code (Stone level) using a mature software application about to be ported
onto a new execution platform. As porting the application on a new platform
using a different compiler may introduce a different behavior in case of errors
such as references to uninitialized variables, reaching the Bronze level seemed
a desirable aim. A significant refactoring effort was required in order to cope
with constructs excluded from the SPARK subset of the Ada language, the most
prominent one being pointers. Thales engineers started using successfully the
refactoring solutions described in the guidance document, but did not manage to
complete refactoring in the expected time frame (5 person-days), due to the
size of the chosen code base (around 300 klocs). This is expected to be
completed in the coming year.

\paragraph{The second case study} focused on programmer proficiency. In that case study the small
subprogram of less than 10 lines of code listed in Figure~\ref{fig:float-case} was given to an experienced Ada
programmer with the goal of performing validation activities, both using the
usual unitary test approach and a contract-based approach. Based on current
tools, it took less than one hour for the experienced software test engineer to
set up a working test environment for the subprogram. On the other hand,
writing relevant contracts on that same subprogram to formally prove properties
took an order of magnitude more time for the same engineer. Interestingly, the
amount of code to implement a contract was in that case as long and complex as
the code to prove. As a consequence, there is no
intention to invest in Gold level verification on numerical computations in the
near future. The lesson is that one should start with the lowest levels of
assurance and work upwards, as practiced in the subsequent case studies.

\begin{figure}
\begin{lstlisting}
   subtype Nb_Type         is Natural     range 0   .. 100;
   subtype D_Time_Type     is Float       range 0.0 .. 1_000.0;
   subtype Delta_Time_Type is D_Time_Type range 0.0 .. 1.0;

   procedure Study_Case (Nb_Of_Fp     : in     Nb_Type;
                         Nb_Of_Pp     : in     Nb_Type;
                         Delta_Time   : in     Delta_Time_Type;
                         Time         : in out Float)
   with
     Pre     => Nb_Of_Pp > 0 and Delta_Time > 0.0 and
                Time >= 0.5 * Float (Nb_Of_Fp + Nb_Of_Pp) * Delta_Time and
                Time < Float'Last - Float (Nb_Of_Fp + Nb_Of_Pp) * Delta_Time,
     Post    => (if Nb_Of_Fp > 0 then Time >= Time'Old)
   is
      D    : D_Time_Type;
      T_Fp : Float;
      T_Pp : Float;
   begin
      D    := Float (Nb_Of_Fp + Nb_Of_Pp) * Delta_Time;
      T_Fp := Time - (D / 2.0);
      T_Pp := T_Fp + Float (Nb_Of_Fp) * Delta_Time;
      Time := T_Pp + 0.5 * Float (Nb_Of_Fp) * Delta_Time;
   end Study_Case;
\end{lstlisting}
\caption{Simple problematic case for formal verification. The initial
  postcondition contained a strict inequality \texttt{Time > Time'Old} that is
  not true for high values of \texttt{Time} where the offset is absorbed. Even
  the fixed postcondition with a non-strict inequality is not provable as it
  depends on the respective magnitures of \texttt{Nb\_Of\_Fp} and \texttt{Nb\_Of\_Pp}
  which are not specified in precondition.}
\label{fig:float-case}
\end{figure}

\paragraph{The third case study} was designed to complement test result artifacts on
automatically generated code. A large amount of the unit's software application source
code relating to data binary serialization and deserialization is automatically
generated. The code generator compiles data models described through a domain
specific language into Ada code. Up to now
the test strategy for the code generator was mostly based on a limited set of
regression tests and the confidence acquired over time as this
technology was deployed across many projects over the last fifteen years. However, a hard to
trigger weakness was lying dormant, which was cleaned up using a Gold
level approach. With the support of SPARK experts, a first stage was to correct
and refactor the code (2 kloc for the runtime and 21 kloc of generated code) to
pass Stone, Bronze and Silver levels. For code written by savvy programmers
making a moderate use of specialized language features such levels are easy
targets, in this case less than half a person-day for a few hundreds lines of
code. Reaching Gold level to prove one property related to buffer overflow
on the generated code
required a larger effort, two person-days in that case, in order to refactor
the code for proof (to avoid the weakness mentioned above related to buffer overflow),
interact with automatic provers through intermediate
assertions and provide the required loop invariants. Given the extra level of
confidence regarding the robustness these changes provide, Thales plans to
deploy them in the next release of the code generator.

\paragraph{The fourth case study} targeted the proof of safety properties in a context where
safety standards apply. Safety properties are usually written as “nothing bad
will ever happen” and, since their scope is usually on a large part of the
code, need to be specified at the highest level of the code, almost at the
entry point. Inside a 70 kloc control commands project, Thales and AdaCore
engineers identified a few units (7 kloc) defining a set of high level automata
where those properties could be specified.  As a first step, the engineers reached the
Stone, Bronze and Silver levels on this code in less than a person-day. Then,
contracts were added on subprograms implementing the automata, mostly to
express the effect of calling each automaton, also in less than a
person-day. Automatic proof was obtained without much difficulty after that,
with no need for intermediate assertions, loop invariants or specific proof
switches. The lessons learned here are that SPARK is expressive enough for
typical safety automata properties, and powerful enough for automatic proof of
such properties.

\paragraph{Lessons learned.}
From an adoption point of view, Thales concluded from this first study that
formal verification as implemented by SPARK 2014 and its associated toolset can
be considered as a toolbox providing various opportunities for subsetting and
tailoring. This flexibility gives the possibility to fine-tune the gradual
insertion of formal verification techniques in existing processes, while
mitigating risks both on their efficiency from a cost and planning point of
view and their ability to output software with a defect density under control.

\subsection{Second Study: Implement and Refine the Adoption Strategy}

In the field of high-security applications, which is particularly important for
Thales, testing represents a considerable part of the software development
process. In addition to unit tests, other principles are implemented such as
enforcing coding rules, peer code reviews and qualimetry surveys with many
tools checking that those principles are strictly followed. One solution to
lighten and improve this process, to produce software of improved quality, is
the use of more suitable tools, such as formal verification tools to replace
part of the tests. Indeed, formal proofs allow a comprehensive checking of
proved parts, unlike testing that can only guarantee a partial checking of the
software.

After a previous internship in 2015 comparing some available environments for formal
verification (eCv~\cite{Crocker2014CMS}, Frama-C~\cite{Kirchner2015}, SPARK),
another six-months internship in 2017 was dedicated to the study of the
benefits of the Ada language and particularly the SPARK language for the
security software developed at Thales. During this internship, Thales evaluated the various advantages of Ada and
SPARK, by implementing the Adacore and Thales adoption guidance on two proofs
of concept in the field of cryptography.

\paragraph{The first case study} was porting from C to Ada, then to SPARK, part of a
cryptographic library which is used as an abstraction layer between a lower
level cryptographic library (also in C) and client applications. This case study
followed the guidance document produced in the previously mentioned first
study, to convert an application from Ada to SPARK.

The preliminary stage consisted of porting the C library code to valid Ada
code. Porting API (.h files) was facilitated by g++ switch ``–fdump-ada-spec''
which produced comprehensive Ada specifications (.ads files) as well as Ada
body skeletons (.adb files) generated automatically with the gnatstub tool. The
body code was completed manually without difficulties as most C idioms are
available with Ada. Interfacing with the C low-level cryptographic library was
essential and was supported natively by Ada. This small step brought
simpler code with pointer-related defensive code eliminated thanks to the use
of handy Ada array attributes and warnings from the Ada compiler.

Firstly, Stone level was reached by transforming the Ada code to be valid SPARK
code. It mostly consisted in suppression of pointers (or at least encapsulating
them in a non-SPARK unit) and transformation of functions with side effects
into procedures (or at least encapsulating them in wrappers within a non-SPARK
unit). Thus, it was possible to make a first analysis of the code with SPARK
tools. This first step didn't require major changes in the code but it
pinpointed parts of the code with potential security vulnerabilities
(pointer casts and side effects in particular).

In a second stage, Bronze level was reached, analyzing the code for data flow
and variable initialization. Data flow (Global) and information flow (Depends)
contracts were added in the code to specify precisely the intended
behavior. The analysis detected unused inputs which could then be removed,
which is useful for maintenance, as well as partially initialized data
structures, which is useful for debugging.

In a third stage, Silver level was reached, ensuring absence of run-time errors
in the code (AoRTE). Preconditions were added in the code, mostly to link the
right algorithm with the right variant of a discriminated structure.

In a fourth stage, Gold level was reached, verifying the functional behavior of
the code.  Preconditions and postconditions were added in the code to specify key
security requirements: cleanup of security-sensitive working variables,
correctness of output value, and consistency between parameters as presented in
Figure~\ref{crypto-case}. At this level, all
the existing defensive code had been replaced by contracts. By achieving
complete proof of these specifications, the propagation of error codes from low-level
subprograms to high-level ones was no longer necessary.

\begin{figure}
\begin{lstlisting}
   procedure computeSha (input         : in     uint8_t_array;
                         inputByteLen  : in     stdint_h.uint32_t;
                         digest        :    out uint8_t_array;
                         outputByteLen : in     stdint_h.uint32_t;
                         hashByteLen   : in     stdint_h.uint32_t)
   with
     Contract_Cases => (hashByteLen = 20 => digest'Length = 20,
                       hashByteLen = 32 => digest'Length = 32,
                       hashByteLen = 48 => digest'Length = 48,
                       hashByteLen = 64 => digest'Length = 64),
     Pre => (inputByteLen = input'Length and
            (hashByteLen = 20 or hashByteLen = 32 or
             hashByteLen = 48 or hashByteLen = 64));
\end{lstlisting}
\caption{Simple case of contract for expressing consistency between parameters,
  checking here that the length of the hashed message \texttt{digest'Length} is
  consistent with the type of hash used \texttt{hashByteLen}.}
\label{fig:crypto-case}
\end{figure}

\paragraph{The second case study} was about producing an API similar to the API ported from C
during the first proof of concept, this time based on a low-level cryptographic
library in Ada, which was also later proved with SPARK. The whole process from
Stone level to Gold level was followed again. New technical
issues arised: the need for loop invariants, contracts on type hierarchies for
subprograms supporting dispatching, visibility of global variables in contracts
of high level subprograms, and non-provable Ada code. Though loop invariants are
the basis of formal proof, they are considered as tricky. Many unproved
properties came mostly from weak preconditions or weak postconditions of
subprograms called inside a loop, which were not obvious to understand. Object
Oriented Programming brings another layer of complexity, with specific rules
for inheriting subprograms and contracts over these subprograms. Global
variables mentioned in data flow contracts propagate to the upper levels of the
call tree, where they may not be visible anymore (due to abstraction mechanisms in Ada),
which required costly
workarounds. A better solution would have been to hide this particular effect
in a low-level non-SPARK package body, or to use the data abstraction feature
available in SPARK. Finally, some idiomatic Ada code did not lead to automatic
proofs in SPARK, which led to changes for
simpler and more readable code.

\paragraph{Lessons learned.}
Thales learned a few lessons from this second study. First, the adoption
guidance document was really helpful: it eased the implementation of SPARK
during the second internship. As a result, it was also refined for future uses
inside and outside Thales. Secondly, as stated in the guidance document,
\textit{Users should refrain from changing the program for the benefit of only
  getting fewer messages from the tool}, a principle that could be phrased as
``do not please the tools''. Of course, it is sometimes adequate to change the
program in a way that will cause some messages about unproved properties to
disappear, provided this favors code quality, readability or
maintenance. Otherwise, tools provide ways to silence messages, that should be
used instead of changing the program.  Thirdly, reaching Gold Level is more
easily achievable when clear and meaningful software specifications are
available. Finally, not all code can be proved but non-provable parts that are
well identified can undergo peer code reviews. For instance, 90\% of the code
was automatically provable in the second case study, after suitable addition of
contracts where necessary.

\section{Related Works}
\label{sec:related-works}

Formal methods have long been considered as a means of compliance to satisfy
verification objectives in critical software development for some certification
domains, for example in railway (EN 50128) and industrial processes (IEC
61508). The avionics standard DO-178C in 2012 has more recently recognized
formal methods as a means of compliance on a par with the dominant technique of
testing. Other certification standards in the domains of automotive (ISO
26262), nuclear (IEC 60880) and space (ECSS-QST-80C) also recognize some uses
of formal methods as verification techniques.

The adequateness of formal methods for certification was thoroughly
investigated by John Rushby in his report for the NASA in 1993 on ``Formal
Methods and the Certification of Critical
Systems''~\cite{Rushby93formalmethods}. Although Rushby's report talks about
``formal methods'', this mostly corresponds to what we call today ``theorem
proving'': model checking techniques are mentioned en passant, and nothing is
said about abstract interpretation techniques, which did not have then the
recognition that they do today. More recently, Graf and Garavel studied
extensively the use of formal methods for developing critical systems, and they
cover in particular the impact of formal methods on development and
verification processes~\cite{GrafGaravel-BSI-2013}. More specific guidance
exists in certain application domains, such as in
avionics~\cite{BrownERTS2010}.

There is on the contrary very little guidance on the use of specific formal
methods and tools. This is somewhat remediated by the availability of tool
specific user guides and publicly available experience
reports~\cite{Woodcock2009}. Company-specific guidance is developed to
carry-over the experience gained from project to project, in the companies
using formal methods, but such guidance is kept confidential. Indeed, the
experience gathered through previous projects is considered as a business
advantage over the competition, and the guidance having been developed in the
specific business context of the company, the information related to formal
methods usage is very tied to other confidential information. In their dual
role of SPARK tools providers and practitioners, Praxis and then Altran have
always been keen on publicizing best practices and lessons learned with formal
verification on industrial
projects~\cite{KingHCP00,Chapman2006CCM,Chapman2014ITP}. The publication of the
guidance co-developed between AdaCore and Thales~\cite{AdaCoreThalesSPARK} on
SPARK adoption follows this lead, which was possible because it was written
since the start with the tool provider. This is similar to the joint effort by
tool provider CEA, certifier Bureau Veritas and user Sirehna to publish
guidelines on the use of Frama-C~\cite{FramaCguidelines}.

Formal methods have been divided between heavyweight and lightweight ones, with
the former being the original formal methods and the latter also being called
the \textit{disappearing} formal methods~\cite{hase00}. SPARK is a case of
lightweight or disappearing formal methods, in which the user does not directly
manipulate the underlying formalism, but instead interacts with tools through
multiple interfaces. Formal methods and tools are usually placed at some point
in between the heavyweight and lightweight extreme points. With the notion of
software assurance levels, we have shown that a given tool can be placed at
multiple places along this axis, and that a project can move between these
places using the same tool.

In particular, it is likely that other formal verification platforms for
software such as Atelier B and Frama-C could similarly define their own
software assurance levels. For example, the plugin structure of Frama-C could
be used to define levels in terms of plugin usage~\cite{Kirchner2015}.

\section{Conclusion}
\label{sec:conclusion}

Formal program verification with SPARK has been used for years at companies
like Altran UK to get as close as possible to zero-defect software. Altran UK
has developed software engineering processes to maximize the costs-benefit
ratio of using SPARK. In particular, it has defined a mapping between levels of
use of SPARK and software assurance targets (SIL/DAL), which is used by all
projects at Altran UK. Altran UK is now investing in its use of SPARK for the
future, by investigating innovative ways to generate tests from contracts, to
combine tests and proofs and to analyze code generated from Simulink.

Other companies like Thales are starting to use SPARK to obtain similar
benefits. We have presented in this article the lessons learned at Thales on
various deployment experiments at different levels of use of SPARK. As for
every promising but complex technology, the success of its deployment is
conditioned by the pace at which adopters can climb the learning curve and
identify relevant insertion points and strategies into established development
processes. While AdaCore expertise was essential in the success of these
experiments, Thales has identified typical use cases where the methodology used
could be replicated without external help. Thales is now aiming at clarifying
how SPARK can be adapted to its internal processes. The guidance document
written as a result of Thales experiments is being used to support adoption of
SPARK in other teams inside Thales and is available for other companies to
start on this path.

We have benefited in multiple ways from the definition of the five software
assurance levels that can be achieved with SPARK. First, the five levels
clarify the verification objectives that can be achieved with formal
verification: not only they provide simple and easy-to-remember names for
communicating between stakeholders, they also make it explicit that upper
levels build on the lower levels, and they provide at each level a clear
identification of the costs and benefits. Secondly, the five levels make it
easier to plan for progressive adoption of higher levels of software assurance,
with lower levels requiring less effort than higher levels, and each level
providing already very valuable benefits. These results could be translated to
other formal methods that similarly provide different depths of use that could
be translated to assurance levels.

\paragraph*{Acknowledgements.}

We would like to thank the anonymous referees for their useful remarks, as well
as our colleagues at AdaCore, Altran and Thales for their reviews on earlier
drafts of this article.

\bibliographystyle{eceasst}
\bibliography{fm2018}

\end{document}
