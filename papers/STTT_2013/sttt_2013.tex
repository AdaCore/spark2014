\documentclass[sttt,draft]{svjour}
% relevant options: draft, referee, final
%
%\usepackage{latexsym}
%\usepackage{graphics}
%\usepackage{amssymb}
%\usepackage{amsfonts,amsmath}
\usepackage[T1]{fontenc}
%
\begin{document}
%
\title{NewSPARK and GNATprove Kick Ass}
\subtitle{Possible Subtitle}
\author{Duc Hoang\inst{1} \and Yannick Moy\inst{1} \and Angela Wallenburg\inst{2}
}                     % Do not remove
\institute{AdaCore, \email{\{duc.hoang, yannick.moy\}@adacore.com}
  \and Altran, \email{angela.wallenburg@altran.com}}

\date{Received: date / Revised version: date}
% The correct dates will be entered by Springer
%
\maketitle
%
\begin{abstract}
Insert your abstract here.
\end{abstract}
%
\section{Introduction}
\label{intro}
Let's mention for example Hi-Lite \cite{hiliteERTS2012} to get the bib in place. This paper is a cool features list with some discussion around the VerifyThis 2012 competition.

\begin{itemize}
\item SPARK 2005: verification platform and industrial use
\item SPARK 2014: complete update of language and tools. main restrictions (no
  pointer, no exception, no task, function without side-effects, no aliasing)
\item Hi-Lite: research project in which Altran and AdaCore collaborated with
  INRIA to create new tool architecture. compiler and verifier based off same
  frontend, which puts all the checks in the AST.
\item GNATprove: name of the formal verification tool. currently prototype. was
  used at VerifyThis competition
\end{itemize}

\section{Key Language Features for Verification}

in Ada 2012 (which was inspired by SPARK 2005):
\begin{itemize}
\item preconditions and postconditions
\item Old and Result attributes
\item new expressions (if/case/quantified)
\item expression functions
\end{itemize}

\noindent
added in SPARK 2014 based on SPARK 2005:
\begin{itemize}
\item pragmas Loop\_Invariant
\item attribute Update
\end{itemize}

\noindent
new in SPARK 2014:
\begin{itemize}
\item attribute Loop\_Entry
\item pragmas Loop\_Variant
\item aspect Contract\_Cases
\end{itemize}

\section{Integrated Testing and Proving}

\begin{itemize}
\item execution semantics for functional properties $\rightarrow$ debug
  annotations like code
\item contracts can be tested
\item same semantics for test and proof (Chalin's work)
\item methodology for mixing test and proof. compiler inserts additional needed
  checks to verify at run time assumptions made during proof.
\end{itemize}

\section{Choosing the Right Semantics for Integers}

example of problem:
\begin{itemize}
\item user wants to add two numbers: \verb|X + Y|
\item user wants to assert that addition cannot overflow:\\
  \verb|with Pre => X + Y in Integer|
\item but this expression may overflow itself!
\end{itemize}

3 overflow checking modes:
\begin{itemize}
\item strict mode: normal overflow checks
\item minimized mode: larger base type (64bits) used when needed
\item eliminated mode: use bignum library in the remaining cases
\end{itemize}

flexible solution:
\begin{itemize}
\item user chooses between 3 modes
\item independent choice for assertions and code
\item same choice for execution and formal verification
\end{itemize}

\section{Making Automatic Verification Work}

\begin{itemize}
\item bulk parallelism
\item use of dependencies and signatures to minimize rework
\item generation of small contexts in VCs (modularity, abstraction in Why3)
\item fast-WP generation (Leino's work)
\item portfolio approach (various provers, allowed with Why3)
\item selection of hypotheses
\item manual axiomatization of critical data structures (e.g. containers)
\item axiomatization as decision procedures (Claire's work as well as François
  Bobot)
\end{itemize}

\section{VerifyThis Competition}

\begin{itemize}
\item solution for challenge 1
\item solution for challenge 2
\item use of execution to debug annotations during competition $\rightarrow$
  prize of user-assistance tool feature
\item snapshot of GNATprove results in GPS
\end{itemize}

\section{Ongoing Work}

\begin{itemize}
\item IDE integration (analyze line/file, show path)
\item dataflow and information flow verification (in particular initialization)
\item proof of code containing containers
\item visualization of assumptions made for proof
\item counterexample generation and visualization for unproved VCs
\item Coq formalization of AST produced by frontend. possible verification of
  the consistency of the generated AST with checks.
\item target parametrization
\item adaptation for non-executable annotations and proof assistants
\end{itemize}

\label{verifythis}
%
\bibliographystyle{alpha}
\bibliography{sttt_2013}

\end{document}


