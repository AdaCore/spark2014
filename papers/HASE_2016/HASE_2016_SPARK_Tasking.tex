\documentclass[conference,compsoc]{IEEEtran}
\usepackage[nocompress]{cite}
\usepackage{todonotes}

\begin{document}
\title{High-Integrity Multitasking in SPARK:\\
  Static Detection of Data Races and Locking Cycles}

\author{\IEEEauthorblockN{S. Tucker Taft}
\IEEEauthorblockA{AdaCore\\
Lexington, MA USA\\
Email: taft@adacore.com}
\and
\IEEEauthorblockN{Florian Schanda}
\IEEEauthorblockA{Altran UL Limited\\
Bath, UK\\
Email: florian.schanda@altran.com}
\and
\IEEEauthorblockN{Yannick Moy}
\IEEEauthorblockA{AdaCore\\
Paris, France\\
Email: moy@adacore.com}}

\maketitle

%
% FAST ABSTRACT PAPER == 2 PAGES
% http://hase2016.org/index.php/submission-details/12-paper-submission-info
%

\begin{abstract}
  SPARK is a subset of Ada designed to enable formal verification. A new
  release of SPARK 2014, based on the Ada 2012 standard, incorporates
  support for multitasking, based on the Ravenscar Profile, which subsets
  the full Ada tasking model to a relatively static, single-level tasking
  model. This paper describes the safety requirements relating to
  multitasking in this version of SPARK, and the corresponding static
  checks performed by the SPARK 2014 toolset.
\end{abstract}

%\IEEEpeerreviewmaketitle

% Keywords (not in IEEE style)
%
% formal verification; multitasking; SPARK subset of Ada; race detection;
% deadlock detection.

%%%%%%%%%%%%%%%%%%%%%%%%%%%%%%%%%%%%%%%%%%%%%%%%%%%%%%%%%%%%%%%%%%%%%%%%%%%%%%
% key review comments

% r1: language safety, not system safety

% r1: brief summary of spark toolset

% r1: are all checks done with why3 (and why)

% r1: more how, not what

% r2: example of annotation

% r2, r3: illustration (e.g. for cyclic locking checks)

% r2: For the locking cycles, an operation can invoke an external call
% based on the data provided on its input (with an if structure for
% instance). Are the domain of data considered to detect cyclic locking? If
% it is not so, a violation of this rule can be triggered while the data
% don't allow to effectively violate this rule in practice. The last remark
% is about the solution proposed: is this solution built specifically for
% the case of multitasking? Does a similar approach was already proposed
% for a monotasking? (and what is the difference with the proposed
% approach)?

% r3: clarify contribution

% r3: future plans

\section{Introduction}
The SPARK language and toolset \cite{spark} are designed to support the
development and formal verification of software systems providing the
highest level of safety and security. The 2014 version of the SPARK
language is based on the Ada 2012 programming language standard \cite{lrm},
subsetted to enable formal verification, and augmented using the Ada 2012
annotation syntax to support more complete behavioral specifications. The
additional annotations allow the full specification of functional effects
and information flow of every component of the software system, which can
then be verified fully statically, or using a combination of static and
dynamic techniques.

The initial release of SPARK 2014 was limited to the sequential constructs
of the language. The latest release incorporates support for real-time
multitasking, including explicit declarations of \emph{task types} and
\emph{task objects} to provide multiple threads of control, and
data-oriented synchronization and coordination capabilities based on the
monitor-like \emph{protected type} feature of Ada.

The formal verification of SPARK programs is supported by a set of tools
that translate the SPARK programs into an intermediate verification
language (Why3 \cite{why3}), which is then transformed into a set of
verification conditions (VCs) that can be discharged by one or more SMT
solvers \cite{smt} (including CVC4, Alt-Ergo, and Z3). In addition, the
SPARK 2014 programs may be compiled by an Ada 2012 compiler to produce an
executable program, allowing the use of run-time checking of assertion
expressions, including preconditions, postconditions, and type invariants,
for parts of the code not yet statically verified.

As part of supporting concurrency in the new release of SPARK 2014, an
additional set of safety requirements with corresponding static checks has
been specified to ensure that there are no data races, no locking cycles,
and no task suspensions while holding a lock, in the resulting programs.
The remainder of this article will describe the particular safety
requirements associated with SPARK 2014 multitasking, and the corresponding
static checks that are performed by the SPARK 2014 toolset.

\section{Multitasking safety requirements}
The Ada language defines a set of rules for the safe use of data objects
shared among multiple tasks \cite{lrm}. Ada does not generally require that
violations of these rules be detected either statically or dynamically,
and, if they are violated, the effects are generally not predictable. By
contrast, the SPARK 2014 toolset statically detects possible violations of
these rules; in other words, the SPARK 2014 toolset performs static
data-race detection.

Ada also provides mechanisms to avoid cyclic locking structures, which can
lead to deadlock. However, these mechanisms are enforced by a
\emph{priority-ceiling} approach, which only provides protection on a
monoprocessor, and are based on run-time checks between the priority of the
task requesting exclusive access to a resource, and the \emph{ceiling
  priority} of the resource \cite{mccormick2011building}. The SPARK 2014
toolset extends this protection against cyclic locking to multiprocessor
contexts, and enforces the rules statically rather than dynamically.

Finally, Ada has a rule disallowing executing a \emph{potentially blocking}
operation while holding a lock. As with data races, this rule is not
required to be enforced by a standard Ada compiler nor by the Ada run-time
system, though if a violation is detected, the run-time system will raise a
run-time exception. By contrast, the SPARK 2014 toolset enforces this rule
statically.

\section{Static multitasking checks}
The multitasking release of SPARK supports both \emph{tasks}, which
represent a separate thread of control, and \emph{protected objects}, which
are resources with \emph{procedure} and \emph{entry} operations that upon
call acquire exclusive read/write access to the object, and \emph{function}
operations that upon call acquire shared read-only access to the object.

Data races are eliminated in SPARK by a simple rule: any global object
referenced from a task shall be marked as \verb|Part_Of| that task, or be a
\emph{synchronized} object. A \emph{synchronized} object is an object that
can support simultaneous access by multiple tasks without incurring a data
race. This includes protected objects, \emph{atomic} objects (all access is
via atomic instructions), and \emph{suspension} objects (a kind of private
semaphore). Normal objects, such as integer or floating point variables,
are not synchronized, and, if global, may be referenced only by the task
they are associated with via a \verb|Part_Of| annotation. If the global
variable has no \verb|Part_Of| annotation, then it may only be referenced
by the \emph{environment task} of the program, that is the \emph{main}
task, the task in which execution of the program begins.

The data race rule is enforced relatively easily in SPARK because all
subprograms must identify the global variables they manipulate with a
Global annotation. Hence, the main body of a task must not call a
subprogram whose Global annotation identifies an object which is not
synchronized and not \verb|Part_Of| the calling task. The Global annotation
of a subprogram includes all indirect references as well as direct
references, so no hidden side effects are possible. Note that, as a
convenience to the user, the SPARK 2014 toolset can also \emph{infer}
Global annotations automatically, but the rule remains effectively the
same, based on the inferred Global annotations.

Cyclic locking, potential violations of ceiling priority rules, and use of
potentially blocking operations while holding a lock, are all checked for
by the SPARK 2014 toolset by propagating across calls information about the
invocation of operations on protected objects. In a later release,
programmers will be able to declare their usage of protected operations and
potentially blocking operations with explicit annotations on the
specification of a subprogram. But in the current release, the toolset
propagates the information automatically, rather than relying on
user-provided annotations.

For cyclic locking, the check is simply that a protected procedure or entry
of a given object must not make an \emph{external} call on a protected
operation of the same object, directly or indirectly. For the ceiling
priority rule, the check is that the priority of a task must be less than
or equal to the ceiling priority of any protected object it operates upon,
directly or indirectly. Similarly, a protected operation whose object has a
given ceiling priority must not call a protected operation on an object
with a lower ceiling priority, directly or indirectly. Finally, for the
potentially blocking rule, the static check is that a protected operation
does not invoke, directly or indirectly, a potentially blocking operation.

The data-race rule is checked in a \emph{modular} fashion, in that it can
be performed without access to the complete program, thanks to the
user-provided Global annotations. By contrast, these locking-related checks
all rely on a propagation of information across calls, globally throughout
the program. This means that these checks are \emph{not} modular and cannot
be performed without access to the entire program. Once the SPARK toolset
supports the explicit specification on a subprogram of the protected and
potentially blocking operations it performs, these checks can be performed
in a modular fashion. Nevertheless, the toolset will still provide support
for \emph{inferring} these annotations, so the programmer can still have
the convenience of omitting these explicit annotations, presuming they do
not have a requirement for modular checking.

\section{Related work}
Detection of race conditions and deadlocks has a relatively long history,
and both static \cite{engler2003racerx} and dynamic \cite{yu2005racetrack}
approaches have been used. A unique feature of the SPARK detection approach
is that it can provide a guarantee that no data races, no cyclic locking,
and no violations of the ceiling priority and potentially-blocking rules
remain in the program, because the language has itself been subsetted to
enable static detection of all such violations. This guarantee is important
to users of the SPARK language, as the applications are often at the
highest level of criticality. In other contexts, identifying race
conditions or deadlocks is seen as simply finding another kind of program
defect. In SPARK, if multitasking is used, then eliminating possible data
races, cyclic locking, and other sources of potential deadlocks is
considered part of ensuring the overall safety of the software.

% \section{Conclusion}
% TODO

\section*{Acknowledgments}
This work was performed as part of a joint development project of AdaCore
and Altran UK.

\bibliographystyle{IEEEtran}
\bibliography{IEEEabrv,HASE_2016_SPARK_Tasking}

\end{document}
