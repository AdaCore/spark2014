\section{Run-Time Checks Verification}
In SPARK GnatProve tool chain, formal verification for the absence of run time
errors for SPARK program relies on the run time check flags that are initially
generated and inserted to SPARK AST by Gnat front end. Gnat front end is
expected to place the correct run time check flags during its static semantic
analysis for SPARK AST. But the fact is that Gnat front end itself is not formallly verified,
and that's why it's often the case that run time check flags are missing or
misplaced in SPARK AST. It leads to the problem that a SPARK program proved to
be free of run-time errors can still gets into a run time error status because
of the missing or incorrect checks placed by Gnat front end. So it's meaningful
to formally verify the run-time check flags with a formal verification way to
make the GnatProve tool chain sound.

To verify the run time check flags, a check-flag-annotated SPARK language
and its corresponding semantics are formally defined, where each expression and
subexpression nodes are attached with a set of run time check flags. Then a
run-time check generator simulating Gnat front end is defined and formally
proved correct with respect to the SPARK reference semantics. It formalizes the
run time-check generation procedure and generates run-time check flags according
to the SPARK semantics by transforming a SPARK program into a
check-flag-annotated SPARK program. Finally, the run-time check flags
generated by Gnat front end are verified against our formally
proved run-time check flags.

\subsection{Check-Flag-Annotated SPARK Language}
\subsubsection{Syntax}
Check-flag-annotated SPARK language is the same as the SPARK language except
that each expression node is annotated with a set of run-time check flags
\textit{check\_flags}, which denotes explicitly what kinds of run time checks
need to be verified during expression evaluation on the annotated expression.

\begin{lstlisting}[escapechar=\#, language=coq, basicstyle=\scriptsize]
Inductive expr_x: Type :=  
| Name_X: astnum -> name_x -> check_flags -> expr_x 
| ...
with name_x: Type := 
| Indexed_Component_X: astnum -> astnum -> idnum -> expr_x -> check_flags -> name_x 
| ...
\end{lstlisting}

\subsubsection{Semantics}
SPARK reference semantics is formalized with run-time checks being always
performed according to the run-time check requirements by SPARK reference
manual. While in the check-flag-annotated SPARK semantics, run-time checks are
triggered to be performed only if the corresponding check flags are set for the
attached expression node. For example, in the following name evaluation for
check-flag-annotated indexed component, range check is required only if the
\textit{Do\_Range\_Check} flag is set for the index expression \textit{e},
otherwise, the index value \textit{i} is used directly as array indexing without
going through range check procedure \textit{do\_range\_check}.

 \begin{lstlisting}[escapechar=\#, language=coq, basicstyle=\scriptsize]
Inducitve eval_name_x: symboltable_x -> stack -> name_x -> Return value -> Prop :=
| Eval_Indexed_Component_X: forall e cks1 cks2 st s i x_ast_num t l u x a v ast_num, 
    exp_check_flags e = cks1 ++ Do_Range_Check :: cks2 ->
    eval_expr_x st s (update_check_flags e (cks1++cks2)) (Normal (BasicV (Int i))) ->
    fetch_exp_type_x x_ast_num st = Some (Array_Type t) ->
    extract_array_index_range_x st t (Range_X l u) ->
    do_range_check i l u Success ->
    fetchG x s = Some (AggregateV (ArrayV a)) ->
    array_select a i = Some v ->
    eval_name_x st s (Indexed_Component_X ast_num x_ast_num x e nil) (Normal (BasicV v))
| ...
\end{lstlisting}

\subsection{Run-Time Checks Generator}
\subsubsection{Check Generator}
Run-time check generator is a translator from a SPARK program to a
check-flag-annotated SPARK program by generating run-time check flags according
to the run-time checking rules required by SPARK reference manual and inserting
these check flags at the corresponding AST node. In expression check generator
\textit{compile2\_flagged\_exp}, \textit{check\_flags} denote the run-time
checks on the expression required by its context, such as range check for
expression used in indexed component, and other expression check flags are
generated according to the operation type to be performed by the expression.

\begin{lstlisting}[escapechar=\#, language=coq, basicstyle=\small] 
Inductive compile2_flagged_exp: symboltable -> check_flags -> 
                                expression -> expression_x -> Props
\end{lstlisting}

\subsubsection{Soundness Proof}
Run-time check generator is proved sound with respect to the SPARK reference
semantics and check-flag-annotated SPARK semantics. For an expression
\textit{e}, if it's evaluated to some value \textit{v} in state \textit{s} by
SPARK reference semantic evaluator \textit{eval\_expr}, and \textit{e'} is the
check-flag-annotated expression generated from \textit{e} by expression check
generator \textit{compile2\_flagged\_exp}, then \textit{e'} should be evaluated
to the same value \textit{v} in check-flag-annotated SPARK semantic evaluator
\textit{eval\_expr\_x}. Similar soundness proof has been done for statement
check generator.

\begin{lstlisting}[escapechar=\#, language=coq, basicstyle=\small]
Lemma expression_checks_soundness: forall e e' st st' s v,
  eval_expr st s e v ->
    compile2_flagged_exp st nil e e' ->
      compile2_flagged_symbol_table st st' ->
        eval_expr_x st' s e' v.
        
Lemma statement_checks_soundness: forall st s stmt s' stmt' st',
  eval_stmt st s stmt s' -> 
    compile2_flagged_stmt st stmt stmt' ->
      compile2_flagged_symbol_table st st' ->
        eval_stmt_x st' s stmt' s'.
\end{lstlisting}

\subsection{Run-Time Check Optimization}
\subsubsection{Optimization Strategy}
We will do some simple optimizations for literal operations and remove those
checks that can be obviously verified at compilation time. It's also the
optimization strategy taken by the Gnat front end.
\subsubsection{Soundness Proof}
Soundess proof for the run-time checks optimization is being carried out.

\subsection{Run-Time Check Verification}
One of the major goals of our formalization work for SPARK language is
to verify the soundness of run-time check flags produced by Gnat front
end. It's done by comparing the Gnat generated run-time check flags with the
expected ones provided by formally verified check generator with respect to the
SPARK semantics. Run-time check flags are verified to be correct if they are
superset of the expected ones required by the SPARK reference manual.











