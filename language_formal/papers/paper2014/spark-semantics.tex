\section{SPARK Formalization}
\subsection{Syntax of SPARK 2014 Subset}
The subset of SPARK 2014 that we have formalized is significant, which includes
array/record (non-nested) and procedure calls. Furthermore, it also supports
some intersting SPARK / Ada language structures, such as nested procedures and
subtypes.

SPARK AST syntax is represented with inductive type definitions in Coq. And each
AST node is annotated with an unique AST number, which will be used to record
the type for each ast node, or it can be used later to track back to the
SPARK source program when an run time error is detected, or it can be used to
locate the position in source program where the run time check flags inserted by
Gnat front end is incorrect.

Here, we list some of SPARK language structures and show how we formalize them
in Coq. Expression (\textit{expr}) can be literal, unary expression, binary
expression or name, and each expression is annotated with an AST number
(\textit{astnum}), which is represented by natural number. For type
\textit{name}, it can be identifier, indexed component or selected component.
Indexed component is constructed with the constructor
\textit{Indexed\_Component}, whose first \textit{astnum} denotes the indexed
component and the second \textit{astnum} denotes the prefix expression
represented by \textit{idnum} and \textit{expr} is for index expression. 

%\begin{quote}
\begin{lstlisting}[escapechar=\#, language=coq, basicstyle=\small]
Inductive expr: Type := 
| Name: astnum -> name -> expr 
| ...
with name: Type := 
| Identifier: astnum -> idnum -> name 
| Indexed_Component: astnum -> astnum -> idnum -> expr -> name 
| Selected_Component: astnum -> astnum -> idnum -> idnum -> name.
\end{lstlisting}

For procedure \textit{Call} in statement \textit{stmt}, its first
\textit{astnum} is the AST number for the procedure call statement, and the
second \textit{astnum} is the AST number for the called procedure represented by 
\textit{procnum} followed by a list of arguments of type \textit{list expr}.

\begin{lstlisting}[escapechar=\#, language=coq, basicstyle=\small]
Inductive stmt: Type := 
| Assignment: astnum -> name -> expr -> stmt 
| Call: astnum -> astnum -> procnum -> list expr -> stmt 
| ...
\end{lstlisting}

Range constrainted scalar types are used commonly in SPARK programs, they can be
declared with either subtype declaration, derived type definition, or integer
type definition. 
A \textit{Subtype} declares a subtype, represented with
\textit{typenum}, of some previously declared \textit{type} with an additional \textit{range}
constraint (e.g. subtype T is Integer range 1 .. 10).
A \textit{Derived\_Type}, whose name is represented by \textit{typenum},
defines a derived type whose characteristics are derived from those of a parent
\textit{type} with an additional \textit{range} constraint (e.g. type U is new
Integer range 1 .. 10).
A \textit{Integer\_Type} defines a new integer type, represented with
\textit{typenum}, with an additional \textit{range} constraint (e.g. type W is
range 0 .. 10).

\textit{Array\_Type} and \textit{Record\_Type} are constructors for defining
aggregate data types array and record.

\begin{lstlisting}[escapechar=\#, language=coq, basicstyle=\small]
Inductive type_decl: Type := 
| Subtype: astnum -> typenum -> type -> range -> type_decl 
| Derived_Type: astnum -> typenum -> type -> range -> type_decl 
| Integer_Type: astnum -> typenum -> range -> type_decl 
| Array_Type: astnum -> typenum -> type -> type -> type_decl 
| Record_Type: astnum -> typenum -> list (idnum*type) -> type_decl.
\end{lstlisting}


\subsection{Run-Time Check Flags}
In SPARK, run time checks flags are automatically inserted at SPARK AST by the
front end during semantic analysis, and their corresponding run time checks are
then discharged by formally verifying their generated verification conditions
with the GnatProve tool chain. So SPARK can guarantee the absence of run time
errors for developing safety critical systems.

For our formalized SPARK 2014 subset, the following check flags are sufficient,
which are enforced on the expression nodes.
\begin{itemize}
\item 
  Do\_Overflow\_Check: This flag is set on an operator where its operation may
  cause overflow, such as binary operators \textit{(+, -, *, /)}, unary operator
  \textit{(-)} and type conversion from one base type to another when the value
  of source base type falls out of domain of the target base type.
\item 
  Do\_Division\_Check: This flag is set on division operators, such as
  \textit{(/, mod, rem)}, to indicate a zero divide check.
\item 
  Do\_Range\_Check: This flag is set on an expression which appears in a
  context where range check is required, such as right hand side of an
  assignment, subscript expression in an indexed component, argument expressions
  for a procedure call and initialization value expression for an object
  declaration.
\end{itemize}

\subsection{Semantical Formalization With Run-Time Checks}
The major semantical difference between SPARK and other programming languages is
that verification for absence of run time errors are required by the languge
itself.
So in our semantical formalization for SPARK language, run time checks is an
important integrant and they are always performed at appropriate points during
the language semantic evaluation. The program will be terminated with a run time
error message once any of its run time checks fails during the program
evaluation.

\subsubsection{Value}
In SPARK semantics, return value for an expression evaluation can be either
a normal value (basic or aggregate value) or a run time error status detected
during expression evaluation.
Similarly, for a well-formed SPARK program, it should either terminate in a
normal state or a detected run time error, which is expected to be detected and
raised during program execution.

\begin{lstlisting}[escapechar=\#, language=coq, basicstyle=\small]
Inductive Return (A: Type): Type :=
| Normal: A -> Return A
| Run_Time_Error: error_type -> Return A.
\end{lstlisting}

\subsubsection{Run Time Check Evaluation}
A small but significant subset of SPARK run time checks are formalized in Coq,
including overflow check, division check and range check. Overflow checks are
performed to check that the result of a given orithmetic operation is within the
bounds of the base type, division checks are performed to prevent divide being
zero, and range checks are performed to check that the evaluation value of an
expression is within bounds of its target type with respect to the context where
it appears. A small fragment for overflow check formalization in Coq is:
%\begin{quote}
\begin{lstlisting}[escapechar = \#, language=coq, basicstyle=\small]
Inductive overflow_check_bin:binop -> value -> value -> status -> Prop :=
| Do_Overflow_Check_On_Binops: forall op v1 v2 v, 
    op = Plus #$\vee$# op = Minus #$\vee$# op = Multiply #$\vee$# op = Divide ->
    Val.binary_operation op v1 v2 = Some (BasicV (Int v)) ->
    (Zge_bool v min_signed) && (Zle_bool v max_signed) = true ->  
    overflow_check_bin op v1 v2 Success
| ...
\end{lstlisting}
%\end{quote}
Now we only model the 32-bit singed integer for SPARK program, where Coq integer
(Z) is used to represent this integer value with a range bound between \textit{min\_signed}
and \textit{max\_signed}. This integer range constraint is enforced through
the above overflow check semantics when we define the semantics for the
language. As we can see, overflow checks are required only for binary operators
\textit{(+, -, *, /)} among the set of binary operators in our formalized SPARK
subset. And it returns either \textit{Success} or \textit{Exception} with
overflow signal.

\subsubsection{Expression Evaluation}
In an expression evaluation, for an arithmetical operation, run time checks are
always performed according to the checking rules required for the arithmetical
operators in SPARK reference manual, and a run time error returns whenever the
check fails, otherwise, a normal operation result is returned. Further checks on
the normal result value maybe required depending on the context where the
expression appears. One such example is that range check should be performed on
the index expression value before it can be used as an index value for an
indexed component.

The following is a snippet of how the expression evaluation is formalized in Coq
with run time checks enforced during its semantics evaluation. For a binary
expression (Binop ast\_num op e1 op e2), if both e1 and e2 are evaluated to some
normal values, then all necessary run time checks required for the operator
\textit{op} are performed, e.g. overflow check for \textit{+} and both overflow
check and division check for \textit{/}, and a normal binary operation result is
returned when the checks succeed. In name evaluation for indexed component, an
additional range check is required to be performed according to the target type
of the array, which is fetched from a preconstructed symbol table.

\begin{lstlisting}[escapechar=\#, language=coq, basicstyle=\small]
Inductive eval_expr:symboltable -> stack -> expr -> Return value -> Prop :=
| Eval_Binop: forall st s e1 v1 e2 v2 ast_num op v,
    eval_expr st s e1 (Normal v1) ->
    eval_expr st s e2 (Normal v2) ->
    do_run_time_check_on_binop op v1 v2 Success ->
    Val.binary_operation op v1 v2 = Some v ->
    eval_expr st s (Binop ast_num op e1 e2) (Normal v)
| ...
with eval_name: symboltable -> stack -> name -> Return value -> Prop :=
| Eval_Indexed_Component_RTE: forall st s e msg ast_num x_ast_num x,
    eval_expr st s e (Run_Time_Error msg) ->
    eval_name st s (Indexed_Component ast_num x_ast_num x e) 
                   (Run_Time_Error msg)
| Eval_Indexed_Component: forall st s e i x_ast_num t l u x a v ast_num, 
    eval_expr st s e (Normal (BasicV (Int i))) ->
    fetch_exp_type x_ast_num st = Some (Array_Type t) ->
    extract_array_index_range st t (Range l u) ->
    do_range_check i l u Success ->
    fetchG x s = Some (AggregateV (ArrayV a)) ->
    array_select a i = Some v ->
    eval_name st s (Indexed_Component ast_num x_ast_num x e) 
                   (Normal (BasicV v))
| ...
\end{lstlisting}

\subsubsection{Statement Evaluation}
In the context of statement evaluation, range checks will be enforced during
statement evaluation for both assignments and procedure calls. For the case of
assignment evaluation, range check for the right hand side expression of the
assignment is enforced if the target type of the left side of the assignment is
some range constrainted type. For the case of procedure calls, range checks are
required to be enforced on arguments against the type of the IN and IN OUT
formal parameters when passing in the arguments before running the called
procedure, and do range checks on values of OUT and IN OUT formal parameters
on the procedure return.

For a normal assignment evaluation, first evaluate its right hand side
expression \textit{e}, if it returns a normal value, then fetch the type of
its left hand side name \textit{x}, perform a range check before updating
its value if it's a range constrainted type.

\begin{lstlisting}[escapechar=\#, language=coq, basicstyle=\small]
Inductive eval_stmt:symboltable -> stack -> stmt -> Return stack -> Prop :=
| Eval_Assignment: forall st s e v x t l u s1 ast_num,
    eval_expr st s e (Normal (BasicV (Int v))) ->
    fetch_exp_type (name_astnum x) st = Some t ->
    extract_subtype_range st t (Range l u) ->
    do_range_check v l u Success ->
    storeUpdate st s x (BasicV (Int v)) s1 ->
    eval_stmt st s (Assignment ast_num x e) s1
| ...
\end{lstlisting}

\subsubsection{Declaration Evaluation}
For an object declaration, range check is required for its initialization
expression if the type of the object being declared is range constrainted.
Type declaration and procedure declaration should have no effect on the final
stack.











