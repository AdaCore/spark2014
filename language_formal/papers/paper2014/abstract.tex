\begin{abstract}
\vspace*{-.5cm} 
%\hspace{.5cm} 
Formal verification and proof is a promising technique to develop 
high integrity systems with a rigorous mathematical analysis
procedure. SPARK 2014 is a programming language designed
to meet the needs of high integrity software development with a 
support of various formal and informal verification and proof tool 
sets. Any formal methods for program specification and verification depend on
the semantics of the language. 
In this paper, we present our major work towards a formal representation of
SPARK 2014 suitable for supporting formally verified static analyses and
translations. In order to do this, we have developed technology for translating
the GNAT compiler's abstract syntax trees into the Coq proof assistant, and we
have formalized in Coq the dynamic semantics for a core subset of the SPARK 2014
language. SPARK 2014 programs must ensure the absence of certain run-time errors 
(for example, those arising while performing division by zero, accessing 
non existing array cells, overflow on integer computation).  
The main novelty in our semantics is the modeling of on-the-fly run-time checks
within the language semantics, which has been used to mechanically prove the
correctness of the run-time checks performed by the compiler and the correctness
of compiler optimizations concerning run-time checks.
This work of run-time checks optimization and verfication can ensure that any
well-formed SPARK programs do not lead to erroneous execution. And the
formalization of SPARK 2014 semantics lays the foundation for future work on
mechanical reasoning about correctness of other verification methods for SPARK
2014 programs.
\end{abstract}