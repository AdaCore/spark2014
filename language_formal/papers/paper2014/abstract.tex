\begin{abstract}
\vspace*{-.5cm}

\hspace{.5cm} We present the first steps of a broad effort to develop a formal
representation of SPARK 2014 suitable for supporting machine-verified 
static analyses and translations. In our initial work, we have developed 
technology for translating the GNAT compiler's abstract syntax trees 
into the Coq proof assistant, and we have formalized in Coq the dynamic
semantics for a toy subset of the SPARK 2014 language [Spark2014].
SPARK 2014 programs must ensure the absence of certain run-time errors 
(for example, those arising while performing division by zero, accessing 
non existing array cells, overflow on integer computation).  The main 
novelty in our semantics is the encoding of (a small part of) the run-time 
checks performed by the compiler to ensure that well-typed and well 
initialized terminating SPARK programs do not lead to erroneous execution. 
This and other results are mechanically proved using the Coq proof assistant. 
The modeling of on-the-fly run-time checks within the semantics lays the 
foundation for future work on mechanical reasoning about SPARK 2014 
program correctness (in the particular area of robustness) and for 
studying the correctness of compiler optimizations concerning run-time 
checks, among others.
\end{abstract} 





% Together with a certified compiler for SPARK, 
% This is part of a larger effort that employs certified compiler

% This provides a crucial component for an eventual system of certified
% compilation that will enable policy conformance at the source level to
% apply to executable.

% proof-carry-code   certificates translated to machine code level
% preserved by a certified compiler.








%%% Local Variables: 
%%% mode: latex
%%% TeX-master: t
%%% End: 
