\begin{abstract}
\vspace*{-.5cm}
\hspace{.5cm} We present the first steps of a broad effort to develop a formal
representation of SPARK 2014 suitable for supporting machine-verified 
static analyses and translations. In our initial work, we have developed 
technology for translating the GNAT compiler's abstract syntax trees 
into the Coq proof assistant, and we have formalized in Coq the dynamic
semantics for a core subset of the SPARK 2014 language.
SPARK 2014 programs must ensure the absence of certain run-time errors 
(for example, those arising while performing division by zero, accessing 
non existing array cells, overflow on integer computation).  The main 
novelty in our semantics is the encoding of (a small but nontrivial part of) the
run-time checks performed by the compiler to ensure that well-formed SPARK
programs do not lead to erroneous execution.
This and other results are mechanically proved using the Coq proof assistant. 
The modeling of on-the-fly run-time checks within the semantics lays the 
foundation for future work on mechanical reasoning about SPARK 2014 
program correctness (in the particular area of robustness) and for 
studying the correctness of compiler optimizations concerning run-time 
checks, among others.
\end{abstract}