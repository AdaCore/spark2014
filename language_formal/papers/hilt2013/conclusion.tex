\section{Conclusion and future work}
We have implemented a prototype tool chain from SPARK 2014 language subset to
Coq and its semantical formalization and proof in Coq. The experiments on running certified 
interpreter show that our formalized SPARK semantics can capture the desired run-time errors.
This's encouraging, but there are still a lot of work needed to do.

Our next step is to prove the correctness of optimizations that remove 
useless run-time checks. 
Our interpreted semantics are parameterized by the set of run-time 
checks to be performed. These semantics may be called with an 
incomplete set of run-time checks, and can evaluate in that case to an 
erroneous execution. A future work could be to formalize some 
optimizations actually performed by the GNAT compiler, and remove those useless 
run-time checks. The idea would be to prove 
these optimizations correct, namely to prove that those executions 
with $\mathit{less}$ run-time checks behave exactly as those following the 
reference semantics, which perform systematically $\mathit{all}$ the checks. 

What's more, we are also interested in adding more SPARK language features, 
such as procedure call, pre/post aspects and loop invariant, to expand our
current SPARK subset and make it more practical. These SPARK formalization work
also paves the way for our further work on machine-verified proof of correctness of
SPARK static analysis and translations.