\section{Formalization And Proof}

\subsection{SPARK Translation Toolchain}
In the long path through the definition of complete semantics for
SPARK 2014, a very important step is to build a tool chain allowing 
the experimentation of the behavior of these semantics on real SPARK
2014 programs. In the front end of this tool chain, as part of the 
Sireum analysis framework \cite{Sireum:URL}, we have developed a tool called Jago
\cite{Jago:URL} that translates XML representation of the GNAT compiler's 
ASTs into a Scala-based 
representation in Sireum.  This open-source framework enables one to build
code translators and analysis tools for SPARK 2014 in Scala.  Scala's 
blending of functional of object-oriented program styles have proven 
quite useful in other contexts for syntax tree manipulation and analysis.  
Integrated into the Jago is a translation of GNAT ASTs into the Coq proof 
assistant. In the backend of the tool chain, a certified interpreter 
encoding run-time checks for the Coq AST has been developed in Coq.

\subsection{Formalizing Language Semantics}
A major difference between SPARK with other programming languages is that 
its semantics requires its language implementation to detect all run-time errors to 
ensure program correct execution. With Coq AST generated from SPARK by Jago, we 
are now developing its semantics capturing run-time errors in Coq and working toward 
adding more and more language features, such as procedure call, in our current 
formalization framework.
At this early stage, our formal semantics consider only a small subset
of SPARK 2014, and only a small subset of run-time errors.  It performs 
appropriate run-time checks as they are specified by the SPARK and Ada 
reference manuals.  We call these semantics, the $\mathit{reference\
semantics}$, and we implemented an interpreter certified with respect to them.
Thus, our reference semantics can be both manually reviewed by SPARK 
experts, and also be experimented on real source code programs.
For those who are interested, we have posted our Coq code on GitHub 
\cite{Formalization:URL}.

\subsection{Program Correctness Proof}
In Coq, we have formalized a well-formed SPARK program as a well-typed, well-defined
and well-checked program. 
A well-typed program has all its language constructs being consistent with respect to the 
typing rules and all its variables have correct in/out mode with respect to their reading 
and writing permissions. 
A well-defined program is a program with all its used variables initialized.
And a well-checked program is a program having right checks inserted at the right places
in AST trees.
It is proved that for any well-formed SPARK programs in our formalized language subset, they will
execute as we expect and will never exhibits undefined behavior.