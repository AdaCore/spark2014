\documentclass{article}

\usepackage[utf8]{inputenc}
\usepackage{listings}
\usepackage{xspace}
\usepackage{url}
\usepackage{fullpage}
\usepackage{tikz}
\usetikzlibrary{arrows,positioning}
\usepackage{pgflibraryshapes} % for ellipse shape

\newcommand{\version}{0.1}

\title{ALFA \version:\\Annotated Language of Functions in Ada}

\usepackage{color}

\newlength\sidebar
\newlength\envrule
\newlength\envborder

\setlength\sidebar{1.5mm}
\setlength\envrule{0.4pt}
\setlength\envborder{2.5mm}

\definecolor{exampleborder}{rgb}{0,0,.7}
\definecolor{examplebg}{rgb}{.9,.9,1}
\definecolor{statementborder}{rgb}{.9,0,0}
\definecolor{statementbg}{rgb}{1,.9,.9}

\newsavebox\envbox

\newcounter{example}

\newenvironment{example}[1][EXAMPLE \theexample]{%
\par
\refstepcounter{example}%
\SpecialEnv{#1}{exampleborder}{examplebg}{}{\theexample}%
}{%
\endSpecialEnv
}

\newenvironment{statement}[1][]{% Default statement has no title
\par
\SpecialEnv{#1}{statementborder}{statementbg}{statementborder}{}%
}{%
\endSpecialEnv
}

\def\Empty{}

% #1 title (if any)
% #2 sidebar (and title bg) color
% #3 background color
% #4 border color (or null for no border)
% #5 Counter, if any.
\newenvironment{SpecialEnv}[5]{%
\par
\def\EnvSideC{#2}% To use later (in end)
\def\EnvBackgroundC{#3}%
\def\EnvFrameC{#4}%
\flushleft
\setlength\leftskip{-\sidebar}%
\addtolength\leftskip{-\envborder}%
\noindent \nobreak
% Check if title is null:
\ifx\delimiter#1\delimiter\else
% If a title is specified, then typeset it in reverse color
\colorbox{\EnvSideC}{%
\hspace{-\leftskip}% usually positive
\hspace{-\fboxsep}%
\footnotesize\sffamily\bfseries\textcolor{white}{#1}%
\hspace{\envborder}}%
\par\nobreak
\setlength\parskip{-0.2pt}% Tiny overlap to counter pixel round-off errors
\nointerlineskip
\fi
% Make side-bar
\textcolor{\EnvSideC}{\vrule width\sidebar}%
% collect body in \envbox:
\begin{lrbox}\envbox
\begin{minipage}{\hsize}%
% insert counter, if any:
\ifx\delimiter#5\delimiter\else
% \theexample. Yannick: i don't like it.
\enspace
\fi
\ignorespaces
}{\par
\end{minipage}\end{lrbox}%
% body is collected. Add background color
\setlength\fboxsep\envborder
\ifx\EnvFrameC\Empty % no frame
\colorbox{\EnvBackgroundC}{\usebox\envbox}%
\else % frame
\setlength\fboxrule\envrule
\addtolength\fboxsep{-\envrule}%
\fcolorbox{\EnvFrameC}{\EnvBackgroundC}{
\usebox\envbox}%
\fi
\nobreak \hspace{-2\envborder}\null
\endflushleft
}

\newcommand{\rationale}{\fbox{rationale}~}
\newcommand{\bnf}[1]{$\mathit{#1}$}
\newcommand{\kw}[1]{\textbf{#1}}
\newcommand{\spark}{$\tau_S$}
\newcommand{\why}{$\tau_W$}
\newcommand{\code}[1]{\texttt{#1}}
\newcommand{\heap}{\code{Heap}\xspace}
\newcommand{\pred}[1]{\ensuremath{\mathit{pred}(#1)}\xspace}
\newcommand{\allwrites}{$\mathcal{W^+}$\xspace}
\newcommand{\Outallwrites}{\ensuremath{\mathcal{W}^{+out}}\xspace}
\newcommand{\Inallwrites}{\ensuremath{\mathcal{W}^{+in}}\xspace}
\newcommand{\inallwrites}[1]{\ensuremath{\mathcal{W}^{+in}(#1)}\xspace}
\newcommand{\outallwrites}[1]{\ensuremath{\mathcal{W}^{+out}(#1)}\xspace}
\newcommand{\writes}{$\mathcal{W}$\xspace}
\newcommand{\Outwrites}{\ensuremath{\mathcal{W}^{out}}\xspace}
\newcommand{\Inwrites}{\ensuremath{\mathcal{W}^{in}}\xspace}
\newcommand{\inwrites}[1]{\ensuremath{\mathcal{W}^{in}(#1)}\xspace}
\newcommand{\outwrites}[1]{\ensuremath{\mathcal{W}^{out}(#1)}\xspace}
\newcommand{\allreads}{$\mathcal{R^+}$\xspace}
\newcommand{\Outallreads}{\ensuremath{\mathcal{R}^{+out}}\xspace}
\newcommand{\Inallreads}{\ensuremath{\mathcal{R}^{+in}}\xspace}
\newcommand{\inallreads}[1]{\ensuremath{\mathcal{R}^{+in}(#1)}\xspace}
\newcommand{\outallreads}[1]{\ensuremath{\mathcal{R}^{+out}(#1)}\xspace}
\newcommand{\reads}{$\mathcal{R}$\xspace}
\newcommand{\Inreads}{\ensuremath{\mathcal{R}^{in}}\xspace}
\newcommand{\Outreads}{\ensuremath{\mathcal{R}^{out}}\xspace}
\newcommand{\inreads}[1]{\ensuremath{\mathcal{R}^{in}(#1)}\xspace}
\newcommand{\outreads}[1]{\ensuremath{\mathcal{R}^{out}(#1)}\xspace}
\newcommand{\union}{~\cup~}
\newcommand{\bigunion}{~\bigcup~}
\newcommand{\inter}{~\cap~}
\newcommand{\biginter}{~\bigcap~}
\newcommand{\minus}{~\backslash~}

\newcommand{\vpath}[1]{\ensuremath{\pi(#1)}\xspace}
\newcommand{\sel}{\ensuremath{\sigma}\xspace}

\newcommand{\etc}{\textit{etc.}\xspace}

\setlength{\parindent}{0.0in}
\setlength{\parskip}{0.1in}

\begin{document}

\maketitle
\sloppy

\newpage

\setcounter{tocdepth}{2}
\tableofcontents

\newpage

\section{Ada 2012}

Ada 2012 is the next version of the Ada standard, expected to be finalized in
2012. It contains many extensions that facilitate the expression of
specifications, for either dynamic or static checking. Its Reference Manual
(RM) in progress is online at
\url{http://www.ada-auth.org/standards/12rm/html/RM-TOC.html}.

\subsection{Introduction to Ada}

\subsubsection{Module System}

Programs in Ada are structured in packages, which consist in:
\begin{itemize}
\item a package specification, which defines the external API of the package,
  consisting of types, variables and subprogram declarations whose lifetime and
  visibility are the same as the one of the package itself;
\item an optional package body, which defines the entities declared in the
  package specification, plus additional entities which are not part of the
  external API of the package.
\end{itemize}

The package specification may be divided in a public part and a private part,
so that the private part is visible to the compiler, for use in separate
compilation, but not visible to the external program. Thus, a type may be
declared as having a private definition, so that the compiler knows its precise
definition (size, initialization, etc.) but the program cannot depend on this
definition outside the package.

Packages may be defined in any declarative section: at the outermost level as
library-level compilation units, or inside other packages and subprogram
bodies. Subprograms too can be defined either as library-level compilation
units, or in local declarative sections. Variables and types can only be
defined in local declarative sections.

Child packages are derived from their parent package with which they have
special visibility relations, which depends on the public/private status of the
child package.

\subsubsection{Type System}

Ada enforces a strong type system by default, although it can be circumvented
by using specific conversion functions (e.g. on addresses), for example to
connect the program to external sensors/actuators.

Integers come in a variety of flavors. New integer types can be defined which
are incompatible, so that type checking uncovers mistakes where different
integer types are mixed, and a subprogram can be overloaded for different
integer types. Subtypes of existing integer types can define a range constraint
which should hold for all initialized values of this subtype, and the compiler
inserts a run-time check anywhere the constraint could be violated. Given a
subtype S of a type T, there exists a base machine type B for both T and S such
that all arithmetic operations on S and/or T are performed in B.  After the
result is computed, it is stored back in a variable of type S or T and the
corresponding range check is performed.

Enumerations define named enumerators which are the only values of this
type. Although pattern-matching is available for all discrete types, it is
mostly used for enumerations. As part of type checking, the compiler checks
that pattern-matching is complete and that no case is redundant.

In GNAT, floating-point types follow the IEEE-754 standard, although this is
not part of the Ada language definition. Fixed-point types limit the precision
of values to a fixed number of figures.

Aggregate types come in two flavors: records aggregate heterogeneous data
components, while arrays aggregate homogeneous data components. Records may
have one or more discriminants: discriminants of enumerated type are used to
provide a form of algebraic datatypes, as the discriminant is used to provide a
variant distinction between different sets of components; discriminants of
integer type are used to provide dependent types, as the discriminant is used
to give the size of some array component.

Pointer types, a.k.a. access types in Ada, follow a set of somewhat complex
rules to prevent dangling pointer references. In the following, we will mostly
ignore pointers.

A non-aggregate type is called an elementary type.

\subsubsection{Metaprogramming}

Generic packages and generic subprograms define template entities that are
instantiated at compile-time. They come with a set of formal parameters for
types, variables, values, subprograms and packages. There are individual
constraints on individual formal parameters and relational constraints between
formal parameters, so that once the compiler has checked a generic entity, any
instantiation with correct arguments will generate a compilable entity.

\subsubsection{Object-Oriented Programming}

Classes are known in Ada as tagged (record) types. Objects are values of tagged
types. Methods are known as primitive operations, which can take their
dispatching type in any position. When the first operand is the dispatching
one, the usual object-dot-method notation is allowed for calls.

An object can be considered either of type T, in which case there is no
dispatching involved when calling a primitive operation of T, or of type
T'Class, in which case all primitive operations on this object are dispatching.

Interfaces can be defined which only introduce primitive operations and no
components.

\subsubsection{Calling Conventions}

Parameters of elementary types are passed by copy.  Parameters of tagged type
are passed by reference. For most other parameters, the compiler is free to
choose between by-reference and by-copy.

Parameters have a mode, which can be by default \emph{in} or else \emph{out}
and \emph{in out}:
\begin{itemize}
\item parameters of mode \emph{in} can only be read;
\item parameters of mode \emph{out} can be both read and written: although the
  user indicates by not making them \emph{in out} that he does not intend to
  read their initial values, he should still be able to read their value after
  writing them, so the compiler cannot check in general this data-flow
  property;
\item parameters of mode \emph{in out} can be both read and written.
\end{itemize}

The Ada standard does not rule out aliasing between parameters or between
parameters and global variables accessed directly in general. However, the
standard rules out any possible different behaviors which could arise from the
compiler choosing to pass some parameters by-reference or by-copy, which would
lead to an ambiguous definition of a program. This is a theoretical rule that
the compiler does not try to enforce, although GNAT will generate warnings for
overlapping exported parameters, following the adoption of ``in out'' formal
parameters for functions in Ada 2012.

\subsection{New in Ada 2012}

The standard defines the following extensions to Ada 2005, which facilitate the
expression of specifications as subprogram contracts or type invariants:

\begin{itemize}
\item AI05-0001: bounded containers
\item AI05-0183: aspect specifications
\item AI05-0147: conditional expressions
\item AI05-0188: case expressions
\item AI05-0177: parameterized expressions
\item AI05-0176: quantified expressions
\item AI05-0146: type invariants
\item AI05-0153: subtype predicates
\end{itemize}

While the ARG website is the final authority on these extensions
(http://www.ada-auth.org/AI05-SUMMARY.HTML), we sketch in the following the
semantics and interest of each one.

Bounded containers introduce bounded versions of the existing generic
containers in Ada 2005 container library (vector, list, hashed set, ordered
set, hashed map, ordered map). The bound on the size of the container is used
to preallocate an array of the desired size, so that dynamic allocation is not
used for these containers. This opens up the possibility to perform proofs on
programs using containers, as the validity of cursors is far simpler in this
new model.

Aspect specifications allow defining contracts for subprograms, a contract
being a pair of a precondition (\verb|Pre|) and a postcondition
(\verb|Post|). Special contracts can also be issued for overriding, so that an
overriding subprogram can only weaken the inherited precondition and strengthen
the inherited postcondition. The standard defines a general syntax for aspects
which allows the definition of compiler-specific aspects, like the test-case
aspect in GNAT described below. In the postcondition, attribute \verb|'Old|
applied on a name indicates the value attached to this name at subprogram
entry, and attribute \verb|'Result| applied to the name of the current function
indicates the result returned by this function. Preconditions and
postconditions may be compiled into executable assertions if the right option
is given to the compiler (-gnata in GNAT).

\begin{example}
A contract for a square-root function is:
\begin{verbatim}
function Sqrt (X : Integer) return Integer with
  Pre  => X >= 0,
  Post => Sqrt'Result >= 0 and then
          Sqrt'Result ** 2 <= X and then
          (Sqrt'Result + 1) ** 2 > X;
\end{verbatim}
\end{example}

Conditional expressions and case expressions allow the use of ``if'' and
``case'' in expressions, which simplifies specifications.

Parameterized expressions are a simple form of function definitions (or lambda
expressions) allowed in the specification part of a package. In particular,
simple boolean predicates on accessors to a (private) type, which are typically
the predicates one needs to write subprogram contracts, can be defined as
parameterized expressions. This has the advantage for proofs that parameterized
expressions (contrary to functions) 1) cannot have write side-effects if they
do not contain a function call, 2) have a much simpler form than the usual
functions, which makes them easier to translate into proof predicates, 3) can
be defined in a package specification which makes them available for proof even
if the package body is not.

Quantified expressions allow the expression of predicates which hold for all
elements in a range or a container, or for some element in a range or a
container.

Type invariants express invariant properties of private types, which should be
typically observed for any value of the type outside its package. However, the
standard only defines specific points at which this property is checked on a
value of the type, like entry and exit points of a subprogram. This does not
enforce by itself that the property always holds.

Subtype predicates express fine-grain properties of subtypes, like the various
enumeration values allowed for the discriminant of a record. Like for type
invariants, the standard only defines specific points at which this property is
checked.

\subsection{GNAT-Specific Extensions}
\label{sub:formal-containers}

GNAT defines an aspect called \verb|Test_Case|, which applies to subprograms
exactly like the standard \verb|Pre| and \verb|Post|. A test-case is an
aggregate with exactly four components, all of which are compulsory:
\begin{enumerate}
\item a \verb|Name| component, of type string, which gives the name of the
  test-case;
\item a \verb|Mode| component, of a predefined enumerated type, which set the
  mode for executing the test-case (\verb|Nominal| or \verb|Robustness|);
\item a \verb|Requires| component, of type boolean, which defines the entry
  condition for the test-case;
\item an \verb|Ensures| component, of type boolean, which defines the exit
  condition for the test-case.
\end{enumerate}

A test-case (N,Req,Ens) is a part of the specification which indicates that
under entry condition Req, the subprogram terminates with condition Ens. Thus,
the \verb|Requires| component bears much resemblance with the precondition, and
the \verb|Ensures| component bears much resemblance with the
postcondition. Indeed, they share the same semantic restrictions
(w.r.t. \verb|'Old| and \verb|'Result|).  More than one test-case can be
defined for a subprogram. No two test-cases on a subprogram should have the
same name.

Contrary to preconditions and postconditions, test-cases are not compiled into
executable assertions by the compiler. GNAT only checks that test-cases are
properly defined. Test-cases should be used by the verification toolkit either
for unit testing or for unit proof.

For unit testing, it is sufficient to write a test procedure which exercises a
test-case to consider this test-case successful. To exercise a test-case, a
test procedure must in order:
\begin{enumerate}
\item generate suitable arguments, by calling one or more subprograms
  called \textit{fixtures};
\item check that the requires is satisfied on these arguments;
\item if the mode is \verb|Nominal|, check that the precondition is satisfied on
  these arguments;
\item call the subprogram tested on these arguments;
\item if the mode is \verb|Nominal|, check that the postconditions is satisfied
  on these arguments;
\item check that the ensures is satisfied.
\end{enumerate}

Note that the test may not execute its ensures or postcondition due to some
exception being raised. A predefined exception raised should count as a failure
(for example \verb|Constraint_Error|) while a user-defined exeption raised
should count as a sucess, as the ensures/postcondition in this case does not
apply.

The part about checking the contract of the subprogram is optional because some
test-cases may correspond to cases beyond the normal behavior of the subprogram
described in its contract. Typically, robustness tests deal with such abnormal
behavior.

For unit proof, it is sufficient to prove that the subprogram implements a
special contract, with 1) the requires, optionally and'ed with the original
precondition, as precondition and 2) the ensures as postcondition.  The
original precondition should be and'ed with the requires for those test-cases
which correspond to normal behavior, and the requires should be the only
precondition for those test-cases which correspond to abnormal behavior.

\begin{example}
  A contract with test-cases for an integer square-root function is:
\begin{verbatim}
function Sqrt (X : Integer) return Integer with
  Pre  => X >= 0,
  Post => Sqrt'Result >= 0 and then
          Sqrt'Result ** 2 <= X and then
          (Sqrt'Result + 1) ** 2 > X,
  Test_Case => (Name     => "test case 1",
                Mode     => Nominal,
                Requires => X = 100, 
                Ensures  => Sqrt'Result = 10),
  Test_Case => (Name     => "test case 2",
                Mode     => Nominal,
                Requires => X < 100, 
                Ensures  => Sqrt'Result >= 0 and then 
                            Sqrt'Result < 10),
  Test_Case => (Name     => "robustness test case",
                Mode     => Robustness,
                Requires => X = -1, 
                Ensures  => Sqrt'Result = 0);
\end{verbatim}
\end{example}

GNAT introduces a new form of containers called the formal containers, to be
used for proof of programs which manipulate containers. These are variants of
Ada 2012 bounded containers, with a different API meant to facilitate proofs.

\section{ALFA}

ALFA is a sub-language of Ada 2012 which identifies which subprograms are fit
for formal verification. ALFA is not a profile, as defined in the Ada standard,
as the complete program does not need to be in ALFA for individual subprograms
to be in ALFA. Rather, subprograms are in ALFA or not depending exclusively on
their specification and body. In particular, the location in the source where a
subprogram is defined (inside a package or another subprogram, in the public or
private part, \etc) should not influence whether it is in ALFA.

ALFA is expected to evolve to support more Ada programs, the current version
described in this document is version \version. The following notes describe
the evolution of ALFA.

\paragraph{Verification Profile} 

This document~\cite{Sango2010RR} written by Marc Sango as part of an internship
describes a possible Ada profile suitable for formal verification by
translation to SPARK. The viewpoint taken implies that the set of features
considered as part of the profile is quite large, as using any feature outside
the profile would abort the possibility of formal verification for the program
(or at least the package). As a result, many translations from Ada to SPARK are
left for a future SPARK roadmap (that is, the SPARK language and toolset should
be extended).

The document outlines the differences between Ada and SPARK, and it
offers a head-to-head comparison of BNF grammars in appendix.

A prototype translation from Ada to SPARK was created to experiment with some
of the translations described in the verification profile. This ASIS-based tool
called Sparkify is available from the Hi-Lite public forge on Open-DO website.

\paragraph{Version \version}

The initial version of ALFA departs from the profile point-of-view, in that
subprograms should be inside or outside ALFA by themselves, which offers a much
finer grain of verification. Translations are described in much more details
than in the verification profile, and they target not only SPARK but also
Why. A sequence of translation passes is defined. All features that were pushed
on the SPARK roadmap in the verification profile are now either in ALFA and a
translation to SPARK/Why is given, or outside ALFA. Practical considerations
also lead to some features excluded from ALFA because not enough beneficial at
this stage, although they could be added later (for example fixed-point types).

\subsection{ALFA Overview}
\label{sec:ALFA-overview}

The high-level goals of ALFA are to:
\begin{enumerate}
\item \label{it:alfa-verif} enable verification of the absence of run-time
  errors and contract verification on individual subprograms;
\item \label{it:alfa-combine} make it possible to combine unit testing and unit
  proof;
\item \label{it:alfa-annot} reduce the annotation burden on the programmer;
\item \label{it:alfa-auto} lead to 100\% automatic proof on subprograms
  whenever it is possible.
\end{enumerate}

Following the approach of SPARK, we aim at proving the absence of reads of
uninitialized data (a part of goal (\ref{it:alfa-verif})) by enforcing and
checking stricter data-flow properties than required in Ada. However, we depart
from SPARK in requiring that the analysis follows initialization at the level
of variable subcomponents instead of entire variables. In ALFA, all syntactic
program paths through a subprogram should initialize individual variable
subcomponents before a subcomponent is read, while other subcomponents of the
same variable may remain uninitialized. Likewise, parameter modes do not impose
initialization of entire variables. Instead, an analysis computes the set of
variable subcomponents only read, only written and both read and written in a
subprogram, which are treated like the ``in'', ``out'' and ``in out''
parameters in a SPARK data-flow analysis:
\begin{itemize}
\item variable subcomponents only read and those both read and written should
  be completely initialized at subprogram entry;
\item variable subcomponents only written should be completely initialized at
  subprogram exit.
\end{itemize}

\begin{example}
  The following subprograms are incorrect in SPARK, both because components
  \verb|X| and \verb|Y| of parameter \verb|P| are initialized separately, and
  because out-parameter \verb|P| is not completely initialized by
  \verb|Init_X|. They are correct in ALFA.
\begin{verbatim}
  type Point is record X, Y : Val; end record;
  
  procedure Init_X (P : out Point) is begin 
     P.X := 0; 
  end Init_X;
  
  procedure Init (P : out Point) is begin 
     Init_X (P);
     P.Y := 0; 
  end Init;
\end{verbatim}
\end{example}

To achieve goal (\ref{it:alfa-annot}), we do not require any user annotations
(a.k.a. frame condition) related to parameters and global variables read or
written, as in SPARK or Why.  Instead, we provide an analysis to compute these
annotations from the source program. This analysis requires that the code of
the subprograms called by the subprograms analyzed is available. This
corresponds to the code which is needed to perform unit testing on a unit,
which is required anyway to achieve goal (\ref{it:alfa-combine}).

In contract-based verification (a part of goal (\ref{it:alfa-verif})), a
subprogram body is verified independently from other subprograms, by relying on
subprogram contracts only to assess the effect of calls. Thus, it is crucial
that subprogram contracts are precise. In particular, the frame condition needs
to be generated as precisely as possible, which is partly achieved by
distinguishing between variable components in the frame condition.

\begin{example}
Consider a procedure which sets a field of an element of an array of points:
\begin{verbatim}
  type Point is record X, Y : Val; end record;
  type Points is array (Index) of Point;

  procedure Set_X_Of_Nth (P : in out Points; V : Val; N : Index) with
     Post => P(N).X = V;
\end{verbatim}

The postcondition of \verb|Set_X_Of_Nth| only says something about the value of
\verb|P(N).X| in the post-state. It says nothing about the value of
\verb|P(N).Y|, or the value of \verb|P(M).X| and \verb|P(M).Y| for some
\verb|M| different from \verb|N|.\\

The only way to add this information in Ada is to explicitly state in the
postcondition that these other values have not changed:
\begin{verbatim}
  Post => P(N).X = V and then P(N).Y = P(N).Y'Old and then
          (for all M in Index => (if M /= N then P(M).X = P'Old(M).X)) and then
          (for all M in Index => (if M /= N then P(M).Y = P'Old(M).Y));
\end{verbatim}

This detailed postcondition is needed if we generate in SPARK/Why an imprecise
frame condition which only says that array \verb|P| is written. But if we
generate a more precise frame condition saying that only field \verb|X| of an
element of \verb|P| is written, then the postcondition is Ada can be reduced,
because it is known in SPARK/Why that no field \verb|Y| has changed:
\begin{verbatim}
  Post => P(N).X = V and then
          (for all M in Index => (if M /= N then P(M).X = P'Old(M).X));
\end{verbatim}
\end{example}

ALFA allows mutual dependencies between packages and (direct or indirect)
recursive calls. Although these should be handled with care to ensure program
termination, the user should deal with termination separately as ALFA does not
target proof of termination.

ALFA does not deal with concurrency at this stage, which means that the
verification results are only valid for sequential programs, or for concurrent
programs if it can be shown that the results are the same as for the underlying
sequential programs (for example, when there is no communication between
threads). The RavenSPARK profile of SPARK, based on the Ravenscar profile of
Ada, suggests restrictions that make formal verification of concurrent programs
feasible using the SPARK toolset. Similar restrictions could be added to ALFA
in a later version to allow dealing with concurrency.

ALFA allows analyzing formally many more constructs than SPARK does, but it
inherits some of its limitations from SPARK, for lack of a simple translation
into SPARK (for example, restrictions on the form of exit statements in loops).

\subsection{Computing the Frame Condition}
\label{sub:data-flow}

In order to define which entities are in ALFA and which are not, we need to
compute the set of components of parameters and global variables which are
read/written by each subprogram, a.k.a. the frame condition of each subprogram.

\subsubsection{Variable Paths}
\label{sub:variable-path}

A variable in ALFA may only have a scalar type (discrete or real) or an
aggregate type (record or array) such that all sub-components have themselves a
scalar type or an aggregate type with the same restriction. Given a variable
$v$ in ALFA, we associate a set of paths \vpath{v} to $v$. More generally, we
associate a set of paths \vpath{v} to an individual path $v$ (which may be a
variable) as follows:
\begin{itemize}
\item if $v$ has a scalar type, $\vpath{v} = \{v\}$;
\item if $v$ has a record type with components $c_i$, 
  $\vpath{v} = \bigunion \vpath{v.c_i}$;
\item if $v$ has an array type whose element type $t$ is a scalar type,
  $\vpath{v} = \{v\}$;
\item if $v$ has an array type whose element type $t$ is a record type with
  components $c_i$, $\vpath{v} = \bigunion \vpath{v.c_i}$. (Here, we do as if $v$
  was of type $t$ so that we can refer to its $c_i$ components.)
\end{itemize}

\begin{example}
\label{ex:variable_path}
  On the following code, there are four variable paths associated to variable
  \verb|X|: \verb|X.B.J|,
  \verb|X.B.K|, \verb|X.A.J| and \verb|X.A.K|.\\

\begin{minipage}{0.65\linewidth}
\begin{verbatim}
type Base is record 
  J : Integer;
  K : Integer range 0 .. 5;
end record;

type Base_Array is array (Boolean) of Base;
\end{verbatim}
\end{minipage}
\begin{minipage}{0.3\linewidth}
\begin{verbatim}
type Ext is record 
  B : Base;
  A : Base_Array;
end record;

X : Ext;
\end{verbatim}
\end{minipage}
\end{example}

A path is a sequence of selectors $\sel_1.\sel_2...\sel_n$.  If any selector
$\sel_i$ on the path refers to an array, we say that the path is an array
path. Otherwise, we say that the path is a record path. A record path refers to
a single value of scalar type, while an array path usually refers to more than
one value of scalar type.

\begin{example}
  In Example~\ref{ex:variable_path}, \verb|X.B.J|, and \verb|X.B.K| are record
  paths; \verb|X.A.J| and \verb|X.A.K| are array paths.
\end{example}

Notice that writing to a record path assigns the new value to the subcomponent
represented by the record path. Writing to an array path only changes part of
the value of the subcomponent represented by the array path, which is partially
written.

\subsubsection{Problem Statement}

For the purpose of data-flow verification, we consider a package body as a
special form of subprogram, where the body of the subprogram is given by the
various initializations performed in the declarations of variables, as well as
the optional sequence of statements.

A data-flow analysis of the source Ada program should associate each subprogram
with three sets of variables paths:
\begin{itemize}
\item the read-set of variable paths possibly read by the subprogram;
\item the write-set of variable paths alway written by the subprogram;
\item the read-write-set of variable paths possibly read and written by
  the subprogram.
\end{itemize}

Note that a variable path which is always written by the subprogram before
being read, so that the initial value of the variable path is never read,
belongs to the write-set, not the read-write-set.

A special global variable called \heap represents all the dynamically allocated
memory plus all variables whose address is taken.

The algorithm should compute various maps from program points $p$ to sets of
(local and global) variable paths:
\begin{itemize}
\item \writes stores variable paths completely written on all program paths
  to $p$;
\item \allwrites stores variable paths partially written on some program
  path to $p$;
\item \reads stores variable paths whose initial value at subprogram entry has
  been partially read on some program path to $p$;
\item \allreads stores variable paths partially read on some program path to
  $p$.
\end{itemize}

As \reads depends on \writes which depends on \allreads, \allreads should be
computed first, then \writes, then \reads. \allwrites can computed at any time, as it is independent from the other maps.

As \writes, \allwrites, \reads and
\allreads take program entities as parameters, they are decomposed into
(\Inwrites,\Outwrites), (\Inallwrites,\Outallwrites), (\Inreads,\Outreads) and
(\Inallreads,\Outallreads) respectively, where the ``in'' part describes the
program point immediately before the entity and the ``out'' part describes the
program point immediately after the entity.

In the following, we give data-flow equations defining each one of \writes,
\allwrites, \reads and \allreads. These equations provide an algorithm for
computing the sets, and since the sets only grow, the algorithm necessarily
reaches a fixpoint and terminates. Note that \writes, \allwrites, \reads and
\allreads will contain path variables for parameters and global variables but
also local variables, even when applied to subprograms.

The control-flow graph of a subprogram allows defining for each statement $s$
the set of its predecessor statements \pred{s}.

\begin{example}
\label{ex:frame-condition}
  We will use the following code, which sets a field of an array element, as
  running example throughout the definition of the algorithms for computing the
  frame condition.
\begin{verbatim}
  type Point is record X, Y : Val; end record;
  type Points is array (Index) of Point;

  procedure Set_X_Of_Nth (P : in out Points; V : Val; N : Index) is begin
     P(N).X := V;
  end Set_X_Of_Nth;
\end{verbatim}
\end{example}

\subsubsection{Computing \allwrites}

There are two sources of partial writes: assignment statements and calls. A
statement $s$ partially writes a variable path $w$ in the following cases:
\begin{itemize}
\item $s$ is an assignment statement to $w$;
\item $s$ is an assignment statement through a dereference, and $w$ is \heap;
\item $s$ contains a call to a subprogram $f$, $w$ is not a local variable path
  of $f$ and $w \in \outallwrites{f}$.
\end{itemize}

Given a statement $s$ partially writing variable path $w$, the following
equations define \allwrites:
\begin{eqnarray*}
\inallwrites{s} &=& \bigunion \outallwrites{\pred{s}}\\
\outallwrites{s} &=& \inallwrites{s} \union \{w\}
\end{eqnarray*}

\begin{example}
  On the code of Example~\ref{ex:frame-condition}, we compute \outallwrites{\mathit{Set\_X\_Of\_Nth}} = \{ \verb|P.X| \}.
\end{example}

\subsubsection{Computing \allreads}

There are two sources of reads: expressions and calls. A statement $s$
partially reads a variable path $r$ in the following cases:
\begin{itemize}
\item $s$ contains an occurence of $r$ that does not count as write, as
  described above;
\item $s$ contains a dereference that does not count as write, as described
  above, and $r$ is \heap;
\item $s$ contains a call to a subprogram $f$, $r$ is not a local variable path
  of $f$ and $r \in \outallreads{f}$.
\end{itemize}

Notice that expressions used in the definition of types, for example dynamic
bounds of a subtype, are also considered here.

Given a statement $s$ partially reading variable paths $r_i$, the following
equations define \allreads:
\begin{eqnarray*}
\inallreads{s} &=& \bigunion \outallreads{\pred{s}}\\
\outallreads{s} &=& \inallreads{s} \union \{r_i\}
\end{eqnarray*}

\begin{example}
  On the code of Example~\ref{ex:frame-condition}, we compute
  \outallreads{\mathit{Set\_X\_Of\_Nth}} = \{ \verb|V| \}.
\end{example}

\subsubsection{Computing \writes}
\label{sec:init-array-through-loop}

There are two natural sources of complete writes: assignment statements and
calls. A statement $s$ completely writes a global variable path $w$ in the
following cases:
\begin{itemize}
\item $s$ is an assignment statement to $w$, and $w$ is a record path;
\item $s$ contains a call to a subprogram $f$, $w$ is not a local variable path
  of $f$ and $w \in \outwrites{f}$.
\end{itemize}
Additionally, if $l_1$ is a loop and $w$ an array path such that:
\begin{enumerate}
\item there is a set of loops $l_1...l_n$ such that each $l_i$ ranges over the
  set of indexes of an array selector in $w$, and there is one loop for every
  array selector;
\item for all $i$, $l_i$ contains
  $l_{i+1}$ as a statement in its outter sequence of statements;
\item $l_1$ does not contain any exit statement (even in inner loops);
\item $w \notin \outallreads{l_1}$ (that is, $w$ is not read in the loop);
\item $l_n$ contains a statement assigning to $w$, where all indexes used for
  array accesses are exactly all the loop variables;
\end{enumerate}
then loop $l_1$ completely writes variable path $w$.

The aim of this special rule for loops is to allow the initialization of array
paths through loops, which is both common in user code and which arises as a
result of expansion, in particular due to default expressions for record
components.

Given a statement $s$ completely writing variable paths $w$, the following
equations define \writes:
\begin{eqnarray*}
\inwrites{s} &=& \biginter \outwrites{\pred{s}}\\
\outwrites{s} &=& \inwrites{s} \union \{w\}
\end{eqnarray*}

\begin{example}
  On the code of Example~\ref{ex:frame-condition}, we compute
 \outwrites{\mathit{Set\_X\_Of\_Nth}} = \{\}.
\end{example}

\subsubsection{Computing \reads}

\reads is computed similarly as \allreads, except it does not count variable
paths completely written on all syntactic program paths before reaching the
current program point. A statement $s$ partially reads the initial value of a
variable path $r$ if $r$ is not in \writes and:
\begin{itemize}
\item $s$ contains an occurence of $r$ that does not count as write, as
  described above;
\item $s$ contains a dereference that does not count as write, as described
  above, and $r$ is \heap;
\item $s$ contains a call to a subprogram $f$, $r$ is not a local variable path
  of $f$ and $r \in \outreads{f}$.
\end{itemize}

Given a statement $s$ partially reading the initial value of variable paths
$r_i$, the following equations define \reads:
\begin{eqnarray*}
\inreads{s} &=& \bigunion \outreads{\pred{s}}\\
\outreads{s} &=& \inreads{s} \union (\{r_i\} \minus \inwrites{s})
\end{eqnarray*}

\begin{example}
  On the code of Example~\ref{ex:frame-condition}, we compute
 \outreads{\mathit{Set\_X\_Of\_Nth}} = \{ \verb|V| \}.
\end{example}

\subsubsection{Computing the read-set, write-set and read-write-set}
\label{sec:frame-sets}

Given subprogram $f$:
\begin{itemize}
\item the initial read-set for $f$ is $\outreads{f} \minus \outallwrites{f}$;
\item the initial write-set for $f$ is $\outwrites{f} \minus \outreads{f}$
  augmented with all local variables of $f$;
\item the initial read-write-set for $f$ is $\outallwrites{f} \minus
  \outwrites{f}$.
\end{itemize}

The read-set is simply the set of variable paths whose initial value may be
read and which are never written. The write-set if the set of variable paths
always written on all program paths to the exit of the subprogram, and whose
initial value is not read. We add all local variables to this set because their
value should not be read before they are assigned, and it does not matter if
they are initialized prior to returning from the function. The read-write-set
is all other variable paths read or written. The frame condition is the
defined by the value of all three sets.

\begin{example}
Consider the following code for setting a field of an array element:
\begin{verbatim}
  type Point is record X, Y : Val; end record;
  type Points is array (Index) of Point;

  procedure Set_X_Of_Nth (P : in out Points; V : Val; N : Index) is begin
     P(N).X := V;
  end Set_X_Of_Nth;
\end{verbatim}

We computed the sets for \verb|Set_X_Of_Nth| as follows:
\begin{itemize}
\item \outallwrites{\mathit{Set\_X\_Of\_Nth}} = \{ \verb|P.X| \}
\item \outallreads{\mathit{Set\_X\_Of\_Nth}} = \{ \verb|V| \}
\item \outwrites{\mathit{Set\_X\_Of\_Nth}} = \{\}
\item \outreads{\mathit{Set\_X\_Of\_Nth}} = \{ \verb|V| \}
\end{itemize}

So that the read-set of \verb|Set_X_Of_Nth| is \{ \verb|V| \}, its write-set is
 \{\} and its read-write-set is  \{ \verb|P.X| \}.
\end{example}

\subsection{ALFA Restrictions}

In the following, we describe the restrictions in ALFA with respect to
packages, types, variables, subprograms, expressions and statements, as well as
their rationale. A precise description of ALFA restrictions based on Ada BNF is
given in Appendix~\ref{sec:bnf-alfa}.

\subsubsection{Packages}

Library-level packages, child packages and generic packages are all in ALFA.

Local packages to other packages or subprograms and formal packages for generic
packages and subprograms are not in ALFA, as they would require quite some work
for no clear benefit.

\subsubsection{Types}

Access types are not in ALFA, because they create aliasing which makes formal
verification intractable.

Floating-point and fixed-point types are not in ALFA, although they could be in
a future version. Floating-point are particularly difficult to translate
accurately, because the IEEE-754 standard is quite complex, but a simpler
(safe) model bounding floating-point values by reals would be possible. Notice
that the current treatment of floating-point numbers as reals in both SPARK and
Frama-C is plain incorrect, and identified as such.

Interfaces are not in ALFA also to simplify the work on the first version.

Controlled types in Ada define special subprograms which are called when
assigning to an object of the type (a controlled object), or when a controlled
object goes out of scope or is released. Controlled types are not in ALFA both
because they are not usually used in critical software, and because they may
lead to the generation of an implicit exception handler, in order to apply
finalization to the controlled object in the cases where an exception has been
raised.

\subsubsection{Variables}

Variables whose address is taken are not in ALFA.

\subsubsection{Subprograms}

Library-level subprograms, package subprograms, local subprograms and generic
subprograms are all in ALFA.

Primitive subprograms of tagged types are in ALFA only if they do not have
global variables in their frame condition. The reason for this restriction is
that these subprograms can be subject to dynamic dispatching, which must
respect the Liskov Substitution Principle to allow formal verification as
described in Section~\ref{sec:LSP}.

\subsubsection{Calls}
\label{ALFA:calls}

\paragraph{Dispatching Calls}

Ada 2012 defines two differents types of contracts on primitive subprograms of
tagged types, that is 
\begin{enumerate}
\item regular \verb|Pre| and \verb|Post| aspects only apply to the subprogram
  on which they are defined;
\item inherited \verb|Pre'Class| and \verb|Post'Class| aspects are inherited by
  the overriding subprograms, in a way that ensures (partly) the Liskov
  Substitution Principle: inherited preconditions are or'ed, which results in a
  weaker precondition; inherited postconditions are and'ed, which results in a
  stronger postcondition.
\end{enumerate}

The or'ing of inherited preconditions does not fit well a user who does not
want to obey the Liskov Substitution Principle (see Section~\ref{sec:LSP} for
the definition of LSP). In this case, the standard mandates that none of the
subprograms in this inheritance hierarchy has a \verb|Pre'Class| aspect. The
\verb|Pre| aspect should be used instead.

In order to perform formal verification of dipatching calls, we require that
the program obeys the Liskov Substitution Principle. As a result, a dispatching
call is in ALFA only if the subprogram associated to the type on which the
dispatch is performed has a \verb|Pre'Class| aspect, directly or inherited.

\subsubsection{Expressions}

All expressions involving pointers (access types, addresses) are not in ALFA.

Quantified expressions are in ALFA only if they appear in predicate position,
that is, at the top-level of the boolean structure of an expression used in
assertions.

\subsubsection{Statements}

Exception handlers are not in ALFA, because they create complex control-flow
which make formal verification intractable. However, defining exceptions and
raising exceptions are in ALFA. As far as formal verification of data-flow,
run-time errors and contracts is concerned, raising an exception is similar to
removing the program path from consideration.

\subsection{ALFA Extensions}

In ALFA, the pragma Assert immediately at the start of a loop body has a
special meaning. It defines a loop-invariant for this loop, which is used for
unit proof.

\begin{example}
  The following implementation of the integer square-root function has a
  loop-invariant:
\begin{verbatim}
function Sqrt (X : Integer) return Integer is
   Res, Two_Res, Res_Square : Integer := 0;
begin
   while Res_Square <= X loop
      pragma Assert (Res >= 0 and then
                     Two_Res = 2 * Res and then
                     Res_Square = Res * Res);
      Res_Square := Res_Square + Two_Res + 1;
      Two_Res    := Two_Res + 2;
      Res        := Res + 1;
  end loop;
  return Res - 1;
end Sqrt;
\end{verbatim}
\end{example}

The loop-invariant to be used for unit proof should be 1) true the first time
the loop is entered and 2) provable from assuming it at some previous iteration
through the loop and examining the effect of the loop body.

\section{Formal Verification}
\label{sec:formal-verification}

We target three types of formal verification goals:
\begin{enumerate}
\item Data-flow verification (DFV) ensures that there are no reads of
  uninitialized data, and that there is no possible unintended aliasing.
\item Contract verification (CV) ensures that subprograms respect their
  contract.
\item Verification of absence of run-time errors (RTE) ensures that subprograms
  are free from most types of run-time errors.
\end{enumerate}

Verification of Liskov Substitution Principle (LSP), for primitive subprograms
of tagged types and dispatching calls, is part of both DFV and CV.

All of data-flow verification, contract verification and verification of
absence of run-time errors can be performed independently. However, there is a
natural ordering which arises from the logical dependencies between them: 
\begin{center}
  DFV $<$ CV $<$ RTE
\end{center}

Indeed, the results of CV are valid only if DFV is verified, and the results of
RTE are valid only if DFV and CV are verified. In practice, this means that
DFV, CV and RTE should be usually performed in this order.

On the one hand, both CV and RTE depend on contracts and intermediate
assertions (loop-invariants and other assertions). For example, loop-invariants
usually contain information on the range of values, which are both used for
proving absence of run-time error and functional properties. Thus, contracts
and intermediate assertions should be assumed for RTE. On the other hand,
proving contracts and intermediate assertions can be performed without assuming
absence of run-time errors. Indeed, what is proved in this case is that,
provided there are no run-time errors, the contracts and intermediate
assertions hold.

It should be possible to perform CV and RTE independently, first because the
user may be interested in only one of these, second because we will aim at CV
principally through Why and RTE principally through SPARK.

\subsection{Data-Flow Verification (DFV)}
\label{sub:DFV}

Data-flow verification should be performed as a static analysis, similar to
what the Examiner does for SPARK code. However, we aim at finer-grain data-flow
verification than the one provided in SPARK. Instead of tracking the
initialization of entire variables, we track the initialization of variable
subcomponents.

Reads of variable subcomponents are only allowed if all syntactic paths through
the subprogram before reaching the read do initialize completely the
corresponding variable subcomponent. This restriction is similar to the SPARK
one, except that it applies to variable subcomponents instead of entire
variables.

Aliasing between parameter subcomponents and global subcomponents accessed by
a subprogram is forbidden, unless all aliasing parameters and globals are
read. This restriction is similar to the SPARK one, except that it applies to
variable subcomponents instead of entire variables.

\subsection{Contract Verification (CV)}

The contract of a subprogram in ALFA is verified by assuming that the
subprograms it calls do respect their contracts, which includes a possible
recursive call (in which case the assumption is made only for the recursive
call). For the purpose of contract verification, possible run-time errors are
ignored, both in the program and in the contracts. Thus, the results of
contract verification hold for all executions which do not raise a run-time
error.

As part of CV, all intermediate assertions and loop-invariants are also
proved. This includes the contracts, intermediate assertions and
loop-invariants that are useful to prove absence of run-time errors, if RTE is
also an objective.

\subsection{Liskov Substitution Principle Verification (LSP)}
\label{sec:LSP}

Dynamic dispatching on tagged types makes it impossible to know the complete
set of subprograms called at a dispatching call site. A solution would be to
require all overriding subprograms to be known to analyze a dispatching
call. This solution is not modular, as it requires re-verifying a program
whenever a type is derived and new overriding subprograms are defined. Instead,
we restrict the behavior of overriding subprograms so that verification results
are valid for any extension of the types which respect the said restrictions.

These restrictions are known as the Liskov Substitution Principle:
\begin{enumerate}
\item \label{it:lsp-pre} The precondition of an overriding subprogram must be
  weaker than (be implied by, include) the precondition of the subprogram
  they override.
\item \label{it:lsp-post} The postcondition of an overriding subprogram must be
  stronger than (imply, be included in) the postcondition of the subprogram
  they override.
\item \label{it:lsp-read} The frame condition of an overriding subprogram
  should put less constraints on the calling context than the frame condition
  of the subprogram they override.
\item \label{it:lsp-write} The frame condition of an overriding subprogram
  should give more guarantees to a caller, after the call, than the frame
  condition of the subprogram they override.
\end{enumerate}

Conditions~\ref{it:lsp-pre} and~\ref{it:lsp-post} are automatically ensured by
the conditions enforced on calls in ALFA, as described in
Section~\ref{ALFA:calls}. These contracts should be verified as part of doing
CV.

Conditions~\ref{it:lsp-read} and~\ref{it:lsp-write} are verified by:
\begin{itemize}
\item checking that the computed frame condition of a primitive subprogram of a
  tagged type both puts less constraints on the calling context and gives more
  guarantees to a caller than the frame condition suggested by its
  specification (``in'', ``out'' and ``in out'' parameters);
\item assuming that each dispatching call has the frame condition suggested by
  the specification of the subprogram on which it dispatches.
\end{itemize}

\begin{example}
  Consider types \verb|T| and \verb|S| derived from \verb|T|, with primitive
  subprogram \verb|F| being overriden for \verb|S|:~\\

\begin{minipage}{0.5\linewidth}
\begin{verbatim}
type T is tagged record 
   X : Integer; 
end record;

type S is new T with record 
   Y : Integer; 
end record;
\end{verbatim}
\end{minipage}
\begin{minipage}{0.5\linewidth}
\begin{verbatim}
procedure F (A : in out T) is
begin
   A.X := 0;
end F;

procedure F (A : in out S) is
begin
   A.Y := A.X + 1;
end F;
\end{verbatim}
\end{minipage}~\\

The frame condition suggested by the specification of the subprogram \verb|F|
applied to type \verb|T| is the read-write-set \{ \verb|A.X| \}.  The computed
frame condition for \verb|F| applied to type \verb|T| is the write-set \{
\verb|A.X| \}.  Since it does not require \verb|A.X| to be initialized at
function call, it puts less constraint on the calling context than its
specification suggests. Since it does not write any variable path not written
in its specification, it provides as much guarantees to the caller as is
specification suggests.\\

The frame condition suggested by the specification of the subprogram \verb|F|
applied to type \verb|S| is the read-write-set \{ \verb|A.X|, \verb|A.Y| \}.
The computed frame condition for \verb|F| applied to type \verb|S| is the
read-set \{ \verb|A.X| \} and the write-set \{ \verb|A.Y| \}. Since it does not
require \verb|A.Y| to be initialized at function call, it puts less constraint
on the calling context than its specification suggests. Since it does not write
any variable path not written in its specification, it provides as much
guarantees to the caller as is specification suggests.~\\

When analyzing a dispatching call to \verb|F| on static type \verb|T'Class|, it
will be assumed that the subprogram called has the frame condition suggested by
\verb|F|'s specification: the read-write-set \{ \verb|A.X|, \verb|A.?| \} where
\verb|A.?| is the component of type \verb|T'Class| denoting all components of
derived types, like \verb|A.Y| for type \verb|S|.
\end{example}

\subsection{Verification of Absence of Run-Time Errors (RTE)}

The fact that a subprogram in ALFA is free from run-time errors is verified by
assuming that the subprograms it calls do respect their contracts and that they
are free from run-time errors, which includes a possible recursive call (in
which case the assumption is made only for the recursive call).

RTE includes proving that the evaluation of assertions (including contracts and
loop-invariants) does not lead to run-time errors. To that end, the checks
inserted by the compiler to prove absence of runt-time error should be proved.

\begin{example}
  Proving RTE on the procedure \verb|Do_Something_On_Array| involves proving
  that evaluating its postcondition cannot lead to a run-time error:
  \verb|I + J| and \verb|A(I) + A(J)| should not overflow; \verb|I|, \verb|J|
  and \verb|I + J| should be allowed indexes for \verb|A|. Proving RTE on a
  caller of procedure \verb|Do_Something_On_Array| involves proving that
  evaluating its precondition cannot lead to a run-time error: \verb|I| and
  \verb|J| should be allowed indexes for \verb|A|.
\begin{verbatim}
procedure Do_Something_On_Array (A : My_Array; I, J : Index) with
   Pre  => A(I) /= A(J),
   Post => A(I + J) = A(I) + A(J);
\end{verbatim}
\end{example}

Although the Examiner already generates VCs for absence of run-time errors on
the SPARK subset of Ada, we will not use these capabilities directly. Instead,
we will rely on the compiler to insert the proper checks, which will be
translate into assertions in SPARK, for which the Examiner also generate
VCs. The benefits of this approach are that 1) the compiler controls which
run-time checks should be proved, which can be modified on option, and reduced
to fewer checks if the compiler is able to prove some of them (typically for
checks that deal with type checking constraints); 2) the same process applies
to checks that the Examiner would generate and checks that the Examiner does
not know about, because they originate in language constructs that are not in
SPARK; 3) the same checks are generated whatever the proof chain used, through
SPARK or through Why.

\subsection{Verification Modes}

CV and RTE require different (although compatible) choices during the
translation.  Therefore, we define three modes for the translation:
\begin{enumerate}
\item mode CV for contract verification;
\item mode RTE for verification of absence of run-time errors;
\item mode CV+RTE for combined contract verification and verification of
  absence of run-time errors.
\end{enumerate}

In all three modes, the only part of the program which is translated
differently is the logic part, having to do with the translation of
assertions. Modes CV and RTE both insert different assertions, which means that
more reads are performed, thus putting more constraints on the data-flow in
terms of non-aliasing and initialization. Thus, it is sufficient to prove DFV
in mode CV+RTE for data-flow to be verified in both modes CV and RTE. Hence DFV
should be performed in mode CV+RTE.

\section{Translation to SPARK/Why}

A subprogram in ALFA should be translated into an intermediate representation
in SPARK or in Why. From this representation, the Examiner or Why tools can
generated Verification Conditions (VCs) to prove using an automatic prover.  In
order to facilitate fine-grain modular proof, each subprogram should lead to
the generation of a separate unit (package in SPARK, module in Why), which can
be proved independently. Notice that the generated SPARK is not executable, and
does not match in general the structure of the source Ada program, even if this
program is written in the SPARK subset of Ada. Ideally, the generated SPARK
should correspond to a simple subset of SPARK (for example no need for
visibility rules in this subset as everything is public). Thus, it should be
easier to formalize this subset and prove the correctness of transformations or
analyses on this subset if needed.

\subsection{Introduction to SPARK/Why}

\subsubsection{Common Basis of SPARK and Why}

SPARK and Why are both programming languages designed for deductive
verification, more than execution. They both mix coding constructs with logic
constructs whose aim is to state invariant properties of the program.

The central logic construct is the contract, which serves to fully describe the
effect of calling a subprogram, for the purpose of separate verification. Each
subprogram in SPARK/Why must be defined with a proper contract:
\begin{enumerate}
\item a precondition describes constraints on the calling context;
\item a frame condition describes both the variables on which the result of the
  subprogram depends (variables read) and the variables which may be modified
  as a result of the call (variables written);
\item a postcondition describes constraints on the result of the subprogram.
\end{enumerate}

Both SPARK and Why define references which are used to pass parameters to
subprogram calls. None defines pointer types. Both SPARK and Why define static
rules to check that the only parameters and globals (mentioned in the frame
condition) which may be aliased are those which are only read.

Both languages come equipped with a verification condition generator (VCGen)
which produces formulas which should be proved (automatically or manually) for
the contracts expressed on the SPARK or Why code to hold. The VCGen for Why is
the Why tool; it can produce VCs in a variety of languages, among which Why
itself and SMTLIB format.  The VCGen for SPARK is the Examiner; it produces VCs
in FDL, which can be translated by the ViCToR tool into a variety of languages,
among which SMTLIB format.

\subsubsection{SPARK Specificities}

SPARK code, when stripped from its logic constructs, is a subset of
Ada. Thus, SPARK inherits many constructs of Ada, and it is of course
executable.

SPARK enforces strict data-flow properties, that are checked statically by the
Examiner tool:
\begin{itemize}
\item to read any part of a variable, the entire variable must be initialized
  on all syntactic program paths from the start of the current subprogram to
  the program point where the read is performed;
\item a parameter of mode ``in'' or ``in out'' in a call counts as a read;
\item every parameter of mode ``out'' in the current subprogram must be
  initialized on all syntactic program paths from the start of the subprogram
  to the end of the subprogram;
\item every parameter of mode ``in'' or ``in out'' in the current subprogram
  must be partially read on at least one syntactic program path from the start
  of the subprogram to the end of the subprogram;
\item global variables mentioned in the frame condition must respect the same
  constraints as parameters.
\end{itemize}

Essentially, these data-flow properties enforce a programming discipline which
ensures that there is no read of an uninitialized value anywhere in the
program, and which the Examiner checks statically.  As part of the same static
analysis, the Examiner also issues errors if some variables or statements are
unused.

As SPARK is a subset of Ada, it knows all about the possible run-time errors in
this subset, and it generates VCs guarding against such errors.

\subsubsection{Why Specificities}

Why is a functional language with imperative features, of an OCaml flavor. Why
allows mixing freely axiomatized entities and defined entities, which means
that Why programs cannot be executed.

Why defines four primitive types:
\begin{itemize}
\item \emph{int} for mathematical integers;
\item \emph{bool} for Booleans;
\item \emph{real} for mathematical real numbers;
\item \emph{unit} for the type of statements.
\end{itemize}

There are no predefined aggregate types (records and arrays) in Why. In
general, these should be defined as abstract types and axiomatized. In this
respect, Why is closer to the translation in FDL of SPARK data structures than
to SPARK itself. However, Why provides a definition for functional arrays in
its prelude (its automatically loaded standard library), and a predefined type
array acting as a mutable reference on functional arrays. This type is
polymorphic, so that it can be instantiated to define arrays over whatever
element type.

Why also provides algebraic types which allow defining enumerations, LISP-like
lists, discriminated unions, \etc, together with pattern-matching expressions
(the equivalent of Ada case-statements and case-expressions) on algebraic
types~\cite{Paskevich09RR}.

Why distinguishes between logic and programs, and between terms and predicates
in the logic. For example, a quantification is of type \verb|prop| which is
different from the Boolean type for terms \verb|bool|.

\subsection{From Ada to Unambiguous Ada}
\label{sub:slicing-function}

Ada functions may write global variables, so that the compiler choice of
evaluation order for expressions may influence the result. Whatever it is, this
choice should be the same for execution and verification, which is obtained by
lifting function calls outside of expressions in the specified order of
evaluation. This consists in a simple walk through the AST following the order
of evaluation, and introducing temporary variables for every function
call. 

\begin{example}
  With a left-to-right evaluation order, the statement
\begin{verbatim}
X := F(X) + G(H(Y),K(Z));
\end{verbatim}
  should be translated into:
\begin{verbatim}
Tmp1 := F(X);
Tmp2 := H(Y);
Tmp3 := K(Z);
Tmp4 := G(Tmp2,Tmp3);
X := Tmp1 + Tmp4;
\end{verbatim}
\end{example}

To simplify further passes, array concatenation \verb|&| should be replaced at
this stage by equivalent concatenation functions, and treated as
such. Similarly, slices used in expressions should be replaced by equivalent
slicing functions, and treated as such.

Likewise, Ada arithmetic expressions may be reordered by the compiler when not
enough parenthesized, so the translation should introduce the necessary
parentheses which force the natural evaluation order given by operators
associativity. 

\begin{example}
  The expression \verb|X + Y + Z| should be translated into \verb|(X + Y) + Z|.
\end{example}

\subsection{From Unambiguous Ada to Expanded Ada}
\label{UnambigAda2ExpandedAda}

This translation should be different for CV and RTE. It is based on the
expansion phase in the GNAT front-end which will simplify the work needed in
subsequent phases:
\begin{itemize}
\item it unfolds all aggregate initializations;
\item it makes all checks for run-time errors explicit;
\item it explicits default expressions for parameters in calls;
\item it removes renamings;
\item it add a tag component to tagged types;
\item \etc [add other expansions here]
\end{itemize}

In particular, types which depend on some dynamic bounds should be translated
into types with static bounds, by creating additional variables to hold the
dynamic bounds. Then, the run-time checks should mention these dynamic bounds.
This concerns discrete subtypes with dynamic bounds and array types with
dynamic bounds. Parameters of array type with dynamic bounds should be
translated into three parameters: one for the array and two for the bounds.

Whenever a subprogram has an in-variable of such a type with dynamic bounds, an
additional precondition should be generated stating that the variable is within
bounds.  Whenever a subprogram has an out-variable of such a type with dynamic
bounds, an additional postcondition should be generated stating that the
variable is within bounds.

Each discriminant of a record type should be translated into a normal
component. When this discriminant controls a variant, all components of the
different cases should be added as components of the record. When this
discriminant controls the size of an array, this array component should be
translated like an array with dynamic bounds.

Declarations of generic entities should be instantiated for each generic
instantiation in the program. Bodies of generic entities should be kept not
instantiated, so that each generic body is verified only once.

A quantified expression over formal containers should be translated into a
quantification over the corresponding cursor type, inserting a guard that the
cursor belongs to the container. This is possible because cursors of formal
containers are simply integer types.

\begin{example}
The quantified expresssion:
\begin{verbatim}
for all X of My_Set => Prop (X)
\end{verbatim}
should be translated into:
\begin{verbatim}
for all C in Cursor => if Has_Element (My_Set, C) then Prop (Element (My_Set, C))
\end{verbatim}
\end{example}

Contracts (precondition and postcondition) in SPARK/Why have a slightly
different semantics than contracts in Ada 2012, because contracts in SPARK/Why
completely ignore the possibility of a run-time error being raised while
evaluating the contract. When checking for absence of run-time errors (which
can be separated from contract checking), the absence of run-time errors in
contracts should also be proved, which requires the extension of contracts with
additional conjuncts in SPARK/Why. This expansion should be performed at this
stage.

\begin{example}
  The following contract in Ada:
\begin{verbatim}
function Get (A : My_Array; X : Integer) return Element
   with Pre => A (X) /= Nil_Element;
\end{verbatim}

is equivalent to the following contract in SPARK:
\begin{verbatim}
function Get (A : My_Array; X : Integer) return Element;
--# pre X in My_Array'Range and then A (X) /= Nil_Element;
\end{verbatim}
\end{example}

\subsection{From Expanded Ada to Extended SPARK}

As a general remark, the translation should be allowed to introduce copies of
any values (for example to translate extended returns), which should not have
any effect on the validity of the proofs.

\subsubsection{Packages}
\label{Ada2SPARK:packages}

\paragraph{Specification and Body}

Packages in Ada are mutually dependent, as package bodies can refer to visible
types, variables and subprograms of any other package. Only package
specifications must respect a partial ordering so that there is no circular
dependencies between package specifications. But circular dependencies between
package specifications become apparent once we express frame conditions on the
subprograms declared in the package: subprogram \verb|P.F| may read variable
\verb|Q.Y| while subprogram \verb|Q.G| reads variable \verb|P.X|.  As SPARK/Why
do not allow expressing circular dependencies (SPARK rules prevent mutual
dependencies between package specifications or bodies; Why prevents circular
inclusion), a package in Ada should be translated into more than one unit in
SPARK/Why. A unit is a package in SPARK and a module in Why.

Visibility should be completely ignored in the translation, so that entities
which are declared public/private in the package specification and entities
which are declared in the package body are treated alike. A package \verb|P| in
Ada (specification+body if present) should be translated into three package
specifications in SPARK: a package specification \verb|P_Type| for its types, a
package specification \verb|P_Data| for its variables and a package
specification \verb|P_Spec| for its subprogram specifications.

\begin{example}
\tikzstyle{arrow}=[->, thick]
The mutual dependence in:
\begin{tikzpicture}[baseline=(current bounding box.west)]
  \node at (0.5,1.2) {\verb|P|};
  \node at (0.5,0.5) (X) {\verb|X|};
  \node at (0.5,0) (F) {\verb|F|};
  \draw (0,-0.5) -- (0,1) -- (1,1) -- (1,-0.5) -- (0,-0.5);

  \node at (2.5,1.2) {\verb|Q|};
  \node at (2.5,0.5) (Y) {\verb|Y|};
  \node at (2.5,0) (G) {\verb|G|};
  \draw (2,-0.5) -- (2,1) -- (3,1) -- (3,-0.5) -- (2,-0.5);

  \draw[arrow] (F) -- (Y);
  \draw[arrow] (G) -- (X);
\end{tikzpicture}
should be translated into: 
\begin{tikzpicture}[baseline=(current bounding box.west)]
  \node at (0.5,1.2) {\verb|P_Data|};
  \node at (0.5,0.5) (X) {\verb|X|};
  \draw (0,0) -- (0,1) -- (1,1) -- (1,0) -- (0,0);

  \node at (0.5,-0.8) {\verb|P_Spec|};
  \node at (0.5,-1.5) (F) {\verb|F|};
  \draw (0,-2) -- (0,-1) -- (1,-1) -- (1,-2) -- (0,-2);

  \node at (2.5,1.2) {\verb|Q_Data|};
  \node at (2.5,0.5) (Y) {\verb|Y|};
  \draw (2,0) -- (2,1) -- (3,1) -- (3,0) -- (2,0);

  \node at (2.5,-0.8) {\verb|Q_Spec|};
  \node at (2.5,-1.5) (G) {\verb|G|};
  \draw (2,-2) -- (2,-1) -- (3,-1) -- (3,-2) -- (2,-2);

  \draw[arrow] (F) -- (Y);
  \draw[arrow] (G) -- (X);
\end{tikzpicture}
\end{example}

Hierarchical relations between packages (child packages, local packages), which
govern special visibility rules, should disappear during the translation, so
that all packages are generated in SPARK at library-level, without any
hierarchical relations between them.

\paragraph{Elaboration}

All initializations and statements which are executed during package
elaboration should be treated like the body of a special elaboration
subprogram, and translated like other subprograms as described in
Section~\ref{Ada2SPARK:subprograms}.

\paragraph{Generics}

Instantiations of generic packages produced during expansion (see
Section~\ref{UnambigAda2ExpandedAda}) should be translated like non-generic
packages, except no subprogram is generated for elaboration.

The generic package itself should be translated specially, so that verification
of the subprogram bodies it defines (including the elaboration subprogram) is
performed only once, instead of being performed for each instantation
separately. A generic package \verb|P| in Ada (specification+body if present)
should be translated into four package specifications in SPARK: package
specifications \verb|P_Type|, \verb|P_Data| and \verb|P_Spec| like previously,
plus a package specification \verb|P_Formal| for its formal part.

Formal parameters should be translated into special entities of
\verb|P_Formal|:
\begin{itemize}
\item A formal object declaration should be translated like an object
  declaration.
\item A formal private type definition should be translated into a private type
  with no definition.
\item A formal derived type definition from a record type should be translated
  into a private derived type from this same record type with no definition.
\item A formal derived type definition from a scalar type should be translated
  into a new derived type from this scalar type.
\item A formal signed integer type definition should be translated into a
  signed integer type with unknown (dynamic) bounds.
\item A formal modular type definition should be translated into a modular type
  with unknown (dynamic) upper bound.
\item A formal floating-point or fixed-point definition is not in ALFA.
\item A formal discrete type definition should be translated into a private
  type with no definition.
\item A formal array type definition should be translated into an array type
  definition.
\item A formal subprogram should be translated into a subprogram specification.
\item A formal package is not in ALFA.
\end{itemize}

See Sections~\ref{Ada2SPARK:types}, \ref{Ada2SPARK:variables}
and~\ref{Ada2SPARK:subprograms} for the precise translation of these formal
parameters.

A generic package instantiation should be translated into a call to a special
procedure, whose parameters are the formal object declarations of the generic,
with the same modes: ``in'' or ``in out''.

\subsubsection{Variables}
\label{Ada2SPARK:variables}

\paragraph{Splitting Subcomponents}

As mentioned in Section~\ref{sec:ALFA-overview}, we aim at using the data-flow
analysis in SPARK to prove the absence of reads of uninitialized data, and the
absence of unintended aliasing. Since we aim at analyze the initialization of
variable subcomponents instead, and since SPARK only analyzes the
initialization of entire variables, we need to split all variables into their
individual subcomponents.

For each variable path as defined in Section~\ref{sub:data-flow}, we introduce
a new path-variable, so that each variable in the source Ada program is
translated into a set of path-variables of elementary or array type (over
elements which have elementary or array types, recursively).

\begin{example}
  Consider the various types and variables below, where \verb|Val| is an
  elementary type:
\begin{verbatim}
  type Point is record X, Y : Val; end record;
  type Points is array (Index) of Point;
  J : Val; A : Point; B : Points;
\end{verbatim}
The variables above should be split into individual subcomponents as follows:
\begin{verbatim}
  type Val_Array is array (Index) of Val;
  J, A_X, A_Y : Val; B_X, B_Y : Val_Array;
\end{verbatim}
\end{example}

\paragraph{Array Variables}

As described in Section~\ref{Ada2SPARK:types} below, array types with dynamic
bounds should be translated into array types with static bounds. In order to
still allow expressing the value of the \verb|'First| and \verb|'Last|
attribute applied to this array, two new variables should be generated for
every dimension of every path-variable of array type, which hold the value of
\verb|'First| and \verb|'Last| applied to this path-variable and this
dimension.

\paragraph{Initialization}

The initialization part of a variable declaration should be translated into an
assignment in the body of the elaboration subprogram (for global variables) or
in the body of the subprogram (for local variables).

\paragraph{Named Numbers and Constants}

Named numbers, which simply provide a proxy name for a literal, should be
inlined.

Constants with a static value, which could be as well named numbers, should be
inlined too.

Constants with a non-static value should be split like variables into
path-variables.

\subsubsection{Types}
\label{Ada2SPARK:types}

\paragraph{Discrete Types}

Enumeration types should be kept unchanged. Character types should be
translated into normal enumeration types, by inserting calls to suitable
conversion functions between characters and the new enumeration types. Boolean
types should be kept unchanged.

Integer types with static bounds should be kept unchanged.

Integer types with dynamic bounds should be translated into the closest
enclosing subtype with static bounds, together with a type invariant to be
asserted in preconditions and postconditions. See
Example~\ref{ex:dynamic-bounds} below. The type invariant states that a value
is within the allowed dynamic bounds.

The type invariant should be asserted:
\begin{itemize}
\item in the precondition of subprogram \verb|F| (the version used for
  verifying the body of \verb|F|) for all path-variables of mode ``in'' and
  ``in out'' in the frame condition of \verb|F|;
\item in the postcondition of subprogram \verb|F| (the version used for
  verifying the body of callers of \verb|F|) for all path-variables of mode
  ``out'' and ``in out'' in the frame condition of \verb|F|;
\end{itemize}

Derived scalar types should be kept (translation into SPARK subtypes is delayed
until Section~\ref{SPARK2SPARK:types}).

\paragraph{Array Types}

One-dimensional array types with static bounds should be kept unchanged.

One-dimensional array types with dynamic bounds should be translated into
arrays types with static bounds by taking the closest enclosing subtype with
static bounds of its index type as new index type, together with special
variables holding the \verb|'First| and \verb|'Last| attributes applied to this
variable (as presented in Section~\ref{Ada2SPARK:variables}).

Multi-dimensional array types should be translated like arrays of arrays, each
array having dimension one only.

\paragraph{Record Types}

As shown in Section~\ref{Ada2SPARK:variables}, record types should disappear
as a result of splitting every composite variable into path-variables.

The discriminants of a discriminant record should be treated like normal
components. Discriminants used to bound array components should be used in
place of the special variables holding the \verb|'First| and \verb|'Last|
attributes applied to this component (as presented in
Section~\ref{Ada2SPARK:variables}). 

\paragraph{Tagged Types}

The type of a tagged type is stored in a tag component, which is inserted by
expansion (see Section~\ref{UnambigAda2ExpandedAda}). This tag component is
required to translate class-wide assignment (there is a check that the source
and destination objects have the same dynamic type) and class-wide structural
equality (objects must be of the same type and their components must be equal
for the objects to be equal). The tag component should be translated like a
mathematical integer (and not an enumerated value, because it would bound the
number of derived types from a type).

\paragraph{Class-wide Types}

A class-wide type \verb|T'Class| denotes type \verb|T| and all types derived
from \verb|T|. It should be translated away like record types into the same set
of components as the underlying record type, plus a special component to denote
all the unknown components of derived types. This special component will not be
accessed directly by the code, but used in frame conditions for dispatching
calls, as described in Section~\ref{sec:LSP}.

\paragraph{Anonymous Types}

All anonymous types should be explicitly named for SPARK/Why. This concerns:
\begin{itemize}
\item anonymous subtypes in type definitions (for example, the bounds in an
  array);
\item anonymous subtypes in object and component declarations;
\item anonymous array types in object declarations.
\end{itemize}

\paragraph{Incomplete Types}

Incomplete type declarations should be dropped, as they are redundant with the
corresponding complete type declarations.

\subsubsection{Subprograms}
\label{Ada2SPARK:subprograms}

 In ALFA, all syntactic
program paths through a subprogram should initialize individual variable
subcomponents before a subcomponent is read, while other subcomponents of the
same variable may remain uninitialized. Likewise, parameter modes do not impose
initialization of entire variables. Instead, an analysis computes the set of
variable subcomponents only read, only written and both read and written in a
subprogram, which are treated like the ``in'', ``out'' and ``in out''
parameters in a SPARK data-flow analysis.

\paragraph{Declarative Sections}

In Ada, subprograms may define several local declarative sections, either at
the start of the subprogram body, or inside a local block-statement. Local
subprograms defined inside these declarative blocks may introduce mutual
dependencies with other packages as seen in
Section~\ref{Ada2SPARK:packages}. Given a subprogram \verb|F| defined in
package \verb|P|, we should extend the solution described in
Section~\ref{Ada2SPARK:packages}, by adding local types to \verb|P_Type|, local
variables to \verb|P_Data| and local subprograms to \verb|P_Spec|. Notice in
particular that local variables are translated into global variables for the
purpose of verification. This translation is correct for deductive
verification, in the sense that every assertion proved on the generated code by
deductive verification is true of the original code too.

\paragraph{Library-level}

Contrary to Ada, SPARK only allows the definition of one library-level
subprogram, which should be the main entry point for the program. In order to
prepare the translation to SPARK, and to provide a uniform solution to the
problem of declarative sections mentioned above, we should translate
library-level subprograms as if they were defined in their own library-level
package.

\paragraph{Body}

The body of a subprogram \verb|F| defined in package \verb|P| should be
translated in its own package \verb|F_Body|, with a package specification
containing the subprogram specification for \verb|F|, and a package body
containing the subprogram body for \verb|F|. Thus, the original subprogram
specification for \verb|F| is split in the generated SPARK code into two
subprogram specifications: one in \verb|P_Spec| and one in \verb|F_Body|. This
separation is necessary to 1) apply deductive verification in SPARK/Why to the
body of function \verb|F| only, in order to get modularity without relying on
specific features of SPARK/Why tools; 2) break dependencies which may arise
from recursion between subprograms, including self and mutual recursion.  Thus,
all calls to \verb|V| in the Ada program should be translated into calls to
\verb|P_Spec.F|, including possible recursive calls in \verb|F_Body.F|.

\paragraph{Overloading}

All names should be made unique in their package. This takes care of
overloading, which is allowed in Ada but not in SPARK or Why. In particular,
operator symbols should be renamed so that operators become regular functions.

\paragraph{Abstract and Null}

An abstract subprogram declaration should be translated into a normal
subprogram declaration, as the constraint that such subprograms are not defined
is a compile-time constraint checked by the compiler.

A null subprogram declaration should be translated into a normal subprogram
declaration and a body with a single null statement. [unless SPARK defines null
procedures as a way to specify axioms or lemmas at the source level, see
J925-004]

\paragraph{Parameters} 

Like shown in Section~\ref{Ada2SPARK:variables} for variables,
each subprogram parameter should be split into a set of path-variables
too. However, only those path-variables which correspond to variable paths in
the frame condition of the subprogram should be added as parameters:
\begin{enumerate}
\item path-variables corresponding to (parameter) variable paths in the
  read-set of subprogram \verb|F| should be added as ``in'' parameters of
  \verb|F|;
\item path-variables corresponding to (parameter) variable paths in the
  write-set of subprogram \verb|F| should be added as ``out'' parameters of
  \verb|F|;
\item path-variables corresponding to (parameter) variable paths in the
  read-write-set of subprogram \verb|F| should be added as ``in out''
  parameters of \verb|F|.
\end{enumerate}

Likewise, globals in the frame condition of subprogram \verb|F| should be
split in those path-variables which correspond to variable paths in
the frame condition of \verb|F|:
\begin{enumerate}
\item path-variables corresponding to (global) variable paths in the
  read-set of subprogram \verb|F| should be added as ``in'' globals of
  \verb|F|;
\item path-variables corresponding to (global and local) variable paths in the
  write-set of subprogram \verb|F| should be added as ``out'' globals of
  \verb|F|;
\item path-variables corresponding to (global) variable paths in the
  read-write-set of subprogram \verb|F| should be added as ``in out''
  globals of \verb|F|.
\end{enumerate}

\begin{example}
  Take a procedure \verb|Set| which only initializes part of its \verb|Point|
  parameter:
\begin{verbatim}
type Point is record X, Y : Val; end record;
procedure Set (P : out Point) is begin P.X := 0; end Set;
\end{verbatim}

  It should be translated into a subprogram which only takes as parameters
  those path-variables that are part of its frame condition:
\begin{verbatim}
procedure Set (P_X : out Val) is begin P_X := 0; end Set;
\end{verbatim}
\end{example}

For each path-variable of mode ``in'' or ``in out'' in the frame condition of
\verb|F| (see Section~\ref{Ada2SPARK:variables}) whose type defines a dynamic
invariant (see Section~\ref{Ada2SPARK:types}), the corresponding dynamic
invariant should be stated in the precondition of \verb|F_Body.F|, so that it
will be assumed during formal verification of the body of~\verb|F|.

For each path-variable of mode ``out'' or ``in out'' in the frame condition of
\verb|F| (see Section~\ref{Ada2SPARK:variables}) whose type defines a dynamic
invariant (see Section~\ref{Ada2SPARK:types}), the corresponding dynamic
invariant should be stated in the postcondition of \verb|P_Spec.F|, so that it
will be assumed during formal verification of the callers of~\verb|F|.

\begin{example}
\label{ex:dynamic-bounds}
The parameters \verb|X| and \verb|Y| below have a type with dynamic bounds:
\begin{verbatim}
subtype Index is Integer range Index_First .. Index_Last;
procedure F (X : in Index; Y : out Index);
\end{verbatim}

They should be translated into parameters of the underlying static subtype,
plus a precondition on \verb|F_Body.F| and a postcondition on \verb|P_Spec.F|:
\begin{verbatim}
procedure F_Body.F (X : in Integer; Y : out Integer);
--# pre X in Index_First .. Index_Last;

procedure P_Spec.F (X : in Integer; Y : out Integer);
--# post Y in Index_First .. Index_Last;
\end{verbatim}
\end{example}

Any parameter of a definite array type should be translated into a parameter of
the corresponding indefinite array type in \verb|P_Spec|, so that it is easier
to translate calls that take slices as actual parameters. Any parameter of an
indefinite array type in the original Ada code should lead to the generation of
two additional parameters to hold the value of the \verb|'First| and
\verb|'Last| attributes applied to this parameter.

\paragraph{Generics} Similarly to what is done for generic packages,
instantiations should be translated like normal subprogram specifications, and
a special package should be generated for the analysis of the generic
subprogram body.

\paragraph{Overriding}

As seen in Section~\ref{Ada2SPARK:subprograms}, inheritance relations disappear
during the translation of subprograms. As indicated in
Section~\ref{sec:LSP}, Liskov Substitution Principle is verified as part of
normal contract verification, so nothing special is required for LSP.

\subsubsection{Calls}

\paragraph{Dispatching Calls}

As indicated in Section~\ref{sec:LSP}, a dispatching call should be analyzed as
a call to the subprogram on which it is dispatching, with the frame condition
suggested by the specification of the subprogram.

\subsubsection{Expressions}

\paragraph{Slices}

The calls to slicing functions introduced in Section~\ref{sub:slicing-function}
for slices in expressions should be translated in calls to procedures
\verb|Create_Slice_T| where \verb|T| is the type of array element. This
procedure takes the following parameters:
\begin{enumerate}
\item a destination array \verb|Dest|;
\item a destination integer variable \verb|Dest_First| for the lower bound of
  \verb|Dest|;
\item a destination integer variable \verb|Dest_Last| for the upper bound of
  \verb|Dest|;
\item a source array \verb|Src|;
\item an integer value \verb|Src_First| for the lower bound of the slice;
\item an integer value \verb|Src_Last| for the upper bound of the slice.
\end{enumerate}

The postcondition of \verb|Create_Slice_T| would ensure that the proper copy is
made:
\begin{verbatim}
type Arr_T is array (Integer range <>) of T;
procedure Create_Slice_T (Dest : out Arr_T; Dest_First, Dest_Last : out Integer;
                          Src  : Arr_T;     Src_First, Src_Last   : Integer);
--# post (for all J in Integer => (Src_First <= J and J <= Src_Last) ->
--#                                  Dest (J) = Src (J)) and
--#      Dest_First = Src_First and Dest_Last = Src_Last;
\end{verbatim}

\begin{example}
The simple slice of array Arr below:
\begin{verbatim}
X := Arr (1 .. 10);
\end{verbatim}

should be translated into a call of \verb|Create_Slice_T| like the following:
\begin{verbatim}
Create_Slice_T (Dest => Tmp, Dest_First => Tmp_First, Dest_Last => Tmp_Last, 
                Src  => Arr, Src_First  => 1,         Src_Last  => 10);
X := Tmp;
\end{verbatim}
\end{example}

Direct writes into a slice should be translated into calls to procedures
\verb|Copy_Slice_T| where \verb|T| is the type of array element. This
procedure takes the following parameters:
\begin{enumerate}
\item a destination array \verb|Dest|;
\item a source array \verb|Src|;
\item an integer value \verb|Src_First| for the lower bound of the slice;
\item an integer value \verb|Src_Last| for the upper bound of the slice.
\end{enumerate}

The postcondition of \verb|Copy_Slice_T| would ensure that the proper copy is
made:
\begin{verbatim}
type Arr_T is array (Integer range <>) of T;
procedure Copy_Slice_T (Dest : out Arr_T; Src : Arr_T; Src_First, Src_Last : Integer);
--# post for all J in Integer => (Src_First <= J and J <= Src_Last) ->
--#                                 Dest (J) = Src (J);
\end{verbatim}

\begin{example}
The simple slice of array Arr below:
\begin{verbatim}
Arr (1 .. 10) := X;
\end{verbatim}

should be translated into a call of \verb|Copy_Slice_T| like the following:
\begin{verbatim}
Copy_Slice_T (Dest => Arr, Src  => X, Src_First  => 1, Src_Last  => 10);
\end{verbatim}
\end{example}

Slices passed in parameters should be translated differently if the parameter
is of a definite or an indefinite array type. If the parameter is of a definite
type, the bounds are known so the entire array should be passed as
parameter. If the parameter is of an indefinite type, the bounds of the slice
should be passed as additional parameters to the call.

\paragraph{Quantified Expressions} 

A quantified expression in Ada should be translated into the corresponding
logic annotation in SPARK (\verb|for all| in Ada becomes \verb|for all| in
SPARK; \verb|for some| in Ada becomes \verb|for some| in SPARK). All quantified
expressions should be over integer ranges at this point. They should be
translated into a quantification over the corresponding static type in SPARK,
inserting a guard that the value respects the dynamic types for the integer
range.

\begin{example}
  The quantified expression over type \verb|Index| with dynamic bounds \verb|A|
  and \verb|B|:
\begin{verbatim}
for all X in Index => Prop (X)
\end{verbatim}
should be translated into:
\begin{verbatim}
for all X : Index range A .. B => Prop (X)
\end{verbatim}
\end{example}


\subsubsection{Statements}

Block statements should be trimmed from their declarative part after the
translation described in Section~\ref{Ada2SPARK:subprograms}.

\subsection{From Extended SPARK to SPARK}

\subsubsection{Packages}
\label{SPARK2SPARK:packages}

\paragraph{SPARK Annotations}

As described in Section~\ref{Ada2SPARK:packages}, a package \verb|P| in Ada may
be translated into three package specifications in SPARK: a package
specification \verb|P_Type| for its types, a package specification
\verb|P_Data| for its variables and a package specification \verb|P_Spec| for
its subprogram specifications. Additional SPARK annotations should be generated
in each package for visibility and data-flow analysis.

First, each package should start with SPARK inherit annotations for all
generated packages from which is refers entities. These should reflect 1) the
with-clauses in the original package; 2) the separation between generated
packages; 3) the data dependencies in the generated frame conditions; 4) the
new location of previously inherited primitive subprograms due to the
translation of derived scalar types into subtypes (see
Section~\ref{SPARK2SPARK:types}). Thus, it will probably be more convenient to
recompute the set of packages which should be inherited.

Second, each data package \verb|P_Data| should include a SPARK own annotation
for all the variables it defines, and a SPARK initialization annotation for
those variables which were initialized at elaboration. In order to detect
elaboration order dependencies, a package \verb|P_Data_Uninit| should be
defined which duplicates \verb|P_Data| without any initialization annotation,
to be referred to from elaboration subprograms. The elaboration subprogram for
package \verb|P| should have all variables initialized at elaboration in its
out global annotation.

\paragraph{Use Clauses}

Use-type-clauses should be generated to restore visibility of predefined
primitive operations for 1) original use-clauses in package \verb|P| which are
removed in SPARK; 2) the new location of previously inherited primitive
operations due to the translation of derived scalar types into subtypes (see
Section~\ref{SPARK2SPARK:types}).

\subsubsection{Variables}
\label{SPARK2SPARK:variables}

Nothing special.

\subsubsection{Types}
\label{SPARK2SPARK:types}

Ada derived scalar types should be translated into SPARK subtypes. Indeed,
after the front-end has checked that derived types are properly used and after
overloading has been resolved, the constraints respected by a derived type are
the same as those respected by a subtype with the same bounds. Notice that this
may require that new packages are withed to make inherited primitive operations
for the parent type available.

\subsubsection{Subprograms}

\paragraph{SPARK Annotations}

SPARK global annotations should be generated from the sets of variable paths
computed in Section~\ref{sub:data-flow}:
\begin{itemize}
\item the global ``in'' annotation should be the read-set;
\item the global ``out'' annotation should be the write-set;
\item the global ``in out'' annotation should be the read-write-set.
\end{itemize}

As local variables were added to the write-set in Section~\ref{sec:frame-sets},
the translation should prevent warnings to be issued because some of these
variable paths may not be written on some path before returning. So an
initialization subprogram should be declared, which takes all local variable
paths in its ``in out'' annotation. This initialization subprogram should be
called prior to returning from the subprogram, which ensures the translated
subprogram respects its data-flow contracts for local variable paths. Only the
absence of any read to some local variable path could be detected and displayed
as an error, which is safe to do.

As said before, a special global variable called \heap represents all the
dynamically allocated memory plus all variables whose address is taken, so that
reads and writes to dynamically allocated memory show in SPARK contracts
as reads and writes to \heap. Notice that without this \heap variable,
contracts would be wrong and break the consistency of the proof system.

\begin{example}
  If reads and writes to \heap were not computed, it would be possible to prove
  that \verb|Problem| below always returns \verb|True|, because \verb|Set|
  would be seen as a noop, and \verb|Get| would be seen as a constant function:

\begin{verbatim}
X : access Integer;

procedure Set is begin X.all := 0; end Set;
function Get return Integer is (X.all);

function Problem return Boolean is
   X1 : Integer := Get;
begin
   Set;
   return X1 = Get;
end Problem;
\end{verbatim}
\end{example}

Ada functions may write global variables, which is not allowed in SPARK. Thus,
functions should be translated into procedures by introducing an additional
``out'' parameter.

\paragraph{Array Initialization Through Loops}

In Section~\ref{sec:init-array-through-loop}, we have described how we can
detect that a loop $l$ completely initializes an array path $w$, without any
reads (direct or indirect) of this array path in the loop. The Examiner does
not perform this kind of fine-grain analysis, and thus it will issue wrong
warnings about the corresponding path-variable not being properly
initialized. To prevent the generation of these warnings, the translation
should insert before the loop a call to a special procedure which has a single
out-global: the path-variable corresponding to $w$.

\subsubsection{Calls}

Nothing special.

\subsubsection{Expressions}

Nothing special.

\subsubsection{Statements}

\paragraph{Return Statements} 

SPARK mandates that procedures do not contain return statements, and that
functions contain a single return statement as last statement. 

Thus, all return
statements of a procedure \verb|P| should be translated into calls to a local
procedure \verb|P_Return| which:
\begin{enumerate}
\item \label{it:no-param} takes no parameters;
\item \label{it:in-globals} has the out-parameters and out-globals of \verb|P|
  as in-globals;
\item \label{it:out-globals} has the local variables of \verb|P| as out-globals;
\item \label{it:pre} has the postcondition of \verb|P| as precondition;
\item \label{it:in-globals-more} has all variables mentioned in the
  postcondition of \verb|P| as in-globals;
\item \label{it:post} has \verb|False| as postcondition.
\end{enumerate}
(\ref{it:in-globals}) ensures that at the calling point of \verb|P_Return|, all
out-variables of \verb|P| are initialized. (\ref{it:out-globals}) ensures that
no flow error will be generated due to some (locally dead) path through this
call not initializing some local variables. (\ref{it:pre}) ensures that at the
calling point of \verb|P_Return|, the postcondition of \verb|P|
holds. (\ref{it:in-globals-more}) adds the variables mentioned in the
precondition of \verb|P_Return| as in-globals, which is compulsory in SPARK.
(\ref{it:post}) ensures that any path through this call is considered as a dead
path when proving VCs.

\begin{example}
  Take procedure \verb|P| with local variable V, which should be declared in
  SPARK as:
\begin{verbatim}
procedure P (X : in T; Y : out T; Z : in out T);
--# global in     A;
--#           out B;
--#        in out C;
--# post Prop (X, Z~, B);
\end{verbatim}
Any return statement in \verb|P| should be translated into a call to local
procedure \verb|P_Return|, declared as:
\begin{verbatim}
procedure P_Return;
--# global in     Y, B,      --  out-variables of P
--#               X, Z_Old;  --  added because used in the pre
--#           out V;
--# pre  Prop (X, Z_Old, B);
--# post False;
\end{verbatim}
where \verb|Z_Old| is a variable declared to hold the value of \verb|Z| at the
entry to \verb|P|.
\end{example}

Likewise, a return statement of a function \verb|F| which is not the final
return statement should be translated into an assignment of the returned value
to a new local variable \verb|F_Result| followed by a call to a local procedure
\verb|F_Return| which:
\begin{enumerate}
\item \label{it:fn-no-param} takes no parameters;
\item \label{it:fn-in-globals} has the out-parameters and out-globals of
  \verb|F| as in-globals;
\item \label{it:fn-out-globals} has the local variables of \verb|F| as
  out-globals;
\item \label{it:fn-pre} has the postcondition of \verb|F| as precondition;
\item \label{it:fn-in-globals-more} has all variables mentioned in the
  postcondition of \verb|P| as in-globals;
\item \label{it:fn-post} has \verb|False| as postcondition.
\end{enumerate}
These conditions provide the same guarantees than in the procedure
case. Additionally, a statement returning \verb|F_Result| should be inserted as
final statement if no final return statement is present. This will lead to a
flow error if-and-only-if there is a path which can reach the exit of the
function without a return statement.

\begin{example}
  Take function \verb|F| with local variable V, which would be declared in
  pseudo-SPARK as:
\begin{verbatim}
function F (X : in T; Y : out T; Z : in out T) return Integer;
--# global in     A;
--#           out B;
--#        in out C;
--# return Res => Prop (X, Z~, B, Res);
\end{verbatim}
  Any non-terminal return statement in \verb|F| should be translated into an
  assignment to \verb|F_Result| followed by a call to local procedure
  \verb|F_Return|, declared as:
\begin{verbatim}
procedure F_Return;
--# global in     Y, B,                --  out-variables of P
--#               F_Result, X, Z_Old;  --  used in the pre
--#           out V;
--# pre  Prop (X, Z_Old, B, Res);
--# post False;
\end{verbatim}
\end{example}

\paragraph{Raise Statements} 

Raising an exception is very similar to returning from a subprogram, except the
postcondition and data-flow contract do not have to be respected in this case.

Thus, all raise statements in a subprogram \verb|S| should be translated into
calls to a local procedure \verb|S_Raise| which:
\begin{enumerate}
\item \label{it:exc-no-param} takes no parameters;
\item \label{it:exc-out-globals} has the local variables of \verb|S| as
  out-globals;
\item \label{it:exc-post} has \verb|False| as postcondition.
\end{enumerate}
(\ref{it:exc-out-globals}) ensures that no flow error will be generated due to
some (locally dead) path through this call not initializing some local
variables. (\ref{it:exc-post}) ensures that any path through this call is
considered as a dead path when proving VCs.

\subsection{From Extended SPARK to Why}

\subsubsection{Packages}

Each package in SPARK should be translated into a module in Why, with
(non-cyclic) visibility relations between packages being translated into
inclusion between modules in Why. 

\subsubsection{Variables}

All variables in Why should be given an initial value. As usual in this case,
we should declare a function returning a value of the appropriate type, which
is used to initialize the variable to any value of the type.

\subsubsection{Types}

\paragraph{Integer Types}

Ada integer types (signed and modular) should be translated into Why abstract
types. Convertion functions in Why between such an abstract type and type
\verb|int|, suitably axiomatized, will provide:
\begin{itemize}
\item a way to enforce the static type range;
\item a representation for the usual arithmetic operations;
\item a representation giving the expected order.
\end{itemize}

Ada integer subtypes (signed and modular) should be translated into Why
abstract types. Convertion functions in Why between such an abstract type and
type \verb|int| provide a way to translate all allowed type conversions in Ada.
 
\paragraph{Enumeration Types}

Ada enumeration types should be translated into Why enumeration
types. Convertions to and from Why \verb|int| type provide a way to get an
order on the enumeration, as well as positions (Ada \verb|'Pos|
attribute). Subtypes of enumerations in Ada should be translated into abstract
types in Why, with suitable conversion functions, to and from their underlying
enumeration type.

The Ada \verb|Boolean| type should be translated into Why predefined
\verb|bool| type in all cases where the features of \verb|Boolean| as an
enumeration type are not used. In the remaining cases, for example to range
over values of type \verb|Boolean|, an enumeration type
\verb|standard__boolean| should be defined in Why.

\paragraph{Array Types}

Ada array types should be translated into instances of the Why array type. 

\subsubsection{Subprograms}

Each subprogram in Extended SPARK should be translated into a corresponding
subprogram in Why.

Why frame conditions should be generated from the sets of variable paths
computed in Section~\ref{sub:data-flow}:
\begin{itemize}
\item the ``reads'' annotation should be the union of the read-set and the
  read-write-set;
\item the ``writes'' annotation should be the union of the write-set and the
  read-write-set.
\end{itemize}

\subsubsection{Calls}

Actual parameters in calls should be given in the order of the definition of
the corresponding formal parameters, as use of parameter associations in the
Ada code may give them outside of their definition order.

\subsubsection{Expressions}

\paragraph{Indexed Components} 

Simple reads and writes of array elements should be translated in the
corresponding calls to \verb|access| and \verb|update| from the prelude of Why,
which encode the usual theory of functional arrays.

\begin{example}
  The following Ada code increments array element \verb|A(I)| through a
  temporary:
\begin{verbatim}
Tmp := A(I); A(I) := Tmp + 1;
\end{verbatim}

Here is its translation in Why:
\begin{verbatim}
Tmp := access (!A, I); A := update (!A, I, Tmp + 1)
\end{verbatim}
\end{example}

Indexed components passed as ``in'' parameters should be translated like simple
reads. Indexed components passed as ``out'' or ``in out'' parameters should be
copied back after the call from a local reference passed in
parameter. Additionally, the local reference should be initialized from the
indexed component before the call for an ``in out'' parameter.

\begin{example}
  The following Ada code sets all elements of an array to 1 through repeated
  calls to \verb|One|:
\begin{verbatim}
procedure One (X : out Integer) is begin X := 1; end One;

procedure All_One (A : in out Arr) is
begin
   for J in Index loop
      One (A(J));
   end loop;
end All_One;
\end{verbatim}

Here is the translation of the loop body in Why:
\begin{verbatim}
let tmp = ref any_integer() in
One (tmp);
A := update (!A, J, tmp);
\end{verbatim}
\end{example}

\paragraph{Slices}

Since the translation to Why does not have to worry about detecting possible
aliasing, slices should be translated similarly as indexed components, by
making the appropriate copies for passing a slice in parameter to a call.

\paragraph{If and Case Expressions} 

Ada if-expressions should be translated into Why if-expressions.

Ada case-expressions over an enumerated type should be translated into Why
pattern-matching expressions. Ada case-expressions over a discrete type which
is not enumerated should be translated as a series of if-expressions.

\paragraph{Quantified Expressions} 

A quantified expression in Ada should be translated into the corresponding
logic proposition in Why (\verb|for all| in Ada becomes \verb|forall| in Why;
\verb|for some| in Ada becomes \verb|exists| in Why). All quantified
expressions should be over integer ranges at this point. They should be
translated into a quantification over the corresponding type in Why, inserting
a guard that the value respects the dynamic types for the integer range.

\begin{example}
The quantified expresssion:
\begin{verbatim}
for all X in Index range A .. B => Prop (X)
\end{verbatim}
should be translated into the following predicate:
\begin{verbatim}
forall X : Index. A <= X <= B -> Prop (X)
\end{verbatim}
\end{example}

\subsubsection{Statements}

\paragraph{If and Case Statements} 

Alternative parts starting with \verb|elsif| in Ada should be translated into
embedded if-statements in Why. 

Ada case-statements over an enumerated type should be translated into Why
pattern-matching expressions. Ada case-statements over a discrete type which is
not enumerated should be translated as a series of embedded if-expressions.

\paragraph{Return and Raise Statements} 

Both should be translated as raising exceptions in Why, with a suitable
exception handler to catch the exception for returns at the end of the
subprogram, before actually returning the value if any.

\bibliographystyle{plain}
\bibliography{alfa}

\newpage
\appendix

\section{BNF-Based Definition of ALFA}
\label{sec:bnf-alfa}

In the following, we refer to entities in the program with the name of the
corresponding non-terminal in the Ada BNF, like \bnf{subprogram\_body} for a
subprogram body. As a general rule, an entity is in ALFA only if all the
sub-entities which define it are in ALFA. As an example, a
\bnf{subprogram\_body} is defined in Ada BNF as:

\hspace*{1cm}\bnf{subprogram\_body} ::=\\
\hspace*{2cm}\lbrack\bnf{overriding\_indicator}\rbrack\\
\hspace*{2cm}\bnf{subprogram\_specification} is\\
\hspace*{3cm}  \bnf{declarative\_part}\\
\hspace*{2cm}  begin\\
\hspace*{3cm}    \bnf{handled\_sequence\_of\_statements}\\
\hspace*{2cm}  end \lbrack\bnf{designator}\rbrack;\\

Thus, a \bnf{subprogram\_body} is in ALFA only if the following are in ALFA:
\begin{itemize}
\item its \bnf{overriding\_indicator} (if present);
\item its \bnf{subprogram\_specification};
\item its \bnf{declarative\_part};
\item its \bnf{handled\_sequence\_of\_statements};
\item its \bnf{designator} (if present).
\end{itemize}

This general rule applies individually to all production rules in the BNF
which define program entities. Thus, some production rules defining a
non-terminal may be in ALFA while others for the same non-terminal are not in
ALFA.

The additional rules below further restrict which entities are in ALFA. 

\subsubsection{Lexical Elements}

ALFA only supports the Latin-1 Character Set, not the full Universal Character
Set (ISO/IEC 10646:2003) supported by Ada.

Based forms of real literals are not in ALFA.

[what about pragmas?]

\subsubsection{Declarations}

All entities related to access types are not in ALFA:
\begin{itemize}
\item the \bnf{aliased} keyword, wherever it appears;
\item \bnf{access\_type\_definition};
\item \bnf{access\_to\_object\_definition};
\item \bnf{general\_access\_modifier};
\item \bnf{access\_to\_subprogram\_definition};
\item \bnf{null\_exclusion};
\item \bnf{access\_definition}.
\end{itemize}

Notice that this rules out the type cursor in Ada 2005 containers and Ada 2012
containers, as it contains a pointer to the underlying container. The formal
containers as described in Section~\ref{sub:formal-containers} define cursors
that are in ALFA.

All entities related to fixed-point types are not in ALFA:
\begin{itemize}
\item \bnf{fixed\_point\_definition};
\item \bnf{ordinary\_fixed\_point\_definition};
\item \bnf{decimal\_fixed\_point\_definition};
\item \bnf{digits\_constraint};
\item \bnf{delta\_constraint}.
\end{itemize}

All entities related to interfaces are not in ALFA:
\begin{itemize}
\item \bnf{interface\_type\_definition};
\item \bnf{interface\_list}.
\end{itemize}

[what about \bnf{aspect\_clause} in \bnf{component\_item}?]\\

Note in particular that a \bnf{default\_expression} in a
\bnf{component\_declaration} is allowed, as well as a \bnf{variant\_part} in
a \bnf{component\_list}.

\subsubsection{Names and Expressions}

An \bnf{identifier} or a \bnf{name} which has a corresponding declaration is in
ALFA if-and-only-if its declaration is in ALFA. As a consequence, a call is in
ALFA only if the declaration of the subprogram called is in ALFA. 

[what about \bnf{aspect\_clause} in \bnf{basic\_declarative\_item}?]

All entities related to access types are not in ALFA:
\begin{itemize}
\item \bnf{explicit\_dereference};
\item \bnf{implicit\_dereference};
\item the \bnf{Access} terminal defining an \bnf{attribute\_designator};
\item the \bnf{null} terminal defining a \bnf{primary};
\item \bnf{allocator}.
\end{itemize}

Operations \bnf{and}, \bnf{or} and \bnf{xor} on Boolean arguments are not in
ALFA. The same operations on arrays of Boolean elements are in ALFA.  Only the
lazy operations \bnf{and\ then} and \bnf{or\ else} on Boolean arguments are 
in ALFA.

Quantified expressions are in ALFA only if they appear in predicate position,
that is, at the top-level of the boolean structure of an expression used in
assertions.

\begin{example}
\label{ex:alfa-quantifiers}
The following quantified expression appearing in predicate position is in ALFA:
\begin{verbatim}
procedure P with Pre => (for all X in Index => Prop (X));
\end{verbatim}

The following quantified expression appearing in term position is not in ALFA:
\begin{verbatim}
procedure P with Pre => if (for all X in Index => Prop (X)) then Y;
\end{verbatim}

The following quantified expression appearing in program position is not in
ALFA:
\begin{verbatim}
B : Boolean := (for all X in Index => Prop (X));
\end{verbatim}
\end{example}

Additionally, an expression in ALFA which appears as part of an assertion
cannot contain calls to functions which write global variables. A syntactic way
to make sure it is the case if to call only parameterized expressions, but
calls to functions which do not write global variables are also in ALFA. A
precondition, a postcondition, a type invariant or a subtype predicate count as
assertions.

\subsubsection{Statements}

\bnf{goto\_statement} is not in ALFA.

In order to be in ALFA, an exit-statement must follow the same restrictions as
in SPARK:
\begin{enumerate}
\item Exit-statements always apply to their innermost enclosing loops.
\item If an exit-statement contains a when clause then its closest-containing
  compound statement must be a loop-statement.
\item If an exit-statement does not contain a when clause then its
  closest-containing compound statement must be an if-statement without an else
  or elsif clause, whose closest-containing compound statement is a
  loop-statement.
\end{enumerate}

\begin{example}
  The following exit-statements from Ada RM are in ALFA:
\begin{verbatim}
for N in 1 .. Max_Num_Items loop
   Get_New_Item(New_Item);
   Merge_Item(New_Item, Storage_File);
   exit when New_Item = Terminal_Item;
end loop;

Main_Cycle:
   loop
      --  initial statements
      exit Main_Cycle when Found;
      --  final statements
   end loop Main_Cycle;
\end{verbatim}
\end{example}

\subsubsection{Subprograms}

A subprogram may have both a \bnf{subprogram\_declaration} and a
\bnf{subprogram\_body}. Such a subprogram is in ALFA only if both its
declaration and body are in ALFA. Additionally, the subprogram should have a
postcondition attached to its declaration. 

\bnf{extended\_return\_statement} is not in ALFA.

\subsubsection{Package Specifications and Declarations}

No special rules. In particular, renamings and the optional statements in a
package body are in ALFA.

\subsubsection{Use Clauses}

No special rules.

\subsubsection{Tasks and Synchronisation}

Most probably, the same restrictions as in RavenSPARK should be enforced for
the sequential verification results to apply to concurrent code.

\subsubsection{Program Structure and Compilation Issues}

No special rules.

\subsubsection{Exceptions}

All entities related to exception handling are not in ALFA:
\begin{itemize}
\item \bnf{exception\_handler};
\item \bnf{choice\_parameter\_specification};
\item \bnf{exception\_choice}.
\end{itemize}

However, defining exceptions and raising exceptions are in ALFA.

\subsubsection{Generic Units}

Generic formal object parameters of mode ``in out'' are not in ALFA.

Generic formal type definitions for fixed-point types are not in ALFA.

Subprograms passed in argument to a generic instantiation as formal subprograms
should write only global variables that none of the generic entity subprograms
writes, neither directy nor indirectly.

\subsubsection{Representation Issues}

[To be discussed.]

\end{document}
