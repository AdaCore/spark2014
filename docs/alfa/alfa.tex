\documentclass[fullpage]{article}

\usepackage[utf8]{inputenc}
\usepackage{listings}
\usepackage{xspace}

\title{ALFA: Annotated Language of Functions in Ada}

\usepackage{color}

\newlength\sidebar
\newlength\envrule
\newlength\envborder

\setlength\sidebar{1.5mm}
\setlength\envrule{0.4pt}
\setlength\envborder{2.5mm}

\definecolor{exampleborder}{rgb}{0,0,.7}
\definecolor{examplebg}{rgb}{.9,.9,1}
\definecolor{statementborder}{rgb}{.9,0,0}
\definecolor{statementbg}{rgb}{1,.9,.9}

\newsavebox\envbox

\newcounter{example}

\newenvironment{example}[1][EXAMPLE]{%
\par
\refstepcounter{example}%
\SpecialEnv{#1}{exampleborder}{examplebg}{}{\theexample}%
}{%
\endSpecialEnv
}

\newenvironment{statement}[1][]{% Default statement has no title
\par
\SpecialEnv{#1}{statementborder}{stateme
ntbg}{statementborder}{}%
}{%
\endSpecialEnv
}

\def\Empty{}

% #1 title (if any)
% #2 sidebar (and title bg) color
% #3 background color
% #4 border color (or null for no border)
% #5 Counter, if any.
\newenvironment{SpecialEnv}[5]{%
\par
\def\EnvSideC{#2}% To use later (in end)
\def\EnvBackgroundC{#3}%
\def\EnvFrameC{#4}%
\flushleft
\setlength\leftskip{-\sidebar}%
\addtolength\leftskip{-\envborder}%
\noindent \nobreak
% Check if title is null:
\ifx\delimiter#1\delimiter\else
% If a title is specified, then typeset it in reverse color
\colorbox{\EnvSideC}{%
\hspace{-\leftskip}% usually positive
\hspace{-\fboxsep}%
\footnotesize\sffamily\bfseries\textcolor{white}{#1}%
\hspace{\envborder}}%
\par\nobreak
\setlength\parskip{-0.2pt}% Tiny overlap to counter pixel round-off errors
\nointerlineskip
\fi
% Make side-bar
\textcolor{\EnvSideC}{\vrule width\sidebar}%
% collect body in \envbox:
\begin{lrbox}\envbox
\begin{minipage}{\hsize}%
% insert counter, if any:
\ifx\delimiter#5\delimiter\else
\theexample.\enspace
\fi
\ignorespaces
}{\par
\end{minipage}\end{lrbox}%
% body is collected. Add background color
\setlength\fboxsep\envborder
\ifx\EnvFrameC\Empty % no frame
\colorbox{\EnvBackgroundC}{\usebox\envbox}%
\else % frame
\setlength\fboxrule\envrule
\addtolength\fboxsep{-\envrule}%
\fcolorbox{\EnvFrameC}{\EnvBackgroundC}{
\usebox\envbox}%
\fi
\nobreak \hspace{-2\envborder}\null
\endflushleft
}

\newcommand{\bnf}[1]{$\mathit{#1}$}
\newcommand{\kw}[1]{\textbf{#1}}
\newcommand{\spark}{$\tau_S$}
\newcommand{\why}{$\tau_W$}
\newcommand{\code}[1]{\texttt{#1}}
\newcommand{\heap}{\code{Heap}\xspace}
\newcommand{\pred}[1]{\ensuremath{\mathit{pred}(#1)}\xspace}
\newcommand{\allwrites}{$\mathcal{W^+}$\xspace}
\newcommand{\Outallwrites}{\ensuremath{\mathcal{W}^{+out}}\xspace}
\newcommand{\Inallwrites}{\ensuremath{\mathcal{W}^{+in}}\xspace}
\newcommand{\inallwrites}[1]{\ensuremath{\mathcal{W}^{+in}(#1)}\xspace}
\newcommand{\outallwrites}[1]{\ensuremath{\mathcal{W}^{+out}(#1)}\xspace}
\newcommand{\writes}{$\mathcal{W}$\xspace}
\newcommand{\Outwrites}{\ensuremath{\mathcal{W}^{out}}\xspace}
\newcommand{\Inwrites}{\ensuremath{\mathcal{W}^{in}}\xspace}
\newcommand{\inwrites}[1]{\ensuremath{\mathcal{W}^{in}(#1)}\xspace}
\newcommand{\outwrites}[1]{\ensuremath{\mathcal{W}^{out}(#1)}\xspace}
\newcommand{\allreads}{$\mathcal{R^+}$\xspace}
\newcommand{\Outallreads}{\ensuremath{\mathcal{R}^{+out}}\xspace}
\newcommand{\Inallreads}{\ensuremath{\mathcal{R}^{+in}}\xspace}
\newcommand{\inallreads}[1]{\ensuremath{\mathcal{R}^{+in}(#1)}\xspace}
\newcommand{\outallreads}[1]{\ensuremath{\mathcal{R}^{+out}(#1)}\xspace}
\newcommand{\reads}{$\mathcal{R}$\xspace}
\newcommand{\Inreads}{\ensuremath{\mathcal{R}^{in}}\xspace}
\newcommand{\Outreads}{\ensuremath{\mathcal{R}^{out}}\xspace}
\newcommand{\inreads}[1]{\ensuremath{\mathcal{R}^{in}(#1)}\xspace}
\newcommand{\outreads}[1]{\ensuremath{\mathcal{R}^{out}(#1)}\xspace}
\newcommand{\union}{~\cup~}
\newcommand{\bigunion}{~\bigcup~}
\newcommand{\inter}{~\cap~}
\newcommand{\biginter}{~\bigcap~}
\newcommand{\minus}{~\backslash~}

\newcommand{\path}[1]{\ensuremath{\pi(#1)}\xspace}
\newcommand{\sel}{\ensuremath{\sigma}\xspace}

\begin{document}

\maketitle
\sloppy
\section{Ada 2012}

Ada 2012 is the next version of the Ada standard, expected to be finalized in
2012. It contains many extensions that facilitate the expression of
specifications, for either dynamic or static checking.

\subsection{Introduction to Ada}

\subsubsection{Module System}

Programs in Ada are structured in packages, which consist in:
\begin{itemize}
\item a package specification, which defines the external API of the package,
  consisting of types, variables and subprogram declarations whose lifetime and
  visibility scope are the same as the one of the package itself;
\item an optional package body, which defines the entities declared in the
  package specification, plus additional entities which are not part of the
  external API.
\end{itemize}

The package specification may be divided in a public part and a private part,
so that the private part is visible to the compiler, for use in separate
compilation, but not visible to the external program. Thus, a type may be
declared as having a private definition, so that the compiler knows its precise
definition (size, initialization, etc.) but the program cannot depend on it
outside the package.

Packages may be defined in any declarative section: at the outtermost level as
library-level compilation units, or inside other packages and subprogram
bodies. Subprograms too can be defined either as library-level compilation
units, or in local declarative sections. Variables and types can only be
defined in local declarative sections.

Child packages are derived from their parent package with which they have
special visibility relations, which depends on the public/private status of the
child package.

\subsubsection{Type System}

Ada enforces a strong type system by default, although it can be circumvented
by using specific conversion functions (e.g. on addresses), for example to
connect the program to external sensors/actuators.

Integers come in a variety of flavors. New integer types can be defined which
are incompatible, so that type checking uncovers mistakes where different
integer types are mixed, and a subprogram can be overloaded for different
integer types. Subtypes of existing integer types can define a range constraint
which should hold for all initialized values of this subtype, and the compiler
inserts a run-time check anywhere the constraint could be violated. Given a
subtype S of a type T, there exists a base machine type B for both T and S such
that all arithmetic operations on S and/or T are performed in B.  After the
result is computed, it is stored back in a variable of type S or T and the
corresponding range check is performed.

Enumerations define named enumerators which are the only values of this
type. Although pattern-matching is available for all discrete types, it is
mostly used for enumerations. As part of type checking, the compiler checks
that pattern-matching is complete and that no case is redundant.

Floating-point types follow the IEEE-754 standard. Fixed-point types limit the
precision of values to a fixed number of figures.

Aggregate types come in two flavors: records aggregate heterogeneous data
components, while arrays aggregate homogeneous data components. Records may
have one or more discriminants: discriminants of enumerated type are used to
provide a form of algebraic datatypes, as the discriminant is used to provide a
variant distinction between different sets of components; discriminants of
integer type are used to provide dependent types, as the discriminant is used
to give the size of some array component.

Pointer types, a.k.a. access types in Ada, follow a set of somewhat complex
rules to prevent dangling pointer references. In the following, we will mostly
ignore pointers.

The set of non-aggregate types are called elementary types.

\subsubsection{Metaprogramming}

Generic packages and subprograms define template entities to instantiate at
compile-time. They come with a set of formal parameters for types, variables,
values and subprograms. Formal parameters declare individual constraints and
relations with other formal parameters, so that once the compiler has checked a
generic entity, any instantiation with correct arguments will generate a
compilable entity.

\subsubsection{Object-Oriented Programming}

Classes are known in Ada as tagged (record) types. Objects are values of tagged
types. Methods are known as primitive operations, which can take their
dispatching type in any position. When the first operand is the dispatching
one, the usual object-dot-method syntax is allowed for calls.

An object can be considered either of type T, in which case there is no
dispatching involved when calling a primitive operation of T, or of type
T'Class, in which case all primitive operations on this object are dispatching.

Interfaces can be defined which only introduce primitive operations, not
components.

\subsubsection{Calling Conventions}

Parameters of elementary types are passed by copy.  Parameters of tagged type
are passed by reference. For most other parameters, the compiler is free to
choose between by-reference and by-copy.

Parameters have a mode, which can be by default \emph{in} or else \emph{out}
and \emph{in out}:
\begin{itemize}
\item parameters of mode \emph{in} can only be read;
\item parameters of mode \emph{out} can be both read and written: although the
  user indicates by not making them \emph{in out} that he does not intend to
  read their initial values, he should still be able to read their value after
  writing them, so the compiler cannot check in general this data-flow
  property;
\item parameters of mode \emph{in out} can be both read and written.
\end{itemize}

The Ada standard does not rule out aliasing between parameters or between
parameters and global variables accessed directly in general. However, the
standard rules out any possible different behaviors which could arise from the
compiler chosing to pass some parameters by-reference or by-copy, which would
lead to an ambiguous definition of a program. This is a theoretical rule that
the compiler does not try to enforce, although GNAT will generate warnings for
overlapping exported parameters, following the adoption of ``in out'' formal
parameters for functions in Ada 2012.

\subsection{New in Ada 2012}

The standard defines the following extensions to Ada 2005, which facilitate the
expression of specifications as subprogram contracts or type invariants:

\begin{itemize}
\item AI05-0001: bounded containers
\item AI05-0183: aspect specifications
\item AI05-0147: conditional expressions
\item AI05-0188: case expressions
\item AI05-0177: parameterized expressions
\item AI05-0176: quantified expressions
\item AI05-0146: type invariants
\item AI05-0153: subtype predicates
\end{itemize}

While the ARG website is the final authority on these extensions
(http://www.ada-auth.org/AI05-SUMMARY.HTML), we sketch in the following the
semantics and interest of each one.

Bounded containers introduce bounded versions of the existing generic
containers in Ada 2005 container library (vector, list, hashed set, ordered
set, hashed map, ordered map). The bound on the size of the container is used
to preallocate an array of the desired size, so that dynamic allocation is not
used for these containers. This opens up the possibility to perform proofs on
programs using containers, as the validity of cursors is far simpler in this
new model.

Aspect specifications allow defining contracts for subprograms, a contract
being a pair of a precondition (Pre) and a postcondition (Post). Special
contracts can also be issued for overriding, so that an overriding subprogram
can only weaken the inherited precondition and strengthen the inherited
postcondition. The standard defines a general syntax for aspects which allows
the definition of compiler-specific aspects, like the test-case aspect in GNAT
described below. In the postcondition, attribute 'Old applied on a name
indicates the value attached to this name at subprogram entry, and attribute
'Result applied to the name of the current function indicates the result
returned by this function. Preconditions and postconditions may be compiled
into executable assertions if the right option is given to the compiler (-gnata
in GNAT).  

\begin{example}
A contract for a square-root function is:
\begin{verbatim}
function Sqrt (X : Integer) return Integer with
  Pre  => X >= 0,
  Post => Sqrt'Result >= 0 and then
          Sqrt'Result ** 2 <= X and then
          (Sqrt'Result + 1) ** 2 > X;
\end{verbatim}
\end{example}

Conditional expressions and case expressions allow the use of 'if' and 'case'
in expressions, which simplifies specifications.

Parameterized expressions are a simple form of function definitions (or lambda
expressions) allowed in the specification part of a package. In particular,
simple boolean predicates on accessors to a (private) type, which are typically
the predicates one needs to write subprogram contracts, can be defined as
parameterized expressions. This has the advantage for proofs that parameterized
expressions (contrary to functions) 1) cannot have write side-effects, 2) have
a much simpler form than the usual functions, which is easier to translate into
proof predicates, 3) can be defined in a package specification which makes them
available for proof even if the package body is not.

Quantified expressions allow the expression of predicates which hold for all
elements in a range or a container, or for some element in a range or a
container.

Type invariants express invariant properties of private types, which should be
typically observed by any value of the type outside its package. However, the
standard only defines specific points at which this property is checked on a
value of the type, like entry and exit points of a subprogram. This does not
enforce by itself that the property always holds.

Subtype predicates express fine-grain properties of subtypes, like the various
enumeration values allowed for the discriminant of a record. Like for type
invariants, the standard only defines specific points at which this property is
checked.

\subsection{GNAT-Specific Extensions}

GNAT defines an aspect called Test\_Case, which applies to subprograms exactly
like the standard Pre and Post. A test-case is an aggregate with exactly three
components, all of which are compulsory:
\begin{itemize}
\item a Name component, of type string, which gives the name of the test-case;
\item a Requires component, of type boolean, which defines the entry condition
  for the test-case;
\item an Ensures component, of type boolean, which defines the exit condition
  for the test-case.
\end{itemize}

A test-case (N,Req,Ens) is a part of the specification which indicates that
under entry condition Req, the subprogram terminates with condition Ens. Thus,
the Requires component bears much resemblance with the precondition, and the
Ensures component bears much resemblance with the postcondition. Indeed, they
share the same semantic restrictions (w.r.t. 'Old and 'Result).  More than one
test-case can be defined for a subprogram. No two test-cases on a subprogram
should have the same name.

Contrary to preconditions and postconditions, test-cases are not compiled into
executable assertions by the compiler. GNAT only checks that test-cases are
properly defined. Test-cases should be used by the verification toolkit either
for unit testing or for unit proof.

For unit testing, it is sufficient to write a test procedure which exercises a
test-case to consider this test-case successful. To exercise a test-case, a
test procedure must in order:
\begin{enumerate}
\item generate suitable arguments, by calling one or more subprograms
  called \textit{fixtures};
\item check that the requires is satisfied on these arguments;
\item optionally check that the precondition is satisfied on these arguments;
\item call the subprogram tested on these arguments;
\item optionally check that the postconditions is satisfied on these arguments;
\item check that the ensures is satisfied.
\end{enumerate}

The part about checking the contract of the subprogram is optional because some
test-cases may correspond to cases beyond the normal behavior of the subprogram
described in its contract. Typically, robustness tests deal with such abnormal
behavior.

For unit proof, it is sufficient to prove that the subprogram implements a
special contract, with:
\begin{itemize}
\item the requires, optionally and'ed with the original precondition, as
  precondition;
\item the ensures as postcondition.
\end{itemize}

The original precondition should be and'ed with the requires for those
test-cases which correspond to normal behavior, and the requires should be the
only precondition for those test-cases which correspond to abnormal behavior.

\begin{example}
  A contract with test-cases for an integer square-root function is:
\begin{verbatim}
function Sqrt (X : Integer) return Integer with
  Pre  => X >= 0,
  Post => Sqrt'Result >= 0 and then
          Sqrt'Result ** 2 <= X and then
          (Sqrt'Result + 1) ** 2 > X,
  Test_Case => (Name     => "test case 1",
                Requires => X = 100, 
                Ensures  => Sqrt'Result = 10),
  Test_Case => (Name     => "test case 2",
                Requires => X < 100, 
                Ensures  => Sqrt'Result >= 0 and then 
                            Sqrt'Result < 10);
\end{verbatim}
\end{example}

\section{ALFA}

ALFA is a sub-language of Ada 2012 which identifies which subprograms are fit
for formal verification. ALFA does not work as an Ada profile, as the complete
program does not need to be in ALFA for individual subprograms to be in
ALFA. Rather, subprograms are in ALFA or not depending exclusively on their
specification and body. In particular, the location in the source where a
subprogram is defined (inside a package or another subprogram, in the public or
private part, etc.) should not influence whether it is in ALFA.

\subsection{ALFA Extensions}

ALFA considers that the pragma Assert immediately at the start of a loop
statement list have a special meaning. They define a loop-invariant for this
loop, which is used for unit proof.  

\begin{example}
  The following implementation of the integer square-root function has a
  loop-invariant:
\begin{verbatim}
function Sqrt (X : Integer) return Integer is
   Res, Two_Res, Res_Square : Integer := 0;
begin
   while Res_Square <= X loop
      pragma Assert (Res >= 0 and then
                     Two_Res = 2 * Res and then
                     Res_Square = Res * Res);
      Res_Square := Res_Square + Two_Res + 1;
      Two_Res    := Two_Res + 2;
      Res        := Res + 1;
  end loop;
  return Res - 1;
end Sqrt;
\end{verbatim}
\end{example}

The loop-invariant to be used for unit proof should be:
\begin{itemize}
\item true the first time the loop is entered;
\item provable from assuming it at some previous iteration through the loop and
  examining the effect of the loop body.
\end{itemize}

ALFA introduces a new form of containers called the formal containers, to be
used for proof of programs which manipulate containers. These are a variant of
bounded containers, with a different API meant to facilitate proofs.

\subsection{ALFA Restrictions}

In the following, we refer to entities in the program with the name of the
corresponding non-terminal in the Ada BNF, like \bnf{subprogram\_body} for a
subprogram body. As a general rule, an entity is in ALFA only if all its
sub-entities which define it are in ALFA. As an example, a
\bnf{subprogram\_body} is defined in Ada BNF as:

\begin{verbatim}
subprogram_body ::=
  [overriding_indicator]
  subprogram_specification is
    declarative_part
  begin
    handled_sequence_of_statements
  end [designator];
\end{verbatim}

Thus, a \bnf{subprogram\_body} is in ALFA only if the following are in ALFA:
\begin{itemize}
\item its \bnf{overriding\_indicator} (if present);
\item its \bnf{subprogram\_specification};
\item its \bnf{declarative\_part};
\item its \bnf{handled\_sequence\_of\_statements};
\item its \bnf{designator} (if present).
\end{itemize}

This general rule applies individually to every production rule in the BNF
which define program entities. Thus, some production rules defining a
non-terminal may be in ALFA while others for the same non-terminal are not in
ALFA.

The additional rules below further restrict which entities are in ALFA. These
rules build on the report written by Marc Sango for his internship on the
definition of a verification profile for Ada. 

\subsubsection{Lexical Elements}

No special rules.

\subsubsection{Declarations}

All entities related to access types are not in ALFA:
\begin{itemize}
\item the \bnf{aliased} keyword, wherever it appears;
\item \bnf{access\_type\_definition};
\item \bnf{access\_to\_object\_definition};
\item \bnf{general\_access\_modifier};
\item \bnf{access\_to\_subprogram\_definition};
\item \bnf{null\_exclusion};
\item \bnf{access\_definition}.
\end{itemize}

\noindent
[what about \bnf{aspect\_clause} in \bnf{component\_item}?]\\

Note in particular that a \bnf{default\_expression} in a
\bnf{component\_declaration} is allowed, as well as a \bnf{variant\_part} in
a \bnf{component\_list}.

\subsubsection{Names and Expressions}

An \bnf{identifier} or a \bnf{name} which has a corresponding declaration is in
ALFA if-and-only-if its declaration is in ALFA. As a consequence, a call is in
ALFA only if the declaration of the subprogram called is in ALFA.\\

\noindent
[what about \bnf{aspect\_clause} in \bnf{basic\_declarative\_item}?]\\

All entities related to access types are not in ALFA:
\begin{itemize}
\item \bnf{explicit\_dereference};
\item \bnf{implicit\_dereference};
\item the \bnf{Access} terminal defining an \bnf{attribute\_designator};
\item the \bnf{null} terminal defining a \bnf{primary};
\item \bnf{allocator}.
\end{itemize}

Uses of the keywords \bnf{and}, \bnf{or} and \bnf{xor} are not in ALFA. Only
the keywords for the lazy operations \bnf{and\ then} and \bnf{or\ else} are in
ALFA.\\

An expression in ALFA which appears as part of an assertion must respect
additional constraints, where an assertion is any of a pragma Assert, a
precondition or postcondition, a type invariant or a subtype predicate:
\begin{enumerate}
\item it may contain calls to parameterized expressions, but no calls to
  functions;
\item it may contain uses of formal containers, but no uses of other
  containers.
\end{enumerate}

\subsubsection{Statements}

\noindent
\bnf{goto\_statement} is not in ALFA.

Note in particular that all possible \bnf{exit\_statement} are in ALFA,
including those which exit an outter loop.

\subsubsection{Subprograms}

A subprogram may have both a \bnf{subprogram\_declaration} and a
\bnf{subprogram\_body}. Such a subprogram is in ALFA only if both its
declaration and body are in ALFA. Additionally, the subprogram should have a
postcondition attached to its declaration. The requirement that a subprogram
has a postcondition ensures that the user states desired properties to prove on
this subprogram, and that callers can rely on a precise indication of what this
subprogram does. Of course, such a postcondition can be simply 'True' in which
case the user chooses not to give any more precise information.

Note that \bnf{extended\_return\_statement} is in ALFA.

\subsubsection{Package Specifications and Declarations}

No special rules. In particular, renamings and the optional statements in a
package body are in ALFA.

\subsubsection{Use Clauses}

No special rules.

\subsubsection{Tasks and Synchronisation}

Most probably, the same restrictions as in RavenSPARK should be enforced for
the sequential verification to apply to the concurrent code.

\subsubsection{Program Structure and Compilation Issues}

No special rules.

\subsubsection{Exceptions}

All entities related to exception handling are not in ALFA:
\begin{itemize}
\item \bnf{exception\_handler};
\item \bnf{choice\_parameter\_specification};
\item \bnf{exception\_choice}.
\end{itemize}

\subsubsection{Generic Units}

No special rules.

\subsubsection{Representation Issues}

[To be discussed.]

\section{Formal Verification}

All of data-flow verification, contract verification and verification of
absence of run-time errors can be performed independently. The results of
contract verification and verification of absence of run-time errors are only
valid if data-flow is correct, which in practice means that data-flow
verification should be performed as a pre-requisite.

\subsection{Data-Flow Verification}

The \bnf{mode} part of a \bnf{parameter\_specification} has a stronger
semantics in ALFA than in Ada, similar to what is found in SPARK:
\begin{itemize}
\item a \bnf{parameter\_specification} of mode \bnf{in} must be initialized at
  subprogram entry, and there must exist at least one syntactic path through
  the subprogram which reads it;
\item a \bnf{parameter\_specification} of mode \bnf{out} must be initialized on
  all syntactic paths through the subprogram before reaching subprogram exit;
\item a \bnf{parameter\_specification} of mode \bnf{in\ out} must be initialized
  at subprogram entry, and there must exist at least one syntactic path through
  the subprogram which assigns to it before reaching subprogram exit.
\end{itemize}

Reads of variables are only allowed if all syntactic paths through the
subprogram before reaching the read do initialize the complete aggregate object
from which a part is read. This restriction is similar to the SPARK one. It
applies both to local variables and parameters.

[need expansion of what is a data-flow error, regarding useless computations]

Aliasing between parameters and globals accessed by a subprogram is forbidden,
unless all aliasing paramaters and globals are read. This restriction is
similar to the SPARK one.

\subsection{Contract Verification}

The contract of a subprogram in ALFA is verified by assuming that the
subprograms it calls do respect their contracts, which includes a possible
recursive call (in which case the assumption is made only for the recursive
call). For the purpose of contract verification, possible run-time errors are
ignored, both in the program and in the contracts. Thus, the results of
contract verification hold for all executions which do not raise a run-time
error.

\subsection{Verification of Absence of Run-Time Errors}

Verifying that a subprogram in ALFA is free from run-time errors is verified by
assuming that the subprograms it calls do respect their contracts and that they
are free from run-time errors, which includes a possible recursive call (in
which case the assumption is made only for the recursive call).

\section{Translation to SPARK / Why}

A subprogram in ALFA should be translated into an intermediate representation
in SPARK or in Why. From this representation, the Examiner or Why tools can
generated Verification Conditions (VCs) to prove using an automatic prover.  In
order to facilitate fine-grain modular proof, each subprogram should lead to
the generation of a separate unit (package in SPARK, module in Why), which can
be proved independently. Notice that the generated SPARK is not executable, and
does not match in general the structure of the source Ada program, even if this
program is written in the SPARK subset of Ada. Ideally, the generated SPARK
should correspond to a simple subset of SPARK (for example no need for
visibility rules in this subset as everything is public). Thus, it should be
easier to formalize this subset and prove the correctness of transformations or
analyzes on this subset if needed.

\subsection{Introduction to SPARK/Why}

\subsubsection{Common Basis of SPARK and Why}

SPARK and Why are both programming languages designed for deductive
verification, more than execution. They both mix coding constructs with logic
constructs whose aim is to state invariant properties of the program.

The central logic construct is the contract, which serves to fully describe the
effect of calling a subprogram, for the purpose of separate verification. Each
subprogram in SPARK/Why must be defined with a proper contract:
\begin{itemize}
\item a precondition describes constraints on the calling context;
\item a frame condition describes both the variables on which the result of the
  subprogram depends (variables read) and the variables which may be modified
  as a result of the call (variables written);
\item a postcondition describes constraints on the result of the subprogram.
\end{itemize}

Both SPARK and Why define references which are used to pass parameters to
subprogram calls. None defines pointer types. Both SPARK and Why define static
rules to check that the only parameters which may be aliased are those which
are only read.

\subsubsection{SPARK Specificities}

SPARK code, when stripped from its logic constructs, is a subset from
Ada. Thus, SPARK inherits many constructs of Ada, and it is of course
executable.

SPARK enforces strict data-flow properties, that are checked statically.

\subsubsection{Why Specificities}

Why is a functional language with imperative features, of an OCaml flavor. Why
allows mixing freely axiomatized entities and defined entities, so that Why
programs cannot be executed.

Why defines four primitive types:
\begin{itemize}
\item \emph{int} for mathematical integers;
\item \emph{bool} for Booleans;
\item \emph{real} for mathematical real numbers;
\item \emph{unit} for the type of statements.
\end{itemize}

There are no aggregate types (records and arrays) in Why. These should be
defined as abstract types and axiomatized. In this respect, Why is closer to
the translation in FDL of SPARK data structures than to SPARK.

\subsection{Generation of Annotations}

A valid SPARK or Why subprogram needs to indicate explicitly which global
variables can be read and/or written during the execution of this
subprogram. As this information is not present in the source Ada program, it
must be generated by our translation. As these reads and writes must account
for direct and indirect accesses, through any number of calls, this global
information must be retrieved by performing a global analysis on the closed set
of subprograms called directly and indirectly.

A special global variable called \heap represents all the dynamically allocated
memory plus all variables whose address is taken, so that reads and writes to
dynamically allocated memory show in SPARK or Why contracts as reads and writes
to \heap. Notice that without this \heap variable, contracts would be wrong and
break the consistency of the proof system. For example, it would be possible to
prove that Problem below always returns True, because Set would be seen as a
noop, and Get would be seen as a constant function:

\begin{verbatim}
X : access Integer;

procedure Set is
   X.all := 0;
end Set;

function Get return Integer is
begin
   return X.all;
end Get;

function Problem return Boolean is
   X1 : Integer := Get;
begin
   Set;
   return X1 = Get;
end Problem;
\end{verbatim}

Contracts (precondition and postcondition) in SPARK and Why have a
slightly different semantics than contracts in Ada 2012, because contracts in
SPARK and Why completely ignore the possibility of a run-time error being
raised while evaluating the contract. When checking for absence of run-time
errors (which can be separated from contract checking), the absence of run-time
errors in contracts should also be proved, which requires the extension of
contracts with additional conjuncts in SPARK and Why. 

\begin{example}
  The following contract in Ada:
\begin{verbatim}
function Get (A : My_Array; X : Integer) return Element
   with Pre => A (X) /= Nil_Element;
\end{verbatim}

would become in SPARK:
\begin{verbatim}
function Get (A : My_Array; X : Integer) return Element;
--# pre X in My_Array'Range and then A (X) /= Nil_Element;
\end{verbatim}
\end{example}

\subsection{General Architecture}

The first requirement is to break the mutual dependencies between packages and
between subprograms in Ada, in order to 1) achieve modular verification at the
subprogram level and 2) prevent circular dependencies between units and between
elements of units, which are either not supported or supported with
restrictions in both SPARK and Why. These mutual dependencies come from
recursion between subprograms as well as cross calls between packages (P.F
calls Q.G which calls P.H) even without recursion. To that end, each
declarative part leads to the generation of two units: one for the data+types
of this declarative part, one for the subprogram specifications of this
declarative part.

As an example, a package P defining data and subprograms should lead to the
generation of a unit P\_Data for its data+types and P\_Spec for its subprogram
specifications. Then, a subprogram P.F in ALFA should be translated into the
only subprogram in unit P\_F, which manipulates data from P\_Data and calls
subprograms from P\_Spec (including potential calls to F in P\_Spec, which
correspond to recursive calls in the source program).

As another example, a subprogram F in ALFA defining local variables and local
subprograms should lead to the generation of a unit F\_Data for its local
variables (including the variables defined in block statements inside F) and
F\_Spec for its local subprogram specifications (including the local
subprograms defined in block statements inside F). Then, F's body should be
translated into the only subprogram in unit F, which manipulates data from
F\_Data and calls subprograms from F\_Spec.

Thus, the generated units should be layered in:
\begin{enumerate}
\item $<$data$>$: units which define global data in SPARK or Why, corresponding
  to either global or local variables in Ada. These units also define types.
\item $<$spec$>$: units which define subprogram specifications in SPARK or Why,
  corresponding to all subprograms specifications and definitions in Ada. These
  units also define parameterized expressions.
\item $<$body$>$: units which define a single subprogram specification and body
  in SPARK or Why from a subprogram in ALFA.
\end{enumerate}

\subsection{Translation From Ada to Unambiguous Ada}

Ada functions may write global variables, so that the compiler choice of
evaluation order for expressions may influence the result. Whatever it is, this
choice should be the same for execution and verification, which is obtained by
lifting function calls outside of expressions in the specified order of
evaluation. This consists in a simple walk through the AST following the order
of evaluation, and introducing temporary variables for every function
call. With a left-to-right evaluation order, the statement
\begin{verbatim}
X := F(X) + G(H(Y),K(Z));
\end{verbatim}
thus translates into:
\begin{verbatim}
Tmp1 := F(X);
Tmp2 := H(Y);
Tmp3 := K(Z);
Tmp4 := G(Tmp2,Tmp3);
X := Tmp1 + Tmp4;
\end{verbatim}

Likewise, Ada arithmetic expressions may be reordered by the compiler when not
enough parenthesized, so the translation will introduce the necessary
parentheses which force the natural evaluation order given by operators
associativity. Thus the expression \verb|X + Y + Z| translates into
\verb|(X + Y) + Z|.

\subsection{Translation From Unambiguous Ada to Lowered Ada}

expanded aggregate initializations

expanded checks

\subsection{Generation of Data-Flow Annotations}
\label{sub:data-flow}

\subsubsection{Variable Paths}

A variable in ALFA may only have a scalar type (discrete or real) or an
aggregate type (record or array) such that all sub-components have themselves a
scalar type or an aggregate type with the same restriction. Given a variable
$v$ in ALFA, we associate a set of paths \path{v} to $v$. More generally, we
associate a set of paths \path{v} to an individual path $v$ (which may be a
variable) as follows:
\begin{itemize}
\item if $v$ has a scalar type, $\path{v} = \{v\}$;
\item if $v$ has a record type with components $c_i$, 
  $\path{v} = \bigunion \path{v.c_i}$;
\item if $v$ has an array type whose element type $t$ is a scalar type,
  $\path{v} = \{v\}$;
\item if $v$ has an array type whose element type $t$ is a record type with
  components $c_i$, $\path{v} = \bigunion \path{v.c_i}$. (Here, we do as if $v$
  was of type $t$ so that we can refer to its $c_i$ components.)
\end{itemize}

\begin{example}
  On the following code, the set of paths of \verb|X| is: \verb|X.B.J|,
  \verb|X.B.K|, \verb|X.A.J|, \verb|X.A.K|.

\begin{verbatim}
type Base is record 
  J : Integer;
  K : Integer range 0 .. 5;
end record;
type Base_Array is array (Boolean) of Base;
type Ext is record 
  B : Base;
  A : Base_Array;
end record;
X : Ext;
\end{verbatim}
\end{example}

A path is a sequence of selectors $\sel_1.\sel_2...\sel_n$.  If any selector
$\sel_i$ on the path refers to an array, we say that the path is an array
path. Otherwise, we say that the path is a record path. A record path refers to
a single value of scalar type, while an array path usually refers to more than
one value of scalar type.
 
\subsubsection{Problem Statement}

For the purpose of data-flow verification, we consider a package body as a
special form of subprogram, where the body of the subprogram is given by the
various initializations performed in the declarations of variables, as well as
the optional sequence of statements.

A data-flow analysis of the source Ada program should associate each subprogram
with three sets of global variables paths:
\begin{itemize}
\item the read-set of global variable paths possibly read by the subprogram;
\item the write-set of global variable paths alway written by the subprogram;
\item the read-write-set of global variable paths possibly read and written by
  the subprogram.
\end{itemize}

Note that a variable path which is always written by the subprogram before
being read, so that the initial value of the variable path is never read,
belongs to the write-set, not the read-write-set.

A special global variable called \heap represents all the dynamically allocated
memory plus all variables whose address is taken.

The algorithm should compute various maps from program points $p$ to sets of
(local and global) variable paths:
\begin{itemize}
\item \writes stores variable paths completely written on all program paths
  to $p$;
\item \allwrites stores variable paths partially written on some program
  path to $p$;
\item \reads stores variable paths whose initial value at subprogram entry has
  been partially read on some program path to $p$;
\item \allreads stores variable paths partially read on some program path to
  $p$.
\end{itemize}

As \reads depends on \writes which depends on \allreads, \allreads should be
computed first, then \writes, then \reads. \allwrites can computed at any time, as it is independent from the other maps.

As \writes, \allwrites, \reads and
\allreads take program entities as parameters, they are decomposed into
(\Inwrites,\Outwrites), (\Inallwrites,\Outallwrites), (\Inreads,\Outreads) and
(\Inallreads,\Outallreads) respectively, where the ``in'' part describes the
program point immediately before the entity and the ``out'' part describes the
program point immediately after the entity.

In the following, we give data-flow equations defining each one of \writes,
\allwrites, \reads and \allreads. These equations provide an algorithm for
computing the sets, and since the sets only grow, the algorithm necessarily
reaches a fixpoint and terminates. Note that \writes, \allwrites, \reads and
\allreads should only contain global variables when applied to subprograms
(local variables are removed in this case).

The control-flow graph of a subprogram allows defining for each statement $s$
the set of its predecessor statements \pred{s}.

\subsubsection{Computing \allwrites}

There are two sources of partial writes: assignment statements and calls. A
statement $s$ partially writes a variable path $w$ in the following cases:
\begin{itemize}
\item $s$ is an assignment statement to $w$;
\item $s$ is an assignment statement through a dereference, and $w$ is \heap;
\item $s$ contains a call to a subprogram $f$, and $w \in \outallwrites{f}$.
\end{itemize}

Given a statement $s$ partially writing variable path $w$, the following
equations define \allwrites:
\begin{eqnarray*}
\outallwrites{s} &=& \inallwrites{s} \union \{w\}\\
\inallwrites{s} &=& \bigunion \outallwrites{\pred{s}}
\end{eqnarray*}

\subsubsection{Computing \allreads}

There are two sources of reads: expressions and calls. A statement $s$
partially reads a variable path $r$ in the following cases:
\begin{itemize}
\item $s$ contains an occurence of $r$ that does not count as write, as
  described above;
\item $s$ contains a dereference that does not count as write, as described
  above, and $r$ is \heap;
\item $s$ contains a call to a subprogram $f$, and $r \in \outallreads{f}$.
\end{itemize}

Given a statement $s$ partially reading variable paths $r_i$, the following
equations define \allreads:
\begin{eqnarray*}
\inallreads{s} &=& \bigunion \outallreads{\pred{s}}\\
\outallreads{s} &=& \inallreads{s} \union \{r_i\}
\end{eqnarray*}

\subsubsection{Computing \writes}

There are two natural sources of complete writes: assignment statements and
calls. A statement $s$ completely writes a global variable path $w$ in the
following cases:
\begin{itemize}
\item $s$ is an assignment statement to $w$, and $w$ is a record path;
\item $s$ contains a call to a subprogram $f$, and $w \in \outwrites{f}$.
\end{itemize}
Additionally, if $l_1$ is a loop and $w$ a variable path such that:
\begin{enumerate}
\item there is a set of loops $l_1...l_n$ such that each $l_i$ ranges over the
  set of indexes of an array selector in $w$, and there is one loop for every
  array selector;
\item for all $i$, $l_i$ contains
  $l_{i+1}$ as a statement in its outter sequence of statements;
\item $l_1$ does not contain any exit statement (even in inner loops);
\item $w \notin \outallreads{l_1}$;
\item $l_n$ contains a statement assigning to $w$, where all indexes used for
  array accesses are exactly all the loop variables;
\end{enumerate}
then loop $l_1$ completely writes variable path $w$.

Given a statement $s$ completely writing variable paths $w$, the following
equations define \writes:
\begin{eqnarray*}
\inwrites{s} &=& \biginter \outwrites{\pred{s}}\\
\outwrites{s} &=& \inwrites{s} \union \{w\}
\end{eqnarray*}

\subsubsection{Computing \reads}

There are two sources of reads: expressions and calls. A statement $s$
partially reads the initial value of a variable path $r$ in the following
cases:
\begin{itemize}
\item $s$ contains an occurence of $r$ that does not count as write, as
  described above;
\item $s$ contains a dereference that does not count as write, as described
  above, and $r$ is \heap;
\item $s$ contains a call to a subprogram $f$, and $r \in \outreads{f}$.
\end{itemize}

Given a statement $s$ partially reading the initial value of variable paths
$r_i$, the following equations define \reads:
\begin{eqnarray*}
\inreads{s} &=& \bigunion \outreads{\pred{s}}\\
\outreads{s} &=& \inreads{s} \union (\{r_i\} \minus \inwrites{s})
\end{eqnarray*}

\subsubsection{Computing the read-set, write-set and read-write-set}

Given subprogram $f$:
\begin{itemize}
\item the initial read-set for $f$ is $\outreads{f} \minus \outallwrites{f}$;
\item the initial write-set for $f$ is $\outwrites{f} \minus \outreads{f}$;
\item the initial read-write-set for $f$ is $\outreads{f} \inter
  \outallwrites{f}$.
\end{itemize}

Then, the write-set of $f$ should be augmented with all the local variable paths
defined in $f$, whether in the declarative part of the subprogram or in a block
statement inside the body.

The goal of this computation is that:
\begin{itemize}
\item for the translation to SPARK:
\begin{itemize}
\item the global ``in'' annotation should be the read-set;
\item the global ``out'' annotation should be the write-set;
\item the global ``in out'' annotation should be the read-write-set.
\end{itemize}
\item and for the translation to Why:
\begin{itemize}
\item the ``reads'' annotation should be the union of the read-set and the
  read-write-set;
\item the ``writes'' annotation should be the union of the write-set and the
  read-write-set.
\end{itemize}
\end{itemize}

The reason for augmenting the write-set with local variable paths is that the
translation will put these local variable paths in another package, and they
will appear as global variable paths from within $f$. Defining these in the
write-set makes sure that SPARK will not consider them as initialized prior to
calling $f$, so that all reads will have to be preceded by appropriate
writes. In order to prevent warnings that some of these variable paths may not
be written on some path before returning, an initialization subprogram should
be declared, which takes all local variable paths in its ``in out''
annotation. This initialization subprogram should be called prior to returning
from the subprogram, which ensures the translated subprogram respects its
data-flow contracts for local variable paths. Only the absence of any read to
some local variable path will be detected and displayed as an error, which is
safe to do.

\subsection{Contract Verification and Verification of Absence of Run-Time Errors}

The two verifications should be performed independently, first because the user
may be interested in only one of these, second because we will aim at contract
verification through Why and verification of absence of run-time errors through
SPARK. The two are nonetheless related, which raises questions as to what is
assumed for each verification, so that performing both effectively gives
assurance that contracts are verified and there are no possible run-time
errors.

On the one hand, both types of verification depend on contracts and
intermediate assertions (loop-invariants and other assertions). For example,
loop-invariants usually contain information on the range of values, which are
both used for proving absence of run-time error and functional
properties. Thus, contracts and intermediate assertions should be assumed for
verification of absence of run-time errors. On the other hand, proving
contracts and intermediate assertions can be performed without assuming absence
of run-time errors. Indeed, what is proved in this case is that, provided there
are no run-time errors, the contracts and intermediate assertions hold.

Thus, the translation should be different for contract verification and for
verification of absence of run-time errors. 

% We denote the former as \why
% (translation to Why) and the latter as \spark (translation to SPARK).

\subsection{Translation From Unambiguous Ada to Extended SPARK}

Additional copied may be introduced (extended returns, etc.)

\subsubsection{Lexical Elements}

\subsubsection{Declarations}



Types which depend on some defining elements determined dynamically should be
translated so as to remove this dynamic element, which should be replaced by
additional dynamic run-time checks. This concerns types with discrete subtypes
with dynamic bounds, record types with discriminants, and all array
types. Although array types can have either static or dynamic bounds, it is
convenient to consider that all array types have potentially dynamic bounds, in
order to translate easily the slices of arrays passed as ``in out'' parameters
to subprograms.

A discrete subtype S of type T with dynamic bounds should be translated into a
record with three components:
\begin{itemize}
\item a component for the value, of type T;
\item a component for the first allowed value S'First;
\item a component for the last allowed value S'Last.
\end{itemize}

An array type with bounds of type S should be translated into a record with
three components:
\begin{itemize}
\item a component for the array, ranging over the first subtype of S which has
  static bounds;
\item a component for the first allowed value S'First;
\item a component for the last allowed value S'Last.
\end{itemize}

Each discriminant of a record type should be translated into a normal
component. When this discriminant controls a variant, all components of the
different cases should be added as components of the record. When this
discriminant controls the size of an array, this array component should be
translated like an array with dynamic bounds.

All anonymous types should be explicitly named. This concerns:
\begin{itemize}
\item anonymous subtypes in type definitions (for example, the bounds in an
  array);
\item anonymous subtypes in object and component declarations;
\item anonymous array types in object declarations.
\end{itemize}

Object declarations with an initialization should be split so that the
initialization is performed instead in assignment statement in the appropriate
subprogram.

Various translations remove entities not defined in SPARK and/or in Why:
\begin{itemize}
\item named numbers should be inlined;
\item the keywords \kw{abstract} and \kw{limited} in a derived type definition
  should be dropped, as they have no dynamic semantics;
\item derived scalar types should be replaced by subtypes with the same bounds;
\item character literals used in enumerations should be renamed into usual
  enumeration values.
\end{itemize}

\subsubsection{Names and Expressions}

All names should be made unique in their package. This takes care of
overloading, which is allowed in Ada but not in SPARK or Why. In particular,
operator symbols should be renamed so that operators become regular functions.

Taking a slice should lead to a temporary update of the first and last allowed
indexes of the corresponding array object, so that, after the call or the
statement, the previous bounds are restored.


\subsubsection{Statements}

\subsubsection{Subprograms}

Various translations remove entities not defined in SPARK and/or in Why:
\begin{itemize}
\item user-defined operators should be translated as regular functions;
\item default expressions for parameters should be inlined at call sites, so
  that they do not appear as default expressions anymore;
\item the \kw{overriding} keyword should be simply removed, as it only serves
  as a visual reminder of the status of the subprogram;
\item parameters in calls should be given in the order of their definition,
  even in the case where use of parameter associations in the Ada code gives
  them outside of their definition order;
\item extended returns should be translated in either:
  \begin{itemize}
  \item a simple return when there is no return statement;
  \item otherwise, the declaration of a corresponding local variable (in the
    $<$data$>$ package) and corresponding initialization code, followed by a
    simple return.
  \end{itemize}
\item null procedure declarations should be translated into procedures with a
  null body statement.
\end{itemize}

\subsubsection{Package Specifications and Declarations}

Entities declared in the package specification and those declared in the
package body should be treated identically. In particular, all entities
declared in the package body should become visible from outside the package in
the generated SPARK code. Likewise, the private part of a package specification
should be merged in its public part, so that all entities become public. As a
consequence, all private views of types should be removed [should they be
kept?].

The set of global variables defined in the package specification and package
body should become the ``own'' global variables of the corresponding $<$data$>$
package.

A subprogram should be generated for the initialization part of variable
declarations, both in the package body and in the package declaration, as well
as the optional sequence of statements in the package body. This subprogram
should have the global annotations computed as described in
section~\ref{sub:data-flow}, and additionally:
\begin{itemize}
\item there should not be any ``in'' or ``in out'' global variables, to prevent
  undue dependencies on the order of elaboration;
\item the ``out'' global variables should be the ``initializes'' global
  variables of the corresponding $<$data$>$ package.
\end{itemize}

\subsubsection{Use Clauses}

Various translations remove entities not defined in SPARK and/or in Why:
\begin{itemize}
\item use clauses (not allowed in SPARK) should be replaced by use-type clauses
  (allowed in SPARK) for all types in the package, so that operations are still
  permitted on the types exported;
\item renamings, which provide a proxy name for an entity, and which are
  strongly limited in SPARK, should be removed and each occurrence of the
  new name should be replaced by the entity being renamed;
\end{itemize}

\subsubsection{Tasks and Synchronisation}

\subsubsection{Program Structure and Compilation Issues}

\subsubsection{Exceptions}

\subsubsection{Generic Units}

\subsubsection{Representation Issues}

\subsection{End of Translation From Extended SPARK to SPARK}

Ada functions may write global variables, which is not allowed in SPARK. Thus,
functions should be translated into procedures by introducing an additional
``out'' parameter.

\subsection{End of Translation From Extended SPARK to Why}

from intermediate SPARK code in a reduced subset of SPARK

\end{document}
