\documentclass{article}

\usepackage{listings}
\usepackage{multicol}
\usepackage{graphicx}
\usepackage[dvipsnames]{xcolor}

\author{Florian Schanda}
\title{Generated Globals With Partitions\\P118-034}

%% Borrowed from the Altran LaTeX style
\lstdefinelanguage{SPARK}{
  language = [95]Ada,
  morekeywords = {pre,post,assert,assume,check,derives,hide,global,depends,inherit,from,own,initializes,main_program,assert,loop_variant,loop_invariant,increases,decreases,ghost,spark_mode,abstract_state,refined_state,refined_global,refined_depends},
  comment=[l][commentstyle]{--\ },
  texcl=true,
  showstringspaces=false
}
\lstdefinestyle{tinystyle}
   {basicstyle=\scriptsize\tt,
    keywordstyle=\color{RoyalBlue},
    commentstyle=\rmfamily\it\color{Sepia},
    captionpos=b,
    caption={},label={},
    numbers=none,
    escapeinside={(*}{*)}}
\lstset{language=SPARK}
\lstset{style=tinystyle}

\begin{document}

\maketitle

\begin{center}
  \color{red} DRAFT - WORK IN PROGRESS
\end{center}

\section{Motivation}
A number of features (tasking, tagged types, generated globals) have been
added to SPARK since the initial addition of flow analysis; which was then
evolved along with these. It is now clear that lack of design causes some
issues, in no order:
\begin{itemize}
\item Maintainability and clarity of code, in particular ability to debug
  issues. We now have more than three graphs dedicated to global analysis,
  in addition to two types of flow graphs, all combined over two phases.
  Interaction between SPARK and non-SPARK code is complex. Generation
  schemes (global, refined global, depends, and refined depends) are a
  16-possibilities case-split. Add to this that results for tasking
  analysis are mixed in here, so even if you don't need to generate globals
  we still do in many cases just to be safe.

\item Performance issues, i.e. analysis of large volumes of sparsely
  annotated code. Another side-effect is that we have high memory
  consumption as the graphs are \emph{huge}; we totally disregard the
  performance benefits abstract state is designed to bring.

\item Correctness issues: incorrect or unhelpful flow results, and cheating
  with accuracy. In particular we do not respect abstraction and
  visibility.
\end{itemize}

A new approach needs to solve all three issues. Abstract states, to some
degree, are the root of all issues; so the solution discussed here is based
on an efficient and \emph{simple} approach to this problem. Untangling
frame condition analysis from tasking analysis is an orthogonal issue.

\pagebreak
\section{Approach}

\subsection{Design criteria}
\begin{itemize}
\item Abstract state should be used as it was intended: to simplify (and
  thus speed up), via \emph{partitioning}, the analysis by reducing the
  size of graphs

\item Generated globals should not break abstraction

\item State of nested packages should not be special cased too much

\item A call-chain from package 1 to 2 back to 1 should work correctly

\item A call-chain involving entering and leaving an nested package should
  work correctly

\item Nested subprograms and packages nested in subprograms should not
  bleed their local globals to the calling environment; this step should
  ideally not be done as a post-processing step

\item There should be no fixed-point iteration

\end{itemize}

\subsection{Overall approach}
\begin{itemize}
\item A package with hidden state but without explicit abstract state will
  have a state synthesized. It may be useful in the future to allow users
  to refer to this state.

\item ALI files contain only the abstract (non-refined) view of a
  subprogram, expressed in terms of abstract state.

\item A limited form of the closure will be computed in phase 1 (to
  generate refined globals), but we only store calls to external procedures
  (internal calls have been summarized), and access to abstract state.

\item Stripping of locals for nested procedures can also be done in phase 1
  instead of phase 2. We can also annotate semi-refined computed globals
  here, but we won't need them.

\item Phase 2 will compute another closure but only in terms of abstract
  state and external calls.

\item Returning less precise results is OK, returning more precise results
  than a global annotation would allow is never OK.
\end{itemize}

\pagebreak
\section{Examples}

\subsection{Inter-package analysis}
Most precise globals possible annotated in comments, assuming the language
extension to summarize hidden state.

\begin{multicols}{2}

\begin{lstlisting}
package P is
   procedure P0;
   --  In\_Out $\rightarrow$ (P, Q)
   procedure P1;
   --  In\_Out $\rightarrow$ (P, Q)
   procedure P4;
   --  Output $\rightarrow$ P
   procedure P5;
   --  Output $\rightarrow$ P
end PACK;
\end{lstlisting}

\vfill\columnbreak

\begin{lstlisting}
package Q is
   procedure P2;
   --  In\_Out $\rightarrow$ (Q, P)
   procedure P3;
   --  Output $\rightarrow$ P
end Q;
\end{lstlisting}

\end{multicols}

\begin{multicols}{2}

\begin{lstlisting}
use Q;
package body P is
   G1 : Boolean;
   G2 : Boolean;

   procedure P0 is begin
      P1;
      G2 := False;
   end P0;
   --  In\_Out $\rightarrow$ (G1, Q)
   --  Output $\rightarrow$ G2

   procedure P1 is begin
      Q.P2;
   end P1;
   --  In\_Out $\rightarrow$ (G1, G2, Q)

   procedure P4 is begin
      P5;
   end P4;
   --  Output $\rightarrow$ (G1, G2)

   procedure P5 is begin
      G1 := True;
      G2 := True;
   end P5;
   --  Output $\rightarrow$ (G1, G2)
end P;
\end{lstlisting}

\vfill\columnbreak

\begin{lstlisting}
use P;
package body Q is
   G3 : Boolean := False;

   procedure P2 is begin
      if G3 then
         P3;
      else
         G3 := True;
      end if;
   end P2;
   --  In\_Out $\rightarrow$ (G3, P)

   procedure P3 is begin
      P.P4;
   end P3;
   --  Output $\rightarrow$ P
end Q;
\end{lstlisting}

\end{multicols}

\pagebreak
\subsubsection{Phase 1 - CFG}
During phase 1 we construct the following information for P and Q:

\begin{multicols}{2}

\begin{description}
\item[P0]
  \begin{tabular}{ll}
    Variables R & \\
    Variables W & G2 \\
    Calls       & P1 (Local) \\
    Maybe       & \\
  \end{tabular}

\item[P1]
  \begin{tabular}{ll}
    Variables R & \\
    Variables W & \\
    Calls       & P2 (Remote) \\
    Maybe       & \\
  \end{tabular}

\item[P2]
  \begin{tabular}{ll}
    Variables R & G3 \\
    Variables W & G3 \\
    Calls       & \\
    Maybe       & P3 (Local) \\
  \end{tabular}

\item[P3]
  \begin{tabular}{ll}
    Variables R & \\
    Variables W & \\
    Calls       & P4 (Remote) \\
    Maybe       & \\
  \end{tabular}

\item[P4]
  \begin{tabular}{ll}
    Variables R & \\
    Variables W & \\
    Calls       & P5 (Local) \\
    Maybe       & \\
  \end{tabular}

\item[P5]
  \begin{tabular}{ll}
    Variables R & \\
    Variables W & G1, G2 \\
    Calls       & \\
    Maybe       & \\
  \end{tabular}
\end{description}

\end{multicols}

\noindent
We then build the following graphs in phase 1.

\begin{figure}[h]
  \begin{center}
    \includegraphics[scale=0.4]{ex1_phase1_p}
  \end{center}
  \caption{Phase 1 for package P}
\end{figure}

\begin{figure}[h]
  \begin{center}
    \includegraphics[scale=0.4]{ex1_phase1_q}
  \end{center}
  \caption{Phase 1 for package Q}
\end{figure}

\subsubsection{Phase 1 - Closure}
For each package we take the closure to eliminate any local calls; so for
package P:

\begin{description}
\item[P0]
  \begin{multicols}{2}
    Raw:
    \begin{tabular}{ll}
      Variables R & \\
      Variables W & G2 \\
      Calls       & P2 \\
      Maybe       & \\
    \end{tabular}

    Projected:
    \begin{tabular}{ll}
      Variables R & P \\
      Variables W & P \\
      Calls       & P2 \\
      Maybe       & \\
    \end{tabular}
  \end{multicols}

\item[P1]
  \begin{multicols}{2}
    Raw:
    \begin{tabular}{ll}
      Variables R & \\
      Variables W & \\
      Calls       & P2 \\
      Maybe       & \\
    \end{tabular}

    Projected:
    \begin{tabular}{ll}
      Variables R & \\
      Variables W & \\
      Calls       & P2 \\
      Maybe       & \\
    \end{tabular}
  \end{multicols}

\item[P4]
  \begin{multicols}{2}
    Raw:
    \begin{tabular}{ll}
      Variables R & \\
      Variables W & G1, G2 \\
      Calls       & \\
      Maybe       & \\
    \end{tabular}

    Projected:
    \begin{tabular}{ll}
      Variables R & \\
      Variables W & P \\
      Calls       & \\
      Maybe       & \\
    \end{tabular}
  \end{multicols}

\item[P5]
  \begin{multicols}{2}
    Raw:
    \begin{tabular}{ll}
      Variables R & \\
      Variables W & G1, G2 \\
      Calls       & P2 \\
      Maybe       & \\
    \end{tabular}

    Projected:
    \begin{tabular}{ll}
      Variables R & \\
      Variables W & P \\
      Calls       & \\
      Maybe       & \\
    \end{tabular}
  \end{multicols}

\end{description}

\noindent
And similarly for package Q:

\begin{description}
\item[P2]
  \begin{multicols}{2}
    Raw:
    \begin{tabular}{ll}
      Variables R & G3 \\
      Variables W & G3 \\
      Calls       & \\
      Maybe       & P4 \\
    \end{tabular}

    Projected:
    \begin{tabular}{ll}
      Variables R & Q \\
      Variables W & Q \\
      Calls       & \\
      Maybe       & P4 \\
    \end{tabular}
  \end{multicols}

\item[P3]
  \begin{multicols}{2}
    Raw:
    \begin{tabular}{ll}
      Variables R & \\
      Variables W & \\
      Calls       & P4 \\
      Maybe       & \\
    \end{tabular}

    Projected:
    \begin{tabular}{ll}
      Variables R & \\
      Variables W & \\
      Calls       & P4 \\
      Maybe       & \\
    \end{tabular}
  \end{multicols}

\end{description}

\pagebreak
\subsubsection{Phase 2}
We can now glue everything together to get global contracts formulated in
terms of abstract state.

\begin{figure}[h]
  \begin{center}
    \includegraphics[scale=0.4]{ex1_phase2}
  \end{center}
  \caption{Phase 2}
\end{figure}

\noindent
The transitive closure of this will match the ideal (most-precise possible
human-written global) contracts:

\begin{table}[h]
  \begin{center}
    \begin{tabular}{lll}
         & Reads & Writes \\
      P0 & P, Q  & P, Q   \\
      P1 & P, Q  & P, Q   \\
      P2 & P, Q  & P, Q   \\
      P3 &       & P      \\
      P4 &       & P      \\
      P5 &       & P      \\
    \end{tabular}
  \end{center}
  \caption{Transitive closure for phase 2}
\end{table}

\noindent
When performing flow analysis in phase 2 for either package we can replace
the abstract state with an appropriate refined view (which may contain
variables we do not use).

\newpage
\subsection{Intra-package analysis}

\begin{multicols}{2}
\begin{lstlisting}
package P with
   Abstract_State => State
is
   procedure P0;

   procedure P2;

   procedure P3;
end P;

package body P with
   Refined_State =>
     (State => G0, G1, G2, Nested)
is
   G0 : Boolean;
   G1 : Boolean;
   G2 : Boolean;

   package Q with
      Abstract_State => Nested
   is
      procedure P1;

      procedure P4;

      procedure P5;
   end Q;
\end{lstlisting}
\vfill\columnbreak
\begin{lstlisting}
   package body Q with
      Refined_State =>
        (Nested => (G3, G4))
   is
      G3 : Boolean := False;
      G4 : Boolean := G1;
   begin
      procedure P1 is begin
         P2;
         G3 := False;
      end P1;

      procedure P4 is begin
         P5;
         G3 := not G3;
      end P4;

      procedure P5 is begin
         G3 := G1;
         G4 := G2;
      end P5;
   end Q;

   procedure P0 is begin
      Q.P1;
      G1 := False;
   end P0;

   procedure P2 is begin
      P3;
      G0 := False;
      G2 := False;
   end P2;

   procedure P3 is begin
      Q.P4;
   end P3;
end P;
\end{lstlisting}
\end{multicols}

\subsubsection{Phase 1 - CFG}
Note calls from P to Q are remote, but calls from Q to P are local.
\begin{multicols}{2}
  \begin{description}
  \item[P0]
    \begin{tabular}{ll}
      Variables R & \\
      Variables W & G1 \\
      Calls       & P1 (Remote) \\
      Maybe       & \\
    \end{tabular}

  \item[P1]
    \begin{tabular}{ll}
      Variables R & \\
      Variables W & G3 \\
      Calls       & P2 (Remote) \\
      Maybe       & \\
    \end{tabular}

  \item[P2]
    \begin{tabular}{ll}
      Variables R & \\
      Variables W & G0, G2 \\
      Calls       & P3 (Local) \\
      Maybe       & \\
    \end{tabular}

  \item[P3]
    \begin{tabular}{ll}
      Variables R & \\
      Variables W & \\
      Calls       & P4 (Remote) \\
      Maybe       & \\
    \end{tabular}

  \item[P4]
    \begin{tabular}{ll}
      Variables R & G3 \\
      Variables W & G3 \\
      Calls       & P5 (Local) \\
      Maybe       & \\
    \end{tabular}

  \item[P5]
    \begin{tabular}{ll}
      Variables R & G1, G2 \\
      Variables W & G3, G4 \\
      Calls       & \\
      Maybe       & \\
    \end{tabular}
  \end{description}
\end{multicols}

\noindent
The graphs produces for P and Q:

\begin{figure}[h]
  \begin{center}
    \includegraphics[scale=0.4]{ex2_phase1_p}
  \end{center}
  \caption{Phase 1 for P}
\end{figure}

\begin{figure}[h]
  \begin{center}
    \includegraphics[scale=0.4]{ex2_phase1_q}
  \end{center}
  \caption{Phase 1 for Q}
\end{figure}

\subsubsection{Phase 1 - Closure}
For each package we take the closure to eliminate any local calls; so for
package P:

\begin{description}
\item[P0]
  \begin{multicols}{2}
    Raw:
    \begin{tabular}{ll}
      Variables R & \\
      Variables W & G1 \\
      Calls       & P1 \\
      Maybe       & \\
    \end{tabular}

    Projected:
    \begin{tabular}{ll}
      Variables R & State \\
      Variables W & State \\
      Calls       & P1 \\
      Maybe       & \\
    \end{tabular}
  \end{multicols}

\item[P2]
  \begin{multicols}{2}
    Raw:
    \begin{tabular}{ll}
      Variables R & \\
      Variables W & G0, G2 \\
      Calls       & P4 \\
      Maybe       & \\
    \end{tabular}

    Projected:
    \begin{tabular}{ll}
      Variables R & State \\
      Variables W & State \\
      Calls       & P4 \\
      Maybe       & \\
    \end{tabular}
  \end{multicols}

\item[P3]
  \begin{multicols}{2}
    Raw:
    \begin{tabular}{ll}
      Variables R & \\
      Variables W & \\
      Calls       & P4 \\
      Maybe       & \\
    \end{tabular}

    Projected:
    \begin{tabular}{ll}
      Variables R & \\
      Variables W & \\
      Calls       & P4 \\
      Maybe       & \\
    \end{tabular}
  \end{multicols}
\end{description}

\noindent
And similarly for Q. For state that is contained in another state, we use
the containing state to denote ``everything except the inner state''.

\begin{description}
\item[P1]
  \begin{multicols}{2}
    Raw:
    \begin{tabular}{ll}
      Variables R & \\
      Variables W & G3 \\
      Calls       & P2 \\
      Maybe       & \\
    \end{tabular}

    Projected:
    \begin{tabular}{ll}
      Variables R & Nested \\
      Variables W & Nested \\
      Calls       & P2 \\
      Maybe       & \\
    \end{tabular}
  \end{multicols}

\item[P4]
  \begin{multicols}{2}
    Raw:
    \begin{tabular}{ll}
      Variables R & G1, G2, G3 \\
      Variables W & G3, G4 \\
      Calls       & \\
      Maybe       & \\
    \end{tabular}

    Projected:
    \begin{tabular}{ll}
      Variables R & State, Nested \\
      Variables W & Nested \\
      Calls       & \\
      Maybe       & \\
    \end{tabular}
  \end{multicols}

\item[P5]
  \begin{multicols}{2}
    Raw:
    \begin{tabular}{ll}
      Variables R & G1, G2 \\
      Variables W & G3, G4 \\
      Calls       & \\
      Maybe       & \\
    \end{tabular}

    Projected:
    \begin{tabular}{ll}
      Variables R & State \\
      Variables W & Nested \\
      Calls       & \\
      Maybe       & \\
    \end{tabular}
  \end{multicols}
\end{description}

\subsubsection{Phase 2}
\begin{figure}[h]
  \begin{center}
    \includegraphics[scale=0.4]{ex2_phase2}
  \end{center}
  \caption{Phase 2}
\end{figure}

\noindent
The transitive closure of this will be an over-approximation of the most
precise human-written globals:

\begin{table}[h]
  \begin{center}
    \begin{multicols}{2}
      \begin{tabular}{lll}
           & Reads & Writes \\
        P0 & S, N  & S, N   \\
        P1 & S, N  & S, N   \\
        P2 & S, N  & S, N   \\
        P3 & S, N  & N      \\
        P4 & S, N  & N      \\
        P5 & S     & N      \\
      \end{tabular}

      \begin{tabular}{lll}
           & Reads     & Writes    \\
        P0 & G1, G2    & S, N      \\
        P1 & G1, G2    & G0, G2, N \\
        P2 & G1, G2    & G0, G2, N \\
        P3 & G1, G2    & N         \\
        P4 & G1, G2    & N         \\
        P5 & G1, G2    & N         \\
      \end{tabular}
    \end{multicols}
  \end{center}
  \caption{Transitive closure for phase 2, and ideal annotations.}
\end{table}


\end{document}
