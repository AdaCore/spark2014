\documentclass[runningheads,a4paper]{llncs}

\usepackage[utf8]{inputenc}

\usepackage{amssymb}
\usepackage{amsmath}
\setcounter{tocdepth}{3}
\usepackage{graphicx}
\usepackage{xspace}
\usepackage{mathpartir}
\usepackage{xcolor}
\usepackage{listings}

\usepackage{url}
\newcommand{\resp}{resp.\xspace}
\newcommand{\etal}{\textit{et al.}\xspace}
\newcommand{\etc}{\textit{etc.}\xspace}
\newcommand{\ie}{\textit{i.e.,}\xspace}
\newcommand{\vs}{\textit{vs.}\xspace}
\newcommand{\eg}{\textit{e.g.,}\xspace}
\newcommand{\wrt}{w.r.t.\xspace}
\newcommand{\keywords}[1]{\par\addvspace\baselineskip
\noindent\keywordname\enspace\ignorespaces#1}

\newcommand{\envL}{\ensuremath{\Gamma_L}\xspace}
\newcommand{\envC}{\ensuremath{\Gamma_C}\xspace}
\newcommand{\eval}{\envL, \envC \vdash}
\newcommand{\where}{\mathit{where}~}
\newcommand{\other}{\mathit{else}}
\newcommand{\somespace}{\vspace{10pt}}
\newcommand{\Fv}{\ensuremath{\mathit{F}_{\varnothing}}\xspace}
\newcommand{\Fc}{\ensuremath{\mathit{Fc}}\xspace}
\newcommand{\FcInv}{\ensuremath{\mathit{Fc}^{-1}}\xspace}
\newcommand{\Fe}{\ensuremath{\mathit{Fe}}\xspace}
\newcommand{\typL}{\ensuremath{\{\Len,\Fc,\Fe\}}\xspace}
\newcommand{\ImFcPrime}{\ensuremath{\mathit{Im}(\Fc')}\xspace}
\newcommand{\ImFc}{\ensuremath{\mathit{Im}(\Fc)}\xspace}
\newcommand{\Img}{\ensuremath{\mathit{Im}}\xspace}
\newcommand{\TypL}{\ensuremath{\mathbb{L}}\xspace}
\newcommand{\TypE}{\ensuremath{\mathbb{E}}\xspace}
\newcommand{\TypD}{\ensuremath{\mathbb{D}}\xspace}
\newcommand{\Nat}{\ensuremath{\mathbb{N}}\xspace}
\newcommand{\Int}{\ensuremath{\mathbb{Z}}\xspace}
\newcommand{\Bool}{\ensuremath{\mathbb{B}}\xspace}
\newcommand{\Elt}{\ensuremath{\mathit{Elt}}\xspace}
\newcommand{\Insert}{\ensuremath{\mathit{Insert}}\xspace}
\newcommand{\Delete}{\ensuremath{\mathit{Delete}}\xspace}
\newcommand{\Replace}{\ensuremath{\mathit{Replace\_Element}}\xspace}
\newcommand{\Copy}{\ensuremath{\mathit{Copy}}\xspace}
\newcommand{\Next}{\ensuremath{\mathit{Next}}\xspace}
\newcommand{\Previous}{\ensuremath{\mathit{Previous}}\xspace}
\newcommand{\First}{\ensuremath{\mathit{First}}\xspace}
\newcommand{\Last}{\ensuremath{\mathit{Last}}\xspace}
\newcommand{\Left}{\ensuremath{\mathit{Left}}\xspace}
\newcommand{\Right}{\ensuremath{\mathit{Right}}\xspace}
\newcommand{\Element}{\ensuremath{\mathit{Element}}\xspace}
\newcommand{\HasElement}{\ensuremath{\mathit{Has\_Element}}\xspace}
\newcommand{\StrictEqual}{\ensuremath{\mathit{Strict\_Equal}}\xspace}
\newcommand{\Empty}{\ensuremath{\mathit{Empty}}\xspace}
\newcommand{\NoElement}{\ensuremath{\mathit{No\_Element}}\xspace}
\newcommand{\Length}{\ensuremath{\mathit{Length}}\xspace}
\newcommand{\Len}{\ensuremath{\mathit{Len}}\xspace}
\newcommand{\Pre}{\ensuremath{\mathit{Pre}}\xspace}
\newcommand{\Post}{\ensuremath{\mathit{Post}}\xspace}

\newcommand{\beforesec}{\vspace{-0.2cm}}
\newcommand{\aftersec}{\vspace{-0.2cm}}
\newcommand{\beforesub}{\vspace{-0.2cm}}
\newcommand{\aftersub}{\vspace{-0.1cm}}

\newcommand{\yannick}[1]{{\color{red}#1}}

\sloppy

\begin{document}

\mainmatter  % start of an individual contribution

% first the title is needed
\title{Correct Code Containing Containers}

% a short form should be given in case it is too long for the running head
\titlerunning{Correct Code Containing Containers}

% the name(s) of the author(s) follow(s) next
%
% NB: Chinese authors should write their first names(s) in front of
% their surnames. This ensures that the names appear correctly in
% the running heads and the author index.
%
\author{Claire Dross\inst{1}\and Jean-Christophe Filli\^ atre\inst{2}\and Yannick Moy\inst{1}}

% the affiliations are given next; don't give your e-mail address
% unless you accept that it will be published
\institute{AdaCore, 46 rue d'Amsterdam, F-75009 Paris (France)\\
\texttt{\{dross,moy\}@adacore.com}
\and 
CNRS, INRIA Saclay Île-de-France, Université Paris-Sud 11\\
\texttt{filliatr@lri.fr}}

\maketitle

\begin{abstract}

  For critical software development, containers such as lists, vectors, sets or
  maps are an attractive alternative to ad-hoc data structures based on
  pointers. 
  As standards like DO-178C put formal verification and testing on an equal
  footing, it is important to give users the ability to apply both to the
  verification of code using containers.
  In this paper,
  we present a definition of containers whose aim is to facilitate their
  use in certified software, using modern proof technology and novel
  specification languages. Correct usage of containers and user-provided
  correctness properties can be checked either by execution during testing
  or by formal proof with an automatic prover.
  We present a formal semantics for containers and an axiomatization of this 
  semantics targeted at automatic provers. We have proved in Coq that the
  formal semantics is consistent and that the axiomatization thereof is correct.

  \keywords{containers, iterators, verification by contracts, annotations,
    axiomatization, API usage verification, SMT, automatic provers}
\end{abstract}

\vspace{-0.5cm}
\section{Introduction}
\aftersec

Containers
%REVIEW 1
%\footnote{We use the word container as a synonym for a homogeneous
%  collection, not to mean an aggregate containing heterogeneous objects as in
%  component architectures.}
are generic data structures offering a high-level
view of collections of objects, while guaranteeing fast access to their content
to retrieve or modify it. The most common containers are lists, vectors, sets
and maps, which are usually defined in the standard library of languages, like
in C++ STL, Ada Standard Libraries or Java JCL, and sometimes even as language
elements, like sets in SETL~\cite{schwartz:1986} or maps in Perl.
In critical software where
verification objectives severely restrict the use of pointers, containers offer
an attractive alternative to pointer-intensive data structures. 
Containers offer both a better defense against errors than low-level code 
manipulating pointers, and a rich high-level API to express properties over 
data. 
This is
particularly evident when the implementation of containers themselves obeys the
coding standards of critical software, with no dynamic allocation and few
pointers, as is the case for the bounded containers defined in the proposed Ada
2012 standard~\cite{ada2012}.

Standards for critical software development define comprehensive verification
objectives to guarantee the high levels of dependability we expect of
life-critical and mission-critical software. All requirements must be shown to
be satisfied by the software, which is a costly activity. In particular,
verification of low-level requirements is usually demonstrated by developing
unit tests, from which high levels of confidence are only obtained at a high
cost. This is the driving force for the adoption of formal verification on an
equal footing with testing to satisfy verification objectives. The upcoming
DO-178C avionics standard states: \textit{Formal methods [..] might
  be the primary source of evidence for the satisfaction of many of the
  objectives concerned with development and verification}.

Although there are some areas where formal verification can be applied
independently~\cite{souyris:2009:fm}, most areas where testing is the main
source of evidence today would benefit from an integration of formal
verification with existing testing practice. At the simplest, this
combination should be provably sound and it should guarantee a coverage of
atomic verifications through formal verification and testing. This is the goal
of project Hi-Lite~\cite{hilite}, a project aiming at combined unit testing and unit proof of
C and Ada programs.
% The core feature enabling this combination is a common
% language of specification (different in C and Ada) that can be both executed
% and formally analyzed. In this paper, we consider exclusively formal deductive
% verification: 1) a program is annotated with preconditions, postconditions, and
% loop invariants; 2) a tool generates verification conditions (VCs), which are
% formulas that must hold for the program to respect its specification; 3) an
% automatic prover discharges these VCs.

In the context of project Hi-Lite, this paper deals with the definition of
suitable containers in Ada, based on Ada standard containers, whose properties
can be both tested dynamically and proved automatically. Properties
over containers offer a high level description of the code,
suitable for expressing partial correctness in the form of code contracts. Therefore, we
are not only interested in correct usage of container APIs, but also in partial
correctness of functional properties of interest to users.
% Common properties
% include invariants over the container content (\textit{\eg all elements satisfy
%   property P}), presence/absence of designated elements, value of maximal
% elements, aggregated value of the container (\textit{\eg the total power
%   consumed by activated devices}).

Before they can be used in the context of avionics (or similar) safety-critical
software, the new library of \textit{formal} containers will need to be certified
and the verification tools we present will have to undergo
qualification~\cite{standard:do-178b}. We present in this paper: (a) a formal
proof of correctness of an implementation of the new library in Coq~\cite{coq}, a
well-known formal programming language; and (b) a formal proof of the new
library properties used in our verification tools, expressed in the Why
language~\cite{why} for formal verification. Thus, this work can be seen as a contribution
to the argument-based approach to certification~\cite{rushby:2010:sss}.

In Section~\ref{sec:formal-containers}, we detail the changes that we introduce
in formal containers \wrt containers as defined in language standards. In the following
sections, we describe in detail formal containers for doubly linked lists, and
we sketch formal containers for vectors, sets and maps: formal semantics in
Section~\ref{sec:description}, formal specification in the Why language in
Section~\ref{sec:axiomatization}, formal proof of correctness in Coq in
Section~\ref{sec:validation}. We finally give a survey of related works in
Section~\ref{sec:related-works}.

%REVIEW 3
A longer version of this article with more details and proofs is available on
the web~\cite{dross:2011:fc-long}.  A web page gives an introduction to the
code for containers' implementation and proofs~\cite{dross:2011:fc-code}, and
instructions to anonymously access the git repository where all the source code
is stored.

% This paper focuses on containers, ie complex mutable data-structures such as
% lists, vectors, sets or maps. It describes the way these containers,
% described in the Ada API, have been adapted to be usable for formal proof and
% the set of axioms that were needed to get SMT automatic provers to carry
% simple proof over programs using them.

\beforesec
\section{Formal Containers}
\label{sec:formal-containers}
\aftersec

We will use the following Ada code as a running example throughout the
section. Procedure \verb|Map_F|
%REVIEW 2
\footnote{Note that \texttt{Map\_F} is client code, not part of the API.}
%
modifies a list in place, by replacing each
element initially contained in the list by its image through function \verb|F|.
\lstset{language=Ada, basicstyle=\small\tt, keywordstyle=\normalfont\textbf}
\begin{lstlisting}
procedure Map_F (L : in out List) is
   Current : Cursor := First (L);
begin
   while Current /= No_Element loop
      Replace_Element (L, Current, F (Element (Current)));
      Next (Current);
   end loop;
end Map_F;
\end{lstlisting}

\beforesub
\subsection{Contracts in Ada 2012}
\aftersub

The forthcoming version of the Ada standard, called Ada 2012~\cite{ada2012}, offers a variety of new
features to express properties of programs. New checks are defined as
\emph{aspects} of program entities, for which the standard defines precisely
the various points at which the check is performed during execution. The most
prominent of these new checks are the \emph{Pre} and \emph{Post} aspects which
define respectively the precondition and postcondition of a subprogram. These
are defined as Boolean expressions over program variables and
functions. 
\ifdefined\longversion
Additionally, the expression in a postcondition can refer to the
value returned by a function \verb|F| as \verb|F'Result|, and to the value in
the pre-state (before the call) of any variable \verb|V| as
\verb|V'Old|.
\fi
% Finally, expressing such properties is greatly facilitated by the
% extended language of expressions in Ada 2012, which comprises if-expressions,
% case-expressions, quantified-expressions and expression-functions, which all
% together define a small Turing-complete purely functional language.
For example, a simple contract on function \verb|Map_F| could specify that its
parameter list should not be empty, and that the call does not modify its
length:
\begin{lstlisting}
procedure Map_F (L : in out List) with 
  Pre  => Length (L) /= 0,
  Post => Length (L) = Length (L'Old);
\end{lstlisting}

Notice that, in the precondition, \verb|L| refers to the list in the pre-state
while, in the postcondition, \verb|L| refers to the list in the post-state, hence
the need to refer to the special attribute \verb|L'Old| in the postcondition, 
which designates the
value of \verb|L| before the call.
\footnote{More
  precisely, attribute \texttt{'Old} can be applied to all \emph{names}
(as defined in Ada standard), so
  that we could use \texttt{Length (L)'Old} instead of \texttt{Length
    (L'Old)}.}  
\ifdefined\longversion
When compiled with checks on, any
failure from the caller to respect the precondition of \verb|Map_F| or from
\verb|Map_F| to respect its postcondition will be detected at run-time and
reported as an error.
\fi
The execution model for these aspects is simply to insert assertions at
appropriate locations, which raise exceptions when violated. For each variable
\verb|V| whose pre-state value may be read in the postcondition (\verb|V'Old|),
the compiler inserts a shallow copy of the variable's value at the beginning of the
subprogram body. It is this copy which is read to give the value of
\verb|V'Old|. 

Expressing properties in contracts is greatly facilitated by the use of
if-expressions, case-expressions, quantified-expressions and expression-functions, 
all defined in Ada 2012. The main objective of these \emph{logic} features is
verification by testing, based on their executable semantics.
In particular, quantified-expressions are always
expressed over finite ranges or (obviously finite) containers, with a loop
through the range or container as execution model:
\verb|for all J in 0 .. 10 => P (J)| is true if-and-only-if the subprogram
\verb|P| returns \verb|True| for every argument, starting from 0 up to 10.

\beforesub
\subsection{Ada Standard Containers}
\aftersub

Like in many other languages, Ada standard containers define two mutually
dependent data structures: containers proper which hold the data, and iterators
(or cursors) which provide handles to individual pieces of data. In function
\verb|Map_F|, the container has type \verb|List| and the iterator has type
\verb|Cursor|, which are both defined by the standard Ada lists. A cursor is
implicitly associated with a container (implemented as a pointer field in the
cursor structure), in which it designates an element. An important feature of Ada
containers is that cursors remain valid as long as the container
exists and the element referenced is not deleted, like
many iterators in other languages (for example those in C++ STL and Java JCL).
This allows modifying a container while iterating over its content with
cursors, without risk of invalidating these cursors.

\ifdefined\longversion
As seen in the example, a cursor \verb|Cu| can be used for three different purposes:
accessing a designated element of the container through \verb|Element (Cu)|;
iterating over the content of a container \verb|L|,
by starting at the first cursor in the container, given by \verb|First (L)|,
and repeatedly reaching to the next cursor, given by \verb|Next (Cu)|;
modifying a container like in the call to \verb|Replace_Element| in the example.
A designated cursor called \verb|No_Element| is the default invalid cursor (it
designates no element).
\fi

\beforesub
\subsection{API Modification: Independent Cursors}
\label{sec:independent-cursor}
\aftersub

\subsubsection{Problem.}

A useful postcondition for \verb|Map_F| is to state how elements are
modified by the call. All cursors are preserved through replacement of an
element in a list. Thus, for every cursor \verb|Cu| that designates an element
\verb|E| in \verb|L| before the call, \verb|Cu| designates \verb|F (E)| in
\verb|L| after the call. It seems like we could express it with a
quantified-expression:
\begin{lstlisting}
procedure Map_F (L : in out List) with 
  Post => (for all Cu in L => Element (Cu) = ???);
\end{lstlisting}

The expression denoted by \verb|???| should designate the value obtained by calling
\verb|F (Element (Cu))| in the pre-state, which could be intuitively written as
\verb|F (Element (Cu))'Old|. Unfortunately, this expression is not valid
because it refers to the value in the pre-state of \verb|Cu|, which is not
defined in the pre-state
%REVIEW 2
since it is quantified.
%
As a side note, notice that, even for a cursor
\verb|Cu| defined outside of the \verb|Map_F| function, \verb|Cu'Old| would be the same as
\verb|Cu| in our example, because the semantics of attribute \verb|'Old| is to
perform a shallow copy, so it does not copy the implicit container in a cursor.

% In order to express such a quantified property over containers, we
% need to be able to refer to the same value for the cursor in the post-state and
% in the pre-state.

\subsubsection{Approach.}
To solve the previous two problems,
we break the implicit link between cursors and containers, so that the same cursor can be used both in
the pre-state and in the post-state. Then, the previous postcondition can be
expressed easily:
\begin{lstlisting}
procedure Map_F (L : in out List) with 
  Post => 
   (for all Cu in L =>
      Element (L, Cu) = F (Element (L'Old, Cu)));
\end{lstlisting}

Notice that we passed an additional argument to function \verb|Element|, to
indicate the container from which the element at cursor \verb|Cu| should be
retrieved. This is true for every function in the API of containers which
previously accessed implicitly the container through a cursor, such as 
\verb|Next| in the example of Section~\ref{sec:formal-containers}.

% so that
% procedure \verb|Map_F| needs to be rewritten as:
% \begin{verbatim}
% procedure Map_F (L : in out List) is
%    Current : Cursor := First (L);
% begin
%    while Current /= No_Element loop
%       Replace_Element (L, Current, F (Element (L, Current)));
%       Next (L, Current);
%    end loop;
% end Map_F;
% \end{verbatim}

This is the only modification to the API of containers that we introduce. The
alternative of using existing containers both greatly complicates the execution
model and the formal verification.

Using existing containers would require a different semantics for the \verb|'Old|
attribute, which would reach to the complete pre-state including the stack and
heap, similar to the work by Kosiuczenko on Java
programs~\cite{kosiuczenko:2010:mpc} that builds a complete history of updates
alongside execution. Our solution has the benefits of sticking to the standard
semantics for \verb|'Old| in Ada 2012, leading to a simple and efficient execution model.

The semantics of standard cursors also leads to more complex verification
conditions to check the correct use of containers' API: each access through
cursor \verb|Cu| to container \verb|Co| is valid only if 1) \verb|Co| is alive,
which can be hard to know if \verb|Co| is implicit, and 2) \verb|Cu| is
associated to container \verb|Co|, which amounts to deciding whether \verb|Co|
is the same as the implicit container in \verb|Cu|. With the semantics of
formal containers, both verification conditions above disappear.

\ifdefined\longversion
As seen from the example, the impact on user code is minimal, easy to automate,
leading to more readable programs. Assuming the constraint that a cursor is
never used to index two different containers, which is almost always true in
code (but not in annotations, as shown in our example), it will be possible to
translate automatically programs that use formal containers into programs that
use standard containers, should the need arise. This translation would require
computing the container associated to each cursor by static analysis.
\fi

\beforesub
\subsection{API Addition: Parts of Containers}
\label{sec:container-parts}
\aftersub

\subsubsection{Problem.}

In order to prove the postcondition of \verb|Map_F| stated above, we need to
annotate the loop in \verb|Map_F| with a loop invariant, which states the
accumulated effect of N iterations through the loop. We would like to state
that \emph{list elements already scanned} have been modified by \verb|F| (as in
the postcondition) and that \emph{list elements not yet scanned} are
unchanged. This pattern of loop invariant, consisting of two parts for the
elements scanned and the elements not yet scanned, is typical of loops that
iterate over a container.

\subsubsection{Approach.}

We introduce two new functions, called \verb|Left| and \verb|Right|, which
return the containers holding respectively the elements preceding (exclusively) or
following (inclusively) a given cursor in the container. With these functions, the effect of
the loop on the elements already scanned resembles the postcondition:
\begin{lstlisting}
pragma Assert 
  (for all Cu in Left (L, Current) => 
    Element (L, Cu) = F (Element (L'Old, Cu)));
\end{lstlisting}
The effect of the loop on the elements not yet scanned is not simply the
equality of right containers. Indeed, equality of lists \verb|L1| and \verb|L2|
only implies that, while iterating separately through each of the lists, the
same elements are encountered in the same order. Here, we also need to be able
to iterate on both lists with the same cursor, so that the first cursors of
\verb|L1| and \verb|L2| should be equal and then each call to \verb|Next|
should return the same cursor on both lists
%REVIEW 2
which is not implied by equality.
% which would be expressed as:
% \begin{verbatim}
% pragma Assert (Right (L, Current) = Right (L'Old, Current));
% \end{verbatim}
% Indeed, the equality operator of the container API defines structural equality,
% which is not enough here. We need to know that the right containers contain the
% same elements \emph{and the same cursors}.
This is expressed with a new
function, called \verb|Strict_Equal|:
\begin{lstlisting}
pragma Assert 
  (Strict_Equal (Right (L, Current), Right (L'Old, Current)));
\end{lstlisting}

% Notice that the operations \verb|Left| and \verb|Right| need to copy the argument
% container, so they should be used only in annotations for efficiency reasons.

% The implementation of \verb|Left| and \verb|Right| simply copies the argument
% container (which preserves cursors, thanks to the modification presented in
% Section~\ref{sec:independent-cursor}), and removes those elements that should
% not be present in the part considered. 

% Yannick: I commented out the following, to save space.

% What is important is that the iteration order over the container should be
% preserved, even in containers such as non ordered sets where the iteration
% order is not always preserved by deletion. Efficiency is not an issue here,
% as these functions are meant to be used in annotations only, and not kept in
% the final executable. The implementation of \verb|Strict_Equal| simply loops
% through the container, so it has the same cost as the equality operator.

\beforesec
\section{Formal Semantics}
\label{sec:description}
\aftersec

In this section, we present a formal semantics for lists. We show
briefly how the formal semantics of other containers compares to the one of
lists in the last subsection.
% We define a formal semantics for doubly linked lists from the Ada standard,
% with the API changes described in Sections~\ref{sec:independent-cursor}
% and~\ref{sec:container-parts}.

\beforesub
\subsection{Syntax of Lists}
\aftersub

A program is a sequence of variable declarations for lists (in \texttt{Lvar})
and cursors (in \texttt{Cvar}) followed by a sequence of instructions (in
\texttt{Instr}). Procedures $\Insert$, $\Delete$ and $\Replace$ modify their
first argument, which must be therefore a variable, and have no return
value. The remaining instructions are assignments. Notice that list assignment
makes an explicit copy of its argument, which prevents aliasing between
lists. \texttt{LExpr} is the set of list expressions. $\Empty$ is the empty
list constant. Functions $\Left$ and $\Right$ return parts of containers as
defined in Section~\ref{sec:container-parts}. \texttt{CExpr} is the set of
cursor expressions. \NoElement is the constant invalid cursor.  Functions
$\First$, $\Last$, $\Next$ and $\Previous$ are used for iterating over
lists. \texttt{EExpr} is the set of element expressions. Function $\Element$
accesses the element designated by a cursor in a list. \texttt{BExpr} is the
set of Boolean expressions. $\HasElement$ checks for the validity of a cursor
in a given container, $=$ is the structural equality and $\StrictEqual$ is the
more constraining equality described in
Section~\ref{sec:container-parts}. Finally, \texttt{IExpr} is the set of
integer expressions. Function $\Length$ returns the length of a list.

\small
\hspace*{-0.8cm}
\begin{minipage}[t]{0.5\linewidth}
  \begin{eqnarray*}
    & \texttt{Instr} := & \Insert (\texttt{LVar},\texttt{CExpr},\texttt{EExpr})\\
    & | & \Delete (\texttt{LVar},\texttt{CVar})\\
    & | & \Replace \\
    &   & \qquad (\texttt{LVar},\texttt{CExpr},\texttt{EExpr})\\
    & | & \texttt{Cvar} := \texttt{CExpr}\\
    & | & \texttt{Lvar} := \Copy(\texttt{LExpr})\\
    & | & \dots\\
    & \texttt{LExpr} := & \texttt{Lvar}\\
    & | & \Empty\\
    & | & \Left (\texttt{LExpr},\texttt{CExpr})\\
    & | & \Right (\texttt{LExpr},\texttt{CExpr})\\
    & \texttt{IExpr} := & \Length (\texttt{LExpr})\\
    & | & \dots
  \end{eqnarray*}
\end{minipage}
\begin{minipage}[t]{0.5\linewidth}
  \begin{eqnarray*}
    & \texttt{CExpr} := & \texttt{Cvar}\\
    & | & \NoElement\\
    & | & \First (\texttt{LExpr})\\
    & | & \Last (\texttt{LExpr})\\
    & | & \Next (\texttt{LExpr},\texttt{CExpr})\\
    & | & \Previous (\texttt{LExpr},\texttt{CExpr})\\
    & \texttt{EExpr} := & \Element (\texttt{LExpr},\texttt{CExpr})\\
    & | & \dots\\
    & \texttt{BExpr} := & \HasElement (\texttt{LExpr},\texttt{CExpr})\\
    & | & \texttt{LExpr}=\texttt{LExpr}\\
    & | & \texttt{CExpr}=\texttt{CExpr}\\
    & | & \StrictEqual (\texttt{LExpr},\texttt{LExpr})\\
    & | & \dots\\
  \end{eqnarray*}
\end{minipage}
\normalsize

For the sake of simplicity, we only list instructions and expressions that are
specific to containers. Thus, we have not included loops or branching
statements in the set of instructions, or arithmetic operations in the set of
integer expressions.

\beforesub
\subsection{Operational Semantics of Lists}
\aftersub

Since lists are generic in the kind of element they contain, the type \TypE of
elements is left unspecified. The type \TypD of cursors can be any countably infinite type.
We add a specific element $\bot$ to this set, $\TypD\cup\bot$ is written $\TypD^\bot$.
There is an environment for lists \envL, and an
environment for cursors \envC:

\begin{eqnarray*}
&&\envL : \texttt{Lvar} \rightarrow \TypL = \{\Len
: \Nat, \Fc : [1 .. \Len] \hookrightarrow \TypD, \Fe : \ImFc
\rightarrow \TypE\} \\
&&\envC : \texttt{Cvar} \rightarrow \TypD^\bot
\end{eqnarray*}
Intuitively, lists can be seen as an array $\Fc$ of cursors with a mapping
$\Fe$ from cursors to elements. \Fc is injective, so \Fc restricted to \ImFc is
bijective, and $\FcInv:\ImFc \rightarrow[1 .. \Len]$ is its inverse. We
extend \FcInv to $\FcInv_+: \ImFc\cup \{\bot\} \rightarrow [1 .. \Len+1]$
with $\FcInv_+(\bot)=\Len+1$.
Given an instruction $I$ in \texttt{Instr}, a list $l$ in \texttt{LExpr}, a
cursor $c$ in \texttt{CExpr}, an expression $e$ in \texttt{EExpr}, a Boolean
expression $b$ in \texttt{BExpr} and an integer expression $i$ in
\texttt{IExpr}, judgments take the following form:
\begin{eqnarray*}
&\eval I \Rightarrow \envL',\envC' 
\qquad \eval l \Rightarrow \TypL
\qquad \eval c \Rightarrow \TypD^\bot\\
&\eval e \Rightarrow \TypE
\qquad \eval b \Rightarrow\Bool
\qquad \eval i \Rightarrow\Int
\end{eqnarray*}
%REVIEW 1
If $\mathit{expr}$ is an expression, $\eval \mathit{expr} \Rightarrow \mathit{val}$
means that $\mathit{expr}$ evaluates in environments $\envL$ and $\envC$ to a
value represented by $\mathit{val}$ in the semantics. 
If $\mathit{instr}$ is an instruction, $\eval \mathit{instr} \Rightarrow \envL',\envC'$
means that $instr$ change the environments $\envL$ and $\envC$ into $\envL'$ and $\envC'$.

Below are the description of the semantics of integer, element and boolean expressions.
The result of function \Length on a list evaluating to \typL
is \Len. Similarly, function \Element returns the value of \Fe on $d$,
where cursor argument $c$ evaluates to $d$. Notice that $\Element(l,c)$ is
defined only when $d\in \ImFc$, which is expressed in the informal semantics as
\emph{$c$ designates an element in $l$}. Indeed, the associated Ada function
will raise a run-time error otherwise. $\HasElement(l,c)$ checks if $c$
effectively designates an element in $l$. Equality over lists ($=$) is the
structural equality. It only implies that the elements in its two list
arguments appear in the same order, \ie the equality of $\Fe\circ\Fc :
[1..\Len] \rightarrow \TypE$. \StrictEqual is stronger than $=$, as expected,
since it also implies the equality of \Fc and \Fe. Equality
of cursors is simply equality of their evaluations.
{\small
\begin{eqnarray*}
&
\inferrule{
\eval l \Rightarrow \typL
}{
\eval \Length(l)\Rightarrow\Len
}\\
&
\inferrule{
\eval l \Rightarrow \typL \\
\eval c \Rightarrow d \\
d\in \ImFc
}{
\eval \Element(l,c)\Rightarrow\Fe(d)
} \\
&
\inferrule{
\eval l \Rightarrow \typL \\
\eval c \Rightarrow d
}{
\eval \HasElement(l,c)\Rightarrow d\in \ImFc
}\\
&
\inferrule{
\eval l_1 \Rightarrow \{\Len_1,\Fc_1,\Fe_1\} \\
\eval l_2 \Rightarrow \{\Len_2,\Fc_2,\Fe_2\} \\
}{
\eval l_1=l_2\Rightarrow\Len_1=\Len_2 \And \Fe_1\circ\Fc_1=\Fe_2\circ\Fc_2
}\\
&
\inferrule{
\eval l_1 \Rightarrow \{\Len_1,\Fc_1,\Fe_1\} \\
\eval l_2 \Rightarrow \{\Len_2,\Fc_2,\Fe_2\} \\
}{
\eval \StrictEqual(l_1,l_2)\Rightarrow\Len_1=\Len_2 \And \Fc_1=\Fc_2 \And \Fe_1=\Fe_2
}\\
&
\inferrule{
\eval c_1 \Rightarrow d_1 \\
\eval c_2 \Rightarrow d_2 \\
}{
\eval c_1=c_2\Rightarrow d_1=d_2
}
\end{eqnarray*}
}%
Below are the description of the semantics of cursor expressions. The special
invalid cursor \NoElement evaluates to $\bot$. This is possible because $\bot$ cannot
appear in \ImFc, as $\Fc : [1..\Len] \rightarrow \TypD$. Therefore $\bot$ is
not a \emph{valid} cursor, \ie it designates no element, in any list. Function \Next is defined
for both valid cursors and \NoElement. It returns \NoElement when applied to a
cursor which has no valid successor (\ie for \NoElement and the last
cursor). \Previous is similar. Function \First is defined on every list. It
returns \NoElement when called on an empty list. \Last is similar.
{\small
\begin{eqnarray*}
&
\inferrule{
}{
\eval \NoElement\Rightarrow\bot
} \\
&
\inferrule{
\eval l \Rightarrow \typL \\
\eval c \Rightarrow d_1 \\
d_1\in \ImFc\cup \{\bot\}
}{
\eval \Next(l,c)\Rightarrow d_2 \\
\where d_2 = \{\Len \neq 0 \And d_1\in \ImFc\backslash \{\Fc(\Len)\} \rightarrow \Fc(\FcInv(d_1)+1),
\other \rightarrow \bot\}
} \\
&
\inferrule{
\eval l \Rightarrow \typL \\
\eval c \Rightarrow d_1 \\
d_1\in \ImFc\cup \{\bot\}
}{
\eval \Previous(l,c)\Rightarrow d_2 \\
\where d_2 = \{\Len \neq 0 \And d_1\in \ImFc\backslash \{\Fc(1)\} \rightarrow \Fc(\FcInv(d_1)-1),
\other \rightarrow \bot\}
} \\
&
\inferrule{
\eval l \Rightarrow \typL
}{
\eval \First(l) \Rightarrow d \\
\where d = \{\Len = 0 \rightarrow \bot, \Len > 0 \rightarrow \Fc(1)\}
} \\
&
\inferrule{
\eval l \Rightarrow \typL
}{
\eval \Last(l) \Rightarrow d \\
\where d = \{\Len = 0 \rightarrow \bot, \Len > 0 \rightarrow \Fc(\Len)\}
}
\end{eqnarray*}
}%
Below are the description of the semantics of list expressions.  The empty
list, returned by $\Empty$, is the only list whose length is null
($\Fv$ is the only function that is defined on the empty set
$\varnothing$).  $\Left$ is defined for both valid cursors and
$\NoElement$. Its evaluation yields a list whose valid cursors are the valid
cursors of the list argument which precede cursor argument $c$ (when $c$ is
\NoElement, that means all cursors). \Right is similar.
{\small
\begin{eqnarray*}
&
\inferrule{
}{
\eval \Empty \Rightarrow \{0,\Fv,\Fv\}
}\\
&
\inferrule{
\eval l \Rightarrow \typL \\
\eval c \Rightarrow d \\
d\in \ImFc\cup \{\bot\} \\
n=\FcInv_+(d)-1
}{
\eval \Left(l,c) \Rightarrow \{n,\Fc',\Fe'\} \\
\where \Fc'=\Fc|_{[1..n]} \\ 
       \Fe'=\Fe|_{\ImFcPrime}
} \\
&
\inferrule{
\eval l \Rightarrow \typL \\
\eval c \Rightarrow d \\
d\in \ImFc\cup \{\bot\} \\
n=\FcInv_+(d)-1
}{
\eval \Right(l,c) \Rightarrow \{\Len-n,\Fc',\Fe'\} \\
\where \Fc'=\lambda i : [1.. \Len-n]. \Fc(n+i) \\ 
       \Fe'=\Fe|_{\ImFcPrime}
}
\end{eqnarray*}
}%
The rules below describe the semantics of instructions.  Rules concerning reads
or assignment of variables are omitted (they are the usual ones). \Insert
modifies the environment so that its list variable argument designates, after
the call, a list where a cursor and an element have been inserted at the proper
place. The cursor argument, which can be either a valid cursor or \NoElement,
encodes the place the new element is inserted. The newly created cursor is not
specified. It should be different from \NoElement and from every valid cursor
in the argument list. \Delete modifies the environment so that its cursor
variable argument (which must reference a valid cursor before the call) is
deleted from the list referenced by its list variable argument. The cursor
variable references the special invalid cursor \NoElement after the call.
\Replace modifies the environment so that, after the call, its cursor argument
(which must be valid) designates its element argument in the list referenced by
its list variable argument.
{\small
\begin{eqnarray*}
&
\inferrule{
\envL(l) = \typL \\
\eval c \Rightarrow d_1 \\
d_1\in \ImFc\cup \{\bot\} \\
n=\FcInv_+(d_1) \\
d_2 \notin \ImFc\cup \{\bot\} \\
\eval e \Rightarrow \Elt
}{
\eval \Insert(l,c,e) \Rightarrow \envL[l\mapsto \{\Len+1,\Fc',\Fe'\}], \envC \\
\where \Fc' = \lambda i : [1 .. \Len+1]. \\
\{ i\in [1..n-1] \rightarrow \Fc(i), i=n \rightarrow d_2, i \in [n+1..\Len+1] 
\rightarrow \Fc(i-1)\}\\
\Fe' = \lambda d : \ImFcPrime. \{ d\in \ImFc \rightarrow \Fe(d), d = d_2
\rightarrow \Elt\}
} \\
&
\inferrule{
\envL(l) = \typL \\
\envC(c) = d \\
d\in \ImFc \\
n=\FcInv(d)
}{
\eval \Delete(l,c) \Rightarrow \envL[l\mapsto \{\Len-1,\Fc',\Fe'\}], \envC[c\mapsto \bot] \\
\where \Fc' = \lambda i : [1 .. \Len-1].
\{ i\in [1..n-1] \rightarrow \Fc(i), i \in [n..\Len-1] \rightarrow \Fc(i+1)\}\\
\Fe' = \Fe|_{\ImFcPrime}
} \\
&
\inferrule{
\envL(l) = \typL \\
\eval c \Rightarrow d_1 \\
d_1\in \ImFc \\
\eval e \Rightarrow \Elt
}{
\eval \Replace(l,c,e) \Rightarrow \envL[l\mapsto \{\Len,\Fc,\Fe'\}], \envC \\
\where \Fe' = \lambda d : \ImFc.
\{ d=d_1 \rightarrow \Elt, d\neq d_1 \rightarrow \Fe(d)\}
}
\end{eqnarray*}
}%

% Yannick: I commented the 2 lines below, as I don't think they add much.

% Notice that elements in a list cannot be
% aliased, as it is copied when inserted or retrieved from the list.

\beforesub
\subsection{Vectors, Sets and Maps}
\aftersub

\subsubsection{Sets.}

Sets do not allow duplication of elements, the order of iteration in a set is
not user-defined and the link between cursors and elements is preserved in most cases.
In the semantics, sets can be modeled as the same
tuples as lists where $\Fe$ is injective: $\{\Len: \Nat, \Fc : [1 .. \Len]
\hookrightarrow \TypD, \Fe : \ImFc \hookrightarrow \TypE\}$.
For non ordered sets, the order of iteration is not specified.
\ifdefined\longversion
{\small
\begin{eqnarray*}
&
\inferrule{
\envL(l) = \typL \\
d_1 \notin \ImFc\cup \{\bot\} \\
\eval e \Rightarrow \Elt \\
\Elt \notin \Img(\Fe)\\
\Fc' : [1 .. \Len+1] \hookrightarrow \ImFc\cup \{ d_1\}
}{
\eval \Insert(l,e) \Rightarrow \envL[l\mapsto \{\Len+1,\Fc',\Fe'\}], \envC \\
\where \Fe' = \lambda d : \ImFcPrime. \{ d\in \ImFc \rightarrow \Fe(d), d = d_1
\rightarrow \Elt\}
}
\end{eqnarray*}
}
\fi
If the set is
ordered, the order of iteration is constrained by the order over elements.
As a consequence, function $\Fe\circ\Fc$ has to preserve order.
\ifdefined\longversion

{\small
\begin{eqnarray*}
&
\inferrule{
\envL(l) = \typL \\
d_1 \notin \ImFc\cup \{\bot\} \\
\eval e \Rightarrow \Elt\\
\Elt \notin \Img(\Fe)\\
n = (max (\{i\in [1..\Len] | \Fe(\Fc(i))<\Elt\}\cup\{0\})) + 1
}{
\eval \Insert(l,e) \Rightarrow \envL[l\mapsto \{\Len+1,\Fc',\Fe'\}], \envC \\
\where \Fc' = \lambda i : [1 .. \Len+1]. \\
\{ i\in [1..n-1] \rightarrow \Fc(i), i=n \rightarrow d_1, i \in [n+1..\Len+1] 
\rightarrow \Fc(i-1)\}\\
\Fe' = \lambda d : \ImFcPrime. \{ d\in \ImFc \rightarrow \Fe(d), d = d_1
\rightarrow \Elt\}
}
\end{eqnarray*}
}
\else
The longer version of this article presents the modified inference rule for 
$\Insert$ for each container.
\fi
\subsubsection{Maps.}

Maps behave just like sets of pairs key/element except that they only constrain
keys: $(k_1,e_1)<(k_2,e_2)\leftrightarrow k_1<k_2$ and
$(k_1,e_1)=(k_2,e_2)\leftrightarrow k_1=k_2$.

\subsubsection{Vectors.}

Vectors do not expect cursors to keep designating the same element in every
case. Instead, as for arrays, elements can be accessed through their position
(an index). As a consequence, we model vectors as tuples $\{\Len : \Nat, \Fc :
[1 .. \Len] \hookrightarrow \TypD, \Fe : [1 .. \Len] \rightarrow \TypE\}$
where $\Fc$ is injective. When an element is inserted or deleted from a vector,
nothing can be said for the cursors that follow the place of insertion/deletion.

\ifdefined\longversion
{\small
\begin{eqnarray*}
&
\inferrule{
\envL(l) = \typL \\
\eval i \Rightarrow n\\
n\in [0..\Len] \\
\eval e \Rightarrow \Elt\\
\Fc_{S} \Rightarrow [n+1..\Len+1] \hookrightarrow \TypD \backslash \Img(\Fc|_{[1..n]})
}{
\eval \Insert(l,i,e) \Rightarrow \envL[l\mapsto \{\Len+1,\Fc',\Fe'\}], \envC \\
\where \Fc' = \lambda j : [1 .. \Len+1]. \\
\{ j\in [1..n] \rightarrow \Fc(j), j \in [n+1..\Len+1] 
\rightarrow \Fc_{S}(j)\}\\
\Fe' = \lambda j : [1 .. \Len+1]. \\
\{ j\in [1..n] \rightarrow \Fe(j), j = n+1
\rightarrow \Elt, j\in [n+2..\Len+1] \rightarrow \Fe(j-1)\}
}
\end{eqnarray*}
}
\fi

\beforesec
\section{Axiomatization}
\label{sec:axiomatization}
\aftersec

In this section, we present an axiomatization of lists in the language Why,
targeted at automatic provers.  We show that this axiomatization is correct
\wrt the formal semantics we gave in
Section~\ref{sec:description}.  We will formally prove it is correct \wrt
formal semantics in Coq in Section~\ref{sec:validation}.

% We define an axiomatization of lists in the language Why. For each construct
% $\inferrule{\Pre_S}{\Post_S}$ in the axiomatic semantics presented in
% Section~\ref{sec:description}, we define propositions $\Pre_A$ and $\Post_A$ in
% Why such that $\Pre_A \Rightarrow \Pre_S$ and $\Post_S \Rightarrow \Post_A$.
% This axiomatic should not be the exact translation of our semantics, but rather
% facilitate the verification of programs with the automatic prover Alt-Ergo.

\beforesub
\subsection{Presentation of Why}
\label{subsec:why}
\aftersub

The Why platform~\cite{why} is a set of tools for deductive program
verification. The first feature of Why is to provide a common
frontend to a wide set of automated and interactive theorem provers. 
Why implements a total, polymorphic, first-order logic, in which the user can
declare types, symbols, axioms and goals. These goals are then
translated to the native input languages of the various supported
provers. In our case, we are using the backends for the Coq proof
assistant~\cite{coq} and the three SMT solvers Alt-Ergo~\cite{alt-ergo},
Z3~\cite{Z3} and Simplify~\cite{Simplify}.
%
For example, here is some Why syntax that declares a
modulo operation over integers, together with some possible axiomatization:
\lstset{language={}, morekeywords={let, logic, axiom, goal, parameter, forall, exists},
 basicstyle=\small\tt, keywordstyle=\normalfont\textbf}
\begin{lstlisting}
logic mod_ : int, int -> int
axiom mod__ : forall a, b : int. 0 < b -> 
  exists q : int. a = b * q + mod_(a, b)
    and 0 <= mod_(a, b) < b 
goal test : mod_(7, 2) = 1
\end{lstlisting}
The second feature of Why we use here is to provide a
verification condition generator for an idealized, alias-free,
Hoare-logic-like programming language. The user declares and
implements programs, which are annotated with pre- and postconditions,
and local assertions such as loop invariants. Verification conditions
are then generated and transmitted to the theorem provers.
Here is for instance the declaration of a program function
\begin{lstlisting}
parameter mod : a : int -> b : int ->
  {0 < b} int {result = mod_(a, b)}
\end{lstlisting}
which takes two integers \texttt{a} and \texttt{b} as arguments, has
precondition \texttt{0 < b}, returns a result of type \texttt{int} and
has postcondition \verb|result = mod_(a, b)|. When used inside programs,
this function will trigger verification conditions (namely, that
its second argument is positive). This is one way to express that
modulo is a partial operation.

% Note: on n'a pas la place de donner un exemple de programme, mais ce*)
% n'est pas la peine car la suite de l'article n'en montre pas

\beforesub
\subsection{Axiomatization of Lists}
\aftersub

Note that this axiomatic is not meant to be the exact translation of our
semantics, but rather is written to facilitate the verification of programs
with automatic provers.

\subsubsection{Types.}

The type of elements is irrelevant, and so it is defined as an abstract
type \verb|element_t|. Cursors and lists are described respectively with abstract types 
\verb|cursor| and \verb|list|, further axiomatized in the following.

\subsubsection{Properties.}

To encode the semantics defined in Section~\ref{sec:description}, we introduce
three logic functions:
\begin{lstlisting}
logic length_   : list -> int
logic position_ : list, cursor -> int
logic element_  : list, cursor -> element_t
\end{lstlisting}
Logic functions \verb|length_| and \verb|element_| define accessors to the
fields \Len and \Fe of a list. The encoding is more complex for the field \Fc,
due to the fact that \Fc is used in three different ways in the specification:
1) directly, 2) through its inverse \FcInv, and 3) through its domain
\ImFc. Function \verb|position_| is the extension of \FcInv to cursors not in
\ImFc, whose image is set to 0. It gives access to both \ImFc ($c\in \ImFc$
$\Leftrightarrow$ \verb|position_(l, c) > 0|) and \FcInv.
We rewrite almost all rules of the semantics to remove
occurrences of \Fc.  For example, in the rule for \Next, $d_2=\Fc(\FcInv(d_1)+1)$
can be rewritten as $\FcInv(d_2)=\FcInv(d_1)+1$. The only rule that cannot be
translated that way is $=$. For this one rule, we can use an
existential quantification, $\forall~ d_1 : \Img(\Fc_1). \exists~ d_2 :
\Img(\Fc_2). \Fc_1^{-1}(d_1)=\Fc_2^{-1}(d_2) \And \Fe_1(d_1)=\Fe_2(d_2)$.
Why functions \verb|length_|, \verb|element_| and \verb|position_|
are related to the semantics as follows:
{\small
\begin{eqnarray*}
  \forall\ \verb|l|,\ \verb|i|,\ \verb|length_(l) = i| &~\Leftrightarrow~& \exists\ \Len,\ \Fc,\ \Fe,\\
  &&\eval \verb|l| \Rightarrow \typL \And \Len = \verb|i|\\
  \forall\ \verb|l|,\ \verb|c|,\ \verb|i|,\ \verb|position_(l, c) = i| &~\Leftrightarrow~& \exists\ \Len,\ \Fc,\ \Fe,\ d,\\
   &&\eval \verb|l| \Rightarrow \typL \And \eval \verb|c| \Rightarrow d\\
  &&  \And \verb|i|\geq 0 \And (\verb|i|=0 \rightarrow d \notin \ImFc) \\
  && \And (\verb|i|>0 \rightarrow d \in \ImFc \And \FcInv(d)=\verb|i|)\\
  \forall\ \verb|l|,\ \verb|c|,\ \verb|e|, \ \verb|position_(l, c) > 0| & \rightarrow &\\
\verb|element_(l, c) = e| &~\Leftrightarrow~&\exists\ \Len,\ \Fc,\ \Fe,\ d,\ \Elt,\\
 && \eval \verb|l| \Rightarrow \typL \And \eval \verb|c| \Rightarrow d\\
 &&  \And \eval \verb|e| \Rightarrow \Elt\\
 && \And (d\in\ImFc\rightarrow\Fe(d)=\Elt)
\end{eqnarray*}
}

\subsubsection{Axioms.}

We encode the semantic properties of functions \verb|length_|
and \verb|position_| into axioms, while ensuring that the axiomatic is not
unnecessarily restrictive (all semantic lists should be also axiomatic lists).
We have four axioms:

{\small
\begin{enumerate}
\item $\forall$ \verb|l, length_(l) >= 0|
\item $\forall$ \verb|l, c, length_(l) >= position_(l, c) >= 0|
\item $\forall$ \verb|l, position_(l, no_element) = 0|
\item $\forall$ \verb|l, c1, c2, position_(l,c1) = position_(l,c2) > 0 |$\rightarrow$\verb| c1 = c2|
\end{enumerate}
}

It is rather straightforward to check that these axioms are implied by the
semantics.
\ifdefined\longversion
We show this translation for the second axiom.

We start with rewriting the axiom to introduce a fresh variable for each
application:
{\small
\begin{center}
$\forall$~\verb|l, c, il, ip, length_(l) = il &  position_(l, c) = ip |$\rightarrow$ \verb| il >= ip >= 0|.
\end{center}
}
We can then translate the two equalities thanks to the equivalences:
{\small
\begin{eqnarray*}
  &&\forall\ \verb|l|,\ \verb|c|,\ \verb|il|,\ \verb|ip|,\\
  &&\quad(\exists\ \Len,\ \Fc,\ \Fe, \quad\eval \verb|l| \Rightarrow \typL \And \Len = \verb|il|) \And \\
  &&\quad(\exists\ \Len,\ \Fc,\ \Fe,\ d, \quad\eval \verb|l| \Rightarrow \typL \And \eval \verb|c| \Rightarrow d\\
  &&\quad\quad \And \verb|ip|\geq0 \And (\verb|ip|=0 \rightarrow d \notin \ImFc) \\
  &&\quad\quad \And (\verb|ip|>0 \rightarrow d \in \ImFc \And \FcInv(d)=\verb|ip|)) \quad \rightarrow\\
  &&\quad\quad\quad \verb|il| \geq \verb|ip| \geq 0
\end{eqnarray*}
}
This can be simplified, by splitting $\verb|ip|\geq0$ into $\verb|ip|=0$ and $\verb|ip|>0$:
{\small
\begin{eqnarray*}
  &&\forall\ \verb|l|,\ \verb|c|,\\
  &&\quad(\exists\ \Len,\ \Fc,\ \Fe,\ d, \quad\eval \verb|l| \Rightarrow \typL \And \eval \verb|c| \Rightarrow d\And\\
  &&\quad\quad\Len \geq 0 \geq 0 \And (d \in \ImFc\rightarrow \Len \geq \FcInv(d) \geq 0)
\end{eqnarray*}
}
The three other axioms may be translated in the same way. Here are the results:
{\small
\begin{enumerate}
\item $\forall\ \verb|l|,\ \exists\ \Len,\ \Fc,\ \Fe, \quad\eval \verb|l| \Rightarrow \typL \And \Len\geq0$
\item $\forall\ \verb|l|,\ \verb|c|,\ \exists\ \Len,\ \Fc,\ \Fe, \ d, \quad\eval \verb|l| \Rightarrow \typL \And \eval \verb|c| \Rightarrow d \And
(d \in \ImFc \rightarrow \Len\geq\FcInv(d)\geq0)$
\item $\forall\ \verb|l|,\ \exists\ \Len,\ \Fc,\ \Fe, \quad\eval \verb|l| \Rightarrow \typL \And \bot \notin \ImFc$
\item $\forall\ \verb|l|,\ \verb|c1|,\ \verb|c2|, \ \exists\ \Len,\ \Fc,\ \Fe,\ d_1,\ d_2, \quad\eval \verb|l| \Rightarrow \typL
\And \eval \verb|c1| \Rightarrow d_1 \And \eval \verb|c2| \Rightarrow d_2 \And
d_1 \in \ImFc \And d_2 \in \ImFc \And \FcInv(d_1)=\FcInv(d_2) \rightarrow d_1 = d_2$
\end{enumerate}
} 
\else
These proofs can be found in the longer version of
this article.
\fi

% Notice that the four axioms described do not imply that there is a cursor for
% each position included in $[1..\Len]$. Rather, this property is implicit in the
% definition of \Next and \First.

\subsubsection{Semantic Rules.}

Each Ada function represented in the semantics is translated into a Why
program. Given the semantic rule $\inferrule{\Pre_S}{\Post_S}$ defining this
function, we define a precondition $\Pre_A$ and a postcondition $\Post_A$ on
the Why program, so that $\Pre_A \Rightarrow \Pre_S$ and $\Post_S \Rightarrow
\Post_A$. In fact, since preconditions are quite simple, we usually have
$\Pre_A = \Pre_S$. We illustrate the general pattern that we applied with function
\Next, which we translate into program \verb|next| in Why:
\begin{lstlisting}
parameter next :
  l:list -> c:cursor ->
  { c = no_element or position_(l, c) > 0 }
  cursor 
  { result = next_(l, c) }
\end{lstlisting}

The precondition of this function states that either the argument cursor
\verb|c| is valid in the argument list \verb|l| (because
\verb|position_(l, c) > 0| $\Leftrightarrow c\in \ImFc$) or the argument
cursor is equal to \NoElement.  This precondition is exactly the condition for
the application of the semantic rule for \Next.

An axiom \verb|next__| defines the behavior of logic function \verb|next_|
over the allowed cases only, leaving the
value of other applications of \verb|next_| unspecified, as
seen in Section~\ref{subsec:why}:

\begin{lstlisting}
axiom next__ :
  forall l:list. forall c:cursor. forall nxt:cursor.
    (length_(l) > position_(l,c) > 0 ->
      position_(l, nxt) = position_(l, c) + 1)
    and (length_(l) > 0 and position_(l, c) = length_(l)
      or c = no_element ->
        nxt = no_element)
\end{lstlisting}

Axiom \verb|next__| is defined as two implications: in the first case, the next
cursor of the cursor argument is valid and we define its position; in the
second case, the result is \NoElement. Intuitively, this is the same as the
semantic rule for \Next. Using the same equivalences as above, we can rewrite
axiom \verb|next__| into a logic formula that is exactly the semantics of
\Next, rewritten to use only \FcInv.
\ifdefined\longversion
Like in the previous paragraph, we start by introducing one fresh
variable per function application and we then use the translations
of each construct to obtain :
\begin{eqnarray*}
  &&\forall \ \verb|il|,\ \verb|ic|,\ \verb|in|,\\
  &&\quad(\exists\ \Len,\ \Fc,\ \Fe, \quad\eval \verb|l| \Rightarrow \typL \And \Len = \verb|il|) \And \\
  &&\quad(\exists\ \Len,\ \Fc,\ \Fe,\ dc, \quad\eval \verb|l| \Rightarrow \typL \And \eval \verb|c| \Rightarrow dc\\
  &&\quad\quad \And \verb|ic|\geq0 \And (\verb|ic|=0 \rightarrow dc \notin \ImFc) \\
  &&\quad\quad \And (\verb|ic|>0 \rightarrow dc \in \ImFc \And \FcInv(dc)=\verb|ic|)) \And\\
  &&\quad(\exists\ \Len,\ \Fc,\ \Fe,\ dn, \quad\eval \verb|l| \Rightarrow \typL \And \eval \verb|nxt| \Rightarrow dn\\
  &&\quad\quad \And \verb|in|\geq0 \And (\verb|in|=0 \rightarrow dn \notin \ImFc) \\
  &&\quad\quad \And (\verb|in|>0 \rightarrow dn \in \ImFc \And \FcInv(dn)=\verb|in|)) \rightarrow\\
  &&\quad\quad\quad\exists\ \Len,\ \Fc,\ \Fe,\ dc,\ dn \quad\eval \verb|l| \Rightarrow \typL\\
  &&\quad\quad\quad\quad\And \eval \verb|c| \Rightarrow dc \And \eval \verb|nxt| \Rightarrow dn \And\\
  &&\quad\quad\quad\quad\quad (\verb|il| > \verb|ic| > 0 \rightarrow \verb|in| = \verb|ic| + 1) \And\\
  &&\quad\quad\quad\quad\quad (\verb|il| > 0 \And \verb|ic| = \verb|il| \ || \ dc = 0 \rightarrow dn = 0)
\end{eqnarray*}

It can be rewritten as:
{\small
\begin{eqnarray*}
  &&\exists\ \Len,\ \Fc,\ \Fe,\ dc,\ dn, \quad\eval \verb|l| \Rightarrow \typL\\
  && \And \eval \verb|c| \Rightarrow dc \And \eval \verb|nxt| \Rightarrow dn \And\\
  &&\quad\quad\quad (dc \in \ImFc \And \Len > \FcInv(dc) \rightarrow\\
  &&\quad\quad\quad\quad\quad\quad dn \in \ImFc \And\FcInv(dn) = \FcInv(dc) + 1) \And\\
  &&\quad\quad\quad (dc \in \ImFc \And \FcInv(dc)=\Len \ ||\ dc = 0\rightarrow dn = 0)
\end{eqnarray*}
}
\else
This proof is presented in the longer version of this article.
\fi

\beforesub
\subsection{Effectiveness}
\aftersub

It is worth noting that the main difficulty we faced, when developing the
axiomatization presented above, was to match the somewhat fuzzy expectations of
SMT automatic provers: some work best with large predicates and few axioms,
some work best with many smaller axioms, \etc
\ifdefined\longversion
 As a point of comparison,
developing the axiomatization took us three times more efforts than the formal
proof of correctness in Coq presented in Section~\ref{sec:validation}.
\fi

We wrote a number of tests (30) to convince ourselves that the axiomatization
of lists presented is effective. With a combination of provers we managed to
prove all the generated verification conditions. Our running example is proved
rather quickly (less than 1s per VC). To facilitate proofs, which impacts both
provability and speed, we defined 15 lemmas. These are not a burden in the
maintainability of the axiomatic and cannot introduce inconsistencies since
they are also automatically proved.

%REVIEW 3
% In Section 4.3 (and Section 4.4) I would like to suggest to add a
%   brief description of the properties that you used as test cases for
%   your axiomatization and, if possible, a brief motivation why you
%   have choosen them. 
Most of the tests correspond to unit-tests for specific properties of list
expressions (especially for complex ones, such as $\Left$) and instructions.
For example, there is a test to check that after replacing the value of an
element in the list, the list indeed contains this value at this position after
the call:
\begin{lstlisting}
let test_replace_element
  (co : list ref) (cu : cursor) (e : element_t) =
    { has_element_ (co, cu) }
    replace_element co cu e
    { element_ (co, cu) = e }
\end{lstlisting}
A few tests, such as our running example, combine expressions and instructions
to validate more complex behaviors of the API. All tests were proved
automatically.

\beforesub
\subsection{Vectors, Sets and Maps}
\aftersub

Here is a table referencing the work done for each container. It contains the
size of its Why file (its number of lines), the number of lemmas given and the
number of tests passed. The code and tests are available
on the web.\\
~\\
\begin{tabular}{|l|c|c|c||}
\hline
~Container & ~Lines~ & ~Lemmas~ & ~Tests~ \\
\hline
~List & 352 & 15 & 30 \\
\hline
~Vectors & 298 & 0 & 22 \\
\hline
~Hashed Sets & 429 & 24 & 35 \\
\hline
\end{tabular}\begin{tabular}{|l|c|c|c|}
\hline
~Container & ~Lines~ & ~Lemmas~ & ~Tests~ \\
\hline
~Ordered Sets & 506 & 30 & 38 \\
\hline
~Hashed Maps & 394 & 27 & 22 \\
\hline
~Ordered Maps~ & 476 & 35 & 25 \\
\hline
\end{tabular}\\

\beforesec
\section{Validation of the Axiomatization}
\label{sec:validation}
\aftersec

We have presented a formal semantics for containers in
Section~\ref{sec:description} and its axiomatization in Why in
Section~\ref{sec:axiomatization}. We have shown a pen-and-paper proof that the
axiomatization presented is correct \wrt formal semantics. Given the size of
the axiomatization, such a manual proof may easily contain errors. In this
section, we describe an implementation in Coq of the formal semantics of
containers, and a formal proof of correctness of the Why axiomatization of
lists \wrt the Coq implementation, (which also implies
the consistency of the axiomatization).

\beforesub
\subsection{Coq Implementation and Proof for Lists}
\aftersub

\subsubsection{Types.}

Our proofs are generic in the element type. For the representation of cursors, we use positive natural
numbers to which we add 0 to model the special cursor $\bot$.
For lists, we model the tuple with a functional list of pairs cursor-element (if \verb|a| is an element of this list, \verb|fst a| refers to the associated cursor
and \verb|snd a| to the element). The field $\Len$ is the length of the
functional list, the field $\Fc$ is the function that, for an integer $i\in[1..\Len]$, returns the cursor of the pair at the $i^{th}$ position
in the list and $\Fe$ returns, for each cursor in $\ImFc$, its first association in the list. Therefore, a list of this kind always defines one and only one
tuple $\typL$.
\lstset{language={}, morekeywords={Definition, Fixpoint, Record, match, end, if, then, else, with},
basicstyle=\small\tt, keywordstyle=\normalfont\textbf}
\begin{lstlisting}
Definition cursor : Set := nat.
Definition Rlist : Set := List.list (cursor*element_t).
\end{lstlisting}
To keep only tuples where $\Fc$ is injective and has value in $\TypD$, we constrain the functional lists with a predicate.
This predicate states that every cursor that appears in a list is different from $\bot$ (positive) and does not appear again in the same list. 
\begin{lstlisting}
Fixpoint well_formed (l : Rlist) : Prop :=
   match l with
        nil => True
      | a :: ls =>    fst a > 0
                   /\ has_element ls (fst a) = false
                   /\ well_formed (ls)
   end.
Record list := {this :> Rlist; wf : well_formed this}.
\end{lstlisting}
The property of well formedness has to be preserved through every modification of the list. With the Coq lists restricted that
way, there is one and only one list per tuple $\{\Len: \Nat, \Fc : [1 .. \Len] \hookrightarrow \TypD, \Fe : \ImFc
\rightarrow \TypE\}$. The two representations are equivalent.

\subsubsection{Axioms.}
Thanks to the \texttt{-coq} option of Why, we translate automatically our Why
axioms into Coq. They can then be proved valid formally.

\subsubsection{Semantic Rules.}

We have written an implementation for each construct of our language. Since Coq
is a pure functional language, the instructions that modify their list
argument, such as \Insert, are modeled by a function that returns a new list.
The implementations are as close as possible to the semantic of their
corresponding construct. Since they are Coq functions, these implementations
have to be total, so we complete them for the cases that are not described in
the semantics. For example, the semantics of \Next is only defined when the
cursor that designates the element to be replaced is valid in the list or equal
to \NoElement\footnote{Using it elsewhere reflects a mistake and is reported as
an error when executed.}. In the Coq implementation below, it returns \NoElement if we are
not in that case (we could have chosen any other return value).
\begin{lstlisting}
Fixpoint next (l : Rlist) (cu : cursor) :=
   match l with
        nil => no_element
      | a :: ls =>
          if beq_nat (fst a) cu then first ls
          else next ls cu
   end.
\end{lstlisting}
The result of the semantic rule for $\Next$ is completely defined. It is easy
to be convinced that, when the cursor given as a parameter is indeed valid in
the list or equal to $\NoElement$, the Coq function \texttt{next}
returns the appropriate cursor. We use this representation to prove formally
that the contracts in our axiomatic are indeed implied by the semantics.

For nearly every rule of the semantics of lists, the result of the modification
is completely defined in terms of the value of the arguments. The only rule
that is not completely determined is $\Insert$, since the value of the new
cursor is not given. For the implementation, we define a function \texttt{new}
that returns a valid cursor. To keep our proofs as general as possible, we
took care to use only the properties of \texttt{new} that were defined in the
semantic (\ie that the result of \texttt{new} $\notin \ImFc\cup \bot$) by
enforcing it thanks to Coq's module system.

\beforesub
\subsection{Vectors, Sets and Maps}
\aftersub

All containers have the same Coq representation. Therefore, some parts of the
proofs are shared. Sets and maps (ordered or not) share the same lemmas and
heavily rely on those of lists. Vectors also rely on the lemmas of lists but
less heavily (they are quite different).
%REVIEW 1
To make the proofs more reliable, general lemmas, which will not be affected
by a slight change in the API, are collected in a separate file named ``Raw''.
%
Here are the size of the files
(the number of lemmas in each file) and the architecture.
\begin{center}
\begin{tabular}{|c|c|c|c|c|c|}
\hline
\multicolumn{6}{|c|}{Raw Lists (154)}\\
\hline
& ~Raw Vectors (108)~ & \multicolumn{4}{|c|}{Raw Sets (139)}\\
\cline{2-6}
~Lists~ & Vectors & Hashed & Ordered & Hashed & Ordered \\
(21) & (29) & ~Sets (64)~ & ~Sets (69)~ & ~Maps (64)~ & ~Maps (70)~\\
\hline
\end{tabular}
\end{center}
Like for the $\Insert$ rule for lists, every unspecified part of the semantic
of every container is kept as general as possible thanks to a sealed Coq module
that only allows proofs to use the specified parts. The whole proof is 16,000+
lines of Coq. All of it plus commented excerpts can be found
on the web.

\beforesec
\section{Related Work}
\label{sec:related-works}
\aftersec

Formal proof over containers is an active area of research.  There are two
important, complementary areas in this domain: certifying user code that uses
containers while assuming that their implementation complies with their
specification (what we are doing) and certifying that an implementation of
containers indeed complies with its specifications.

On the one hand, Bouillaguet \etal~\cite{bouillaguet:2007:vmcai} focus on
verifying that a container's implementation indeed complies with its
specifications. They use resolution based first-order theorem provers to verify
that the invariants of data structures such as sets and maps are preserved when
considering operations on their encodings as arrays and trees.  Zee
\etal~\cite{zee:2008:pldi} even presented the first verification of full
functional correctness for some linked data structure implementations. Unlike
Bouillaget \etal, they use interactive theorem provers as well to discharge
their verification conditions. Since they aim at certifying an implementation
once, it does not seem to be too heavy a burden.
% define the Jahob system, a combined approach to
% verify invariants of linked data structures such as lists and trees, where VCs
% are discharged by a combination of specialized decision procedures, first-order
% theorem provers and interactive theorem provers.

On the other hand, Gregor and Schupp~\cite{gregor:2006:spe} focus, like we do,
on the certification of user programs. They present an unsound static analysis
of C++ programs using STL containers. They generate partially executable
assertions in C++ to express the constraints over containers' usage, in
particular a non-executable \texttt{foreach} quantifier to iterate over all objects of a given
type in the current memory state.  Blanc \etal~\cite{blanc:2007:ase} also work
on certifying user code using the C++ STL
containers. Just as we did in this work, they axiomatize the containers and
then construct some preconditions (\resp postconditions) that are more (\resp
less) constraining than those of the semantics. Their work is still
substantially different from what we did since they only check that the
containers are properly used and they have no annotation language to allow the
user to express other properties.
% axiomatize containers from the C++ Standard
% Template Library, translate this axiomatization back in C++ and apply model
% checking to the combination of user code + axiomatization. They do not provide
% an annotation language for the user, and it is not clear what bounds are used
% for the maximum size of containers and their elements, as they use bounded
% model checking with SatAbs.
Dillig \etal~\cite{dillig:2011:popl} present a static analysis for reasoning
precisely over the content of containers. While they assume that we can analyze
the code that fills the containers to provide constraints over the
containers' content, we rely instead on user annotations. This gives 
us the possibility to verify user properties expressed in the same 
annotation language.

% A bit farther from what we have done,
% Suter \etal~\cite{suter:2010:popl} define decision procedures for the
% underlying algebraic data types inside functional containers, which supports
% the addition of abstraction functions over these data types provided they are
% suitably restricted. They use executable annotation to enable users to debug
% their annotations using testing approaches.

% The use of SMT solvers to reason about containers has also be done before.  For
% example, Kawaguchi \etal~\cite{kawaguchi:2009:pldi} present a mixed approach to
% prove that refinements of basic data structures in a functional language
% satisfy additional invariants, using both rich typing and an SMT solver.

\beforesec
\section{Conclusion}
\label{sec:conclusion}
\aftersec

We have presented a library of \textit{formal containers}, a slightly modified
version of the standard Ada containers. The aim was to make them usable in
programs annotated with properties of interest to the user, that can be both
tested and formally proved automatically.  Although we have limited experience
with using this library, our experiments so far indicate that most user-defined
annotations can now be expressed with few quantifiers, leading to automatic proofs of
rich properties with SMT provers. We are now looking forward to working
with our industrial partners in project Hi-Lite to develop large use-cases with
formal containers.

We have given a formal semantics for these containers, and we have proved that
this semantics is consistent by implementing it in Coq. We have developed an
axiomatization of these containers in the language Why, targeted at automatic
proofs with SMT provers, and we have proved in Coq that this axiomatization is
correct \wrt the formal semantics of containers. On the one hand, this
formalization is an essential step towards an argument-based certification of
the library of formal containers, for their use in safety-critical software
development. On the other hand, the proof of correctness of the axiomatization
used in automatic provers is a very strong assurance against inconsistencies in
proofs, which are a sour point of formal methods in industry.

Formal containers have been implemented in Ada, and could be included in any Ada
compiler's library.
%REVIEW 3
They have been included in the standard library of the not yet released GNAT
6.5 Ada compiler.
%REVIEW 1
Theoretically, the Why axiomatization could be reused to model containers in
other languages, with a few modifications to comply with the particularities of
their respective APIs.
%
The implementation we provide for Ada containers should be correct \wrt the
formal semantics in Coq, but we have not proved it formally. This is an
interesting (but difficult) problem for the future.

\subsubsection*{Acknowledgement}
We would like to thank Ed Schonberg, David Lesens and the anonymous referees
for their useful reviews of this paper.

\beforesec
\bibliographystyle{plain}
\bibliography{main}

% \begin{thebibliography}{4}

% \end{thebibliography}

\end{document}