\documentclass[]{easychair}
\usepackage[english]{babel}
\usepackage{mathpartir}
\usepackage{amssymb}
\usepackage{amsmath}
\usepackage{stmaryrd}
\usepackage{graphicx}
\usepackage{xspace}
\usepackage{xcolor}
\usepackage{listings}
\usepackage{enumerate}
%\usepackage{mathptmx}
%\usepackage[scaled=.82]{beramono} % for typewriter font

\newcounter{tmp}

\newcommand{\terms}{\mathcal{T}}
\newcommand{\Fm}[1]{\llb #1 \rrb^-}
\newcommand{\Fp}[1]{\llb #1 \rrb^+}
\newcommand{\Lift}{{\mathcal{L}\mathit{ift}(\mathbf{S})}}
\newcommand{\GS}{\mathbf{S}}
\newcommand{\K}{\mathit{known}}
\newcommand{\reach}{\mathit{reach}}
\newcommand{\next}{\mathit{next}}
\newcommand{\update}{\mathit{update}}
\newcommand{\access}{\mathit{access}}
\newcommand{\uncyclic}{\mathit{acyclic}}
\newcommand{\option}{\mathit{option}}
\newcommand{\farray}{\mathit{array}}
\newcommand{\Some}{\mathit{Some}\ }
\newcommand{\None}{\mathit{None}}
\newcommand{\bb}{\omega}

\newcommand{\FOL}{\mathrm{FOL}}
\newcommand{\FOLT}{\mathrm{FOL}^\star}
\newcommand{\bp}{\mathbf{BP}}

\newcommand{\iFOLT}{\vdash^\star}
\newcommand{\iFOL}{\vdash}
\newcommand{\cls}{\mathcal{C}\mathit{ls}}
\newcommand{\sko}{\mathcal{S}\mathit{ko}}
\newcommand{\mgu}{\mathrm{mgu}}

\newcommand{\limp}{\supset} \newcommand{\lequ}{\equiv}
\newcommand{\la}{\langle} \newcommand{\ra}{\rangle}
\newcommand{\llb}{\llbracket} \newcommand{\rrb}{\rrbracket}
\newcommand{\eqdef}{\:\triangleq\:}
\newcommand{\sep}{\;|\;}
\newcommand{\ie}{{\em i.e.},\xspace}
\renewcommand{\Im}{\mathit{Image}}

\newcommand{\beforesec}{\vspace{0cm}}
\newcommand{\aftersec}{\vspace{0cm}}
\newcommand{\beforesub}{\vspace{0cm}}
\newcommand{\aftersub}{\vspace{0cm}}


\makeatletter  % makes '@' an ordinary character
\renewcommand{\verbatim@font}{\ttfamily\small}
\makeatother   % makes '@' a special symbol again

\newtheorem{theorem}{Theorem}[section]
\newtheorem{lemma}{Lemma}[section]
\newtheorem{definition}{Definition}[section]
\newtheorem{corollary}{Corollary}[section]

%opening
\title{Reasoning with Triggers}
\titlerunning{Reasoning with Triggers}
% \author{
% Claire Dross \and
% Sylvain Conchon\inst{1,2} \and
% Johannes Kanig\inst{3} \and
% \mbox{Andrei Paskevich\inst{1,2}}}
\author{
Claire Dross\\
\affiliation{LRI, Universit\'e Paris-Sud 11,}\\
\affiliation{CNRS, Orsay F-91405}\\
\affiliation{INRIA Saclay-\^Ile de France,}\\
\affiliation{ProVal, Orsay F-91893}\\
\affiliation{AdaCore, Paris F-75009}
\and
Sylvain Conchon\\
\affiliation{LRI, Universit\'e Paris-Sud 11,}\\
\affiliation{CNRS, Orsay F-91405}\\
\affiliation{INRIA Saclay-\^Ile de France,}\\
\affiliation{ProVal, Orsay F-91893}\and
Johannes Kanig\\
\affiliation{AdaCore, Paris F-75009} \and
Andrei Paskevich\\
\affiliation{LRI, Universit\'e Paris-Sud 11,}\\
\affiliation{CNRS, Orsay F-91405}\\
\affiliation{INRIA Saclay-\^Ile de France,}\\
\affiliation{ProVal, Orsay F-91893}}

\authorrunning{C. Dross \and S. Conchon \and J. Kanig \and A. Paskevich}

% \institute{
% LRI, Universit\'e Paris-Sud 11, CNRS, Orsay F-91405 \and
% INRIA Saclay-\^Ile de France, ProVal, Orsay F-91893 \and
% AdaCore, Paris F-75009
% }

\begin{document}
\maketitle

\begin{abstract}
SMT solvers can decide the satisfiability of ground formulas
modulo a combination of built-in theories. Adding a built-in theory to a given
SMT solver is a complex and time consuming task that requires internal
knowledge of the solver. However, many theories (arrays~\cite{nelson-arrays},
reachability~\cite{lahiri2008back}), can be easily expressed using
first-order formulas. Unfortunately,
since universal quantifiers are not handled in a complete way by SMT solvers,
these axiomatics cannot be used as decision procedures.

In this paper, we show how to extend a generic SMT solver to accept a
custom theory description and behave as a decision procedure for that
theory, provided that the described theory is complete and terminating
in a precise sense. The description language consists of first-order
axioms with triggers, an instantiation mechanism that is found in many
SMT solvers. This mechanism, which usually lacks a clear semantics in
existing languages and tools, is rigorously defined here; this
definition can be used to prove completeness and termination of the
theory. We demonstrate using the theory of arrays, how such proofs can
be achieved in our formalism.
\end{abstract}

% \begin{abstract}
%   Instantiation patterns, also known as triggers, are a commonly used
%   heuristic to guide quantifier handling in SMT solvers. Being a
%   heuristic, they have no commonly accepted formal semantics. This
%   makes it difficult for a user to find good triggers and be sure that
%   the chosen SMT solver works as intended. For some
%   axiomatic theories, carefully chosen triggers can make an SMT solver
%   behave as a decision procedure. Thus, it is particularly important
%   to have full control over instantiation.
% 
%   In this paper, we promote triggers from a heuristic to a construct with
%   formal semantics, by introducing a first-order logic which includes
%   triggers. Our formalism gives a precise meaning to the notion of
%   ``known term'', \ie~a term that enables an instantiation.
%   We prove that instantiation restricted to known terms has
%   the same deductive power in classical first-order logic with
%   equality as the unrestricted instantiation.
%   Then we show how to extend a generic ground SMT solver to
%   our logic with triggers, and we study an example of axiomatic theory with
%   triggers on which this abstract solver behaves as a decision procedure.
% \end{abstract}

\beforesec
\section{Introduction}
\aftersec

SMT solvers are sound, complete, and efficient tools for deciding the
satisfiability of ground formulas modulo combinations of built-in
theories such as linear arithmetic, arrays, bit-vectors etc. Usually, they work
on top of a SAT solver which handles propositional formulas. Assumed literals
are then handed to dedicated solvers for theory reasoning. These solvers are
complete. Adding a new theory to the framework is a complex and time consuming
task that requires internal knowledge of the solver. For some theories however,
it is possible to give a first-order axiomatization. Unfortunately, even if a
few SMT solvers also handle first-order formulas, for example,
Simplify~\cite{Simplify}, CVC3~\cite{CVC3}, Z3~\cite{Z3} and
Alt-Ergo~\cite{altergo}, these axiomatizations cannot be used as theories.
Indeed, these solvers are not complete when quantifiers are involved, even in
the absence of theory reasoning.

SMT solvers handle universal quantifiers through an
instantiation mechanism. They maintain a set of ground formulas (without
quantifiers) on
which theory reasoning is done. This set is periodically augmented by
heuristically chosen instances of universally quantified formulas.

The heuristics for choosing new instances differ between SMT
solvers.
Nevertheless, it is commonly admitted that user guidance is
useful in this matter~\cite{Simplify,moskal-2009}. The choice of instances can
be influenced by manually adding instantiation
patterns, also known as \emph{triggers}. These patterns are used to restrict
instantiation to \emph{known} terms that have a given form.
Here is an example of a universally quantified formula with
a trigger in SMT-LIB~\cite{smtlib} notation:
\begin{verbatim}
      (forall ((x Int)) (! (= (f x) c) :pattern ((g x))))
\end{verbatim}
The syntax for triggers includes a bang (a general syntax for annotating
formulas) before the restricted formula \texttt{(=
(f x) c)} and the keyword \texttt{:pattern} to introduce the trigger
\texttt{(g x)}.
  The commonly agreed meaning of the above formula can be stated as
  follows:

  \begin{quote}
    \emph{Assume} \texttt{(= (f t) c)} \emph{only for terms}
    \texttt{t} \emph{such that} \texttt{(g t)} \emph{is known}.
  \end{quote}
%
  Intuitively, a term is \emph{known} when it appears in a ground fact
  assumed by the solver. However, that rule is quite vague and does not include
answers to the
following questions: when does a term become known? Is that notion to be
considered modulo equality and modulo the built-in theories, and
finally, when is this rule applied exactly, and to which formulas?
Different provers have found different answers to these questions,
consequence of the fact that triggers are considered a heuristics and
not a language feature with precise semantics.

We give a proper semantics for first-order formulas with
triggers. In this semantics, instantiation of universally quantified
formulas is restricted to known terms. This makes it possible to extend a
generic SMT solver so that it behaves as a decision procedure on an
axiomatization representing a custom theory, provided the theory is complete in
our framework. This enables non-expert users to add their own decision
procedure to SMT solvers. Unlike first-order axiomatization in SMT solvers
handling quantifiers, a proof of completeness and termination of the decision
procedure can be attempted and, unlike manual implementation of decision
procedures inside SMT solvers, it does not require internal knowledge of the
solver.

  In Sect.~\ref{labeled-calculus}, we introduce a formal semantics for
  first-order logic with a notation for instantiation patterns that
  restrict instantiation. It formalizes both the notion of
  trigger and the dual notion of known term. We show that this extension
  of first-order logic is conservative: formulas without triggers
  preserve their satisfiability under this semantics. We present in
  Sect.~\ref{prover} a theoretical way of extending a
  generic ground SMT solver so that it can turn an axiomatization T with
triggers into a decision procedure, provided that T has
some additional properties.
  Finally, in Sect.~\ref{case-study}, we
  demonstrate on the non-extensional theory of arrays how our framework can be
  used to demonstrate that an axiomatization with triggers indeed fulfills its
  requirements.

\paragraph{Related Work.}

Triggers are a commonly used heuristic in SMT solvers that handle
quantifiers. User manuals usually
explain how they should be used to achieve the best
performance~\cite{Simplify,moskal-2009,CVC3}. Triggers can be automatically
computed by the solvers. A lot of work has also been
done on defining an efficient mechanism for finding the instances
allowed by a trigger. These techniques, called $E$-matching, are
described in~\cite{Simplify,nelson-arrays} for Simplify,
in \cite{Z3-e-matching} for Z3, and in \cite{CVC3} for CVC3.
% For review 1
Other heuristics for generating instances include model-based quantifier
instantiation~\cite{MBQI} and saturation processes closed to the superposition
calculus~\cite{saturation}.
%

In this paper, triggers are not handled in the usual manner. On the one
hand, since SMT solvers are not complete in general when quantifiers are
involved, they favor efficiency over completeness in the treatment of
triggers. For example, they usually do not attempt to match triggers modulo
underlying theories. On the other hand, in our framework, triggers are used
to define theories, and they need therefore to be handled in a complete way.

Triggers can also be used in complete first-order theorem provers to guide 
the proof search and improve the solver's efficiency. This is done on a
complete solver for a subset of first-order logic with linear arithmetics
based on a sequent calculus in~\cite{rummer2012matching}. 

As for using an SMT solver as a decision procedure, the related idea that a
set of first-order formulas can be saturated with
a finite set of ground instances has been explored previously.
For example, in~\cite{kuncak}, decision procedures for
universally quantified properties of functional programs are designed
using local model reasoning.
% review 3: a comparison to the "complete instantiation" method by
% Ge and de Moura should be given
In the same way, Ge and de Moura describe
fragments of first-order logic that can be decided modulo theory 
by saturation \cite{MBQI}. Both of these works define a restricted class of
universally quantified formulas that can be finitely instantiated. We do not
impose such restrictions a priori but rather require a dedicated proof of
completeness.


%

% \beforesec
% \section{\label{section-ex}Motivating Examples}
% \aftersec
% 
% In this section, we give examples to emphasize the need for a
% formal framework for reasoning about triggers. We show that the
% current (under-specified) practice of determining known terms in SMT
% solvers makes them behave non-intuitively.
% 
% \beforesub
% \subsection{Known Terms}
% \aftersub
% 
% Let us repeat: a term is known by a solver when it \emph{appears} in a ground
% fact \emph{assumed} by the solver. In this subsection, we show what makes
% these concepts non-trivial, using a hypothetical solver whose handling of
% triggers is representative of existing SMT solvers. Similar behavior can be
% observed in several of the existing SMT solvers with support for quantifiers
% % review 1 : The hypothetical solver described on Section 2.1 is not
% %representative of existing SMT solvers.
% \footnote{This behavior can only be observed if triggers are the only
% heuristic used by the solver. To reproduce with Z3 for example, model-based
% quantifier instantiation has to be disabled with \texttt{(set-option :mbqi
% false)}.}.
% %
% 
% Firstly, the asymmetric treatment of disjunctions in DPLL-based solvers
% makes it difficult to predict the evolution of the set of known
% terms. Our hypothetical solver interprets $A\vee B$ as the propositionally
% equivalent formula $A\vee(\neg A\wedge B)$, thus \emph{assuming} $\neg
% A$ in the second branch. For example, consider:
% \begin{verbatim}
%   (or (= (g c) b) (= (f c) b))
%   (not (= (f c) c))
%   (forall ((x Int)) (! (= (f x) c) :pattern ((g x))))
% \end{verbatim}
% The term \texttt{(g c)} is known in both branches of the disjunction,
% once in the fact \texttt{(= (g c) b)} and once in \texttt{(not (= (g
%   c) b))}. Therefore, the trigger \texttt{(g x)} allows the
% instantiation of the universally quantified formula with \texttt{c} in
% both branches. The solver deduces that this set of formulas is
% unsatisfiable.
% 
% Let us now consider the case where the formulas of the disjunction are
% exchanged: \texttt{(or (= (f c) b) (= (g c) b))}. Neither \texttt{(= (g
%   c) b)}, nor its negation is assumed in the first branch.
% Thus, the term \texttt{(g c)} is not known and the quantified
% formula cannot be instantiated. As a consequence, the solver
% terminates and returns a candidate model for the set of formulas.
% 
% Secondly, equalities are taken into account to determine which terms
% \emph{appear} in a given set of facts. The following set of formulas,
% for example, is detected as unsatisfiable, because the term \texttt{(f (g 0))}
% can be constructed by replacing \texttt{c} by \texttt{(g 0)} in the known term
% \texttt{(f c)}:
% \begin{verbatim}
%   (forall ((x Int)) (! (= (f (g x)) 0) :pattern ((f (g x)))))
%   (not (= (f c) 0))
%   (= (g 0) c)
% \end{verbatim}
% 
% Not every deducible equality is taken into account though. For example,
% equalities implied by the theory of arithmetics are not used. On the following
% set of formulas, the solver cannot deduce the unsatisfiability:
% \begin{verbatim}
%   (forall ((x Int)) (! (= (f (+ x 1)) (+ (f x) 1))
%                             :pattern ((f (+ x 1)))))
%   (= (f 0) 0)
%   (not (= (f 1) 1))
% \end{verbatim}
% The term \texttt{(f (+ 0 1))} is not known here, even though the term
% \texttt{(f 1)} is known and the equality
% between \texttt{1} and \texttt{(+ 0 1)} can be deduced by the solver.
% 
% Finally, given a formula $F$, a solver may not be able to deduce that
% $\{F,\neg F\}$ is unsatisfiable. Indeed, in the formulas
% \begin{verbatim}
%   (forall ((x Int)) (! (= (f x) c) :pattern ((g x))))
%   (not (forall ((x Int)) (! (= (f x) c) :pattern ((g x)))))
% \end{verbatim}
% the solver drops the trigger \texttt{(g x)} under the existential
% quantifier introduced by the negation in the second formula and no term
% \texttt{(g a)} becomes known when this formula is assumed. As a consequence,
% the first formula cannot be instantiated.
% 
% \beforesub
% \subsection{Over-restrictive and Over-permissive Triggers}
% \aftersub
% 
% Given a universally quantified formula, its triggers should be one or
% several terms that contain all the quantified variables. Usually,
% triggers are chosen among the sub-terms of the formula. Consider for
% instance the following formula $F$:
% \begin{verbatim}
%   (forall ((x Int) (y Int)) (= (g (f x) y) (g x y)))
% \end{verbatim}
% The set of its potential triggers contains the terms \texttt{(g (f
% x) y)} and \texttt{(g x y)}. Even for this tiny formula, it is difficult to
% find a suitable trigger.
% 
% We can show that \texttt{(g (f x) y)} is too restrictive by exhibiting
% a ground formula whose unsatisfiability cannot be proved if we take
% this trigger. Let $R$ be the following conjunction of ground
% equalities:
% \begin{verbatim}
%   (and (= (f 1) (f 0)) (not (= (g 1 0) (g 0 0))))
% \end{verbatim}
% If we erase the trigger, we can show that
% $\{R,F\}$ is unsatisfiable by instantiating $F$ with the substitutions
% $\{ x \mapsto 1, y\mapsto 0\}$ and $\{ x \mapsto 0, y\mapsto 0 \}$.
% Unfortunately, the trigger \texttt{(g (f x) y)} prevents these
% instantiations since neither \texttt{(g (f 1) 0)} nor \texttt{(g (f 0)
%   0)} are known from $R$.
% 
% We can show that \texttt{(g x y)} is too permissive by exhibiting a
% \emph{matching loop}, \ie an infinite chain of instantiations. As
% soon as a term \texttt{(g t1 t2)} is known, the trigger \texttt{(g x
%   y)} allows $F$ to be instantiated and to produce the term \texttt{(g
%   (f t1) t2)}, then \texttt{(g (f (f t1)) t2)} etc.
% 
% A good trigger for $F$ could be the two terms \texttt{(g x y)} and
% \texttt{(f x)} together. Demonstrating that such a trigger is neither too
% permissive nor too restrictive requires a proper framework.

\beforesec
\section{\label{labeled-calculus}First-Order Logic with Triggers and Witnesses}
\aftersec

In this section, we extend classical first-order logic, denoted
$\FOL$, with constructions to specify instantiation patterns and
known terms. The semantics of this extension, denoted $\FOLT$,
is defined through an encoding into usual first-order logic.
In the rest of the article, we write formulas in standard
mathematical notation.

\beforesub
\subsection{\label{syntax} Syntax}
\aftersub
Informally, a trigger is a guard that prevents the usage of a formula until
the requested term is known. We write it $[t]\,F$, which should be read
\emph{if the term $t$ and all its sub-terms are known then assume $F$}.
Note that we do not require a trigger
to be tied to a quantifier. We separate the actual instantiation of a
universal formula from the decision to use its result.

A dual construct for $[t]~F$, which we call \emph{witness}, is written
$\langle t\rangle~F$ and is read \emph{assume that the term $t$ and all
its sub-terms are known and assume $F$}.
This new construction explicitly updates the set of known terms,
something for which there is no proper syntax in existing languages.

The extended syntax of formulas can be summarized as follows:
\begin{equation*}
   F ::= A \sep F_1\land F_2 \sep F_1\lor F_2 \sep \forall x.~F \sep \exists
   x.~F \sep \langle t\rangle~F \sep [t]~F\sep\neg F
\end{equation*}
We treat implication ($\to$) and equivalence ($\leftrightarrow$) as
abbreviations, in a standard fashion.

\beforesub
\subsection{\label{semantics}Denotational Semantics}
\aftersub

We define the semantics of our language via two encodings $\Fp{~}$ and
$\Fm{~}$ into first-order language, given in Fig.~\ref{fig-sem}.
The notation $\llb~\rrb^\pm$ is used when the rule is the same for
both polarities and the polarity of the sub-formulas does not change.
We introduce a fresh unary predicate symbol $\K$ which denotes the
fact that a term is known.
Given a term $t$ or an atomic formula $A$, we denote with $\terms(t)$
(respectively, $\terms(A)$) the set of all the non-variable sub-terms of $t$
(resp.~$A$). The expression $\K(\terms(t))$ stands for the conjunction
$\bigwedge_{t'\in\terms(t)}\K(t')$.

\begin{figure}[tb]
\begin{align*}
 \llb F_1\land F_2\rrb^\pm &\eqdef \llb F_1\rrb^\pm\land \llb F_2\rrb^\pm &
 \llb \langle t\rangle~F\rrb^\pm &\eqdef \K(\terms(t))\land\llb F\rrb^\pm\\
 \llb F_1\lor F_2\rrb^\pm &\eqdef \llb F_1\rrb^\pm\lor\llb F_2\rrb^\pm &
 \llb [t]~F\rrb^\pm &\eqdef \K(\terms(t)) \to\llb F\rrb^\pm\\
 \llb \forall x.~F\rrb^\pm &\eqdef \forall x.~\K(x)\to\llb F\rrb^\pm &
 \llb \exists x.~F\rrb^\pm &\eqdef \exists x.~\K(x)\land\llb F\rrb^\pm\\
 \Fp{\neg F} &\eqdef\neg \Fm{F}& \Fm{\neg F} &\eqdef \neg \Fp{F}\\
 \Fp{A} &\eqdef \K(\terms(A))\to A &
 \Fm{A} &\eqdef \K(\terms(A))\land A
\end{align*}
\caption{\label{fig-sem}Semantics of $\FOLT$-formulas
($\llb\,\rrb^\pm$ denotes either $\Fp{\,}$ or $\Fm{\,}$)}
\end{figure}

We require universally quantified formulas to be instantiated
with known terms. This is consistent with the standard use of triggers:
indeed, SMT solvers require (or compute) a trigger containing each
quantified variable for every universal quantifier.
Then every term that replaces a universally quantified variable is
necessarily known, since sub-terms of a known term are known, too.
Dually, every existentially quantified variable is assumed to be
known. This is necessary in order to allow instantiation
with a witness from an existential formula.

To maintain the invariant that the sub-terms of a known term are
also known, our interpretation of $\la t\ra~F$ implies the presence
of every non-variable sub-term of $t$ (the presence of variables
is assured by interpretation of the quantifiers). Dually, $[t]~F$
requires the presence of every non-variable sub-term of $t$;
due to the mentioned invariant, this is not a restriction.

Finally, whenever we encounter an atomic formula, regardless
of its polarity, we assume the presence of its sub-terms.
This is also in agreement with the standard use of triggers.

We define entailment in $\FOLT$ as follows:
$$F \iFOLT G \quad \eqdef \quad \K(\bb),\,\Fm{F}\iFOL\Fp{G}$$
where $\bb$ is an arbitrary fresh constant supposed to be known
\emph{a priori}, and $\iFOL$ stands for entailment in $\FOL$.

A peculiar aspect of $\FOLT$ is the cut rule is not admissible in it.
Indeed, one cannot prove
$\forall x\,.\,[f(x)]\:P(f(x)),\,
\forall x\,.\,[f(g(x))]\:(P(f(g(x))) \to Q(x)) \,\iFOLT\, Q(c)$,
since the term $f(g(c))$ is not known and neither of the premises
can be instantiated.
However, $Q(c)$ is provable via an intermediate lemma
$P(f(g(c))) \to Q(c)$.

\beforesub
\subsection{\label{ssec:example} Example}
\aftersub

Consider the following set of formulas $R$ from the previous section:
\[R=\left\{\ f(0) \approx 0,\quad f(1) \not\approx 1,\quad \forall
x.[f(x+1)]~f(x+1) \approx f(x)+1\ \right\}\]

We want to show that $R$ is unsatisfiable in $\FOLT$, that is to say,
$R\iFOLT\bot$. By definition, we have to prove that
$\K(\bb),\,\Fm{R}\iFOL\bot$.
\[
\Fm{R} =
\left\{\begin{array}{l}
    \K(\terms(f(0)\approx 0)) \wedge f(0)\approx 0, \\[0.3em]
    \K(\terms(f(1)\not\approx 1)) \wedge f(1) \not\approx 1, \\[0.3em]
    \forall x.~ \K(x) \to \K(\terms(f(x+1))) \to {} \\
    \qquad  \K(\terms(f(x+1)\approx f(x)+1)) \wedge f(x+1)
\approx f(x)+1\end{array}\ \right\}\]

The set of formulas $\Fm{R}$ is unsatisfiable in first-order logic with
arithmetic. Therefore, in our framework, the initial set $R$ is
unsatisfiable.

\beforesub
\subsection{The Extension of First-Order Logic is Conservative}
\aftersub

Even if a formula does not contain triggers or witnesses,
our encoding modifies it to restrict instantiation of universal formulas.
However, it preserves satisfiability
of formulas in classical first-order logic with equality.

\begin{theorem}[Soundness]
For every first-order formula $F$, if
we have $F\iFOLT\bot$, then we also have $F\iFOL\bot$.
\end{theorem}
\begin{proof}
 Since $\K$ is a fresh predicate symbol, for every model $M$ of $F$, there is a
model $M'$ of $F$ such that $M'$ only differs from $M$ in the interpretation of
$\K$ and $M'\iFOL\forall x.\K(x)$. By immediate induction, $(\forall
x.\K(x))\land
F\iFOL\Fm{F}$. As a consequence, $M'$ is a model of $\Fm{F}$ and
$F\not\iFOLT\bot$. Thus, by contra-position, if $F\iFOLT\bot$ then there is no
model of $F$ and $F\iFOL\bot$.
\end{proof}

\begin{theorem}[Completeness]\label{thm-folt}
For any first-order formula $F$, if $F\iFOL\bot$ then $F\iFOLT\bot$.
\end{theorem}

The proof, based on inference trees in a certain paramodulation calculus, can
be found in the technical report~\cite{tech-report}.

\beforesec
\section{\label{prover}Adding a Customizable Theory to a SMT Solver for Ground
Formulas}
\aftersec

In this section, we define a wrapper over a generic SMT solver for ground
formulas that accepts a theory written as a set of formulas with triggers. This
solver is a theoretical model
and it is not meant to be efficient. We prove it sound with respect to our
framework.

It is easy to
show that conversion to NNF does not change the semantics of a $\FOLT$-formula.

\begin{definition} We define a \emph{skolemization transformation} $\sko_T$
for $\FOLT$-for\-mu\-las in negative normal form.
Given a formula $F=\exists x.\,G$, we have
$\sko_T(F)\eqdef\la c(\overline y)\ra~\sko_T(G[x\leftarrow c(\overline y)])$,
where $\overline y$ is the set of free variables of $F$, and $c$ is a fresh
function symbol.
\end{definition}
We put the witness $\la c(\overline y)\ra$ to
preserve satisfiability. Indeed, $\sko(\exists x.\,[x]\,\bot)$
is $[c]\,\bot$ which is satisfiable, while $\exists x.\,[x]\,\bot$ is not.
In the following, we work with $\FOLT$-formulas in Skolem negative normal form.

\beforesub
\subsection{A Solver for Ground Formulas}
\aftersub

To reason about $\FOLT$-formulas, we use a solver $\GS$ for ground formulas.
\begin{definition} We denote implication over ground formulas with theories
$\vdash_o$ to distinguish it from implication in first-order logic with theories
$\iFOL$.
\end{definition}

We make a few assumptions about the interface of the ground solver $\GS$:
\begin{itemize}
 \item It returns \emph{Unsat$(U)$} when called on an unsatisfiable set of
ground
formulas $R$ where $U$ is an unsatisfiable core of $R$. We assume that $U$ is a
set of formulas that occur in $R$ such that $R\vdash_o U$
and $U\vdash_o\bot$.
 \item It returns \emph{Sat$(M)$} when called on a satisfiable set of ground
formulas $R$ where $M$ is a model of $R$. We assume that $M$ is a set of
literals of $R$ such that $M\vdash_o R$.
\end{itemize}
We write $R\rightsquigarrow_\GS\mathit{Unsat}(U)$
(resp.~$R\rightsquigarrow_\GS \mathit{Sat}(M)$)
to express that the solver $\GS$ returns
\emph{Unsat$(U)$} (resp.~\emph{Sat$(M)$})
when launched on a set of ground formulas $R$.

\beforesub
\subsection{Deduction Rules for First-Order Formulas with Triggers}
\aftersub

The solver $\Lift$ takes a set of formulas with triggers $T$ and a
set of ground formulas $S$ as input and decides whether $S$ is satisfiable
modulo $T$. It is constructed on top of a
solver for ground formulas $\GS$ and works on a set of ground formulas $R$
that is augmented incrementally. While the solver $\GS$ returns a model $M$
of $R$, new facts are deduced from $M$ and added to $R$.

The set $R$ initially contains the formulas from the input $S$ as well as those
from the theory $T$ where literals,
quantified formulas, and formulas with triggers or witnesses are replaced
by fresh atoms. The atom
replacing a formula $F$ is written $\boxed{F}$ and is called a
\emph{protected formula.}
\begin{definition} We say that a model $M$ \emph{produces} a pair $F,t$ of a
formula $F$ and a term $t$ if either $F$ is the atom $\top$ and there is a
literal $L$ in $M$ from $S$ such that $t\in\terms(L)$, $F$ is a protected
witness $\boxed{\langle s\rangle~G}\in M$ and $t\in\terms(s)$, or
$F$ a protected literal $\boxed{L}\in M$ and $t\in\terms(L)$. We write it
$M\uparrow F,t$.
\end{definition}
The following deduction rules are used to retrieve information from
the protected formulas of a model $M$:

\begin{small}
\[\begin{array}{c}
 \inferrule[Pos Unfold]{\boxed{\langle t\rangle~F}\in M}
{\boxed{\langle t\rangle~F}\to F}\hspace{2cm}
 \inferrule[Lit Unfold]{\boxed{L}\in M}
{\boxed{L}\to L}
\\[1em]
 \inferrule[Neg Unfold]{\boxed{[t]~F}\in M\\ M\uparrow G,t'\\
M\cup\{t\not\approx
t'\}\rightsquigarrow_\GS\mathit{Unsat}(U\cup\{t\not\approx t'\})}
{\boxed{[t]~F}\wedge G\wedge U\to F}\\[1em]
 \inferrule[Inst]{\boxed{\forall x.~F}\in M\\ M\uparrow G,t\\
M\cup\{\neg F[x\leftarrow t]\}\rightsquigarrow_\GS\mathit{Sat}(M')
}
{\boxed{\forall x.~F}\wedge G\to F[x\leftarrow
t]}\end{array}
\]
\end{small}

Rule {\sc Inst} adds to $R$ an instantiation of a universal
formula with a known term. It is restricted by the premise $M\cup\{\neg
F[x\leftarrow t]\}\rightsquigarrow_\GS\mathit{Sat}(M')$ so that it does not
instantiate a quantified formula if the result of the instantiation is already
known. Rule {\sc Pos Unfold} (resp.~{\sc Lit Unfold}) unfolds
formulas with witnesses (resp.~literals).
Rule {\sc Neg Unfold} removes a trigger when it is equal to a
known term. Note that every deduction rule returns an implication: in a model
where, say, $\la t\ra~F$ is not true, $F$ does not need to be true either.

The solver for $\FOLT$-formulas $\Lift$ returns \emph{Unsat} on $R$, as
soon as $\GS$ returns \emph{Unsat} on the current set of formulas.
It returns \emph{Sat} on $R$ if the ground solver $\GS$ returns a model $M$
from which nothing new can be deduced by the above deduction rules.

Here is an example of execution of the solver $\Lift$ on the set of ground
formulas $S$ modulo the theory $T$:
\[S =\{f(0)\approx 0,\ f(1)\not\approx 1\}\]
\[T =\{\forall x.[f(x+1)]~f(x+1)\approx f(x)+1\}\]
Let us show how the solver $\Lift$ can deduce that
\[R_0 = \left\{\begin{array}{l}
               f(0)\approx 0,\ f(1)\not\approx 1,\\
               \boxed{\forall x.[f(x+1)]~f(x+1)\approx f(x)+1}
              \end{array}\right\}
\]
is unsatisfiable.
\begin{enumerate}
 \item The ground solver returns the only possible model $M_0$ of $R_0$, namely
$R_0$ itself. Since $f(0)\approx 0\in M_0$, $M_0$ produces the pair $\top,0$. As
a consequence, the rule {\sc Inst} can instantiate $x$ with $0$ in the universal
formula:
\[R_1 = R_0\cup\left\{\begin{array}{l}
\boxed{\forall
x.~[f(x+1)]~f(x+1)\approx f(x)+1}\wedge\top\to {}\\
\hspace{1cm}\boxed{[f(0+1)]~f(0+1)\approx f(0)+1}
\end{array}\right\}\]
 \item The solver returns the model $M_1 =
M_0\cup\{\boxed{[f(0+1)]~f(0+1)\approx f(0)+1}\}$ of $R_1$. Since
$f(1)\not\approx 1\in M_1$, $M_1$ produces the pair $\top,f(1)$.
Based on results from the
theory of arithmetics, the ground solver can
deduce that $f(0+1)\not\approx f(1)$ is unsatisfiable. Thus the rule
{\sc Neg Unfold} can add another formula to $R_1$:
\[R_2 = R_1\cup\left\{\begin{array}{l}
\boxed{[f(0+1)]~f(0+1)\approx f(0)+1}\wedge\top\to {}\\
\hspace{1cm}\boxed{f(0+1)\approx f(0)+1}
\end{array}\right\}\]
 \item The ground solver returns the model
$M_2 =M_1\cup\{\boxed{f(0+1)\approx f(0)+1}\}$ of $R_2$. The rule {\sc
Lit Unfold} can
now unfold the protected literal $\boxed{f(0+1)\approx f(0)+1}$:
\[R_3 = R_2\cup\{\boxed{f(0+1)\approx f(0)+1}\to f(0+1)\approx f(0)+1
\}\]
 \item Any model of
$R_3$ contains $f(0+1)\approx f(0)+1$, $f(0)\approx 0$ and $f(1)\not\approx 1$.
The ground solver
returns \emph{Unsat$($\_$)$} on $R_3$. As expected, the initial set $S$ is
reported to be unsatisfiable modulo $T$.
\end{enumerate}

\beforesub
\subsection{Properties}
\aftersub

In this section, we prove that our solver is sound and complete on a particular
class of axiomatics. In the following section, we demonstrate on an example how
our framework can be used to check that a given axiomatics is in this class.
\begin{description}
\item[Completeness] We say that a set of formulas with triggers $T$ is
\emph{complete} if, for every finite set of literals $G$,
$\Fm{G\cup T}$ and  $G\cup T$, triggers being ignored, are equisatisfiable in
$FOL$.
\item[Termination] We say that a set of formulas with triggers $T$ is
\emph{terminating} if, from every finite set of literals $G$,
there can only be a finite number of instances of formulas of $T$. In our
framework, we enforce three rules to enable reasoning about termination:
\begin{itemize}
 \item instantiation is always done with known terms
 \item new known terms cannot be deduced if they are protected by a trigger
 \item an instance of a formula $F$ with a term $t$ is not generated if an
instance of $F$ has already been generated with $t'$ equal to $t$.
\end{itemize}
\end{description}
Our solver is sound and complete if it works modulo a complete and terminating
theory $T$:
\begin{theorem} \label{unsat}
If $\Lift$ returns \emph{Unsat} on a set of ground
formulas $S$ modulo a theory $T$ then $S\cup T$, triggers being
ignored, is unsatisfiable in $\FOL$.
\end{theorem}
\begin{theorem} \label{sat}
If $\Lift$ returns \emph{Sat} on a set of ground
formulas $S$ modulo a complete theory $T$ then $S\cup T$, triggers being
ignored, is satisfiable in $\FOL$.
\end{theorem}
\begin{theorem} \label{termination}
If the theory $T$ is terminating, then the solver $\Lift$ will terminate on
any set of ground literal $S$.
\end{theorem}
The proofs of these three theorems may be found in the technical
report~\cite{tech-report}.

\beforesec
\section{\label{case-study}Completeness and Termination of a theory}
\aftersec

Within our framework, we can reason about a theory $T$ written as a set of
formulas with triggers and demonstrate that it has the requested properties for
our solver $\Lift$ to be sound and complete. This section demonstrates how it
can be done on an axiomatization of the non-extensional theory of arrays.

\setcounter{equation}{0}
% review 3 :The usefulness of the array case study is a bit questionable, since
% it mostly rephrases proofs that have already been given in other contexts.
We show that Greg Nelson's
proof of completeness for his decision procedure for
arrays~\cite{nelson-arrays} can be turned into a proof of completeness of our
solver on an axiomatization with carefully chosen triggers.
%
Another example is given in the technical
report~\cite{tech-report}.
For terms $a$, $x$ and $v$,
we write $\access(a,x)$ the $\access$ in the array $a$ at the index $x$ and
$\update(a,x,v)$ the $\update$ of the array $a$ by the element $v$
at the index $x$. The following set of first-order formulas $T$ is an
axiomatization of the classical theory from McCarthy:
\begin{eqnarray}
&&\forall a, x, v.~[\update(a,x,v)]~\access(\update(a,x,v),x)\approx v\\
&&\forall a, x_1, x_2, v.~[\access(\update(a,x_1,v),x_2)]~x_1\not\approx
x_2\to\nonumber \\
&&\hspace{4cm}\access(\update(a,x_1,v),x_2)\approx \access(a,x_2)\\
&&\forall a, x_1,
x_2, v.~[\access(a,x_2)]~[\update(a,x_1,v)]~x_1\not\approx
x_2\to\nonumber \\
&&\hspace{4cm}\access(\update(a,x_1,v),x_2)\approx \access(a,x_2)
\end{eqnarray}
Note that (2) and (3) are in fact duplications of the same first order
formula with different triggers\footnote{Most provers have a dedicated syntax
for using several triggers for the same axiom.}. Both of them are needed for
completeness.
For example, without (2) (resp.~(3)), the set of formulas
$\{y\not\approx x,\ \access(\update(a,y,v_1),x)\not\approx
\access(\update(a,y,v_2),x)\}$ (resp. the set of
formulas $\{y\not\approx x,\ \access(a_1,x)\not\approx\access(a_2,x),\
\update(a_1, y,v)=\update(a_2,y,v)\}$) cannot
be proven unsatisfiable.

We prove that this axiomatics is complete and terminating.
\paragraph{Termination:}
If $G$ is a set of ground literals, there can only be a finite number of
instances from $G$ and $T$. From (1), at most
one $\access$ term $\access(\update(a,x,v),x)$ can be created per
\emph{update} term $\update(a,x,v)$ of $G$. No new $\update$ term
can be created, so there will be only one instantiation of (1) per $\update$
term of $G$. Equations (2) and (3) can create at
most one $\access$ term per couple comprising an index term (sub-term of an
\emph{access} term at the rightmost position) and an $\update$ term. We deduce
that at most one term per couple comprising the equality classes of an index
term and an $\update$ term of $G$ can be deduced.
\paragraph{Completeness:}
The set of formulas $T$ gives a complete axiomatics. We prove that for every
set of ground formulas $G$ such that $\Fm{G\cup T}$
is satisfiable, $\Fm{G\cup T}\cup\forall t.\K(t)$ is also satisfiable. Since
assuming $\K(t)$
for every term $t$ removes triggers and witnesses, this shows that $G\cup T$
is satisfiable, triggers being ignored.

The proof is similar to the proof of Greg Nelson's decision procedure for
arrays~\cite{nelson-arrays}. We first define the set of every array term $a'$
such that $\access(a',x)$ is equated to a given term $\access(a,x)$ by (2) or
(3):
\begin{definition} For a set of formulas $S$ and two terms $a$ and $x$ known in
$S$, we define the set $S_{a,x}$ to be the smallest set of terms
containing $a$ and closed by
\begin{enumerate}[(i)]
 \item if $a'\in S_{a,x}$ then every known term
$\update(a',y,v)$ such that $S\nvdash y \approx x$ is in $S_{a,x}$ and
 \item for
every term $\update(a',y,\_)\in S_{a,x}$,
if $S\nvdash y \approx x$ then $a'$ is in $S_{a,x}$.
\end{enumerate}
\end{definition}

We now prove that, for every $\access$ or $\update$ term $t$, if $S$ is a
satisfiable set of ground formulas saturated by $T$ then it can be
extended to another satisfiable set $S'$ saturated by $T$ that contains $t=t$.
Since, by definition of $\Fm{\,}$, $\Fm{t=t}$ is equivalent to
$\bigwedge_{t'\in\terms(t)}\K(t)$, this is enough to have the completeness of
$T$.

This proof is an induction over the size of $t$. We assume that every
sub-term of $t$ has already been added. If $\K(t)$ is already implied by
$\Fm{S}$, then we are done. If $t$ is neither an $\access$ nor an $\update$
term, then assuming the presence of $t$ does not allow any new deduction.

Assume $t$ is an $\update$ term $\update(a,x,v)$. With the presence of $t$,
(1) deduces the literal $\access(t, x)\approx v$. This
cannot lead to an inconsistency since nothing can be known about $t$ otherwise
$t$ would be known in $\Fm{S}$.
Equations (2) and (3) deduce $\access(t, y)=\access(a', y)$ for all terms $a'$
and $y$ such that $\Fm{S}\vdash\K(\access(a',y))$ and $t\in S_{a',y}$. Like the
first one, this
deductions cannot cause an inconsistency. The new set $S'$ obtained by adding
these literals to $S$ is saturated by $T$. Indeed, if it is not, one of the
formulas of $T$ can deduce something that is not in $S'$. It cannot be (1) since
we have applied it to the only new $\update$ term of $S'$. If it is (2) or (3)
then it comes from a term $\access(a',y)\in S'$ and $S'_{t,y}=S'_{a',y}$. By
construction of $S'$, the result is in $S'$.

Assume $t$ is an $\access$ term $\access(a,x)$. With the presence of $t$,~(2)
and~(3) deduce $t=\access(a', x)$ for every $a'\in S_{a,x}$. This
deduction cannot cause an inconsistency. Indeed, nothing can be known about
$\access(a',x)$ otherwise $t$ would have been known in $S$ by~(2) and~(3). The
new set
$S'$ obtained by adding these literals to $S$ is saturated by $T$. Indeed, if it
is not, one of the formulas of $R$ can deduce something that is not in $S'$. It
cannot be (1) since there is no new $\update$ term in $S'$. If it is (2) or (3)
then it comes from a term $\access(a',y)\in S'$ and $S'_{a,y}=S'_{a',y}$. By
construction of $S'$, the result is in $S'$.

\beforesec
\section{Conclusion}
\aftersec

We have presented a new first-order logic with a syntax for triggers and given
it a clear semantics. We have shown that a solver accepting a theory written as
a set of formulas with triggers $T$ can be implemented on top of an
off-the-shelf SMT solver, and we have identified properties requested from $T$
for the resulting solver to be sound and complete on ground formulas. Finally,
we have demonstrated, on the
non-extensional theory of arrays, that our framework can be used to
prove that a theory expressed as a set of first-order formulas with triggers
indeed has the requested properties.

In future work, we would like to integrate our technique of quantifier
handling directly inside a DPLL(T)-based solver. Once a solver implementing our
semantics exists, a
static analysis could be done to detect too restrictive or too
permissive axiomatizations, and matching loops. We believe that such an
analysis will help theory designers avoid common pitfalls when writing
axiomatizations.


\beforesec
\bibliographystyle{plain}
\bibliography{triggers_SMT_2012}
\end{document}
